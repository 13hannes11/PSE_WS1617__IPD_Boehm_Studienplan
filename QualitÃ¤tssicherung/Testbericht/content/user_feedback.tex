\FloatBarrier
\section{Nutzer-Feedback}

Während der Qualitätssicherungsphase wurden einige externe Nutzer von uns zur Bedienbarkeit der Software befragt. Die hierbei gesammelten Verbesserungsvorschläge sind im Folgenden aufgelistet.

\begin{longtable}{| >{\hspace{0pt}} p{.80\textwidth} | >{\hspace{0pt}} p{.19\textwidth} | }
	\hline
	\textbf{Vorschlag} & \textbf{Status} \\ 
	\hhline{|=|=|}
	\endfirsthead
	\endhead
	%%
	Bei gedrückter Maus visuelle Rückmeldung für den Nutzer, wohin er Module ziehen kann 
	& ERLEDIGT \\
	\hline
	GUI-Elemente mit Tab-Taste navigierbar machen 
	& ABGELEHNT (zu aufwändig) \\
	\hline 
	Beschreibungstext für Hauptansicht 
	& ERLEDIGT \\
	\hline
	Verbesserte Fehlermeldung im Falle doppelter Modulplatzierung
	& ERLEDIGT \\
	\hline
	Aktive Filter sollten durch eine leichte KIT-farbene Blässe hervorgehoben werden
	& ABGELEHNT (zu aufwändig) \\
	\hline
	Orientierungsprüfungen bei Überprüfung/Generierung berücksichtigen; Bachelor-Thesis erst zum Schluss erlauben
	& ABGELEHNT (In Datensatz nicht vorgesehen) \\
	\hline
	Hilfe-Funktion, die dem Nutzer Anleitungen für das Produkt bereitstellt: Anwendung zwar übersichtlich, aber es fehlt anfangs der Ansatz, was als nächstes zu tun sei.
	& ERLEDIGT \\
	\hline
	Plannamen anklickbar machen (daraufhin Anzeigen des Plans)
	& ERLEDIGT \\
	\hhline{|=|=|}
\end{longtable}