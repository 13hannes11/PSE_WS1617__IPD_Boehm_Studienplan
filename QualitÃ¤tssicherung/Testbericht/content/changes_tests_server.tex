\FloatBarrier
\subsection{Server}

\subsubsection*{Unit-Tests}

\begin{longtable}{| >{\hspace{0pt}} p{.26\textwidth} | >{\hspace{0pt}} p{.45\textwidth} | >{\hspace{0pt}} p{.19\textwidth} |}
	\hline
	\textbf{Testklasse} & \textbf{Beschreibung} & \textbf{Status} \\ 
	\hhline{|=|=|=|}  
	\endfirsthead
	\endhead
	CategoryTest & Getter und Setter für \texttt{Category} getestet & ERFOLGREICH \\
	\hline
	DisciplineTest & Getter und Setter für \texttt{Discipline} getestet & ERFOLGREICH \\
	\hline
	SemesterTest & Test der Semester-Abstandsberechnung zwischen:
	\begin{itemize}
		\item zwei Wintersemestern
		\item zwei Sommersemestern
		\item zwischen Winter- und Sommersemester
		\item zwischen Sommer- und Wintersemester
	\end{itemize}
	Test der compareTo-Methoden & ERFOLGREICH \\
	\hline
	StandardVerifierTest & Testet, ob der \texttt{StandardVerifier} folgendes erkennt:
	\begin{itemize}
		\item fehlende Pflichtmodule
		\item Verletzung von Rule-Group"=Bedingungen
		\item Verletzung von Bereichsbedingungen (Fields)
	\end{itemize} & ERFOLGREICH \\
	\hline
	StandardVerifierConstraintTest & Testet ob der die Verletzung folgender \texttt{ConstraintTypes} in unterschiedlichen Ausartungen erkennt:
	\begin{itemize}
		\item Verletzung von Überlappung 
		\item Verletzung von Plan"=Zusammengehörigkeit 
		\item Verletzung von Semester"=Zusammengehörigkeit 
		\item Verletzung von Voraussetzungen
	\end{itemize}
	Dadurch wurden auch die \texttt{isValid}-Methoden der entsprechenden \texttt{ConstraintTypes} getestet & ERFOLGREICH \\
	\hline
	SimpleGeneratorTest & Testet einzelne Methoden des \texttt{SimpleGenerator}. 
	Ein Plan mit einem einzelnen \texttt{ModuleEntry} wird:
	\begin{itemize}
		\item anhand der Benutzer"=Präferenzen und \texttt{Constraint}s verschiedener Art vervollständigt und modifiziert
		\item anhand einer Zielfunktion optimiert
	\end{itemize}
	Dafür werden sog. Mock-Objekte benutzt, die nur zum Testen erstellt wurden, d.h. bei der echten Verwendung des Systems werden diese nicht benutzt werden, sondern stattdessen die Daten aus der Datenbank. Das Verhalten von Methoden der Klassen \texttt{ModuleDao} und \texttt{Plan} wurde hierfür mittels Mockito angepasst. & ERFOLGREICH\\
	\hline
	NodesListTest & Testet die topologische Sortierung der Klasse \texttt{NodesList}. & ERFOLGREICH\\
	\hline
	VerificationManager & Getter für \texttt{Verifier} getestet. & ERFOLGREICH\\
	\hline
	ConditionTest & Factory-Methoden getestet & ERFOLGREICH \\
	\hline
	ContainsFilterDescriptorTest, ListFilterDescriptorTest, RangeFilterDescriptorTest, FilterDescriptorTest,  & Korrekte JSON-Serialisierung, Parsen der Filterdaten aus einem Request getestet; dabei auch ungültige Eingaben überprüft & ERFOLGREICH \\
	\hline
	FilterDescriptorProviderTest & Invarianten des Providers getestet (untersch. URI"=Identifier u. IDs, jeder veröffentlichte Filter kann mit dem Identifier gefunden werden) & ERFOLGREICH \\
	\hline
	MultiFilterTest & Korrekte Listenglättung der Sub-Condition-Listen getestet & ERFOLGREICH \\
	\hline
	TrueFilterTest & Rückgabe einer leeren Condition-Liste getestet & ERFOLGREICH \\
	\hhline{|=|=|=|}  
\end{longtable}

\subsubsection*{Integrations-Tests}

\begin{longtable}{| >{\hspace{0pt}} p{.26\textwidth} | >{\hspace{0pt}} p{.45\textwidth} | >{\hspace{0pt}} p{.19\textwidth} |}
	\hline
	\textbf{Testklasse} & \textbf{Beschreibung} & \textbf{Status} \\ 
	\hhline{|=|=|=|}  
	\endfirsthead
	\endhead
	DisciplinesResourceIntegrationTest & Überprüft die korrekte JSON-Serialisierung von GET /disciplines. & ERFOLGREICH \\
	\hline
	FieldsResourceIntegrationTest & Überprüft die korrekte JSON-Serialisierung von GET /fields. & ERFOLGREICH \\
	\hline
	ObjectiveFunctionResourceIntegrationTest & Überprüft die korrekte JSON-Serialisierung von GET /objective-functions. & ERFOLGREICH \\
	\hline
	FilterResourceIntegrationTest & Überprüft die korrekte JSON-Serialisierung von GET /filters (die drei Typen range, list und contains samt zugehöriger Spezifikation). & ERFOLGREICH \\
	\hline
	StudentResourceIntegrationTest & Überprüft, ob der Nutzer beim Neuanlegen korrekt initialisiert ist; das Setzen von Fachrichtung und Studienstart funktioniert; fehlerhafte Fachrichtungen/Studienstarts (zukünftig/prähistorisch) abgefangen werden; bestandene Module korrekt gesetzt werden; fehlerhafte bestandene Module (falsches Modul, falsches Semester) abgefangen werden; die JSON-Rückgaben korrekt serialisiert werden und eine Autorisierung vor dem Zugriff auf die Ressourcen verlangt wird. & ERFOLGREICH \\
	\hline
	ModuleResourceIntegrationTest & Überprüft, ob GET /modules/id korrekt als JSON serialisiert wird und auch die DB-Daten enthält; testet mehrere Filteraufrufe auf korrekte Rückgabe sowie fehlerhafte Eingabe. & ERFOLGREICH \\
	\hline
	
	& & ERFOLGREICH \\
	
	\hhline{|=|=|=|}  
\end{longtable}