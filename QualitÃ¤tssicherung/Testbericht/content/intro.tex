\section{Einleitung}

In diesem Testbericht wird die Qualitätssicherungsphase des Projekts „Studienplanung als Generierung von Workflows mit Compliance-Anforderungen: Planerstellung und Visualisierung“ vorgestellt. Während dieser Phase wurde mithilfe von Testfällen und -szenarien aus dem Pflichtenheft die Qualität der Software überprüft und verbessert.

Zuerst wurden anhand der Anwendungsfälle des Pflichtenhefts Fehler im Produkt und Abweichungen von der Produktdefinition gesammelt und festgestellt. Anschließend wurden diese behoben, wobei aufgrund der geänderten Anforderungen auch Anpassungen an den Anwendungsfällen vorgenommen wurden. Zudem wurden mit weiteren Unit- und Integrationstests die Qualität der Software verbessert. Außerdem haben wir Nutzer-Feedback zur Bedienbarkeit gesammelt, welches wir der Nachwelt nicht vorenthalten möchten.