\FloatBarrier
\subsection{Geänderte Anwendungsfälle}

Durch die während der Entwicklung geänderten Anforderungen erfordern manche Anwendungsfälle zusätzliche Anpassungen, welche hier aufgelistet sind. Gestrichen wurden Anwendungsfälle, deren zugrundeliegende Produktfunktionalität entfernt wurde.

\begin{longtable}{| >{\hspace{0pt}} p{.32\textwidth} | >{\hspace{0pt}} p{.63\textwidth} |}
	\hline
	\textbf{Anwendungsfall} & \textbf{Änderung} \\ 
	\hhline{|=|=|}  
	\endfirsthead
	\endhead
	%%
	A10: Erstanmeldung / \newline A20: Login 
	& Auf Einbindung des Shibboleth Identity Providers wurde (wie angekündigt) verzichtet. \\
	\hline
	A40: Profil bearbeiten
	& Eingabe von Studienbeginn/Studiengang nur bei Erstanmeldung möglich; „Konto speichern“ kehrt nicht zur letzten offenen Seite zurück. Der Nutzer kann außerdem sein Konto löschen. \\
	\hline
	A50: Neuen Studienplan anlegen
	& Das Namens-Popup hat keine Voreinstellung. \\
	\hline
	A70: Studienplan duplizieren
	& Es wird nach einem neuen Namen für den Plan gefragt. (Benutzerfreundlicher) \\
	\hline
	A85: Studienplan exportieren
	& Es wird eine HTML-Zusammenfassung des Studienplans im Browser angezeigt. \\
	\hline
	A90: Mehrere Studienpläne löschen
	& Duplizieren und Teilen wurden entfernt und das Aktions"=Wahlfeld durch eine einzige Schaltfläche ersetzt. \\ 
	\hline
	A100: Studienpläne vergleichen
	& Entfernt. \\
	\hline
	A110: Module filtern
	& Filtern nach Fachrichtung und Vorkommen im Plan entfernt; Filtern nach Bereich hinzugefügt. \\
	\hline
	A130: Modulinfo-Leiste
	& Möglichkeit zur Bewertung entfernt; Modul"=Constraints werden nun angezeigt. \\
	\hline
	A140: Modul einfügen
	& Module können auch in abgeschlossene Semester eingefügt werden. \\
	\hline
	A170: Änderung rückgängig machen
	& Entfernt. \\
	\hline
	A210: Semesterzeile löschen
	& Da Zeilen nicht explizit gespeichert werden und leere Zeilen gar nicht: Entfernt. \\
	\hline
	A215: Abgeschlossene Semester ein-/ausblenden
	& Entfernt. \\
	\hline
	A220: Studienplan überprüfen
	& Es gibt keine Tooltips beim Drüberfahren mit der Maus, stattdessen werden die Verifizierungsergebnisse gesammelt als Popup angezeigt. \\
	\hhline{|=|=|}
\end{longtable}

\subsubsection*{Testergebnisse}

\begin{longtable}{| >{\hspace{0pt}} p{.73\textwidth} | >{\hspace{0pt}} p{.20\textwidth} |}
	\hline
	\textbf{Anwendungsfall} & \textbf{Status} \\ 
	\hhline{|=|=|}  
	\endfirsthead
	\endhead
	%%
	A10: Erstanmeldung & ERFOLGREICH \\
	\hline
	A20: Login & ERFOLGREICH \\
	\hline
	A30: Logout & ERFOLGREICH \\
	\hline
	A40: Profil bearbeiten & ERFOLGREICH \\
	\hline
	A50: Neuen Studienplan anlegen & ERFOLGREICH \\
	\hline
	A55: Studienplan anzeigen / A57: schließen & ERFOLGREICH \\
	\hline
	A60: Plan umbenennen & ERFOLGREICH \\
	\hline
	A70: Plan duplizieren & ERFOLGREICH \\
	\hline
	A80: Plan löschen & ERFOLGREICH \\
	\hline
	A85: Plan exportieren & ERFOLGREICH  \\
	\hline
	A90: Mehrere Studienpläne löschen & ERFOLGREICH \\
	\hline
	A110: Module filtern & ERFOLGREICH  \\
	\hline
	A130: Modulinfo anzeigen & ERFOLGREICH  \\
	\hline
	A140: Modul einfügen & ERFOLGREICH  \\
	\hline
	A150: Modul löschen & ERFOLGREICH  \\
	\hline
	A160: Modul verschieben & ERFOLGREICH  \\
	\hline
	A180/190: Modul positiv/negativ bewerten & ERFOLGREICH  \\
	\hline
	A200: Semesterzeile einfügen & ERFOLGREICH  \\
	\hline
	A220: Plan überprüfen & ERFOLGREICH \\
	\hline
	A230: Plan vervollständigen & ERFOLGREICH \\
	\hhline{|=|=|}
\end{longtable}

