\FloatBarrier
\subsection{Server-Coverage-Daten}

Im Folgenden sind Coverage-Statistiken zur Server-Überdeckung aufgeführt. 

\paragraph{Unit-Tests} Für das rest-Package wurden keine Unit-Tests verfasst, weshalb es hier auch nicht aufgeführt ist; die REST-Schnittstellen werden sinnvollerweise in den Integrationstests abgedeckt. Da der Plugin-Manager weitestgehend kastriert wurde, erschien eine größere Abdeckung nicht mehr nötig; genauso auch wie bei vielen primitiven Gettern und Settern im Model.

\begin{longtable}{| >{\hspace{0pt}} p{.26\textwidth} | >{\hspace{0pt}} p{.45\textwidth} | }
	\hline
	\textbf{Package} & \textbf{Zeilenüberdeckung} \\ 
	\hhline{|=|=|}  
	\endfirsthead
	\endhead
	%%
	filter & 81 \% \\
	\hline
	generation & 67 \% \\
	\hline
	model & 36 \% \\
	\hline
	pluginmanager & 14 \% \\
	\hline
	verification & 85 \% \\
	\hhline{|=|=|} 
	\textbf{Gesamt} & \textbf{57 \%} \\
	\hhline{|=|=|} 
	\caption{Unit-Test-Überdeckung Server}
\end{longtable}

\paragraph{Integrations-Tests} Aufgrund von Schwierigkeiten beim Zusammenspiel von JaCoCo und unserer Tomcat"=Serverumgebung sind für Integrationstests keine Coverage-Daten verfügbar. Als groben Überblick diene stattdessen die manuell bestimmte Methodenüberdeckung für die einzelnen REST"=Ressourcen. Auf Integrationstests für Verifzierung und Generierung wurde verzichtet, da diese schon ausreichend mit Unit-Tests abgedeckt sind.

\begin{longtable}{| >{\hspace{0pt}} p{.26\textwidth} | >{\hspace{0pt}} p{.45\textwidth} | }
	\hline
	\textbf{REST-Ressource} & \textbf{Methodenüberdeckung} \\ 
	\hhline{|=|=|}  
	\endfirsthead
	\endhead
	%%
	/disciplines & 100 \% \\
	\hline
	/fields & 100 \% \\
	\hline
	/filters & 100 \% \\
	\hline
	/modules & 100 \% \\
	\hline
	/objective-functions & 100 \% \\
	\hline
	/plans & 72 \% \\
	\hline
	/student & 100 \% \\
	\hhline{|=|=|} 
	\textbf{Gesamt} & \textbf{ 96 \%} \\
	\hhline{|=|=|} 
	\caption{Integrations-Test-Überdeckung Server}
\end{longtable}