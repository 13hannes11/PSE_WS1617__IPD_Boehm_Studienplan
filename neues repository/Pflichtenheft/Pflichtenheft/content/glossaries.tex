%
% % Automatisch generiertes Glossar
%
%\glsaddall % das sorgt dafür, dass alles Glossareinträge gedruckt werden, nicht nur die verwendeten. Das sollte nicht nötig sein!
%
% % Glossareinträge
%
\newglossaryentry{Rest}
{
	name=REST,
	description={Abk. für Representational State Transfer, Programmierparadigma für \glspl{Webservice} auf Basis des HTTP-Protokolls}
}

\newglossaryentry{ECTS-Punkte}
{
	name=ECTS-Punkte,
	description={Leistungspunkte, die für ein erfolgreich absolviertes \gls{Modul} von der Hochschule auf Basis des ECTS"=Punktesystems vergeben werden. Durch diese wird der Arbeitsaufwand gemessen},
}

\newglossaryentry{Generierungs-Tool}
{
	name=Generierungs-Tool,
	description={Tool, für die automatische Erstellung bzw. Vervollständigung von \glslink{Studienplan}{Studienplänen}, siehe \gls{Generierung}},
}


\newglossaryentry{Webservice}
{
	name=Webservice,
	plural=Webservices,
	description={Softwareanwendung, die über ein Netzwerk bereitgestellt wird}
}

\newglossaryentry{Internetbrowser}
{
	name={Internetbrowser},
	description={Programm, mit dem Websites gefunden, gelesen und verwaltet werden können, mit aktiviertem JavaScript}
}

\newglossaryentry{Online-Shop}
{
	name={Online-Shop},
	description={Internetseite, die Produkte zum Kauf anbietet}
}

\newglossaryentry{KIT}
{
	name=KIT,
	description={Das Karlsruher Institut für Technologie ist die Forschungsuniversität in der Helmholtz"=Gemeinschaft. Hauptstandort der Universität ist Karlsruhe}
}

\newglossaryentry{SCC}
{
	name=SCC,
	plural=SCC,
	description={Das Steinbuch Center for Computing ist ein Institut und das zentrale Rechenzentrum des \gls{KIT}}
}

\newglossaryentry{Benutzer}
{
	name=Benutzer,
	plural=Benutzer,
	description={Ein am \gls{KIT} eingeschriebener Studierender, der über ein gültiges \gls{SCC}-Benutzerkonto verfügt}
}

\newglossaryentry{Wizard}
{
	name=Wizard,
	plural=Wizards,
	description={Ein Wizard ist ein Subsystem, welches einen \gls{Benutzer} visuell durch eine Systemfunktionalität führt und dabei vom \gls{Benutzer} bestimmte Interaktionen mit dem System fordert}
}

\newglossaryentry{Drag-and-Drop}
{
	name=Drag-and-Drop,
	description={(deutsch: „Ziehen und Ablegen“) Eine Methode zur Bedienung grafischer Benutzeroberflächen, bei der grafische Elemente mittels eines Mauszeigers bewegt werden}
}

\newglossaryentry{Shibboleth Identity Provider}
{
	name=Shibboleth Identity Provider,
	description={Ein genau spezifiziertes System zum Login mittels einer von einer dritten Instanz bereitgestellten Identität (in diesem Fall vom \gls{SCC})}
}

\newglossaryentry{Studiengang}
{
	name=Studiengang,
	description={Ein vom KIT angebotener, auf einer Studien"= und Prüfungsordnung und einem Modulhandbuch basierender Studiengang}
}

\newglossaryentry{Semester des Studienbeginns}
{
	name=Semester des Studienbeginns,
	description={Das Semester, in welchem der \gls{Benutzer} im ersten Fachsemester des \glslink{Studiengang}{Studiengangs} immatrikuliert war}
}

\newglossaryentry{Modul}
{
	name=Modul,
	plural=Module,
	description={Ein Modul ist ein Teilblock des Studiums, welchen man \glslink{Modul abgeschlossen}{bestehen} kann und für welchen man nach Ablegung eventueller \glspl{Modulpruefung} eine festgelegte Anzahl an ECTS-Punkten erhält}
}
\newglossaryentry{Modulpruefung}
{
	name=Modulprüfung,
	plural=Modulprüfungen,
	description={Eine Modulprüfung ist eine Prüfung, welche abgelegt werden muss, um ein Modul \glslink{Modul abgeschlossen}{abzuschließen}}
}
\newglossaryentry{Zur Pruefung angetreten}
{
	name=Zur Prüfung angetreten,
	description={Ein \gls{Benutzer} ist zu einer \gls{Modulpruefung} angetreten, wenn er sich fristgerecht für selbige angemeldet und nicht fristgerecht abgemeldet hat}
}

\newglossaryentry{Modul abgeschlossen}
{
	name=Modul abgeschlossen,
	description={Ein \gls{Modul} gilt als abgeschlossen, wenn der \gls{Benutzer} alle nach Modulhandbuch notwendigen \glspl{Modulpruefung} bestanden hat}
}

\newglossaryentry{Studienplan}
{
	name=Studienplan,
	plural=Studienpläne,
	description={Eine Zuordnung von Modulen zu Semestern, in welcher enthalten ist, wann welches Modul planmäßig \glslink{Modul abgeschlossen}{abgeschlossen} werden soll}
}
\newglossaryentry{Weboberflaeche}{
	name={Weboberfl{\"a}che},
	description={Eine auf HTML, CSS und JavaScript basierende, mit einem \gls{Internetbrowser} anzeigbare Benutzeroberfläche}
}
\newglossaryentry{Popup}
{
	name=Popup,
	description={Ein auf der \gls{Weboberflaeche} eingeblendetes Fenster, das vom \gls{Benutzer} Informationen abfragen kann, indem es Knöpfe oder Eingabefelder anbietet}
}

\newglossaryentry{Benachrichtigung}
{
	name=Benachrichtigung,
	plural=Benachrichtigungen,
	description={Eine Benachrichtigung an den \gls{Benutzer}, die auf der \gls{Weboberflaeche} am Seitenrand eingeblendet wird und mit der Maus weggeklickt werden kann bzw. nach vorgegebenem Zeitlimit von selbst verschwindet}
}

\newglossaryentry{Tooltip}
{
	name=Tooltip,
	description={Ein kleines rahmenloses Fenster mit Informationen, das auf der \gls{Weboberflaeche} eingeblendet wird, falls mit der Maus über ein bestimmtes Bedienelement gefahren wird}
}
\newglossaryentry{modular}
{
	name=modular,
	description={Das Aufspalten von Software in austauschbare Teile, die durch klar definierte Schnittstellen verbunden sind. Auch neue Optionen können an einer solchen Schnittstelle mit wenig Aufwand hinzugefügt werden.}
}
	
\newglossaryentry{Generierung}
{
	name=Generierung,
	plural=Generierungen,
	description={Ein \gls{Studienplan} kann "generiert", das heißt automatisiert den \glspl{Constraint} entsprechend vervollständigt werden.}
}

\newglossaryentry{Verifizierung}
{
name=Verifizierung,
plural=Verifizierungen,
description={Ein \gls{Studienplan} kann \enquote{verifiziert} werden. Das heißt, es wird automatisiert geprüft, ob der Studienplan einen erfolgreichen Studienabschluss unter Einhaltung aller gegebenen \glspl{Constraint} ermöglicht}
}

\newglossaryentry{Constraint}
{
name=Constraint,
plural=Constraints,
description={Als Constraints werden die Anforderungen bezeichnet, die ein \gls{Studienplan} erfüllen soll. Diese müssen auch von der \gls{Generierung} berücksichtigt werden. Zwingende Constraints sind die \glspl{System-Constraint}. Optionale (bei der \gls{Generierung} zu optimierende) Constraints sind die \glspl{Nutzer-Constraint} sowie die \gls{Nutzer-Zielfunktion}}
}

\newglossaryentry{System-Constraint}{
	name={System-Constraint},
	plural={System-Constraints},
	description={System-Constraints sind die vom Modul-Handbuch vorgegebenen Abhängigkeiten und Vorgaben, die von jedem \gls{Studienplan} erfüllt werden müssen}
}
\newglossaryentry{Nutzer-Constraint}{
	name={Nutzer-Constraint},
	plural={Nutzer-Constraints},
	description={Nutzer-Constraints sind die vom \gls{Benutzer} für einen spezifischen \gls{Studienplan} vorgegebenen Bedingungen, die der Studienplan bei der \gls{Generierung} erfüllen muss. Hierzu gehört der Wunsch in einem gegebenen Semester eine bestimmte Vorlesung hören zu wollen, aber auch Dinge wie die maximale Anzahl an ECTS-Punkten pro Semester}
}

\newglossaryentry{Nutzer-Zielfunktion}{
	name={Nutzer-Zielfunktion},
	plural={Nutzer-Zielfunktionen},
	description={Die Nutzer-Zielfunktion ist eine den Studienplan nach gegebenen Parametern bewertende Funktion, die bei der \gls{Generierung} von Heuristiken zu optimieren ist}
}

\newglossaryentry{Parallelstudium}
{
name=Parallelstudium,
plural=Parallelstudien,
description={Ein Parallelstudium ist ein Studium, welches zeitgleich zu einem anderen, eigenständigen Studiengang absolviert wird.}
}