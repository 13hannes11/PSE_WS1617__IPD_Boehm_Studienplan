\section{Systemmodelle}
Das System basiert auf einer Thick-Client\footnote{Es wird kein klassischer Thick-Client eingesetzt, sondern vielmehr eine JavaScript-basierte \gls{Weboberflaeche}, die auf viele Nutzeraktionen ohne ein Neuladen der Seite reagieren kann.}-Server-Architektur mit einer starken Trennung zwischen der Benutzerschnittstelle und dem Anwendungsserver. Der \gls{Benutzer} gibt die benötigten Daten über die Benutzerschnittstelle ein. Die Verarbeitung findet serverseitig statt. Die \gls{Weboberflaeche} sendet hierfür eine Anfrage über einen \gls{Rest}-\gls{Webservice} und erhält über diese Schnittstelle eine Antwort zurück. \\
Auf dem Anwendungsserver werden die notwendigen Berechnungen durchgeführt, sowie die Produktdaten verarbeitet und gesichert. \\
Dieser Aufbau fördert die Modularität des Systems, da über die \gls{Rest}-Architektur offene Schnittstellen zur Verarbeitung, Speicherung und Abfrage von Nutzer- und Anwendungsdaten bereitgestellt werden, die auch in Zukunft von alternativen Benutzerschnittstellen, wie beispielsweise eine App, genutzt werden können. Zudem verkürzen sich die Interaktionszeiten des \glslink{Benutzer}{Benutzers}, da aufwendige Berechnungen auf einem leistungsfähigen Server durchgeführt werden können. Durch den Thick-Client werden die Interaktionszeiten weiterhin verkürzt, da Eingabevalidierungen clientseitig ausgeführt werden können. Aufgrund der Leistungsfähigkeit heutiger Geräte beschleunigt dies die Anwendung, denn es muss nicht auf den Server gewartet werden. Zudem entsprechen JavaScript-basierte Webclients den heutigen Industriestandards. \\
Eine Visualisierung dieses Systemmodells findet sich in Abbildung \ref{system_model:overview}.