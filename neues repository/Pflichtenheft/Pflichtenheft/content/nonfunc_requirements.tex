\section{Nichtfunktionale Anforderungen}
\subsection{Muss-Anforderungen}
\begin{itemize}[nosep]
	\item[NF10]
		Die Nutzeroberfläche des Systems muss intuitiv bedienbar sein und auch ohne eine Schulung verwendet werden können.
	\item[NF20]
		Dem Nutzer muss es möglich sein bei der Plan-Erstellung einfach mit gegebenen Suchkriterien nach einem \gls{Modul} zu suchen und dieses dem Plan hinzuzufügen.
	\item[NF30]
		Der \gls{Benutzer} muss jederzeit einen guten Überblick über die Vollständigkeit und Korrektheit seiner Pläne erhalten können.
	\item[NF40]
		Das System muss modular sein: Es muss gut möglich sein das System zukünftig auf weitere Anwendungsfälle zu erweitern.
	\item[NF50]
		Das System muss über eine gute Dokumentation verfügen.
	\item[NF60]
		Die Suche nach Modulen darf bei einer LAN-Internetverbindung innerhalb des KIT-Netzes nicht länger als 800ms benötigen.
	\item[NF70]
		Es darf einer dritten Person (also nur dem \gls{Benutzer} sowie dem Systemadministrator) nicht möglich sein Daten über einen \gls{Benutzer} einzusehen.
	\item[NF80]
		Bei jeder Aktion muss der \gls{Benutzer} eine verständliche Rückmeldung vom System erhalten.
	\item[NF90]
		Das System muss nach dem objektorientierten Programmierparadigma entwickelt werden.
\end{itemize}
\subsection{Kann-Anforderungen}
\label{subsec:nonfunc_requirements-kann}
\begin{itemize}[nosep]
	\item[NF100]
		Dem \gls{Benutzer} muss es möglich sein jederzeit Informationen über ein Modul in einem seiner \glspl{Studienplan} abzurufen.
	\item[NF110]
		Die Suche nach Modulen darf bei einer LAN-Internetverbindung innerhalb des KIT-Netzes nicht länger als 100ms benötigen.
	\item[NF120]
		Das System muss eine optisch ansprechende Benutzeroberfläche besitzen.
	\item[NF130]
		Das serverseitige System muss mittels eines Load-Balancers auf mehrere Server skalierbar sein, um die Last zu verteilen.
\end{itemize}
