% scenario environment
% #1: Scenario title
\newenvironment{scenario}[1]{
	\vspace{-\baselineskip}
	\subsubsection*{#1} 
	\vspace{-\baselineskip}
	\begin{enumerate}[nosep]
}{
	\end{enumerate}
}

% scenario* environment
% #1: Scenario title
% #2: Initial description
\newenvironment{scenario*}[2]{
	\vspace{-\baselineskip}
	\subsubsection*{#1} \vspace{-\baselineskip}
	\hspace{0pt}#2  \vspace{-\baselineskip}
	\begin{enumerate}[nosep]
}{
	\end{enumerate}
}

%%%%%%%%%%%%%%%%%%%%%%%%%%%%%%%%%%%%%

\FloatBarrier
\subsection{Geänderte Testszenarien}

Im Folgenden eine Auflistung der konsequenterweise ebenfalls angepassten Testszenarien.

\begin{scenario}{T1: Erststart mit „halbherziger Bedienung“}
	\item A10: Erstanmeldung (ohne Angabe bereits bestandener {Module})
	\item A50: Neuen Studienplan anlegen
	\item A220: Studienplan auf Korrektheit überprüfen – Ergebnis „fehlerhaft“ (da unvollständig)
	\item A230: Studienplan vervollständigen lassen mit anschließendem Verwerfen
	\item A57: Schließen der Studienplan-Ansicht
	\item A80: Studienplan löschen 
	\item A30: Logout
\end{scenario}

\begin{scenario}{T2: Einfache Vervollständigung}
	\item A20: Login
	\item A40: Profil bearbeiten – erste zwei Semester anschließend belegt
	\item A50: Neuen Studienplan anlegen
	\item A230: Studienplan vervollständigen lassen mit anschließendem Übernehmen des {Studienplans}
	\item A60: Studienplan umbenennen 
	\item A57: Schließen der Studienplan-Ansicht
	\item A85: Studienplan exportieren
	\item A30: Logout
\end{scenario}

\begin{scenario*}{T3: Bearbeitung eines Studienplans}
	{{Nutzer} ist bereits eingeloggt und hat mind. einen {Studienplan} angelegt.}
	\item A55: Studienplan anzeigen
	\item A110: Module filtern 
	\item A130: Modulinfo anzeigen und wieder schließen
	\item A140: Modul einfügen
	\item A150: Modul löschen
	\item A160: Modul verschieben
	\item A160: Modul negativ bewerten
	\item A180: Modul positiv bewerten (selbes {Modul} – es ist dann positiv bewertet) 
	\item A140: Modul einfügen
	\item A57: Schließen der Studienplan-Ansicht
\end{scenario*}

\begin{scenario*}{T4: Profil bearbeiten}
	{{Nutzer} ist bereits eingeloggt und hat mind. einen {Studienplan} angelegt.}
	\item A55: Studienplan anzeigen
	\item A40: Profil bearbeiten – dabei Änderung der Semester-Belegung
	\item Anschließend sollte die Änderung im Studienplan gezeigt werden.
\end{scenario*}

\begin{scenario}{T5: Vervollständigung mit mehreren Alternativen}
	\item A20: Login
	\item A50: Neuen Studienplan anlegen
	\item A110: Module filtern
	\item A140: Modul einfügen
	\item Schritte 3–4 mehrmals wiederholen, sodass Abhängigkeitsfehler vorhanden sind und der {Studienplan} noch unvollständig ist
	\item A220: Studienplan auf Korrektheit überprüfen: Es werden Abhängigkeitsfehler gemeldet
	\item Mittels \enquote{A150: Modul löschen} und \enquote{A160: Modul verschieben} Abhängigkeitsfehler beheben
	\item A220: Studienplan auf Korrektheit überprüfen: Der Studienplan ist unvollständig
	\item A230: Studienplan vervollständigen lassen mit anschließendem Speichern unter neuem Namen
	\item A57: Schließen der Studienplan-Ansicht
	\item A55: Studienplan anzeigen (den in Schritt 2 erstellten Studienplan)
	\item A230: Studienplan vervollständigen lassen mit anderen Zielkriterien als in Schritt 9 und anschließendem Speichern unter neuem Namen
	\item A57: Schließen der Studienplan-Ansicht
\end{scenario}

\begin{scenario}{T6: Studienpläne duplizieren und löschen}
	\item A20: Login
	\item A50: Neuen Studienplan anlegen
	\item A57: Schließen der Studienplan-Ansicht
	\item A70: Studienplan duplizieren
	\item Mehrmaliges Wiederholen von Schritt 4.
	\item A90: Mehrere Studienpläne löschen: Alle Studienpläne löschen.
\end{scenario}

\subsubsection*{T7: Semester"=Zeilen anpassen}
\vspace{-\baselineskip}
Entfällt. 

\subsubsection*{Testergebnisse}

\begin{longtable}{| >{\hspace{0pt}} p{.73\textwidth} | >{\hspace{0pt}} p{.20\textwidth} |}
	\hline
	\textbf{Testszenario} & \textbf{Status} \\ 
	\hhline{|=|=|}  
	\endfirsthead
	\endhead
	%%
	T1: Erststart mit „halbherziger Bedienung“ & ERFOLGREICH \\
	\hline
	T2: Einfache Vervollständigung & ERFOLGREICH \\
	\hline
	T3: Bearbeitung eines Studienplans & ERFOLGREICH \\
	\hline
	T4: Profil bearbeiten & ERFOLGREICH \\
	\hline
	T5: Vervollständigung mit mehreren Alternativen & ERFOLGREICH \\
	\hline
	T6: Studienpläne duplizieren und löschen & ERFOLGREICH \\
	\hhline{|=|=|}
\end{longtable}