% ---------------------------------------------------------------------------
% TexDoc macros start - everything below this point should be copied to your
% own document and adapted to your style/language if needed
% ---------------------------------------------------------------------------


% Environment used to simulate html <p> </p>
\newenvironment{texdocp}{}{

}
% Environment for packages
\newenvironment{texdocpackage}[1]{%
	%\newpage{}
	%\gdef\packagename{#1}
	\subsection*{\Large Package \texttt{#1}}
	\addcontentsline{toc}{subsection}{Package #1}  % only temporarily
	%\rule{\hsize}{.7mm}
}{}

% Environment for classes, interfaces
% Argument 1: "class" or "interface"
% Argument 2: the name of the class/interface
\newenvironment{texdocclass}[2]{%
	%\gdef\classname{#2}
	\subsubsection*{\Large \texttt{#1 \textbf{#2}}}
}{
%\newpage{}
}

% Environment for package description
\newenvironment{texdocpackageintro}{
	\paragraph{Beschreibung}
}{
}

% Environment for class description
\newenvironment{texdocclassintro}{%\subsubsection*{Beschreibung}
	\paragraph{Beschreibung}
	%\miniheading{Beschreibung}
}{%\vspace{3\baselineskip}
}

% Environment around class fields
\newenvironment{texdocclassfields}{%
	\paragraph{Attribute}
	\begin{itemize}
}{%
	\end{itemize} 
}

% Environment around class methods
\newenvironment{texdocclassmethods}{%
	\paragraph{Methoden}
	\begin{itemize}
}{%
	\end{itemize}
}

% Environment around class Constructors
\newenvironment{texdocclassconstructors}{%
	\paragraph{Konstruktoren}
	\begin{itemize}
}{%
	\end{itemize}
}

% Environment around enum constants
\newenvironment{texdocenums}{%
	\paragraph{Enum-Konstanten}
	\begin{itemize}
}{%
	\end{itemize}
}

% Environment around "See also"-Blocks (\texdocsee invocations)
%  Argument 1: Text preceding the references  		% Really nice, texdoclet. What about localization?
\newenvironment{texdocsees}[1]{%\textbf{#1:} 
	\paragraph{Siehe auch: }
	\begin{itemize}
}{%
	\end{itemize}
}
% Formats a single field
%  Argument 1: modifiers
%  Argument 2: type
%  Argument 3: name
%  Argument 4: Documentation text
\newcommand{\texdocfield}[4]{\item \texttt{#1 #2 \textbf{#3}} \\ #4}

% Formats an enum element
%  Argument 1: name
%  Argument 2: documentation text
\newcommand{\texdocenum}[2]{\item \texttt{\textbf{#1}} \\ #2}

% Formats a single method
%  Argument 1: modifiers
%  Argument 2: return type
%  Argument 3: name
%  Argument 4: part after name (parameters)
%  Argument 5: Documentation text
%  Argument 6: Documentation of parameters/exceptions/return values
\newcommand{\texdocmethod}[6]{\item \texttt{#1 #2 \textbf{#3}#4} \par #5#6}

% Formats a single constructor
%  Argument 1: modifiers
%  Argument 2: name
%  Argument 3: part after name (parameters)
%  Argument 4: Documentation text
%  Argument 5: Documentation of parameters/exceptions/return values
\newcommand{\texdocconstructor}[5]{\item \texttt{#1 \textbf{#2}#3} \\ #4#5}

% Inserted when @inheritdoc is used
%  Argument 1: Class where the documentation was inherited from
%  Argument 2: Documentation
\newcommand{\texdocinheritdoc}[2]{#2 (\textit{Dokumentation von \texttt{#1} geerbt})}

% Formats a single see-BlockTag
%  Argument 1: text
%  Argument 2: reference label
\newcommand{\texdocsee}[2]{\item \texttt{\hyperref[#2]{#1}}}

% Environment around \texdocparameter invocations
\newenvironment{texdocparameters}{%\vspace{-5pt}\minisec{Parameter}\vspace{2pt}
	\vspace{-\baselineskip}\subparagraph{Parameter} \hspace{0pt} \\*
	\begin{tabular}{>{\hspace{5pt}}ll}
}{%
	\end{tabular} 
}

% Environment around \texdocthrow invocations
\newenvironment{texdocthrows}{% \vspace{-5pt}\minisec{Ausnahmen}\vspace{2pt}
	\vspace{-\baselineskip}\subparagraph{Ausnahmen} \hspace{0pt} \\*
    \begin{tabular}{>{\hspace{5pt}}ll}
}{%
    \end{tabular}
}

\newcommand{\texdocreturn}[1]{\vspace{-\baselineskip}\subparagraph{Rückgabe} #1	}%\vspace{-1\baselineskip}\subparagraph{Rückgabe} #1

% Formats a parameter (this gets put inside the input of a \texdocmethod or 
% \texdocconstructor macro)
%  Argument 1: name
%  Argument 2: description text
\newcommand{\texdocparameter}[2]{$\,$\texttt{\textbf{#1}} & \begin{minipage}[t]{0.8\textwidth}#2\end{minipage} \\}

% Formats a throws tag
%  Argument 1: exception name
%  Argument 2: description text
\newcommand{\texdocthrow}[2]{\texttt{\textbf{#1}} & \begin{minipage}[t]{0.6\textwidth}#2\end{minipage} \\}

% Used to simulate html <br/>
\newcommand{\texdocbr}{\mbox{}\newline{}}

% Used to simulate html <h[1-9]> - </h[1-9]>
% Argument 1: number of heading (5 for a <h5>)
% Argument 2: heading text
\newcommand{\headref}[2]{\minisec{#2}} 

\newcommand{\refdefined}[1]{
\expandafter\ifx\csname r@#1\endcsname\relax
\relax\else
{$($ in \ref{#1}, page \pageref{#1}$)$}
\fi}

\newlength{\oldparsep}
\setlength\oldparsep\parsep
\newlength{\oldparskip}
\setlength\oldparskip\parskip


% ---------------------------------------------------------------------------
% TexDoc macros end
% ---------------------------------------------------------------------------


