\documentclass[titlepage=true, parskip=full]{scrartcl}
\usepackage[utf8]{inputenc} % use utf8 file encoding for TeX sources
\usepackage[T1]{fontenc}    % avoid garbled Unicode text in pdf
\usepackage{palatino}	      % because "Computer Modern" standard font is illegible
\usepackage{mathpazo}
\usepackage[ngerman]{babel}  % german hyphenation, quotes, etc
\usepackage{hyperref}       % detailed hyperlink/pdf configuration
\hypersetup{                % ‘texdoc hyperref‘ for options
	pdftitle={PSE: Entwurfsdokumentation},%
	bookmarks=true,%
}
\usepackage{graphicx}       % provides commands for including figures
\usepackage{csquotes}       % provides \enquote{} macro for "quotes"
\usepackage[nonumberlist, numberedsection]{glossaries}     % provides glossary commands
\usepackage{enumitem}
\usepackage{tikz}
\usepackage{tikz-uml}
\usepackage{tikz-er2}
\usepackage{comment}
\usepackage{array}
\usepackage{longtable}
\usepackage{hhline}
\usepackage{placeins}
\usepackage{needspace}
\usepackage[toc, page]{appendix}
\usepackage{tabularx}

\usepackage{microtype}
\usepackage{MnSymbol}
\usetikzlibrary{positioning}
\begin{document}
	\subsection{Einstieg}
Wir nehmen zunächst an, dass eine Menge an Modulen $M$ in der Form $(id_{modul}, ects, constraints)$ gegeben ist.
Hierbei besteht $constraints$ aus einer Menge an Constraints der Form $(id_{modul1}, id_{modul2}, typ)$ mit $typ \in \{prerequisite, both, symmetry, antisymmetry\}$\\
Wir beschreiben einen \textit{Studienplan} als eine Menge $P$ von Tupeln der Form $(num_{semester}, modul)$ wobei $num_{semester} \in \mathbb{N} \cup \{\infty\}$ ($\infty$ ist hierbei für ewige Studenten, welche Module schieben wollen) und $modul$ wie oben beschrieben. Eine \textit{Familie} beschreibt eine Menge von \textit{Studienplänen}.\\
Weiter existiert eine Zielfunktion über die Menge aller Studienpläne $\Psi$: $\sigma\colon \Psi \to [0,1]$
	\subsection{Genetischer Algorithmus}
Der Algorithmus erhält die oben beschriebene Menge $M$ und generiert mit Hilfe von  \ref{subsec:genalg-random-generation} eine \textit{Famile} $F$ mit $|F| = O \in \mathbb{N}$ \textit{Studienplänen}.\\
Anschließend wählt er ein beliebiges $p \in \{q \in F | \sigma(q)=max \{\sigma(t)|t \in F \} \}$.\\
Mittels \ref{subsec:genalg-random-modification} wird auf Basis von $p$ eine neue Familie $F'$ mit $|F'| = O$ erstellt, aus welcher erneut ein $p' \in \{q \in F' | \sigma(q)=max \{\sigma(t)|t \in F' \} \}$ gewählt wird. Der letzte Schritt wird mit $p := p'$ dann $R \in \mathbb{N}$ mal wiederholt.
Anschließend wird das letzte generierte $p'$ als generierter Studienplan ausgegeben.
	\subsection{Zufällige Generierung}
	\label{subsec:genalg-random-generation}
	Zunächst generieren wir auf Basis der übergebenen Module $m \in M$ einen Graphen $G = (V, E)$ mit
	\[V = \{M \cup \{m | \exists n \in M: es\ ex.\ ein\ Constraint\\\] \[ der\ Form\ (id_m, id_n, prerequisite),\ (id_m, id_n, both)\ oder\ (id_n, id_m, both) \}\]
	\[E = { (m, n) | es\ ex.\ ein\ Constraint\ der\ Form\ (id_m, id_n, prerequisite)\ oder\ (id_m, id_n, both)}\].
	Bei \textit{both} ist zu beachten, dass m oder m und n bestandene Module sein können, nicht jedoch nur n.
	\begin{equation}
	\Sigma_{m\in V}m.ects \ge 180
	\label{equ:genalg-enough-ects}
	\end{equation}
	Zunächst definieren wir die Menge $Z=\emptyset$ aller zufällig hinzugefügten Module.
	Ist die Ungleichung erfüllt so ist der Studienplan damit fertiggestellt und kann mittels \ref{subsec:genalg-contract-topolog-sort} der Plan fertig gestellt werden.
	Wenn dem nicht so ist wird für ein zufälliges Modul m des Vertiefungs-Fachs gesetzt:
	\[Z' = Z \cup \{m\}\]
	\[V' = V \cup \{m\} \cup \{n | es\ ex.\ ein\ Constraint\\\]\[der\ Form\ (id_m, id_n, prerequisite),\ (id_m, id_n, both)\}
	\]
	\[E' = E \cup \{ (n, m) | es\ ex.\ ein\ Constraint\ der Form\ (id_n, id_m, prerequisite)\ oder\ (id_n, id_m, both)\}\]
	Anschließend wird erneut (\ref{equ:genalg-enough-ects}) überprüft und erneut wie oben beschrieben vorgegangen.
	\subsection{Zufällige Modifizierung}
	\label{subsec:genalg-random-modification}
	Für einen beliebigen wie in \ref{subsec:genalg-random-generation} als Graph dargestellten Plan, wähle man zufällig ein Modul $m \in Z$ und lösche diese samt seiner nur für dieses Modul $m$ benötigten Abhängigkeiten. Diesen Schritt führe man $l \in \mathbb{N}$ mal aus.\\
	Anschließend füge man wie in \ref{subsec:genalg-random-generation} beschrieben erneut Module hinzu bis (\ref{equ:genalg-enough-ects}) wieder erfüllt wird.
	
	\subsection{Kontraktion der topologischen Sortierung}
	\label{subsec:genalg-contract-topolog-sort}
	Zunächst wird der Graph mittels Tiefensuche topologisch sortiert.
	Anschließend paralallelisierung mit dem unten beschriebenen Algorithmus.
	Hierbei werden die bereits gesetzen Module zunächst in die vom Nutzer vorgegebenen Semester gesetzt.
	Hier würde ich dann aber tatsächlich den Pseudocode hinsetzen:
	https://docs.google.com/document/d/1GjRJpvLLVv-inBgDgmrVt-f3btAeqErk6v5kuAnb7Yk/edit?usp=sharing
	
\end{document}