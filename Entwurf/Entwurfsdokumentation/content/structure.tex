\section{Aufbau}

\subsection{Architektur}

\subsubsection{Server}

\subsubsection{Client}
Der Client besteht aus einer leicht abgewandelten Model-View-Controller-Architektur (MVC-Architektur), die auf Backbone.js \cite{backbone} basiert.
Das System besteht aus 4 Haupt-Paketen: Storage, Model, View und Router.\\
Storage übernimmt die Kommunikation mit dem REST-Webservice sowie die Speicherung in Cookies. Somit bildet Storage die Zugriffsschicht für das Model-Paket. \\
Model bildet die Daten aus den Cookies und dem REST-Webservice auf Klassen ab und stellt Möglichkeiten zum Abruf sowie zur Speicherung zur Verfügung, die auf Storage basieren.\\
View zeigt die Daten aus dem Model an und aktualisiert im Fall einer Änderung des Models die Anzeige. Auch fängt das View-Paket Interaktionen mit der Benutzeroberfläche (wie Klicks oder Drag and Drop) ab und verarbeitet diese entsprechend.\\
Router übernimmt schließlich die Aufgaben des Controllers in klassischen MVC-Architekturen. Der Router wird bei Änderungen des URI Fragments aufgerufen und kann so beim Wechsel auf eine andere Seite der WebApp die neuen View-Klassen initialisieren und die zugehörigen Inhalte anzeigen.

\subsection{Klassendiagramm}  % necessary?