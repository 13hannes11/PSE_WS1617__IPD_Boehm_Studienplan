\subsection{Generierungs-Algorithmus}

\newcommand{\id}[1]{\textit{id}_{#1}}
\newenvironment{nstabbing}{
	\setlength{\topsep}{-\parskip}%
	\setlength{\partopsep}{0pt}%
	\tabbing
}{\endtabbing}

\subsubsection{Einstieg}
Wir nehmen zunächst an, dass eine Menge an Modulen $M$ in der Form $(\id{\textit{modul}}, \textit{ects}, \textit{constraints})$ gegeben ist.
Hierbei besteht \textit{constraints} aus einer Menge an Constraints der Form $(\id{\textit{modul1}}, \id{\textit{modul2}}, \textit{typ})$ mit $\textit{typ} \in \{\textit{prerequisite}, \textit{both}, \textit{symmetry}, \textit{antisymmetry}\}$. \\
Constraints der Form $(\id{\textit{modul1}}, \id{\textit{modul2}}, \textit{both})$ und $(\id{\textit{modul2}}, \id{\textit{modul1}}, \textit{both})$ gelten im Folgenden als äquivalent.\\
Wir sehen $ \mathcal{M} $ als die Menge aller vorhandenen Module an.
Wir beschreiben einen \textit{Studienplan} als eine Menge $P$ von Tupeln der Form $(\textit{num}_{\textit{semester}}, \textit{modul})$, wobei $\textit{num}_{\textit{semester}} \in \NN \cup \{\infty\}$\footnote{$\infty$ steht hierbei für ewige Studenten, welche Module schieben wollen.} und \textit{modul} wie oben beschrieben. Eine \textit{Familie} beschreibt eine Menge von \textit{Studienplänen}. \\
Weiter existiert über der Menge aller Studienpläne $\Psi$ eine Zielfunktion  $\sigma : \Psi \to [0,1]$.

\subsubsection{Genetischer Algorithmus}
Der Algorithmus erhält die oben beschriebene Menge $M$ und generiert mit Hilfe von  \ref{subsec:genalg-random-generation} eine \textit{Famile} $F$ mit $|F| = O \in \NN$ \textit{Studienplänen}.\\
Anschließend wählt er ein beliebiges $p \in \{q \in F \mid \sigma(q)=\max \{\sigma(t) \mid t \in F \} \}$.\\
Mittels \ref{subsec:genalg-random-modification} wird auf Basis von $p$ eine neue Familie $F'$ mit $|F'| = O$ erstellt, aus welcher erneut ein $p' \in \{q \in F' \mid \sigma(q) = \max \{\sigma(t) \mid t \in F' \} \}$ gewählt wird. Der letzte Schritt wird mit $p := p'$ dann $R \in \NN$ mal wiederholt.
Anschließend wird das letzte generierte $p'$ als generierter Studienplan ausgegeben.

\subsubsection{Zufällige Generierung}
\label{subsec:genalg-random-generation}
Zunächst generieren wir auf Basis der übergebenen Module $m \in M$ einen Graphen $G = (V, E)$ mit 
\begin{nstabbing}
	\quad $V := M$\\
	\quad $V' := V \cup \{m \in \mathcal{M} \mid \exists n \in V: $ \= es ex. ein Constraint der Form $ (\id{m}, \id{n}, \textit{prerequisite}), $ \\ 
	\> $ (\id{m}, \id{n}, \textit{both}) \} $ oder $ (\id{m}, \id{n}, \textit{symmetry}) \} $, 
\end{nstabbing}
\begin{nstabbing}
	\quad $E := \{ (m, n) \in V' \times V' \mid $ \= es ex. ein Constraint der Form $ (\id{m}, \id{n}, \textit{prerequisite}) $ \\
	\> oder $ (\id{m}, \id{n}, \textit{both})  \} $.
\end{nstabbing}
Der Prozess zur Generierung von $V'$ wird solange weitergeführt, bis alle Abhängigkeiten enthalten sind.
Bei \textit{both} ist zu beachten, dass $ m $ oder $ m $ und $ n $ bestandene Module sein können, nicht jedoch nur $ n $.
\begin{equation}
\sum_{m\in V}m.ects \geqslant 180
\label{equ:genalg-enough-ects}
\end{equation}
Zunächst definieren wir die Menge $Z=\emptyset$ aller zufällig hinzugefügten Module.
Ist die Ungleichung \ref{equ:genalg-enough-ects} erfüllt, so kann mittels \ref{subsec:genalg-contract-topolog-sort} der Plan fertiggestellt werden.
Ist die Ungleichung nicht erfüllt, wird für ein zufälliges Modul $m$ des gewählten Vertiefungs-Fachs gesetzt:
\begin{nstabbing} 
	\quad $Z' := Z \cup \{m\}$ \\
	\quad $V' := V' \cup \{m\}$ \\
	\quad $V' := V' \cup \{ n \in \mathcal{M} \mid \exists l \in V':$ \= es ex. ein Constraint der Form  $(\id{l}, \id{n}, \textit{prerequisite}), $ \\
	\>  $(\id{l}, \id{n}, \textit{both})$  oder $(\id{l}, \id{n}, \textit{symmetry}) \}$  \\
	\quad $E' := E \cup \{ (n, m) \in V' \times V' \mid$ \= es ex. ein Constraint der Form $(\id{n}, \id{m}, \textit{prerequisite})$,\\
	\> $(\id{n}, \id{m}, \textit{both})  \}$
\end{nstabbing}
Auch hier wird der Prozess zur Erweiterung von $V'$ iterativ vorgenommen, bis alle Constraint-Module in $V'$ aufgenommen wurden.
Anschließend wird erneut Ungleichung (\ref{equ:genalg-enough-ects}) überprüft und wie oben beschrieben vorgegangen.

\subsubsection{Zufällige Modifizierung}
\label{subsec:genalg-random-modification}
Für einen beliebigen wie in \ref{subsec:genalg-random-generation} als Graph dargestellten Plan, wähle man zufällig ein Modul $m \in Z$ und lösche diese samt seiner nur für dieses Modul $m$ benötigten Abhängigkeiten. Diesen Schritt führe man $l \in \mathbb{N}$ mal aus.\\
Anschließend füge man -- wie in \ref{subsec:genalg-random-generation} beschrieben -- erneut Module hinzu, bis Ungleichung (\ref{equ:genalg-enough-ects}) wieder erfüllt wird.

\subsubsection{Kontraktion der topologischen Sortierung}
\label{subsec:genalg-contract-topolog-sort}
Zunächst wird der Graph mittels Tiefensuche topologisch sortiert und
anschließend  mit unten beschriebenen Algorithmus parallelisiert.
Hierbei werden die bereits gesetzen Module zunächst in die vom Nutzer vorgegebenen Semester gesetzt.

Sei also $G = (V,E)$ ein DAG und $s: \{1,..,|V|\} \to V$ seine topologische Sortierung, $c: V \to [0,1]$ die Gewichtungsfunktion. Sei weiter $P = V \times V$ eine symmetrische, transitive Relation von Knoten, welche nicht parallel gesetzt werden dürfen.
Sei weiter $ k \in \NN $ das Maximum pro parallelem Schritt.
Weiter sei $ p: V \times \NN \to \{\KwTrue, \KwFalse\} $ eine Funktion, die angibt, ob ein gegebener Knoten in ein gegebenes Semester gesetzt werden darf (Winter- und Sommersemester-Regelung sowie (Anti-)Symmetrie-Constraints).

\begin{algorithm}
	\LComment{Initialisiere:}
	$\Var{bucketAllocation} : \KwArray [1...|V|] \KwOf \NN $ \;
	$\Var{bucketSum} : \KwArray [1...|V|] \KwOf \NN $ \;
	$\Var{minPos} := \langle 1,...,1 \rangle: \KwArray [1...|V|] \KwOf \NN $ \;
	\BlankLine
	\LComment{Iteriere über alle $v \in V$ in topologischer Sortierung}
	\For{$i := 1$ \KwTo $|V|$}{
		$v := s(i)$ \;
		$\Var{set} := \KwFalse$ \;
		\For{$j := \Var{minPos}[i]$ \KwTo $|V|$}{
			\If{$c(v) + \Var{bucketSum}[j] \leqslant k $ \\
				\Indp \KwAnd $\forall(v,u) \in P: \Var{bucketAllocation}[s^{-1} (u)] \ne j $ \\ 
				\KwAnd $p(v,j) $ \\ }{
				$\Var{bucketAllocation}[i] := j $ \RCommentNoBreak{setze v in Bucket j} \;
				$\Var{bucketSum}[j] \pluseq c(v)$ \;
				\ForEach{$(v,u) \in E$}{
					$ \Var{minPos}[s^{-1} (u)] := \max \{j + 1, \Var{minPos}[i]\} $ \;
				}
				$\Var{set} := \KwTrue$ \;
				\KwBreak \;
			}
		}
		\If{$\KwNot \Var{set}$}{
			\KwThrow Exception  \RCommentNoBreak{Knoten zu groß} \;
		} 
	} 
\end{algorithm}

% https://docs.google.com/document/d/1GjRJpvLLVv-inBgDgmrVt-f3btAeqErk6v5kuAnb7Yk/edit?usp=sharing