% lblstatus command
% #1: status code number
\def\lblstatus#1 {\label{STATUS-#1}\textbf{#1} }

%%%%%%%%%%%%%%%%%%%%%%%%%%%%%%%%%%%%%%%%%%%%%%%%%%%%%%%%%%%%%%%%%%%%%%%

\FloatBarrier
\subsection{Spezifikation der HTTP-Statuscodes}

Abschließend folgt die Spezifikation der für die REST-Kommunikation verwendeten HTTP-Statuscodes. Außer bei den Statuscodes \refstatus{200} und \refstatus{201} werden -- abgesehen vom Header -- keine weiteren Daten als Antwort gesendet.

\begin{longtable}{| >{\hspace{0pt}} p{.25\textwidth} | >{\hspace{0pt}} p{.70\textwidth} |}
	\hline
	\textbf{Statuscode} & \textbf{Beschreibung}  \\ 
	\hhline{|=|=|}  
	\endfirsthead
	
	\hline
	\textbf{Statuscode} & \textbf{Beschreibung}  \\ 
	\hhline{|=|=|}    
	\endhead
	
	\hhline{|=|=|}
	\endlastfoot
	%===================================================================
	\lblstatus 200 OK & Die Anfrage konnte korrekt entgegengenommen, ausgeführt und ggfs. beantwortet werden. \\ 
	\hline
	\lblstatus 201 Created & Die Anfrage konnte korrekt entgegengenommen und ausgeführt werden. Eine oder mehrere Ressourcen wurden erfolgreich auf dem Server neu angelegt. \\*
	\hhline{|=|=|} 
	\lblstatus 400 Bad Request & Die Anfrage-Parameter sind fehlerhaft oder unvollständig. Die Anfrage wurde daher nicht ausgeführt. \\
	\hline
	\lblstatus 401 Unauthorized & Für die Anfrage ist erst eine Authentifizierung erforderlich (siehe Kap. \ref{subsec:api-auth}). \\
	\hline
	\lblstatus 404 Not Found & Eine oder mehrere angefragte Ressourcen sind mit vorliegenden Rechten nicht zugänglich, da sie nicht existieren oder der Zugriff verboten ist. \\
	\hline
	\lblstatus 405 Method Not Allowed & Die Zugriff auf die URI mittels dieser HTTP-Methode ist nicht zulässig. \\
	\hline
	\lblstatus 422 Unprocessable Entity & Die Anfrage-Parameter sind syntaktisch korrekt, allerdings ist die Anfrage aufgrund semantischer Konflikte nicht ausführbar. \\*
	\hhline{|=|=|} 
	\lblstatus 500 Internal Server Error & Es wurde versucht, die Anfrage auszuführen. Dabei ist serverseitig ein interner Fehler aufgetreten. \\
\end{longtable}

