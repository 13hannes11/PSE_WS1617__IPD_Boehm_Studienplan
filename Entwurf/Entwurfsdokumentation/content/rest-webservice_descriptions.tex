\section{REST-Webservice Spezifikation}
\subsection{Konvention}
Atomare Werte werden mittels \jsonatom{name} angegeben, wobei name ein eindeutiger Bezeichner für den Wert ist.\\
Zusammengesetzte JSON-Datenklassen werden mittels \jsonobj{name} angegeben, wobei name ein eindeutiger Bezeichner für das JSON-Datenklassen ist.\\
Teile einer zusammengesetzten JSON-Datenklasse werden mittels \jsonpart{name}{prop1, prop2, prop3} angegeben, wobei prop1, prop2, prop3 die einzigen Werte sind, welche angezeigt werden.
blub
\subsubsection{Module}
\begin{lstlisting}[language=json,firstnumber=1]
(*\jsonobj{Module}*) = {
	"id": (*\jsonatom{Modul-ID}*),
	"name": (*\jsonatom{Modul-Name}*),
	"category" : (*\jsonatom{Modul-Category-Name}*),
	"completed": (*\jsonatom{Modul-Completed}*)
	"semester": (*\jsonatom{Modul-Semester}*),
	"creditpoints": (*\jsonatom{Modul-Creditpoints}*),
	"lecturer": (*\jsonatom{Modul-Dozent}*),
	"preference" (*\jsonatom{Modul-Preferenz}*),
	"description": (*\jsonatom{Modul-Beschreibung}*),	
	"constrains": [(*\jsonobj{Constraint}*), ...]	
}
\end{lstlisting}

\subsubsection{Constraint}
\begin{lstlisting}[language=json,firstnumber=1]
(*\jsonobj{Constraint}*) = {
	"name": (*\jsonatom{Constraint-Name}*),
	"first": (*\jsonpart{Modul}{id}*),
	"second": (*\jsonpart{Modul}{id}*),
	"type": (*\jsonatom{Constraint-Typ}*),
	"descriptionFirst": (*\jsonatom{Constraint-Beschreibung}*)
	"descriptionSecond": (*\jsonatom{Constraint-Beschreibung}*)	
}
\end{lstlisting}

\subsubsection{Filter}
\begin{lstlisting}[language=json,firstnumber=1]
(*\jsonobj{Filter}*) = {
	"id": (*\jsonatom{Filter-ID}*),
	"name":(*\jsonatom{Filter-Name}*),
	"value": (*\jsonatom{Filter-Value}*),
	"tooltipp": (*\jsonatom{Filter-Tooltipp}*),
	"specification": (*\jsonobj{Filter-Eigenschaften}*)
}
\end{lstlisting}
\begin{lstlisting}[language=json,firstnumber=1]
(*\jsonobj{Filter-Eigenschaften}*) = {
	"type": (*\jsonatom{Filter-Typ}*),
	"value": (*\jsonatom{Filter-Value}*)

	EITHER (
		"max": (*\jsonatom{Filter-ID}*),
		"min": (*\jsonatom{Filter-Name}*),
	) OR (
		"items": [(*\jsonatom{Selectable-Item}*), ...]
	)
}
\end{lstlisting}

\subsubsection{Objective Function}
\begin{lstlisting}[language=json,firstnumber=1]
(*\jsonobj{Objective}*) = {
	"id": (*\jsonatom{Zielfunktion-ID}*),
	"name": (*\jsonatom{Zielfunktion-Name}*),
	"description": (*\jsonatom{Zielfunktion-Beschreibung}*)
}
\end{lstlisting}

\subsubsection{Studienplan}
\begin{lstlisting}[language=json,firstnumber=1]
(*\jsonobj{Studienplan}*) = {
	"id": (*\jsonatom{Studienplan-ID}*),
	"status": (*\jsonatom{Studienplan-Status}*),
	"creditpoints_sum":(*\jsonatom{Studienplan-Creditpoints-Sum}*),
	"name": (*\jsonatom{Studienplan-Name}*),
	"modules":[(*\jsonpart{Modul}{id, name, completed, semester, creditpoints, lecturer}*), ...]
	"violations": (*\jsonpart{Modul}{id, constrains}*)	
}
\end{lstlisting}

\subsubsection{CopyLink}
\begin{lstlisting}[language=json,firstnumber=1]
(*\jsonobj{CopyLink}*) = {
	"id": (*\jsonatom{Studienplan-ID}*),
	"token": (*\jsonatom{CopyToken}*)
}
\end{lstlisting}


\subsection{REST-Json}

\subsubsection{/plans}
POST:
\begin{lstlisting}[language=json,firstnumber=1]
{
	"plan": (*\jsonpart{Studienplan}{name}
}
\end{lstlisting}
POST return:
\begin{lstlisting}[language=json,firstnumber=1]
{
	"plan": (*\jsonpart{Studienplan}{name}
}
\end{lstlisting}
GET:
\begin{lstlisting}[language=json,firstnumber=1]
{
	"plans": [(*\jsonpart{Studienplan}{id, status, creditpoints-sum, name}*), ...]
}
\end{lstlisting}

\subsubsection{/plans/\{plan\_id\}}
GET, PUT
\begin{lstlisting}[language=json,firstnumber=1]
{
	"plan": (*\jsonpart{Studienplan}*)
}
\end{lstlisting}
PATCH
\begin{lstlisting}[language=json,firstnumber=1]
{
	"plan": (*\jsonpart{Studienplan}{name}*)
}
\end{lstlisting}
DELETE return
\begin{lstlisting}[language=json,firstnumber=1]
{
	"successful": <Delete-Success-Message>
}
\end{lstlisting}
POST
\begin{lstlisting}[language=json,firstnumber=1]
{
	"plan": (*\jsonpart{Studienplan}{name}*)
}
\end{lstlisting}
GET, PUT, POST, PATCH return
\begin{lstlisting}[language=json,firstnumber=1]
{
	"plan": (*\jsonpart{Studienplan}{id, status, creditpoints-sum, name}*)
}
\end{lstlisting}

\subsubsection{/plans/\{plan\_id\}/modules/}
Es sollten nur noch nicht abgeschlossene  Module zurückgegeben werden.\linebreak
GET (list of applied filters and their respective value as URL parameters):
\begin{lstlisting}[language=json,firstnumber=1]
{
	"modules":[(*\jsonpart{Modul}{id, name, creditpoints, lecturer, preference}*), ...]
}	
\end{lstlisting}

\subsubsection{/plans/\{plan\_id\}/modules/\{modul\_id\}}
GET:
\begin{lstlisting}[language=json,firstnumber=1]
{
	"module": (*\jsonobj{Modul}*)
}
\end{lstlisting}
PUT, PATCH:
\begin{lstlisting}[language=json,firstnumber=1]
{
	"module": (*\jsonpart{Modul}{id, semester}*)
}
\end{lstlisting}
DELETE
\begin{lstlisting}[language=json,firstnumber=1]
{
	"module": (*\jsonpart{Modul}{id}*)
}
\end{lstlisting}

\subsubsection{/plans/\{plan\_id\}/modules/\{modul\_id\}/preference}
PUT
\begin{lstlisting}[language=json,firstnumber=1]
{
	"module": (*\jsonpart{Modul}{id, preference}*)
}
\end{lstlisting}

\subsubsection{/plans/\{plan\_id\}/verification}
GET
\begin{lstlisting}[language=json,firstnumber=1]
{
	"plan": (*\jsonpart{Studienplan}{id,status, violations}*)
}
\end{lstlisting}

\subsubsection{/plans/\{plan\_id\}/proposal/\{objective\_id\}}
GET
\begin{lstlisting}[language=json,firstnumber=1]
{
	"plan": (*\jsonpart{Studienplan}{id, modules}*)
}
\end{lstlisting}

\subsubsection{/plans/\{plan\_id\}/pdf}
GET (just return pdf file)

\subsubsection{/plans/\{plan\_id\}/share}
GET
\begin{lstlisting}[language=json,firstnumber=1]
{
	"plan":(*\jsonobj{CopyLink}*)
}
\end{lstlisting}

\subsubsection{/filters}
GET
\begin{lstlisting}[language=json,firstnumber=1]
{
	"filters":[(*\jsonobj{Filter}*), ...]
}
\end{lstlisting}

\subsubsection{/objective-functions}
GET
\begin{lstlisting}[language=json,firstnumber=1]
{
	"functions":[(*\jsonobj{Objective}*), ...]
}
\end{lstlisting}
\subsection{Authentifizierung}
\label{subsec:api-auth}
\paragraph{Zugrundeliegende Spezifikation} Die Authentifizierung am REST-Webservice erfolgt über eine OAuth 2.0 Schnittstelle nach RFC 6749 (siehe \cite{rfc6749}).
Für das momentane System ist hierbei nur die Authentifizierung über \textit{Implicit-Grant} \cite[Kap. 4.2]{rfc6749} notwendig, jedoch sollte das System mindestens um \textit{Authorization-Code-Grant} \cite[Kap. 4.1]{rfc6749} und \textit{Resource-Owner-Password-Credentials-Grant} \cite[Kap. 4.3]{rfc6749} erweitert werden können, weshalb die Basis für diesen Authentifizierungstyp ebenfalls implementiert wird. Bei dem Versuch eine Authentifizierung mittels \textit{Authorization-Code-Grant} oder \textit{Resource-Owner-Password-Credentials-Grant} auszuführen, wird dann eine Fehlermeldung zurück gegeben.\\
Die zugrundeliegende Spezifikation verteilt klare Rollen an die verschiedenen Akteure, welche an der Interaktion der Systeme beteiligt sind. In Tabelle \ref{tab:api-auth-roles} ist angegeben, welches Subsystem die jeweiligen Rollen bei der OAuth-Authentifizierung in unserem System übernimmt

\begin{table}
	\begin{tabularx}{\textwidth}{@{} | X | X | @{}}
		\hline
		\textbf{Rolle} & \textbf{Subsystem}\\ \hline \hline
		\textit{resouce owner} & Benutzer \\ \hline
		\textit{resource server} & REST-Webservice \\ \hline
		\textit{client} & Aktuell ausschließlich die WebApp. Zukünftig vielleicht weitere Systeme, welche auf den REST-Webservice zugreifen \\ \hline
		\textit{authorization server} & REST-Webservice \\ \hline
		\textit{user agent} & Web-Browser des Benutzers \\
		\hline
	\end{tabularx}
	\caption{Rollen in der OAuth 2.0 Spezifikation}
	\label{tab:api-auth-roles}
\end{table}

\paragraph{Registrierung der Klienten}
Klienten sind bei der Authentifizierung alle auf den REST-Webservice zugreifenden Systeme, welche sich dafür im Namen eines Nutzers beim REST-Webservice authentifizieren.\\
Ein Beispiel für einen solchen Klienten ist die WebApp.\\
Alle Klienten werden in der Datenbank gespeichert.\\
Wir unterscheiden zwischen zwei Typen von Klienten: \textit{Vertrauenswürdige} sowie \textit{öffentliche} Klienten. Die Definitionen hierzu finden sich auch in \cite[Kap. 2.1]{rfc6749}. Bei der WebApp handelt es sich auf Grund der Zugänglichkeit des Codes und der Daten um einen öffentlichen Klienten.
\subparagraph{Vertrauenswürdiger Klient}
Für einen vertrauenswürdigen Klienten werden die Informationen aus Tabelle \ref{tab:api-auth-confidential-client-data} gespeichert.\\
\begin{table}
	\begin{tabularx}{\textwidth}{@{} | X | X | @{}}
		\hline
		\textbf{Name} & \textbf{Beschreibung}\\ \hline \hline
		\textit{api\_key} & Eindeutige, öffentliche ID des Klienten. Beginnt mit dem Präfix \enquote{key-} \\ \hline
		\textit{api\_secret} & Nur dem Klienten bekannte Kennung. Beginnt mit dem Präfix \enquote{secret-} \\ \hline
		\textit{scope} & Berechtigungen (auch mehrere) die der Klient anfordern darf. Vorerst immer nur \enquote{student} \\ \hline
		\textit{redirect\_url} & URL an welche beim \textit{Implicit-Grant} bzw. \textit{Authorization-Code-Grant} weitergeleitet wird. \\ \hline
		\textit{origin} & Domains, von welchen aus der Klient auf die Ressourcen zugreifen darf. Wird als Regulärer Ausdruck angegeben. Dieser Wert wird bei Anfragen mit Hilfe des \textit{Referer-Header} überprüft. \\
		\hline
	\end{tabularx}
\caption{Daten eines vertrauenswürdigen Klienten}
\label{tab:api-auth-confidential-client-data}
\end{table}

\subparagraph{Öffentlicher Klient}
Für einen öffentlichen Klienten werden die Informationen aus Tabelle  \ref{tab:api-auth-public-client-data} gespeichert. \\
\begin{table}
	\begin{tabularx}{\textwidth}{@{} | X | X | @{}}
		\hline
		\textbf{Name} & \textbf{Beschreibung}\\ \hline \hline
		api\_key & Eindeutige, öffentliche ID des Klienten. Beginnt mit dem Präfix \enquote{key-} \\ \hline
		api\_secret & NULL (wird nicht benötigt) \\ \hline
		scope & Berechtigungen (auch mehrere) die der Klient anfordern darf. Vorerst immer nur \enquote{student} \\ \hline
		redirect\_url & URL an welche beim \textit{Implicit-Grant} weitergeleitet wird. \\ \hline
		origin & Domains, von welchen aus der Klient auf die Ressourcen zugreifen darf. Wird als Regulärer Ausdruck angegeben. Dieser Wert wird bei Anfragen mit Hilfe des \textit{Referer-Header} überprüft. \\
		\hline
	\end{tabularx}
\caption{Daten eines öffentlichen Klienten}
\label{tab:api-auth-public-client-data}
\end{table}
\paragraph{Schnittstellen}
Für die Authentifizierung werden die Schnittstellen aus Tabelle \ref{tab:api-auth-endpoints} verwendet.
Beim Login wird hierbei der Web-Browser des Benutzers, welcher die WebApp geöffnet, und auf den Button \enquote{Login} geklickt hat, an den URI /auth/login des REST-Webservice weitergeleitet. Auf dieser Seite kann sich der Nutzer authentifizieren. Anschließend wird er auf die Hauptseite der WebApp weitergeleitet. Die Kommunikation der Authentifizierungsdaten zwischen Klient und REST-Webservice verläuft über den \textit{Query} \cite[Kap 3.4]{rfc3986} bzw. das \textit{Fragment} \cite[Kap. 3.5]{rfc3986} des URI.

\begin{table}
	\begin{tabularx}{\textwidth}{@{} | X | X | X | @{}}
		\hline
		\textbf{Methode} & \textbf{URL} & \textbf{Beschreibung} \\ \hline \hline
		GET & /auth/login & Der in RFC 6749, Kapitel 3.1 definierte \textit{Authorization Endpoint} \\ \hline
		POST & /auth/login & Der in RFC 6749, Kapitel 3.1 definierte \textit{Authorization Endpoint} (für \textit{Resource-Owner-Password-Credentials-Grant}) \\ \hline
		POST & /auth/token & Der in RFC 6749, Kapitel 3.2 definierte \textit{Token Endpoint} \\ \hline
		GET & /auth/logout & Devalidiert das genutze Access-Token \\ \hline
	\end{tabularx}
\caption{Schnittstellen für die OAuth 2.0 Kommunikation}
\label{tab:api-auth-endpoints}
\end{table}

\paragraph{Authentifizierungs-Vorgang}
Es wird hier lediglich die Authentifizierung mittels \textit{Implicit-Grant} beschrieben, da andere Authentifizierungs-Typen nicht implementiert werden.\\
Zunächst leitet der Klient den Web-Browser des Nutzers beim Login auf den URI /auth/login des REST-Webservice weiter. Hierbei werden im \textit{Query}  die Parameter aus Tabelle \ref{tab:api-auth-login-req-params} übergeben.\\
Wenn die \textit{client\_id} einem Klienten des REST-Webservice zuordenbar ist, wird der Web-Browser des Nutzers nach der Authentifizierung an den \textit{redirect\_url} weitergeleitet. Ist die \textit{client\_id} nicht zuordenbar, so wird ein \verb|420 Policy Not Fulfilled| Status-Code mit einer Fehlermeldung zurückgegeben.
Die Authentifizierung schlägt in folgenden Fällen mit den Parametern aus Tabelle \ref{tab:api-auth-login-res-error} im \textit{Fragment} fehl (siehe hierfür auch \cite[Kap. 4.2.2.1]{rfc6749}):
\begin{itemize}
	\item[invalid\_request] Fehlen von in Tabelle \ref{tab:api-auth-login-req-params} spezifizierten Parametern
	\item[unsupported\_response\_type] Nicht unterstützter \textit{response\_type} (alles außer \enquote{token})
	\item[invalid\_scope] Nicht unterstützter \textit{scope} (momentan ist der einzige unterstützte Scope \enquote{student})
	\item[server\_error] Fehler des Servers
\end{itemize}
Ist die Authentifizierung erfolgreich so werden bei der Weiterleitung die Parameter aus Tabelle \ref{tab:api-auth-login-res-success} im \textit{Fragment} übergeben.
\begin{table}
	\begin{tabularx}{\textwidth}{@{} | X | X | @{}}
		\hline
		\textbf{Parameter} & \textbf{Beschreibung}\\ \hline \hline
		\textit{response\_type} & Für die Authentifizierung mittels \textit{Implicit-Grant} der Wert \enquote{token}. Für Authentifizierung mittels \textit{Authorization-Code-Grant} der Wert \enquote{code}. Für Authentifizierung mittels \textit{Resource-Owner-Password-Credentials-Grant} der Wert \enquote{password}. \\ \hline
		\textit{client\_id} & Den \textit{api\_key} des Klienten \\ \hline
		\textit{scope} & In den ersten Versionen des Systems immer \enquote{student}. Später möglicherweise auch andere Werte. \\ \hline
		\textit{state} & Ein im Web-Browser gespeicherter, nicht von dritten veränderbarer Schlüssel, der vom REST-Webservice in der Antwort mitgesendet wird.\\
		\hline
	\end{tabularx}
	\caption{Übergebene Parameter bei der Weiterleitung an /auth/login}
	\label{tab:api-auth-login-req-params}
\end{table}

\begin{table}
	\begin{tabularx}{\textwidth}{@{} | X | X | @{}}
		\hline
		\textbf{Parameter} & \textbf{Beschreibung}\\ \hline \hline
		\textit{error} & Ein den Fehler identifizierenden Schlüssel (siehe hierfür auch \cite[Kap. 4.2.2.1]{rfc6749})\\ \hline
		\textit{state} & Das bei der Anfrage vom Klienten übergebene \textit{state}-Parameter\\
		\hline
	\end{tabularx}
	\caption{Übergebene Parameter bei einer fehlgeschlagenen Authentifizierung}
	\label{tab:api-auth-login-res-error}
\end{table}

\begin{table}
	\begin{tabularx}{\textwidth}{@{} | X | X | @{}}
		\hline
		\textbf{Parameter} & \textbf{Beschreibung}\\ \hline \hline
		\textit{access\_token} & Ein Token, mit welchem der Klient im Namen des Nutzers auf den REST-Webservice zugreifen kann\\ \hline
		\textit{token\_type} & \enquote{Bearer} (siehe \cite[Kap. 7.1]{rfc6749}) \\ \hline
		\textit{expires\_in} & Dauer, welche das \textit{access\_token} gültig ist. \\ \hline
		\textit{scope} & In den ersten Versionen des Systems immer \enquote{student}. Später möglicherweise auch andere Werte.\\ \hline
		\textit{state} & Das bei der Anfrage vom Klienten übergebene \textit{state}-Parameter\\
		\hline
	\end{tabularx}
	\caption{Übergebene Parameter bei einer erfolgreichen Authentifizierung}
	\label{tab:api-auth-login-res-success}
\end{table}

\paragraph{Ressourcen-Zugriff}
Beim Zugriff auf die Ressourcen des REST-Webservice übergibt der Klient das \textit{access\_token} über den Header-Wert \enquote{Authorization: Bearer <\textit{access\_token}>}.
Der Zugriff wird gestattet, wenn das \textit{access\_token} gültig und nicht abgelaufen ist, und der Referer-Header-Wert dem regulären Ausdruck \textit{origin} des Klienten entspricht.
\subsection{Atomare Werte}

\subsection{Spezifikation der JSON-Datenklassen}
\begin{lstlisting}[language=json,firstnumber=-3]
(*\jsonobj{Modul}*) = {
	"menu": {
		"id": (*\jsonatom{Modul-ID}*),
		"value": "File",
		"popup": {
			"menuitem": [
				{"value": "New", "onclick": "CreateNewDoc()"},
				{"value": "Open", "onclick": "OpenDoc()"},
				{"value": "Close", "onclick": "blub"}
			]
		}
	},
	"foo" : "bar",
	"bar" : "foo"
}
\end{lstlisting}
\jsonpart{Modul}{menu, foo} wäre die JSON-Datenklasse \jsonobj{Modul} ohne bar.
\subsection{Zugriffsstruktur}

\begin{longtable}{| >{\hspace{0pt}} p{.11\linewidth} | >{\hspace{0pt}} p{.27\linewidth} | >{\hspace{0pt}} p{.27\linewidth} | >{\hspace{0pt}} p{.29\linewidth} | }
	\hline
	\textbf{Methode} & \textbf{URL} & \textbf{Beschreibung} & \textbf{\hspace{0pt}Kommunikationsdaten (Note to self: finde schöneres Wort!)} \\ \hline  \hline
	\\ \hline
	& & & \\ \hline
	GET & /student & Lese Informationen über Student & tbd	\\ \hline
	PUT & /student & Ersetze Informationen über Student & tbd
	\\ \hline
	DELETE & /student & Löscht Student & tbd \\ \hline
	& & & \\ \hline
	POST & /plans & Erstellt neuen Studienplan & tbd \\ \hline
	GET & /plans & Lese Planliste & tbd \\ \hline
	GET & /plans/\{plan\_id\} & Lese Plan & tbd \\ \hline
	PUT & /plans/\{plan\_id\} & Ersetze Plan (inkl. Verifikations-Informationen) & tbd \\ \hline
	PATCH & /plans/\{plan\_id\} & Bearbeite Plan: Name ändern (setzte Verifizierung zurück) & tbd \\ \hline
	DELETE & /plans/\{plan\_id\} & Lösche Plan & tbd \\ \hline
	POST & /plans/\{plan\_id\} & Dupliziere Plan & tbd \\ \hline
	GET & /plans/\{plan\_id\}/\+modules & Lese Modul-Liste (gefiltert) & tbd \\ \hline
	GET & /plans/\{plan\_id\}/\+modules/\{module\_id\} & Lese Modul mit \{module\_id\} & tbd \\ \hline
	PUT & /plans/\{plan\_id\}/\+modules/\{module\_id\} & Setze Modul in Plan in gegebenes Semester, setzte Verifizierung zurück & tbd \\ \hline
	PATCH & /plans/\{plan\_id\}/\+modules/\{module\_id\} & Bearbeite Modul in Plan (also setze in anderes Semester), setze Verifizierung zurück & tbd \\ \hline
	DELETE & /plans/\{plan\_id\}/\+modules/\{module\_id\} & Lösche Modul aus Plan, setzte Verifizierung zurück & tbd \\ \hline
	PUT & /plans/\{plan\_id\}/\+modules/\{module\_id\}/\+preference & Setze Bewertung für Modul & tbd \\ \hline
	& & & \\ \hline
	GET & /plans/\{plan\_id\}/\+verification & Verifiziere den Plan, gebe das Ergebnis zurück und speichere es in der Datenbank & tbd \\ \hline
	& & & \\ \hline
	GET & /plans/\{plan\_id\}/\+proposal & Erstelle und Erhalte einen auf Basis des Plans generierten (neuen) Plan & tbd \\ \hline
	& & & \\ \hline
	GET & /plans/\{plan\_id\}/pdf & Lese PDF-Version des Plans & tbd \\ \hline
	GET & /plans/\{plan\_id\}/\+share & Lese Link zum teilen des Plans & tbd \\ \hline
	& & & \\ \hline
	GET  & /filters & Lade Filtertypen und Beschränkungen & tbd \\ \hline
	GET & /objective-functions & Lese Liste mit allen vorhandenen Zielfunktionen & tbd \\ \hline
\end{longtable}


\subsection{Spezifikation der Statuscodes}