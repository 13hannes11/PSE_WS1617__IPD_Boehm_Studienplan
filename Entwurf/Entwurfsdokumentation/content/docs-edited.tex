\begin{texdocpackage}{client.model.modules}
\label{texdoclet:edu.kit.informatik.studyplan.client.model.modules}

\begin{texdocclass}{class}{Module}
\label{texdoclet:edu.kit.informatik.studyplan.client.model.modules.Module}
\begin{texdocclassintro}
Diese Klasse repräsentiert ein Modul, welches vom Server geladen wird. Das
 Modul wird als Element zunächst nur teilweise geladen. Mittels eines Aufrufs
 von fetch() werden dann gegebenfalls weitere notwendige Daten nachgeladen.\end{texdocclassintro}
\begin{texdocclassconstructors}
\texdocconstructor{public}{Module}{(Object params)}{}{}
\end{texdocclassconstructors}
\end{texdocclass}


\begin{texdocclass}{class}{ModuleCollection}
\label{texdoclet:edu.kit.informatik.studyplan.client.model.modules.ModuleCollection}
\begin{texdocclassintro}
Eine Sammlung von Modulen, welche vom Server abgerufen werden kann\end{texdocclassintro}
\begin{texdocclassconstructors}
\texdocconstructor{public}{ModuleCollection}{(Object params)}{}{}
\end{texdocclassconstructors}
\begin{texdocclassmethods}
\texdocmethod{public abstract}{String}{url}{()}{Generiert abhängig von der PlanId den URL für den Abruf der Module}{\texdocreturn{Der generierte URL}
}
\end{texdocclassmethods}
\end{texdocclass}


\begin{texdocclass}{class}{ModuleConstraint}
\label{texdoclet:edu.kit.informatik.studyplan.client.model.modules.ModuleConstraint}
\begin{texdocclassintro}
Diese Klasse beschreibt einen zu einem Modul gehörenden System-Constraint
 (erweitert Backbone.Model)\end{texdocclassintro}
\begin{texdocclassconstructors}
\texdocconstructor{public}{ModuleConstraint}{(Object params)}{}{}
\end{texdocclassconstructors}
\end{texdocclass}


\begin{texdocclass}{class}{Preference}
\label{texdoclet:edu.kit.informatik.studyplan.client.model.modules.Preference}
\begin{texdocclassintro}
Diese Klasse beschreibt eine zu einem Modul gehörende gesetzte Präferenz
 (positiv$/$neutral$/$negativ)\end{texdocclassintro}
\begin{texdocclassconstructors}
\texdocconstructor{public}{Preference}{(Object params)}{}{}
\end{texdocclassconstructors}
\begin{texdocclassmethods}
\texdocmethod{public}{String}{url}{()}{Generiert die URL für das speichern$/$abrufen der Präferenz abhängig von
 planId und moduleId}{\texdocreturn{Den generierten URL}
}
\end{texdocclassmethods}
\end{texdocclass}


\end{texdocpackage}



\begin{texdocpackage}{client.model}
\label{texdoclet:edu.kit.informatik.studyplan.client.model}

\end{texdocpackage}



\begin{texdocpackage}{client.model.plans}
\label{texdoclet:edu.kit.informatik.studyplan.client.model.plans}

\begin{texdocclass}{class}{Plan}
\label{texdoclet:edu.kit.informatik.studyplan.client.model.plans.Plan}
\begin{texdocclassintro}
Klasse, welche einen Studienplan beschreibt\end{texdocclassintro}
\begin{texdocclassconstructors}
\texdocconstructor{public}{Plan}{(Object params)}{}{}
\end{texdocclassconstructors}
\begin{texdocclassmethods}
\texdocmethod{public}{boolean}{containsModule}{(int moduleId)}{Eine Methode, welche zurück gibt, ob das Modul mit der gegebenen ID im
 Plan vorhanden ist.}{\begin{texdocparameters}
\texdocparameter{moduleId}{Die ID des Moduls nach welchem gesucht werden soll}
\end{texdocparameters}
\texdocreturn{True, wenn das Modul vorhanden ist, False wenn nicht.}
}
\texdocmethod{public}{void}{loadVerification}{()}{Lädt die Informationen zur Verifizierung und speichert diese im Plan.}{}
\end{texdocclassmethods}
\end{texdocclass}


\begin{texdocclass}{class}{PlanCollection}
\label{texdoclet:edu.kit.informatik.studyplan.client.model.plans.PlanCollection}
\begin{texdocclassintro}
Klasse, welche eine Collection von Studienplänen beschreibt\end{texdocclassintro}
\begin{texdocclassconstructors}
\texdocconstructor{public}{PlanCollection}{(Object params)}{}{}
\end{texdocclassconstructors}
\end{texdocclass}


\begin{texdocclass}{class}{ProposedPlan}
\label{texdoclet:edu.kit.informatik.studyplan.client.model.plans.ProposedPlan}
\begin{texdocclassintro}
Klasse, welche einen vorgeschlagenen Plan repräsentiert\end{texdocclassintro}
\begin{texdocclassconstructors}
\texdocconstructor{public}{ProposedPlan}{(Object params)}{}{}
\end{texdocclassconstructors}
\begin{texdocclassmethods}
\texdocmethod{public}{void}{save}{(Object options)}{Methode, welche den Plan ggf. speichert}{\begin{texdocparameters}
\texdocparameter{options}{Alle Optionen von Backbone.Model.save() zusätzlich:
            \begin{itemize}
            \item Boolean newPlan true falls, der Plan als neuer Plan
            geschrieben werden soll, false falls der Plan überschrieben
            werden soll. Standard ist false
            \item String planName Name des Plans, falls newPlan = true
            \end{itemize}
}
\end{texdocparameters}
}
\end{texdocclassmethods}
\end{texdocclass}


\begin{texdocclass}{class}{Semester}
\label{texdoclet:edu.kit.informatik.studyplan.client.model.plans.Semester}
\begin{texdocclassintro}
Klasse, welcher ein Semester als Collection von Modulen beschreibt.\end{texdocclassintro}
\begin{texdocclassconstructors}
\texdocconstructor{public}{Semester}{(Object params)}{}{}
\end{texdocclassconstructors}
\begin{texdocclassmethods}
\texdocmethod{public}{String}{url}{()}{}{}
\end{texdocclassmethods}
\end{texdocclass}


\begin{texdocclass}{class}{SemesterCollection}
\label{texdoclet:edu.kit.informatik.studyplan.client.model.plans.SemesterCollection}
\begin{texdocclassintro}
Eine Collection von Semestern, welche von einem Plan genutzt wird. (Erbt von
 Backbone.Collection)\end{texdocclassintro}
\begin{texdocclassconstructors}
\texdocconstructor{public}{SemesterCollection}{(Object params)}{}{}
\end{texdocclassconstructors}
\end{texdocclass}


\end{texdocpackage}



\begin{texdocpackage}{client.model.system}
\label{texdoclet:edu.kit.informatik.studyplan.client.model.system}

\begin{texdocclass}{class}{CookieModel}
\label{texdoclet:edu.kit.informatik.studyplan.client.model.system.CookieModel}
\begin{texdocclassintro}
Ein Model, welches in einem Cookie gespeichert wird (Erweitert von
 Backbone.Model)\end{texdocclassintro}
\begin{texdocclassconstructors}
\texdocconstructor{public}{CookieModel}{(Object params)}{}{}
\end{texdocclassconstructors}
\end{texdocclass}


\begin{texdocclass}{class}{Discipline}
\label{texdoclet:edu.kit.informatik.studyplan.client.model.system.Discipline}
\begin{texdocclassintro}
Klasse, welche ein Fach repräsentiert\end{texdocclassintro}
\begin{texdocclassconstructors}
\texdocconstructor{public}{Discipline}{(Object params)}{}{}
\end{texdocclassconstructors}
\end{texdocclass}


\begin{texdocclass}{class}{DisciplineCollection}
\label{texdoclet:edu.kit.informatik.studyplan.client.model.system.DisciplineCollection}
\begin{texdocclassintro}
Klasse, welche eine Collection von Fächern repräsentiert\end{texdocclassintro}
\begin{texdocclassconstructors}
\texdocconstructor{public}{DisciplineCollection}{(Object params)}{}{}
\end{texdocclassconstructors}
\end{texdocclass}


\begin{texdocclass}{class}{EventBus}
\label{texdoclet:edu.kit.informatik.studyplan.client.model.system.EventBus}
\begin{texdocclassintro}
Dieses Singleton wird als Event-Bus zwischen verschiedenen Komponenten der
 Applikation verwendet. Jedes Objekt kann mittels getInstance() auf den
 EventBus zugreifen und sich als Listener registrieren oder einen Event
 auslösen.\texdocbr{}

 Ein Beispiel für einen solchen Event wäre die Anzeige eines Moduls in der
 Seitenleiste beim Klick auf das Modul. (Erbt von Backbone.Event)\end{texdocclassintro}
\begin{texdocclassconstructors}
\texdocconstructor{private}{EventBus}{()}{}{}
\end{texdocclassconstructors}
\begin{texdocclassmethods}
\texdocmethod{public static}{EventBus}{getInstance}{()}{Methode zum erhalten einer Instanz des Event Busses}{\texdocreturn{Die Event-Bus Instanz}
}
\end{texdocclassmethods}
\end{texdocclass}


\begin{texdocclass}{class}{Filter}
\label{texdoclet:edu.kit.informatik.studyplan.client.model.system.Filter}
\begin{texdocclassintro}
Klasse welcher die Daten zu einem Filter speichert\end{texdocclassintro}
\begin{texdocclassconstructors}
\texdocconstructor{public}{Filter}{(Object params)}{}{}
\end{texdocclassconstructors}
\begin{texdocclassmethods}
\texdocmethod{public}{String}{getParams}{()}{Konvertiert die Filter-Daten in GET-Query-Parameter}{\texdocreturn{Die GET-Query-Parameter}
}
\end{texdocclassmethods}
\end{texdocclass}


\begin{texdocclass}{class}{FilterCollection}
\label{texdoclet:edu.kit.informatik.studyplan.client.model.system.FilterCollection}
\begin{texdocclassintro}
Eine Collection von Filtern\end{texdocclassintro}
\begin{texdocclassconstructors}
\texdocconstructor{public}{FilterCollection}{(Object params)}{}{}
\end{texdocclassconstructors}
\end{texdocclass}


\begin{texdocclass}{class}{LanguageManager}
\label{texdoclet:edu.kit.informatik.studyplan.client.model.system.LanguageManager}
\begin{texdocclassintro}
Singleton, welches die sprachlichen Ausgaben des Programms verwaltet. Hierbei
 lässt sich diei Sprache in welcher die Nachrichten ausgegeben werden mittels
 setLanguage statisch einstellen. Die Nachrichten werden bereits zur
 Compile-Zeit in den Initialisierungscode des Programms geladen und sind
 deshalb von Beginn an verfügbar.\end{texdocclassintro}
\begin{texdocclassconstructors}
\texdocconstructor{private}{LanguageManager}{()}{}{}
\end{texdocclassconstructors}
\begin{texdocclassmethods}
\texdocmethod{public static}{LanguageManager}{getInstance}{()}{Gibt die aktuelle Instanz des LanguageManagers zurück}{\texdocreturn{Die aktuelle Instanz des Sprachmanagers}
}
\texdocmethod{public}{String}{getMessage}{(String key)}{Methode zum erhalten einer Nachricht in der aktuellen Sprach auf Basis
 eines keys}{\begin{texdocparameters}
\texdocparameter{key}{Der key, welcher den auszugebenden String identifiziert}
\end{texdocparameters}
\texdocreturn{Den String, falls dieser vorhanden ist, sonst den key selbst}
}
\texdocmethod{public static}{void}{setLanguage}{(String langCode)}{Setzt die Sprache des LanguageManagers fest}{\begin{texdocparameters}
\texdocparameter{langCode}{Der Code der Sprache, welche gesetzt werden soll}
\end{texdocparameters}
}
\end{texdocclassmethods}
\end{texdocclass}


\begin{texdocclass}{class}{Notification}
\label{texdoclet:edu.kit.informatik.studyplan.client.model.system.Notification}
\begin{texdocclassintro}
Klasse, welche eine Benachrichtigung repräsentiert (erweitert Backbone.Model)\end{texdocclassintro}
\begin{texdocclassconstructors}
\texdocconstructor{public}{Notification}{(Object params)}{}{}
\end{texdocclassconstructors}
\end{texdocclass}


\begin{texdocclass}{class}{NotificationCollection}
\label{texdoclet:edu.kit.informatik.studyplan.client.model.system.NotificationCollection}
\begin{texdocclassintro}
Singleton-Collection, welche alle Benachrichtigungen enthält \texdocbr{}

 (erweitert Backbone.Collection)\end{texdocclassintro}
\begin{texdocclassconstructors}
\texdocconstructor{private}{NotificationCollection}{()}{}{}
\end{texdocclassconstructors}
\begin{texdocclassmethods}
\texdocmethod{public}{void}{add}{(Notification notification)}{Methode zum hinzufügen einer neuen Benachrichtigung}{\begin{texdocparameters}
\texdocparameter{notification}{Die Benachrichtigung, welche hinzugefügt werden soll}
\end{texdocparameters}
}
\texdocmethod{public static}{NotificationCollection}{getInstance}{()}{Methode zum erhalten der aktuellen Instanz der NotificationCollection}{\texdocreturn{Die aktuelle Instanz von NotificationCollection}
}
\end{texdocclassmethods}
\end{texdocclass}


\begin{texdocclass}{class}{OAuthCollection}
\label{texdoclet:edu.kit.informatik.studyplan.client.model.system.OAuthCollection}
\begin{texdocclassintro}
Diese Klasse repräsentiert eine Collection, welche als Resource in der
 REST-Schnittstelle gespeichert ist, und bei welcher sich man beim Zugriff
 über OAUTH authentifizieren muss. (Erweitert Backbone.Collection)\end{texdocclassintro}
\begin{texdocclassconstructors}
\texdocconstructor{public}{OAuthCollection}{(Object params)}{}{}
\end{texdocclassconstructors}
\end{texdocclass}


\begin{texdocclass}{class}{OAuthModel}
\label{texdoclet:edu.kit.informatik.studyplan.client.model.system.OAuthModel}
\begin{texdocclassintro}
Diese Klasse repräsentiert ein Model, welches als Resource in der
 REST-Schnittstelle gespeichert ist, und bei welchem man sich beim Zugriff
 über OAUTH authentifizieren muss. (Erweitert Backbone.Model)\end{texdocclassintro}
\begin{texdocclassconstructors}
\texdocconstructor{public}{OAuthModel}{(Object params)}{}{}
\end{texdocclassconstructors}
\end{texdocclass}


\begin{texdocclass}{class}{ObjectiveFunction}
\label{texdoclet:edu.kit.informatik.studyplan.client.model.system.ObjectiveFunction}
\begin{texdocclassintro}
Klasse, welche eine Objective Function repräsentiert\end{texdocclassintro}
\begin{texdocclassconstructors}
\texdocconstructor{public}{ObjectiveFunction}{(Object params)}{}{}
\end{texdocclassconstructors}
\end{texdocclass}


\begin{texdocclass}{class}{ObjectiveFunctionCollection}
\label{texdoclet:edu.kit.informatik.studyplan.client.model.system.ObjectiveFunctionCollection}
\begin{texdocclassintro}
Klasse, welche eine Collection von Objective Functions repräsentiert\end{texdocclassintro}
\begin{texdocclassconstructors}
\texdocconstructor{public}{ObjectiveFunctionCollection}{(Object params)}{}{}
\end{texdocclassconstructors}
\end{texdocclass}


\begin{texdocclass}{class}{ProposalInformation}
\label{texdoclet:edu.kit.informatik.studyplan.client.model.system.ProposalInformation}
\begin{texdocclassintro}
Die Methoden welche im Zusammenhang mit der Generierung eines Studienplans
 abgespeichert und zum Server geschickt werden. (Erweitert Backbone.Model)\end{texdocclassintro}
\begin{texdocclassconstructors}
\texdocconstructor{public}{ProposalInformation}{(Object params)}{}{}
\end{texdocclassconstructors}
\end{texdocclass}


\begin{texdocclass}{class}{SearchCollection}
\label{texdoclet:edu.kit.informatik.studyplan.client.model.system.SearchCollection}
\begin{texdocclassintro}
Die Seach-Collection, welche das Lazy-Loading der Module in der Suche
 übernimmt. Hierbei werden in zehner Schritten nach und nach mehr Module
 geladen und in einzelne SearchModuleCollections gespeichert. Beim Abruf
 mittels getModules() werden diese dann in eine einzige Collection
 zusammengeführt.\end{texdocclassintro}
\begin{texdocclassconstructors}
\texdocconstructor{public}{SearchCollection}{(Object params)}{}{}
\end{texdocclassconstructors}
\begin{texdocclassmethods}
\texdocmethod{public}{void}{setFilters}{(FilterCollection filters)}{Methode, welche die Filter neu setzt und im Zuge desse, die alten
 SearchModuleCollections löscht und highestLoaded zurücksetzt.
 Anschließend werden erneut Module geladen.}{\begin{texdocparameters}
\texdocparameter{filters}{Die Filter, welche bei der Suche berücksichtigt werden sollen.}
\end{texdocparameters}
}
\texdocmethod{public}{String}{url}{()}{}{}
\end{texdocclassmethods}
\end{texdocclass}


\begin{texdocclass}{class}{SubjectDiscipline}
\label{texdoclet:edu.kit.informatik.studyplan.client.model.system.SubjectDiscipline}
\begin{texdocclassintro}
Klasse, welche ein Vertiefungsfach repräsentiert\end{texdocclassintro}
\begin{texdocclassconstructors}
\texdocconstructor{public}{SubjectDiscipline}{(Object params)}{}{}
\end{texdocclassconstructors}
\end{texdocclass}


\begin{texdocclass}{class}{SubjectDisciplineCollection}
\label{texdoclet:edu.kit.informatik.studyplan.client.model.system.SubjectDisciplineCollection}
\begin{texdocclassintro}
Klasse, welche eine Collection von Vertiefungsfächern repräsentiert\end{texdocclassintro}
\begin{texdocclassconstructors}
\texdocconstructor{public}{SubjectDisciplineCollection}{(Object params)}{}{}
\end{texdocclassconstructors}
\end{texdocclass}


\begin{texdocclass}{class}{TemplateManager}
\label{texdoclet:edu.kit.informatik.studyplan.client.model.system.TemplateManager}
\begin{texdocclassintro}
Singleton, welches die Templates für die Views verwaltet. Jedes Template
 lässt sich über einen key abrufen.\texdocbr{}

 Die Templates werden zur Compile-Zeit in den Initialisierungscode
 hineingeladen und sind somit bereits beim Start der Applikation im Template
 Manager vorhanden. Der Vorteil dieses Vorgehens ist, dass man beim Laden der
 einzelnen Views nicht auf eine Antwort vom Server (welche das Template
 enthält) warten muss.\end{texdocclassintro}
\begin{texdocclassconstructors}
\texdocconstructor{private}{TemplateManager}{()}{}{}
\end{texdocclassconstructors}
\begin{texdocclassmethods}
\texdocmethod{public static}{TemplateManager}{getInstance}{()}{Methode zum erhalten der aktuellen Instanz des TemplateManagers}{\texdocreturn{Gibt die aktuelle Instanz des TemplateManagers zurück}
}
\texdocmethod{public}{Object}{getTemplate}{(String key)}{Methode zum erhalten eines Templates, welches unter dem key gespeichert
 ist}{\begin{texdocparameters}
\texdocparameter{key}{Der Key zur identifizierung des Templates}
\end{texdocparameters}
\texdocreturn{Gibt eine Funktion (ein Underscore Template) zurück, die mit
         Parametern aufgerufen werden kann und deren Rückgabewert ein
         befülltes Template als String ist}
}
\end{texdocclassmethods}
\end{texdocclass}


\end{texdocpackage}



\begin{texdocpackage}{client.model.user}
\label{texdoclet:edu.kit.informatik.studyplan.client.model.user}

\begin{texdocclass}{class}{PassedModuleCollection}
\label{texdoclet:edu.kit.informatik.studyplan.client.model.user.PassedModuleCollection}
\begin{texdocclassintro}
Die Collection aller bestandenen Module, welche zu einem Nutzer gehört\end{texdocclassintro}
\begin{texdocclassconstructors}
\texdocconstructor{public}{PassedModuleCollection}{(Object params)}{}{}
\end{texdocclassconstructors}
\begin{texdocclassmethods}
\texdocmethod{public}{String}{url}{()}{}{}
\end{texdocclassmethods}
\end{texdocclass}


\begin{texdocclass}{class}{SessionInformation}
\label{texdoclet:edu.kit.informatik.studyplan.client.model.user.SessionInformation}
\begin{texdocclassintro}
Die Informationen zur aktuellen Session der WebApp, welche in einem Cookie
 gespeichert werden. Diese Informationen werden in einem Singleton
 gespeichert, da es nicht mehr als ein SessionInformation-Objekt geben darf.\end{texdocclassintro}
\begin{texdocclassconstructors}
\texdocconstructor{private}{SessionInformation}{()}{}{}
\end{texdocclassconstructors}
\begin{texdocclassmethods}
\texdocmethod{public}{void}{generateState}{()}{Methode, die ein neues StateAttribut generiert}{}
\texdocmethod{public static}{SessionInformation}{getInstance}{()}{Methode welche die aktuelle Instanz der SessionInformation zurück gibt.}{\texdocreturn{}
}
\end{texdocclassmethods}
\end{texdocclass}


\begin{texdocclass}{class}{Student}
\label{texdoclet:edu.kit.informatik.studyplan.client.model.user.Student}
\begin{texdocclassintro}
Klasse, welche den Studenten beschreibt\end{texdocclassintro}
\begin{texdocclassconstructors}
\texdocconstructor{public}{Student}{(Object params)}{}{}
\end{texdocclassconstructors}
\end{texdocclass}


\end{texdocpackage}



\begin{texdocpackage}{client.router}
\label{texdoclet:edu.kit.informatik.studyplan.client.router}

\begin{texdocclass}{class}{MainRouter}
\label{texdoclet:edu.kit.informatik.studyplan.client.router.MainRouter}
\begin{texdocclassintro}
Klasse, welche das Routing zu den verschiedenen Seiten hin übernimmt. Die
 Methoden stoßen dann jeweils die entsprechenden Änderungen im View an.\end{texdocclassintro}
\begin{texdocclassconstructors}
\texdocconstructor{public}{MainRouter}{()}{}{}
\end{texdocclassconstructors}
\begin{texdocclassmethods}
\texdocmethod{public}{void}{comparisonPage}{(String plan1Id, String plans2Id)}{Methode zum anzeigen der Pläne-Vergleichen-Seite\texdocbr{}

 \textit{$/$compare$/$$\{$plan1Id$\}$$/$$\{$plan2Id$\}$}}{\begin{texdocparameters}
\texdocparameter{plan1Id}{Die ID des ersten Plans}
\texdocparameter{plans2Id}{Die ID des zweiten Plans}
\end{texdocparameters}
}
\texdocmethod{public}{void}{editPage}{(String planId)}{Methode zum anzeigen der Plan-Bearbeiten-Seite\texdocbr{}

 \textit{$/$plan$/$$\{$planId$\}$}}{\begin{texdocparameters}
\texdocparameter{planId}{Die ID des Plans}
\end{texdocparameters}
}
\texdocmethod{public}{void}{generationWizard}{(int planId)}{Methode zum anzeigen des Generierungs-Wizards\texdocbr{}

 \textit{$/$plan$/$$\{$planId$\}$$/$generate}}{}
\texdocmethod{public}{void}{handleLogin}{()}{Methode zum verarbeiten der OAuth Antwort vom Server. Diese Methode
 leitet dann an die entsprechende Seite (Hauptseite oder Login) weiter.
 \texdocbr{}

 \textit{$/$processLogin}}{}
\texdocmethod{public}{void}{loginPage}{()}{Methode zum anzeigen der Login-Seite\texdocbr{}

 \textit{$/$login}}{}
\texdocmethod{public}{void}{mainPage}{()}{Methode zur Anzeige der Hauptseite\texdocbr{}

 \textit{$/$}}{}
\texdocmethod{public}{void}{showProfile}{()}{Methode zum anzeigen des Profils\texdocbr{}

 \textit{$/$profile}}{}
\texdocmethod{public}{void}{signUpWizard}{()}{Methode zum anzeigen des Registrierungs-Wizards\texdocbr{}

 \textit{$/$signup}}{}
\end{texdocclassmethods}
\end{texdocclass}


\end{texdocpackage}



\begin{texdocpackage}{client.storage}
\label{texdoclet:edu.kit.informatik.studyplan.client.storage}

\begin{texdocclass}{class}{Sync}
\label{texdoclet:edu.kit.informatik.studyplan.client.storage.Sync}
\begin{texdocclassintro}
Dieses Objekt kapselt die Methoden zur Synchronisierung mit Server oder
 Cookies, welche von den verschiedenen Model-Typen verwendet werden.\texdocbr{}

 Die Methoden sind von Backbone spezifiziert und werden jeweils Submethoden
 enthalten, welche die einzelnen Methoden behandeln\end{texdocclassintro}
\begin{texdocclassconstructors}
\texdocconstructor{public}{Sync}{()}{}{}
\end{texdocclassconstructors}
\begin{texdocclassmethods}
\texdocmethod{public}{void}{CookieSync}{(String method, Object model, Object options)}{}{\begin{texdocparameters}
\texdocparameter{method}{}
\texdocparameter{model}{}
\texdocparameter{options}{}
\end{texdocparameters}
}
\texdocmethod{public}{void}{OAuthSync}{(String method, Object model, Object options)}{Methode zur synchronisierung mittels OAuth-REST-API}{\begin{texdocparameters}
\texdocparameter{method}{Die Methode "create", "read", "update" oder "delete"}
\texdocparameter{model}{Das zu speicherende Model}
\texdocparameter{options}{Die Optionen, welche bei Backbone.sync verfügbar sind}
\end{texdocparameters}
}
\end{texdocclassmethods}
\end{texdocclass}


\end{texdocpackage}



\begin{texdocpackage}{client.view.components.filter}
\label{texdoclet:edu.kit.informatik.studyplan.client.view.components.filter}

\begin{texdocclass}{class}{FilterComponent}
\label{texdoclet:edu.kit.informatik.studyplan.client.view.components.filter.FilterComponent}
\begin{texdocclassintro}
Klasse, welche die Anzeige eine eines Filters kapselt.\end{texdocclassintro}
\begin{texdocclassconstructors}
\texdocconstructor{public}{FilterComponent}{(Object params)}{}{}
\end{texdocclassconstructors}
\begin{texdocclassmethods}
\texdocmethod{public}{Filter}{getFilter}{()}{Methode zum erhalten des Filter-Modells}{\texdocreturn{}
}
\texdocmethod{public}{String}{getId}{()}{Methode zum erhalten der ID des Filters}{\texdocreturn{Die ID des Filters}
}
\texdocmethod{public abstract}{void}{onSelect}{()}{Methode welche bei der Auswahl eines Filterobjekts ausgeführt wird}{}
\end{texdocclassmethods}
\end{texdocclass}


\begin{texdocclass}{class}{ModuleFilter}
\label{texdoclet:edu.kit.informatik.studyplan.client.view.components.filter.ModuleFilter}
\begin{texdocclassintro}
Ein Filter, welcher die Menge der angezeigten Module auf Basis von Filtern
 einschränkt\end{texdocclassintro}
\begin{texdocclassconstructors}
\texdocconstructor{public}{ModuleFilter}{(Object params)}{}{}
\end{texdocclassconstructors}
\begin{texdocclassmethods}
\texdocmethod{public}{FilterCollection}{getFilters}{()}{Methode, die die ausgewählten Filter zurückgibt}{}
\texdocmethod{public}{void}{onSearch}{()}{Methode, die bei der Suche ausgeführt wird. Diese kann über den
 Konstruktor als Callback gesetzt werden}{}
\end{texdocclassmethods}
\end{texdocclass}


\begin{texdocclass}{class}{RangeFilter}
\label{texdoclet:edu.kit.informatik.studyplan.client.view.components.filter.RangeFilter}
\begin{texdocclassintro}
Filter, welcher durch die Auswahl eines Bereichs aktiviert wird\end{texdocclassintro}
\begin{texdocclassconstructors}
\texdocconstructor{public}{RangeFilter}{(Object params)}{}{}
\end{texdocclassconstructors}
\begin{texdocclassmethods}
\texdocmethod{public}{void}{onSelect}{()}{Methode welche bei der Auswahl eines Filterobjekts ausgeführt wird}{}
\end{texdocclassmethods}
\end{texdocclass}


\begin{texdocclass}{class}{SelectFilter}
\label{texdoclet:edu.kit.informatik.studyplan.client.view.components.filter.SelectFilter}
\begin{texdocclassintro}
Filter, welcher durch die Auswahl eines Elements aktiviert wird\end{texdocclassintro}
\begin{texdocclassconstructors}
\texdocconstructor{public}{SelectFilter}{(Object params)}{}{}
\end{texdocclassconstructors}
\begin{texdocclassmethods}
\texdocmethod{public}{void}{onSelect}{()}{Methode welche bei der Auswahl eines Filterobjekts ausgeführt wird}{}
\end{texdocclassmethods}
\end{texdocclass}


\begin{texdocclass}{class}{TextFilter}
\label{texdoclet:edu.kit.informatik.studyplan.client.view.components.filter.TextFilter}
\begin{texdocclassintro}
Filter, welcher durch die Eingabe eines Suchtexts aktiviert wird\end{texdocclassintro}
\begin{texdocclassconstructors}
\texdocconstructor{public}{TextFilter}{(Object params)}{}{}
\end{texdocclassconstructors}
\begin{texdocclassmethods}
\texdocmethod{public}{void}{onSelect}{()}{Methode welche bei der Auswahl eines Filterobjekts ausgeführt wird}{}
\end{texdocclassmethods}
\end{texdocclass}


\end{texdocpackage}



\begin{texdocpackage}{client.view.components.uielement}
\label{texdoclet:edu.kit.informatik.studyplan.client.view.components.uielement}

\begin{texdocclass}{class}{ModuleBox}
\label{texdoclet:edu.kit.informatik.studyplan.client.view.components.uielement.ModuleBox}
\begin{texdocclassintro}
Diese Klasse repräsentiert eine einzelne Box, welche ein Modul darstellt\end{texdocclassintro}
\begin{texdocclassconstructors}
\texdocconstructor{public}{ModuleBox}{(Object params)}{}{}
\end{texdocclassconstructors}
\begin{texdocclassmethods}
\texdocmethod{public}{void}{click}{()}{Methode, welche beim Klick auf das Modul ausgeführt wird.}{}
\texdocmethod{public}{void}{removeModule}{()}{Methode, welche das Modul entfernt}{}
\texdocmethod{public}{void}{setRedBorder}{(boolean setBorder)}{Möglichkeit einen roten Rand um das Modul zu setzten}{\begin{texdocparameters}
\texdocparameter{setBorder}{Ob der Rahmen gesetzt werden soll (true) oder nicht (false)}
\end{texdocparameters}
}
\end{texdocclassmethods}
\end{texdocclass}


\begin{texdocclass}{class}{ModuleFinder}
\label{texdoclet:edu.kit.informatik.studyplan.client.view.components.uielement.ModuleFinder}
\begin{texdocclassintro}
Klasse, welche den Modul Finder repräsentiert. Dieser erlaubt es nach Modulen
 zu suchen und diese (wenn es sich um eine Sidebar handelt) in einen Plan zu
 ziehen.\end{texdocclassintro}
\begin{texdocclassconstructors}
\texdocconstructor{public}{ModuleFinder}{(Object params)}{}{}
\end{texdocclassconstructors}
\begin{texdocclassmethods}
\texdocmethod{public}{void}{onSearch}{()}{Methode die bei der Suche nach Modulen aufgerufen wird.}{}
\end{texdocclassmethods}
\end{texdocclass}


\begin{texdocclass}{class}{ModuleInfoSidebar}
\label{texdoclet:edu.kit.informatik.studyplan.client.view.components.uielement.ModuleInfoSidebar}
\begin{texdocclassintro}
Klasse, welche die Modul-Informations-Sidebar kapselt.\end{texdocclassintro}
\begin{texdocclassconstructors}
\texdocconstructor{public}{ModuleInfoSidebar}{(Object params)}{}{}
\end{texdocclassconstructors}
\begin{texdocclassmethods}
\texdocmethod{public}{void}{onChange}{()}{Methode, die bei Veränderungen des Moduls aufgerufen wird (notwendig
 wegen dynamischem Laden des Moduls).}{}
\texdocmethod{public}{void}{onClose}{()}{Methode, die beim Schließen der ModuleInfoSidebar aufgerufen wird.}{}
\end{texdocclassmethods}
\end{texdocclass}


\begin{texdocclass}{class}{ModuleList}
\label{texdoclet:edu.kit.informatik.studyplan.client.view.components.uielement.ModuleList}
\begin{texdocclassintro}
\end{texdocclassintro}
\begin{texdocclassconstructors}
\texdocconstructor{public}{ModuleList}{(Object params)}{}{}
\end{texdocclassconstructors}
\end{texdocclass}


\begin{texdocclass}{class}{NotificationBox}
\label{texdoclet:edu.kit.informatik.studyplan.client.view.components.uielement.NotificationBox}
\begin{texdocclassintro}
Klasse welche eine Benachrichtigungsbox kapselt\end{texdocclassintro}
\begin{texdocclassconstructors}
\texdocconstructor{public}{NotificationBox}{(Object params)}{}{}
\end{texdocclassconstructors}
\begin{texdocclassmethods}
\texdocmethod{public}{void}{blurOut}{()}{Die Methode, welche nach einem gegebenen Zeitintervall die
 Benachrichtigung ausblendet}{}
\texdocmethod{public}{void}{onClose}{()}{Die Methode, welche die Benachrichtigungsbox schließt}{}
\end{texdocclassmethods}
\end{texdocclass}


\begin{texdocclass}{class}{PassedModulePlan}
\label{texdoclet:edu.kit.informatik.studyplan.client.view.components.uielement.PassedModulePlan}
\begin{texdocclassintro}
Plan der bestandenen Module\end{texdocclassintro}
\begin{texdocclassconstructors}
\texdocconstructor{public}{PassedModulePlan}{(Object params)}{}{}
\end{texdocclassconstructors}
\end{texdocclass}


\begin{texdocclass}{class}{Plan}
\label{texdoclet:edu.kit.informatik.studyplan.client.view.components.uielement.Plan}
\begin{texdocclassintro}
Klasse, welche die Ansicht des Studienplans kapselt\end{texdocclassintro}
\begin{texdocclassconstructors}
\texdocconstructor{public}{Plan}{(Object params)}{}{}
\end{texdocclassconstructors}
\begin{texdocclassmethods}
\texdocmethod{public}{void}{addSemester}{()}{Methode zum hinzufügen neuer Semester}{}
\texdocmethod{public}{void}{onChange}{()}{Methode, welche bei Änderungen des Plans aufgerufen wird, und welche auch
 die onChange() Methoden der Semester aufruft.}{}
\end{texdocclassmethods}
\end{texdocclass}


\begin{texdocclass}{class}{PlanHeadBar}
\label{texdoclet:edu.kit.informatik.studyplan.client.view.components.uielement.PlanHeadBar}
\begin{texdocclassintro}
Klasse, welche die Kopfzeile des Plans kapselt\end{texdocclassintro}
\begin{texdocclassconstructors}
\texdocconstructor{public}{PlanHeadBar}{(Object params)}{}{}
\end{texdocclassconstructors}
\begin{texdocclassmethods}
\texdocmethod{public}{void}{onChange}{()}{Die Methode, welche bei einer Änderung des Plans aufgerufen wird.}{}
\end{texdocclassmethods}
\end{texdocclass}


\begin{texdocclass}{class}{PlanListElement}
\label{texdoclet:edu.kit.informatik.studyplan.client.view.components.uielement.PlanListElement}
\begin{texdocclassintro}
Klasse welche ein Element einer PlanList kapselt\end{texdocclassintro}
\begin{texdocclassconstructors}
\texdocconstructor{public}{PlanListElement}{(Object params)}{}{}
\end{texdocclassconstructors}
\begin{texdocclassmethods}
\texdocmethod{public}{void}{delete}{()}{Methode, welche beim Klick auf "löschen" aufgerufen wird}{}
\texdocmethod{public}{void}{duplicate}{()}{Methode, welche beim Klick auf "duplizieren" aufgerufen wird}{}
\texdocmethod{public}{void}{export}{()}{Methode, welche beim Klick auf "exportieren" aufgerufen wird}{}
\texdocmethod{public}{void}{show}{()}{Methode, welche beim Klick auf "anzeigen" aufgerufen wird}{}
\end{texdocclassmethods}
\end{texdocclass}


\begin{texdocclass}{class}{ProposedHeadBar}
\label{texdoclet:edu.kit.informatik.studyplan.client.view.components.uielement.ProposedHeadBar}
\begin{texdocclassintro}
Kopfzeile, welche angezeigt wird, wenn ein vorgeschlagener Plan angezeigt
 wird\end{texdocclassintro}
\begin{texdocclassconstructors}
\texdocconstructor{public}{ProposedHeadBar}{(Object params)}{}{}
\end{texdocclassconstructors}
\end{texdocclass}


\begin{texdocclass}{class}{RegularHeadBar}
\label{texdoclet:edu.kit.informatik.studyplan.client.view.components.uielement.RegularHeadBar}
\begin{texdocclassintro}
Kopfzeile welche angezeigt wird, wenn ein regulärer Plan angezeigt wird\end{texdocclassintro}
\begin{texdocclassconstructors}
\texdocconstructor{public}{RegularHeadBar}{(Object params)}{}{}
\end{texdocclassconstructors}
\begin{texdocclassmethods}
\texdocmethod{public}{void}{generate}{()}{Methode, welche die Generierung eines neuen Plans anstößt}{}
\texdocmethod{public}{void}{rename}{()}{Methode, welche die Umbenennung eines Plans ausführt}{}
\texdocmethod{public}{void}{verify}{()}{Methode, welche die Verifizierung eines neuen Plans anstößt}{}
\end{texdocclassmethods}
\end{texdocclass}


\begin{texdocclass}{class}{Semester}
\label{texdoclet:edu.kit.informatik.studyplan.client.view.components.uielement.Semester}
\begin{texdocclassintro}
Klasse, welche ein Semester innerhalb eines Studienplans kapselt\end{texdocclassintro}
\begin{texdocclassconstructors}
\texdocconstructor{public}{Semester}{(Object params)}{}{}
\end{texdocclassconstructors}
\begin{texdocclassmethods}
\texdocmethod{public}{void}{onDrop}{(Event event, Object ui)}{Methode, welche beim Drop eines Moduls auf das Semester das hinzufügen
 ausführt}{\begin{texdocparameters}
\texdocparameter{event}{Der JQuery Event}
\texdocparameter{ui}{Das JQuery UI Objekt}
\end{texdocparameters}
}
\texdocmethod{public}{void}{removeSemester}{()}{Methode zum löschen des aktuellen Semesters}{}
\texdocmethod{public}{void}{scrollLeft}{()}{Methode zum Links-Scrollen der Semester Leiste}{}
\texdocmethod{public}{void}{scrollRight}{()}{Methode zum Rechts-Scrollen der Semester Leiste}{}
\end{texdocclassmethods}
\end{texdocclass}


\end{texdocpackage}



\begin{texdocpackage}{client.view.components.uipanel}
\label{texdoclet:edu.kit.informatik.studyplan.client.view.components.uipanel}

\begin{texdocclass}{class}{ComparisonView}
\label{texdoclet:edu.kit.informatik.studyplan.client.view.components.uipanel.ComparisonView}
\begin{texdocclassintro}
Anzeige, welche 2 Pläne vergleicht\end{texdocclassintro}
\begin{texdocclassconstructors}
\texdocconstructor{public}{ComparisonView}{(Object params)}{}{}
\end{texdocclassconstructors}
\end{texdocclass}


\begin{texdocclass}{class}{GenerationWizardComponent1}
\label{texdoclet:edu.kit.informatik.studyplan.client.view.components.uipanel.GenerationWizardComponent1}
\begin{texdocclassintro}
Klasse, welche die erste Seite des Generierungs-Wizards kapselt.\end{texdocclassintro}
\begin{texdocclassconstructors}
\texdocconstructor{public}{GenerationWizardComponent1}{(Object params)}{}{}
\end{texdocclassconstructors}
\begin{texdocclassmethods}
\texdocmethod{public}{WizardComponent}{next}{()}{Gibt Objekt vom Typ GenerationWizardComponent2 zurück}{\texdocreturn{Objekt vom Typ GenerationWizardComponent2 (zweite Seite des
         Wizards)}
}
\texdocmethod{public}{void}{onChange}{()}{Methode, welche bei Veränderungen des Seiteninhalts aufgerufen wird.}{}
\end{texdocclassmethods}
\end{texdocclass}


\begin{texdocclass}{class}{GenerationWizardComponent2}
\label{texdoclet:edu.kit.informatik.studyplan.client.view.components.uipanel.GenerationWizardComponent2}
\begin{texdocclassintro}
\end{texdocclassintro}
\begin{texdocclassconstructors}
\texdocconstructor{public}{GenerationWizardComponent2}{(Object params)}{}{}
\end{texdocclassconstructors}
\begin{texdocclassmethods}
\texdocmethod{public}{WizardComponent}{next}{()}{Gibt Objekt vom Typ GenerationWizardComponent3 zurück}{\texdocreturn{Objekt vom Typ GenerationWizardComponent3 (dritte Seite des
         Wizards)}
}
\texdocmethod{public}{void}{onChange}{()}{Methode, welche bei Veränderungen des Seiteninhalts aufgerufen wird.}{}
\end{texdocclassmethods}
\end{texdocclass}


\begin{texdocclass}{class}{GenerationWizardComponent3}
\label{texdoclet:edu.kit.informatik.studyplan.client.view.components.uipanel.GenerationWizardComponent3}
\begin{texdocclassintro}
\end{texdocclassintro}
\begin{texdocclassconstructors}
\texdocconstructor{public}{GenerationWizardComponent3}{(Object params)}{}{}
\end{texdocclassconstructors}
\begin{texdocclassmethods}
\texdocmethod{public}{WizardComponent}{next}{()}{Gibt null zurück, da dies die letzte Seite des Wizards ist.}{\texdocreturn{null}
}
\texdocmethod{public}{void}{onChange}{()}{Methode, welche bei Veränderungen des Seiteninhalts aufgerufen wird.}{}
\end{texdocclassmethods}
\end{texdocclass}


\begin{texdocclass}{class}{NotificationCentre}
\label{texdoclet:edu.kit.informatik.studyplan.client.view.components.uipanel.NotificationCentre}
\begin{texdocclassintro}
Klasse, welche die Anzeige von Benachrichtigungen kapselt\end{texdocclassintro}
\begin{texdocclassconstructors}
\texdocconstructor{public}{NotificationCentre}{(Object params)}{}{}
\end{texdocclassconstructors}
\begin{texdocclassmethods}
\texdocmethod{public}{void}{onChange}{()}{Methode, welche aufgerufen wird, wenn sich die notificationCollection
 verändert.}{}
\end{texdocclassmethods}
\end{texdocclass}


\begin{texdocclass}{class}{PlanList}
\label{texdoclet:edu.kit.informatik.studyplan.client.view.components.uipanel.PlanList}
\begin{texdocclassintro}
Klasse, welche eine Liste von Plänen kapselt\end{texdocclassintro}
\begin{texdocclassconstructors}
\texdocconstructor{public}{PlanList}{(Object params)}{}{}
\end{texdocclassconstructors}
\begin{texdocclassmethods}
\texdocmethod{public}{void}{onActionSelection}{()}{Listener, der aufgerufen wird, wenn eine Aktion für einen$/$mehrere Pläne
 aufgerufen wird.}{}
\texdocmethod{public}{void}{onChange}{()}{Methode, welche bei einer Veränderung der Pläne aufgerufen wird}{}
\end{texdocclassmethods}
\end{texdocclass}


\begin{texdocclass}{class}{ProposalSidebar}
\label{texdoclet:edu.kit.informatik.studyplan.client.view.components.uipanel.ProposalSidebar}
\begin{texdocclassintro}
Sidebar, welche angezeigt wird, wenn ein generierter Plan angezeigt wird.\end{texdocclassintro}
\begin{texdocclassconstructors}
\texdocconstructor{public}{ProposalSidebar}{(Object params)}{}{}
\end{texdocclassconstructors}
\begin{texdocclassmethods}
\texdocmethod{public}{void}{delete}{()}{Methode, welche aufgerufen wird, wenn der Plan gelöscht werden soll}{}
\texdocmethod{public}{void}{save}{()}{Methode, welche aufgerufen wird, wenn der Plan gespeichert werden soll}{}
\texdocmethod{public}{void}{saveAs}{()}{Methode, welche aufgerufen wird, wenn der Plan unter anderem Namen
 gespeichert werden soll}{}
\end{texdocclassmethods}
\end{texdocclass}


\begin{texdocclass}{class}{SignUpWizardComponent1}
\label{texdoclet:edu.kit.informatik.studyplan.client.view.components.uipanel.SignUpWizardComponent1}
\begin{texdocclassintro}
Klasse, welche die erste Seite des Registrierungs-Wizards kapselt\end{texdocclassintro}
\begin{texdocclassconstructors}
\texdocconstructor{public}{SignUpWizardComponent1}{(Object params)}{}{}
\end{texdocclassconstructors}
\begin{texdocclassmethods}
\texdocmethod{public}{WizardComponent}{next}{()}{Gibt ein Objekt vom Typ SignUpWizardComponent2 zurück}{\texdocreturn{Objekt vom Typ SignUpWizardComponent2 (zweite Seite des Wizards)}
}
\texdocmethod{public}{void}{onChange}{()}{Methode, welche bei Veränderungen des Seiteninhalts aufgerufen wird.}{}
\end{texdocclassmethods}
\end{texdocclass}


\begin{texdocclass}{class}{SignUpWizardComponent2}
\label{texdoclet:edu.kit.informatik.studyplan.client.view.components.uipanel.SignUpWizardComponent2}
\begin{texdocclassintro}
Klasse, welche die zweite Seite des Registrierungswizards kapselt\end{texdocclassintro}
\begin{texdocclassconstructors}
\texdocconstructor{public}{SignUpWizardComponent2}{(Object params)}{}{}
\end{texdocclassconstructors}
\begin{texdocclassmethods}
\texdocmethod{public}{WizardComponent}{next}{()}{Gibt null zurück, da dies die letzte Seite des SignUpWizards ist}{\texdocreturn{null}
}
\texdocmethod{public}{void}{onChange}{()}{Methode, welche bei Veränderungen des Seiteninhalts aufgerufen wird.}{}
\end{texdocclassmethods}
\end{texdocclass}


\begin{texdocclass}{class}{WizardComponent}
\label{texdoclet:edu.kit.informatik.studyplan.client.view.components.uipanel.WizardComponent}
\begin{texdocclassintro}
Klasse, welche eine Komponente eines Wizards kapselt. Die Wizards sind nach
 dem Konzept des State-Machine-Patterns aufgebaut\end{texdocclassintro}
\begin{texdocclassconstructors}
\texdocconstructor{public}{WizardComponent}{(Object params)}{}{}
\end{texdocclassconstructors}
\begin{texdocclassmethods}
\texdocmethod{public abstract}{WizardComponent}{next}{()}{Methode, welche den View-Component der nächsten Seite des Wizard zurück
 gibt}{\texdocreturn{Den View-Component der nächsten Siete des Wizard}
}
\end{texdocclassmethods}
\end{texdocclass}


\end{texdocpackage}



\begin{texdocpackage}{client.view}
\label{texdoclet:edu.kit.informatik.studyplan.client.view}

\begin{texdocclass}{class}{MainView}
\label{texdoclet:edu.kit.informatik.studyplan.client.view.MainView}
\begin{texdocclassintro}
Der MainView kapselt die Benutzeroberfläche, er kann mittels gegebener
 Funktionen mit beliebigem Inhalt in Form von Objekten des Typs Backbone.View
 befüllt werden.\end{texdocclassintro}
\begin{texdocclassconstructors}
\texdocconstructor{public}{MainView}{()}{}{}
\end{texdocclassconstructors}
\begin{texdocclassmethods}
\texdocmethod{public}{void}{setContent}{(BackboneView content, Object options)}{Mit dieser Funktion kann man den Content-Bereich des MainViews setzen.}{\begin{texdocparameters}
\texdocparameter{content}{Der zu setztende Content-Bereich}
\texdocparameter{options}{Objekt mit welchem der Header initialisiert werden soll}
\end{texdocparameters}
}
\texdocmethod{public}{void}{setHeader}{(BackboneView header, Object options)}{Mit dieser Methode kann man den Header des MainViews setzen}{\begin{texdocparameters}
\texdocparameter{header}{Der zu setzende Header}
\texdocparameter{options}{Objekt mit welchem der Header initialisiert werden soll}
\end{texdocparameters}
}
\end{texdocclassmethods}
\end{texdocclass}


\end{texdocpackage}



\begin{texdocpackage}{client.view.subview}
\label{texdoclet:edu.kit.informatik.studyplan.client.view.subview}

\begin{texdocclass}{class}{ComparisonPage}
\label{texdoclet:edu.kit.informatik.studyplan.client.view.subview.ComparisonPage}
\begin{texdocclassintro}
Klasse, welche die Vergleichsseite kapselt\end{texdocclassintro}
\begin{texdocclassconstructors}
\texdocconstructor{public}{ComparisonPage}{(Object params)}{}{}
\end{texdocclassconstructors}
\end{texdocclass}


\begin{texdocclass}{class}{Header}
\label{texdoclet:edu.kit.informatik.studyplan.client.view.subview.Header}
\begin{texdocclassintro}
Klasse, welche den Header der Seite kapselt\end{texdocclassintro}
\begin{texdocclassconstructors}
\texdocconstructor{public}{Header}{(Object params)}{}{}
\end{texdocclassconstructors}
\end{texdocclass}


\begin{texdocclass}{class}{LoginPage}
\label{texdoclet:edu.kit.informatik.studyplan.client.view.subview.LoginPage}
\begin{texdocclassintro}
Klasse, welche die Login-Seite kapselt\end{texdocclassintro}
\begin{texdocclassconstructors}
\texdocconstructor{public}{LoginPage}{(Object params)}{}{}
\end{texdocclassconstructors}
\end{texdocclass}


\begin{texdocclass}{class}{MainPage}
\label{texdoclet:edu.kit.informatik.studyplan.client.view.subview.MainPage}
\begin{texdocclassintro}
Klasse, welche die Hauptseite kapselt\end{texdocclassintro}
\begin{texdocclassconstructors}
\texdocconstructor{public}{MainPage}{(Object params)}{}{}
\end{texdocclassconstructors}
\begin{texdocclassmethods}
\texdocmethod{public}{void}{addPlan}{()}{Methode zum hinzufügen eines Plans}{}
\end{texdocclassmethods}
\end{texdocclass}


\begin{texdocclass}{class}{PlanEditPage}
\label{texdoclet:edu.kit.informatik.studyplan.client.view.subview.PlanEditPage}
\begin{texdocclassintro}
Klasse, welche die Plan-Bearbeitungsansicht kapselt\end{texdocclassintro}
\begin{texdocclassconstructors}
\texdocconstructor{public}{PlanEditPage}{(Object params)}{}{}
\end{texdocclassconstructors}
\begin{texdocclassmethods}
\texdocmethod{public}{void}{hideModuleDetails}{()}{Methode zum schließen einer ModuleInfoSidebar}{}
\texdocmethod{public}{void}{showModuleDetails}{(Module module)}{Methode zum aufrufen einer ModuleInfoSidebar}{\begin{texdocparameters}
\texdocparameter{module}{}
\end{texdocparameters}
}
\end{texdocclassmethods}
\end{texdocclass}


\begin{texdocclass}{class}{ProfilePage}
\label{texdoclet:edu.kit.informatik.studyplan.client.view.subview.ProfilePage}
\begin{texdocclassintro}
Klasse, welche die Profil-Seite kapselt\end{texdocclassintro}
\begin{texdocclassconstructors}
\texdocconstructor{public}{ProfilePage}{(Object params)}{}{}
\end{texdocclassconstructors}
\begin{texdocclassmethods}
\texdocmethod{public}{void}{close}{()}{Leitet zurück zur Hauptseite (MainPage) und speichert die Änderungen.}{}
\texdocmethod{public}{void}{hideModuleDetails}{()}{Methode zum schließen einer ModuleInfoSidebar}{}
\texdocmethod{public}{void}{showModuleDetails}{(Module module)}{Methode zum aufrufen einer ModuleInfoSidebar}{\begin{texdocparameters}
\texdocparameter{module}{}
\end{texdocparameters}
}
\end{texdocclassmethods}
\end{texdocclass}


\begin{texdocclass}{class}{WizardPage}
\label{texdoclet:edu.kit.informatik.studyplan.client.view.subview.WizardPage}
\begin{texdocclassintro}
Diese Klasse kapselt die Anzeige von Wizards in einem State-Machine-Pattern.
 curView wird bei der Initialisierung gesetzt. WizardPage kann deshalb für
 jedwede Wizards verwendet werden.\texdocbr{}

 WizardPage implementiert Backbone.Event, sodass man sich als Observer der
 WizardPage registrieren kann. Hier kann man auf den "wizardComplete"-Event
 reagieren, welcher bedeutet, dass der Wizard beendet ist.\end{texdocclassintro}
\begin{texdocclassfields}
\texdocfield{public}{WizardComponent}{curView}{Der aktuell angezeigte WizardComponent}
\end{texdocclassfields}
\begin{texdocclassconstructors}
\texdocconstructor{public}{WizardPage}{(Object params)}{}{}
\end{texdocclassconstructors}
\begin{texdocclassmethods}
\texdocmethod{public}{void}{next}{()}{Methode, welche die nächste Seite des Wizards aufruft oder diesen beendet.}{}
\end{texdocclassmethods}
\end{texdocclass}


\end{texdocpackage}



\begin{texdocpackage}{server.filter}
\label{texdoclet:edu.kit.informatik.studyplan.server.filter}

\begin{texdocclass}{interface}{AttributeFilter}
\label{texdoclet:edu.kit.informatik.studyplan.server.filter.AttributeFilter}
\begin{texdocclassintro}
Repräsentiert einen Filter für ein bestimmtes Module-Attribut.\end{texdocclassintro}
\begin{texdocclassmethods}
\texdocmethod{public}{FilterDescriptor}{getDescriptor}{()}{Liefert die zum AttributeFilter gehörende Filterbeschreibung.}{\texdocreturn{die Filterbeschreibung}
}
\texdocmethod{public}{FilterType}{getFilterType}{()}{Liefert den Filter-Typ des AttributeFilters.}{\texdocreturn{der Filter-Typ}
}
\end{texdocclassmethods}
\end{texdocclass}


\begin{texdocclass}{class}{CategoryFilter}
\label{texdoclet:edu.kit.informatik.studyplan.server.filter.CategoryFilter}
\begin{texdocclassintro}
Repräsentiert einen Kategorie-Wahlfilter mit den Modulkategorien als Wahlmöglichkeiten.\end{texdocclassintro}
\begin{texdocclassconstructors}
\texdocconstructor{public}{CategoryFilter}{(int selection)}{Erzeugt einen neuen Kategorie-Wahlfilter mit gegebener festgelegter Auswahl.}{\begin{texdocparameters}
\texdocparameter{selection}{die Nummer des ausgewählten Elements}
\end{texdocparameters}
}
\end{texdocclassconstructors}
\begin{texdocclassmethods}
\texdocmethod{public}{FilterDescriptor}{getDescriptor}{()}{}{}
\texdocmethod{public}{List\textless{}String\textgreater{}}{getItems}{()}{}{}
\end{texdocclassmethods}
\end{texdocclass}


\begin{texdocclass}{class}{CompulsoryFilter}
\label{texdoclet:edu.kit.informatik.studyplan.server.filter.CompulsoryFilter}
\begin{texdocclassintro}
Repräsentiert einen Pflicht-$/$Wahlmodul-Auswahlfilter mit Filterung nach
 Pflicht-, Wahlmodulen oder beidem als Wahlmöglichkeiten.\end{texdocclassintro}
\begin{texdocclassconstructors}
\texdocconstructor{public}{CompulsoryFilter}{(int selection)}{Erzeugt einen neuen Pflicht-$/$Wahlmodul-Auswahlfilter mit gegebener festgelegter Auswahl.}{\begin{texdocparameters}
\texdocparameter{selection}{die Nummer des ausgewählten Elements}
\end{texdocparameters}
}
\end{texdocclassconstructors}
\begin{texdocclassmethods}
\texdocmethod{public}{FilterDescriptor}{getDescriptor}{()}{}{}
\texdocmethod{public}{List\textless{}String\textgreater{}}{getItems}{()}{}{}
\end{texdocclassmethods}
\end{texdocclass}


\begin{texdocclass}{class}{ContainsFilter}
\label{texdoclet:edu.kit.informatik.studyplan.server.filter.ContainsFilter}
\begin{texdocclassintro}
Repräsentiert einen Textsuch-Attribut-Filter.\end{texdocclassintro}
\begin{texdocclassfields}
\texdocfield{protected}{String}{substring}{Der Suchstring.}
\end{texdocclassfields}
\begin{texdocclassconstructors}
\texdocconstructor{protected}{ContainsFilter}{(String substring)}{Erzeugt einen neuen Textsuch-Filter mit gegebenem Suchstring.}{\begin{texdocparameters}
\texdocparameter{substring}{der Suchstring}
\end{texdocparameters}
}
\end{texdocclassconstructors}
\begin{texdocclassmethods}
\texdocmethod{public}{Condition}{getCondition}{()}{Liefert eine Filterbedingung, die das Vorkommen des Substrings im Attributswert fordert.}{\texdocreturn{die Filterbedingung als jOOQ-Condition-Objekt}
}
\texdocmethod{public abstract}{FilterDescriptor}{getDescriptor}{()}{}{}
\texdocmethod{public}{FilterType}{getFilterType}{()}{}{}
\texdocmethod{public}{String}{getSubstring}{()}{Liefert den Substring, nach welchem gefiltert werden soll.}{\texdocreturn{der Substring}
}
\end{texdocclassmethods}
\end{texdocclass}


\begin{texdocclass}{class}{CreditPointsFilter}
\label{texdoclet:edu.kit.informatik.studyplan.server.filter.CreditPointsFilter}
\begin{texdocclassintro}
Repräsentiert einen ECTS-Intervall-Filter.\end{texdocclassintro}
\begin{texdocclassconstructors}
\texdocconstructor{public}{CreditPointsFilter}{(int lower, int upper, int min, int max)}{Erzeugt einen neuen ECTS-Intervall-Filter mit gegebenen Schranken.}{\begin{texdocparameters}
\texdocparameter{lower}{untere Schranke des Filters}
\texdocparameter{upper}{obere Schranke des Filters}
\texdocparameter{min}{minimale untere Schranke des Filters}
\texdocparameter{max}{maximale obere Schranke des Filters}
\end{texdocparameters}
}
\end{texdocclassconstructors}
\begin{texdocclassmethods}
\texdocmethod{public}{FilterDescriptor}{getDescriptor}{()}{}{}
\end{texdocclassmethods}
\end{texdocclass}


\begin{texdocclass}{class}{CycleTypeFilter}
\label{texdoclet:edu.kit.informatik.studyplan.server.filter.CycleTypeFilter}
\begin{texdocclassintro}
Repräsentiert einen Turnus-Auswahlfilter mit Filterung nach Winter-,
 Sommersemester oder beidem als Wahlmöglichkeiten.\end{texdocclassintro}
\begin{texdocclassconstructors}
\texdocconstructor{public}{CycleTypeFilter}{(int selection)}{Erzeugt einen neuen Turnus-Auswahlfilter mit gegebener festgelegter Auswahl.}{\begin{texdocparameters}
\texdocparameter{selection}{die Nummer des ausgewählten Elements}
\end{texdocparameters}
}
\end{texdocclassconstructors}
\begin{texdocclassmethods}
\texdocmethod{public}{FilterDescriptor}{getDescriptor}{()}{}{}
\texdocmethod{public}{List\textless{}String\textgreater{}}{getItems}{()}{}{}
\end{texdocclassmethods}
\end{texdocclass}


\begin{texdocclass}{class}{DisciplineFilter}
\label{texdoclet:edu.kit.informatik.studyplan.server.filter.DisciplineFilter}
\begin{texdocclassintro}
Repräsentiert einen Fachrichtungs-Wahlfilter mit den Fachrichtungen als Wahlmöglichkeiten.\end{texdocclassintro}
\begin{texdocclassconstructors}
\texdocconstructor{public}{DisciplineFilter}{(int selection)}{Erzeugt einen neuen Fachrichtungs-Auswahlfilter mit gegebener festgelegter Auswahl.}{\begin{texdocparameters}
\texdocparameter{selection}{die Nummer des ausgewählten Elements}
\end{texdocparameters}
}
\end{texdocclassconstructors}
\begin{texdocclassmethods}
\texdocmethod{public}{FilterDescriptor}{getDescriptor}{()}{}{}
\texdocmethod{public}{List\textless{}String\textgreater{}}{getItems}{()}{}{}
\end{texdocclassmethods}
\end{texdocclass}


\begin{texdocclass}{interface}{Filter}
\label{texdoclet:edu.kit.informatik.studyplan.server.filter.Filter}
\begin{texdocclassintro}
Repräsentiert einen Filter für Module über eine Filterbedingung.\end{texdocclassintro}
\begin{texdocclassmethods}
\texdocmethod{public}{Condition}{getCondition}{()}{Gibt die Filterbedingung als jOOQ-Condition-Objekt zurück.}{\texdocreturn{die Filterbedingung}
}
\end{texdocclassmethods}
\end{texdocclass}


\begin{texdocclass}{enum}{FilterDescriptor}
\label{texdoclet:edu.kit.informatik.studyplan.server.filter.FilterDescriptor}
\begin{texdocclassintro}
Beschreibungen der nach außen sichtbaren Filterklassen für den Client.\end{texdocclassintro}
\begin{texdocenums}
\texdocenum{CATEGORY}{Beschreibt den Kategorie-Wahlfilter.}
\texdocenum{COMPULSORY}{Beschreibt den Pflicht-$/$Wahlveranstaltungs-Wahlfilter.}
\texdocenum{CREDIT\_POINTS}{Beschreibt den ECTS-Intervall-Filter.}
\texdocenum{CYCLE\_TYPE}{Beschreibt den Turnus-Wahlfilter.}
\texdocenum{DISCIPLINE}{Beschreibt den Fachrichtungs-Wahlfilter.}
\texdocenum{NAME}{Beschreibt den Modulnamen-Textfilter.}
\texdocenum{TYPE}{Beschreibt den Modultyp-Wahlfilter.}
\end{texdocenums}
\begin{texdocclassmethods}
\texdocmethod{public}{String}{attributeName}{()}{Liefert den Namen des Module-Attributs, das durch beschriebenen Filter gefiltert werden soll.}{\texdocreturn{der Attribut-Name}
}
\texdocmethod{public}{AttributeFilter}{defaultFilter}{()}{Liefert ein Default-Objekt des beschriebenen Filtertyps.}{\texdocreturn{das Default-Objekt}
}
\texdocmethod{public}{String}{filterName}{()}{Liefert den Namen des Filters (für die Benutzeroberfläche).}{\texdocreturn{den Filternamen}
}
\texdocmethod{public}{int}{id}{()}{Liefert die ID des Filters.}{\texdocreturn{die ID des Filters}
}
\texdocmethod{public}{Field\textless{}Object\textgreater{}}{toField}{()}{Gibt ein jOOQ-Field-Objekt zurück, das des Filters attributeName kapselt.}{\texdocreturn{das jOOQ-Field-Objekt}
}
\texdocmethod{public}{JSONObject}{toJson}{()}{Liefert eine JSON-Repräsentation des beschriebenen Filters für den Client.}{\texdocreturn{eine JSON-Repräsentation des beschriebenen Filters}
}
\texdocmethod{public}{String}{tooltip}{()}{Liefert den zum Filter gehörenden Tooltip (für die Benutzeroberfläche).}{\texdocreturn{das Tooltip zum Filter}
}
\texdocmethod{public static}{FilterDescriptor}{valueOf}{(String name)}{}{}
\texdocmethod{public static}{FilterDescriptor}{values}{()}{}{}
\end{texdocclassmethods}
\end{texdocclass}


\begin{texdocclass}{enum}{FilterType}
\label{texdoclet:edu.kit.informatik.studyplan.server.filter.FilterType}
\begin{texdocclassintro}
Aufzählung der verschiedenen AttributeFilter-Typen.\end{texdocclassintro}
\begin{texdocenums}
\texdocenum{CONTAINS}{Repräsentiert den Filtertyp ContainsFilter (siehe \ref{texdoclet:edu.kit.informatik.studyplan.server.filter.ContainsFilter}).}
\texdocenum{LIST}{Repräsentiert den Filtertyp ListFilter (siehe \ref{texdoclet:edu.kit.informatik.studyplan.server.filter.ListFilter}).}
\texdocenum{RANGE}{Repräsentiert den Filtertyp RangeFilter (siehe \ref{texdoclet:edu.kit.informatik.studyplan.server.filter.RangeFilter}).}
\end{texdocenums}
\begin{texdocclassmethods}
\texdocmethod{public abstract}{JSONObject}{defaultJsonValue}{(AttributeFilter defaultFilter)}{Liefert eine JSON-Repräsentation der Werte des übergebenen Default-Filters.}{\begin{texdocparameters}
\texdocparameter{defaultFilter}{das Filter-Objekt mit Default-Werten}
\end{texdocparameters}
\texdocreturn{die Werte des Default-Filters}
}
\texdocmethod{public abstract}{JSONObject}{toJsonSpecification}{(AttributeFilter defaultFilter)}{Liefert den Spezifikation-Abschnitt der JSON-Repräsentation des übergebenen Default-Filters.}{\begin{texdocparameters}
\texdocparameter{defaultFilter}{der Default-Filter}
\end{texdocparameters}
\texdocreturn{die Spezifikation des Filters als JSON-Objekt}
}
\texdocmethod{public static}{FilterType}{valueOf}{(String name)}{}{}
\texdocmethod{public static}{FilterType}{values}{()}{}{}
\end{texdocclassmethods}
\end{texdocclass}


\begin{texdocclass}{class}{ListFilter}
\label{texdoclet:edu.kit.informatik.studyplan.server.filter.ListFilter}
\begin{texdocclassintro}
Repräsentiert einen Listenauswahl-Attribut-Filter.\end{texdocclassintro}
\begin{texdocclassfields}
\texdocfield{protected}{int}{selection}{Die Nummer des ausgewählten Elements.}
\end{texdocclassfields}
\begin{texdocclassconstructors}
\texdocconstructor{protected}{ListFilter}{(int selection)}{Erzeugt einen neuen Auswahlfilter mit gegebener festgelegter Auswahl.}{\begin{texdocparameters}
\texdocparameter{selection}{die Nummer des ausgewählten Elements}
\end{texdocparameters}
}
\end{texdocclassconstructors}
\begin{texdocclassmethods}
\texdocmethod{public}{Condition}{getCondition}{()}{Liefert eine Filterbedingung, die die Übereinstimmung des untersuchten Attributswertes mit der festgelegten
 Auswahl (selection) fordert.}{\texdocreturn{die Filterbedingung als jOOQ-Condition-Objekt}
}
\texdocmethod{public abstract}{FilterDescriptor}{getDescriptor}{()}{}{}
\texdocmethod{public}{FilterType}{getFilterType}{()}{}{}
\texdocmethod{public abstract}{List\textless{}String\textgreater{}}{getItems}{()}{Liefert alle Wahlmöglichkeiten dieses Auswahl-Filters als Strings.}{\texdocreturn{die Wahlmöglichkeiten des Auswahl-Filters}
}
\texdocmethod{public}{int}{getSelection}{()}{Liefert die Nummer des selektierten Elements.}{\texdocreturn{die Element-Nummer}
}
\end{texdocclassmethods}
\end{texdocclass}


\begin{texdocclass}{class}{MultiFilter}
\label{texdoclet:edu.kit.informatik.studyplan.server.filter.MultiFilter}
\begin{texdocclassintro}
Bündelt mehrere Filter zu einem einzigen mittels UND-Verknüpfung der Filterbedingungen.\end{texdocclassintro}
\begin{texdocclassconstructors}
\texdocconstructor{public}{MultiFilter}{(List\textless{}Filter\textgreater{} filters)}{Erzeugt einen neuen MultiFilter aus gegebenen Unterfiltern.}{\begin{texdocparameters}
\texdocparameter{filters}{die Unterfilter, die gebündelt werden sollen}
\end{texdocparameters}
}
\end{texdocclassconstructors}
\begin{texdocclassmethods}
\texdocmethod{public}{Condition}{getCondition}{()}{Gibt die UND-Verknüpfung der Filterbedingungen der gebündelten Filter
 als Filterbedingung zurück, oder eine konstant wahre Filterbedingung,
 falls filters leer ist.}{\texdocreturn{Die neue Filterbedingung als jOOQ-Condition-Objekt}
}
\texdocmethod{public}{List\textless{}Filter\textgreater{}}{getFilters}{()}{Gibt alle Filter zurück, die von diesem MultiFilter gebündelt werden.}{\texdocreturn{die gebündelten Filter}
}
\end{texdocclassmethods}
\end{texdocclass}


\begin{texdocclass}{class}{NameFilter}
\label{texdoclet:edu.kit.informatik.studyplan.server.filter.NameFilter}
\begin{texdocclassintro}
Repräsentiert einen Modulnamen-Textsuchfilter.\end{texdocclassintro}
\begin{texdocclassconstructors}
\texdocconstructor{public}{NameFilter}{(String substring)}{Erzeugt einen neuen Modulnamen-Textsuchfilter mit gegebenem Suchstring.}{\begin{texdocparameters}
\texdocparameter{substring}{der Suchstring}
\end{texdocparameters}
}
\end{texdocclassconstructors}
\begin{texdocclassmethods}
\texdocmethod{public}{FilterDescriptor}{getDescriptor}{()}{}{}
\end{texdocclassmethods}
\end{texdocclass}


\begin{texdocclass}{class}{RangeFilter}
\label{texdoclet:edu.kit.informatik.studyplan.server.filter.RangeFilter}
\begin{texdocclassintro}
Repräsentiert einen Intervall-Beschränkungs-Filter für ganzzahlige Attribute.\end{texdocclassintro}
\begin{texdocclassfields}
\texdocfield{protected}{int}{lower}{Die untere Schranke des Filters.}
\texdocfield{protected}{int}{max}{Die maximale untere Schranke des Filters.}
\texdocfield{protected}{int}{min}{Die minimale untere Schranke des Filters.}
\texdocfield{protected}{int}{upper}{Die obere Schranke des Filters.}
\end{texdocclassfields}
\begin{texdocclassconstructors}
\texdocconstructor{protected}{RangeFilter}{(int lower, int upper, int min, int max)}{Erzeugt einen neuen Intervall-Filter mit gegebenen Schranken.}{\begin{texdocparameters}
\texdocparameter{lower}{untere Schranke des Filters}
\texdocparameter{upper}{obere Schranke des Filters}
\texdocparameter{min}{minimale untere Schranke des Filters}
\texdocparameter{max}{maximale obere Schranke des Filters}
\end{texdocparameters}
}
\end{texdocclassconstructors}
\begin{texdocclassmethods}
\texdocmethod{public}{Condition}{getCondition}{()}{Liefert eine Filterbedingung, die vom Attributswert das Einhalten der festgelegten Intervall-Grenzen fordert.}{\texdocreturn{die Filterbedingung als jOOQ-Condition-Objekt}
}
\texdocmethod{public abstract}{FilterDescriptor}{getDescriptor}{()}{}{}
\texdocmethod{public}{FilterType}{getFilterType}{()}{}{}
\texdocmethod{public}{int}{getLower}{()}{Liefert die festgelegte untere Schranke des Filters.}{\texdocreturn{die untere Schranke}
}
\texdocmethod{public}{int}{getMax}{()}{Liefert den Maximalwert der oberen Schranke des Filters.}{\texdocreturn{den Maximalwert der oberen Schranke}
}
\texdocmethod{public}{int}{getMin}{()}{Liefert den Mindestwert der unteren Schranke des Filters.}{\texdocreturn{der Mindestwert der unteren Schranke}
}
\texdocmethod{public}{int}{getUpper}{()}{Liefert die festgelegte obere Schranke des Filters.}{\texdocreturn{die obere Schranke}
}
\end{texdocclassmethods}
\end{texdocclass}


\begin{texdocclass}{class}{TrueFilter}
\label{texdoclet:edu.kit.informatik.studyplan.server.filter.TrueFilter}
\begin{texdocclassintro}
Stellt einen Filter dar, der alle Module zulässt (und dessen Filterbedingung
 daher konstant wahr ist).\end{texdocclassintro}
\begin{texdocclassconstructors}
\texdocconstructor{public}{TrueFilter}{()}{Erzeugt einen neuen TrueFilter.}{}
\end{texdocclassconstructors}
\begin{texdocclassmethods}
\texdocmethod{public}{Condition}{getCondition}{()}{Gibt eine konstant wahre Filterbedingung zurück.}{\texdocreturn{die Filterbedingung als jOOQ-Condition-Objekt}
}
\end{texdocclassmethods}
\end{texdocclass}


\begin{texdocclass}{class}{TypeFilter}
\label{texdoclet:edu.kit.informatik.studyplan.server.filter.TypeFilter}
\begin{texdocclassintro}
Repräsentiert einen Modultyp-Wahlfilter mit den Modultypen als Wahlmöglichkeiten.\end{texdocclassintro}
\begin{texdocclassconstructors}
\texdocconstructor{public}{TypeFilter}{(int selection)}{Erzeugt einen neuen Modultyp-Wahlfilter mit gegebener festgelegter Auswahl.}{\begin{texdocparameters}
\texdocparameter{selection}{die Nummer des ausgewählten Elements}
\end{texdocparameters}
}
\end{texdocclassconstructors}
\begin{texdocclassmethods}
\texdocmethod{public}{FilterDescriptor}{getDescriptor}{()}{}{}
\texdocmethod{public}{List\textless{}String\textgreater{}}{getItems}{()}{}{}
\end{texdocclassmethods}
\end{texdocclass}


\end{texdocpackage}



\begin{texdocpackage}{server.generation}
\label{texdoclet:edu.kit.informatik.studyplan.server.generation}

\begin{texdocclass}{interface}{Generator}
\label{texdoclet:edu.kit.informatik.studyplan.server.generation.Generator}
\begin{texdocclassintro}
Das Interface Generator bietet die allgemeine Struktur eines Generierers.\end{texdocclassintro}
\begin{texdocclassmethods}
\texdocmethod{public}{Plan}{generate}{(PartialObjectiveFunction objectiveFunction, Plan currentPlan, ModuleDao moduleDAO)}{Die Methode generate generiert einen vollständigen, 
 optimierten und korrekten Studienplan. Hierzu nimmt sie 
 einen angefangenen Studienplan entgegen, vervollständigt diesen zunächst nach
 System-Constraints und Zufall und optimiert ihn dann mit 
 Hilfe der übergebenen Zielfunktion objectiveFunktion. Die benötigten Module
 werden über übergebenen ModuleDao erreicht.}{\begin{texdocparameters}
\texdocparameter{objectiveFunction}{Die Zielfunktion, anhand der optimiert werden soll}
\texdocparameter{currentPlan}{der bereits bestehende Plan}
\texdocparameter{moduleDAO}{die Module}
\end{texdocparameters}
\texdocreturn{Zurückgegeben wird ein vollständiger, korrekter und 
 optimierter Studienplan vom Typ Plan}
}
\end{texdocclassmethods}
\end{texdocclass}


\end{texdocpackage}



\begin{texdocpackage}{server.generation.objectivefunction}
\label{texdoclet:edu.kit.informatik.studyplan.server.generation.objectivefunction}

\begin{texdocsees}{See also}
\texdocsee{edu.kit.informatik.studyplan.server.generation.objectivefunction.ObjectiveFunction}{texdoclet:edu.kit.informatik.studyplan.server.generation.objectivefunction.ObjectiveFunction}
\end{texdocsees}
\begin{texdocclass}{class}{AtomObjectiveFunction}
\label{texdoclet:edu.kit.informatik.studyplan.server.generation.objectivefunction.AtomObjectiveFunction}
\begin{texdocclassintro}
AtomObjectiveFunction ist eine Teilzielfunktion, die nur eine Eigenschaft
 berücksichtigt.\end{texdocclassintro}
\begin{texdocclassconstructors}
\texdocconstructor{public}{AtomObjectiveFunction}{()}{}{}
\end{texdocclassconstructors}
\begin{texdocclassmethods}
\texdocmethod{public abstract}{double}{evaluate}{(Plan plan)}{}{}
\end{texdocclassmethods}
\end{texdocclass}


\begin{texdocclass}{class}{AverageObjectiveFunction}
\label{texdoclet:edu.kit.informatik.studyplan.server.generation.objectivefunction.AverageObjectiveFunction}
\begin{texdocclassintro}
Es wird bei der Auswertung der Teilzielfunktionen immer der durchschnitt genomen.\end{texdocclassintro}
\begin{texdocclassconstructors}
\texdocconstructor{public}{AverageObjectiveFunction}{()}{}{}
\end{texdocclassconstructors}
\begin{texdocclassmethods}
\texdocmethod{public}{double}{evaluate}{(Plan plan)}{}{}
\end{texdocclassmethods}
\end{texdocclass}


\begin{texdocclass}{class}{MinimalECTSAtomObjectiveFunction}
\label{texdoclet:edu.kit.informatik.studyplan.server.generation.objectivefunction.MinimalECTSAtomObjectiveFunction}
\begin{texdocclassintro}
Je geringer die Gesamtanzahl der ECTS in einem Studienplan, desto besser ist
 die Bewertung von MinimalECTSAtomObjectiveFunction. Zu beachten ist jedoch,
 dass sobald ein gewisser Schwellwert unterschritten wird alle Studienpläne
 die Bestnote erhalten.\end{texdocclassintro}
\begin{texdocclassconstructors}
\texdocconstructor{public}{MinimalECTSAtomObjectiveFunction}{()}{Setzt den Schwellwert auf 0, der Studienplan mit den
 wenigsten ECTS wird somit immer am besten bewertet.}{}
\texdocconstructor{public}{MinimalECTSAtomObjectiveFunction}{(int threshold)}{}{\begin{texdocparameters}
\texdocparameter{minimalECTS}{ist die Menge an ECTS ab der diese Funktion den Bestwert
          zurückgibt - niedrieger bringt also in diesem Fall dann nichts.}
\end{texdocparameters}
}
\end{texdocclassconstructors}
\begin{texdocclassmethods}
\texdocmethod{public}{double}{evaluate}{(Plan plan)}{}{}
\end{texdocclassmethods}
\end{texdocclass}


\begin{texdocclass}{class}{MinimalSemestersAtomObjectiveFunction}
\label{texdoclet:edu.kit.informatik.studyplan.server.generation.objectivefunction.MinimalSemestersAtomObjectiveFunction}
\begin{texdocclassintro}
Minimiert nach der Semesteranzahl. Ein Studienplan mit einem Semester erhält 
 demzufolge die beste Bewertung.\end{texdocclassintro}
\begin{texdocclassconstructors}
\texdocconstructor{public}{MinimalSemestersAtomObjectiveFunction}{()}{}{}
\end{texdocclassconstructors}
\begin{texdocclassmethods}
\texdocmethod{public}{double}{evaluate}{(Plan plan)}{}{}
\end{texdocclassmethods}
\end{texdocclass}


\begin{texdocclass}{class}{MinimalStandardAverageDeviationECTSAtomObjectiveFunction}
\label{texdoclet:edu.kit.informatik.studyplan.server.generation.objectivefunction.MinimalStandardAverageDeviationECTSAtomObjectiveFunction}
\begin{texdocclassintro}
Minimiert die durchschnittliche Standardabweichung pro Semestester des Palns. 
 Ideal sind deshalb pläne bei denen jedes Semester genau gleich viele ECTS beinhaltet.\end{texdocclassintro}
\begin{texdocclassconstructors}
\texdocconstructor{public}{MinimalStandardAverageDeviationECTSAtomObjectiveFunction}{()}{}{}
\end{texdocclassconstructors}
\begin{texdocclassmethods}
\texdocmethod{public}{double}{evaluate}{(Plan plan)}{}{}
\end{texdocclassmethods}
\end{texdocclass}


\begin{texdocclass}{class}{ModulePreferencesAtomObjectiveFunction}
\label{texdoclet:edu.kit.informatik.studyplan.server.generation.objectivefunction.ModulePreferencesAtomObjectiveFunction}
\begin{texdocclassintro}
Optimiert nach den Nutzerbewertungen. Ideal ist, wenn alle Nutzerbewertungen berücksichtigt wurden, 
 also alle positiv bewerteten Module im Studienplan zu finden sind und kein negativ bewertetes.\end{texdocclassintro}
\begin{texdocclassconstructors}
\texdocconstructor{public}{ModulePreferencesAtomObjectiveFunction}{()}{}{}
\end{texdocclassconstructors}
\begin{texdocclassmethods}
\texdocmethod{public}{double}{evaluate}{(Plan plan)}{}{}
\end{texdocclassmethods}
\end{texdocclass}


\begin{texdocclass}{class}{MultiplicationObjectiveFunction}
\label{texdoclet:edu.kit.informatik.studyplan.server.generation.objectivefunction.MultiplicationObjectiveFunction}
\begin{texdocclassintro}
Multipliziert die Ergebnise der Auswertung Teilzielfunktionen aufeinander auf.\end{texdocclassintro}
\begin{texdocclassconstructors}
\texdocconstructor{public}{MultiplicationObjectiveFunction}{()}{}{}
\end{texdocclassconstructors}
\begin{texdocclassmethods}
\texdocmethod{public}{double}{evaluate}{(Plan plan)}{}{}
\end{texdocclassmethods}
\end{texdocclass}


\begin{texdocclass}{class}{ObjectiveFunction}
\label{texdoclet:edu.kit.informatik.studyplan.server.generation.objectivefunction.ObjectiveFunction}
\begin{texdocclassintro}
Eine Zielfunktion dient zur Sammlung von Teilzielfunktionen, die dann alle ausgewertet werden.\end{texdocclassintro}
\begin{texdocclassconstructors}
\texdocconstructor{public}{ObjectiveFunction}{()}{}{}
\end{texdocclassconstructors}
\begin{texdocclassmethods}
\texdocmethod{public}{void}{add}{(PartialObjectiveFunction objective)}{Fügt eine Teilzielfunktion zu dieser Zielfunktion hinzu.}{\begin{texdocparameters}
\texdocparameter{objective}{die hinzuzufügende Teilzielfunktion}
\end{texdocparameters}
}
\texdocmethod{public abstract}{double}{evaluate}{(Plan plan)}{}{}
\texdocmethod{public}{Collection\textless{}PartialObjectiveFunction\textgreater{}}{getSubFunctions}{()}{Getter für subFunctions.}{\texdocreturn{die Collection subFunctions, bestehend aus
         PartialObjectiveFunctions}
}
\texdocmethod{public}{PartialObjectiveFunction}{remove}{(PartialObjectiveFunction objective)}{Entfernt eine TeilZielfunktion von dieser Zielfunktion}{\begin{texdocparameters}
\texdocparameter{objective}{die zu entfernende Zielfunktion}
\end{texdocparameters}
\texdocreturn{die entfernte Zielfunktion}
}
\end{texdocclassmethods}
\end{texdocclass}


\begin{texdocclass}{interface}{PartialObjectiveFunction}
\label{texdoclet:edu.kit.informatik.studyplan.server.generation.objectivefunction.PartialObjectiveFunction}
\begin{texdocclassintro}
"PartialObjectiveFunction" ist eine Teilzielfunktion. Da jede Zielfunktion
 teil einer anderen Zielfunktion sein kann, ist jede Zielfunktion auch eine
 Teilzielfunktion.\end{texdocclassintro}
\begin{texdocclassmethods}
\texdocmethod{public}{double}{evaluate}{(Plan plan)}{Evaluate wertet einen Studienplan aus und gibt dementsprechend eine Zahl
 zwischen 0 und 1 zurück.}{\begin{texdocparameters}
\texdocparameter{plan}{der zu bewertende Plan}
\end{texdocparameters}
\texdocreturn{Wert zwischen 0 und 1 der den Plan evaluiert, wobei ein Plan mit
         der Bewertung 1 ein idealer Plan ist.}
}
\end{texdocclassmethods}
\end{texdocclass}


\begin{texdocclass}{class}{ThresholdObjectiveFunction}
\label{texdoclet:edu.kit.informatik.studyplan.server.generation.objectivefunction.ThresholdObjectiveFunction}
\begin{texdocclassintro}
Wertet alle Teilzielfunktionen erst ab der Überschreitung des Schwellwertes.
 Mittelmäßige Studienpläne können so auf sehr schlechte Werte gesetzt werden.
 Bessere Studienpläne heben sich so stärker heraus.\end{texdocclassintro}
\begin{texdocclassconstructors}
\texdocconstructor{public}{ThresholdObjectiveFunction}{()}{Setzt den Schwellwert auf 0.5}{}
\texdocconstructor{public}{ThresholdObjectiveFunction}{(int threshold)}{Setzt den Schwellwert auf den Angegebenen Wert}{\begin{texdocparameters}
\texdocparameter{threshold}{}
\end{texdocparameters}
}
\end{texdocclassconstructors}
\begin{texdocclassmethods}
\texdocmethod{public}{double}{evaluate}{(Plan plan)}{}{}
\end{texdocclassmethods}
\end{texdocclass}


\end{texdocpackage}



\begin{texdocpackage}{server.generation.standard}
\label{texdoclet:edu.kit.informatik.studyplan.server.generation.standard}

\begin{texdocclass}{class}{Node}
\label{texdoclet:edu.kit.informatik.studyplan.server.generation.standard.Node}
\begin{texdocclassintro}
Die abstrakte Klasse Node stellt Knoten des Graphen da.\end{texdocclassintro}
\begin{texdocclassconstructors}
\texdocconstructor{public}{Node}{()}{}{}
\end{texdocclassconstructors}
\begin{texdocclassmethods}
\texdocmethod{public}{void}{addParent}{(Node node)}{Die Methode addParent fügt der Liste parents vom Typ node 
 den übergebenen node hinzu.}{\begin{texdocparameters}
\texdocparameter{node}{}
\end{texdocparameters}
}
\texdocmethod{public}{void}{addSubNode}{(Node node)}{Die Methode addSubNode fügt dem innersten Knoten 
 den übergebenen Knoten als inneren Knoten hinzu.}{\begin{texdocparameters}
\texdocparameter{node}{der neue innerste Knoten.}
\end{texdocparameters}
}
\texdocmethod{public}{Module}{getModule}{()}{Die Methode gibt das Modul zurück, welches der Knoten darstellt.}{\texdocreturn{Das Modul (Typ Module), welches der Knoten repräsentiert.}
}
\texdocmethod{public}{Collection\textless{}Node\textgreater{}}{getParents}{()}{Die Methode getPartens gibt die Liste parents zurück.}{\texdocreturn{parent-moduls}
}
\texdocmethod{public}{Node}{getSubNodes}{()}{die Methode getSubNodes gibt eventuell enthaltene weitere Knoten zurück.}{\texdocreturn{enthaltene Knoten}
}
\end{texdocclassmethods}
\end{texdocclass}


\begin{texdocclass}{class}{NodeWithoutOutput}
\label{texdoclet:edu.kit.informatik.studyplan.server.generation.standard.NodeWithoutOutput}
\begin{texdocclassintro}
Die Klasse NodeWithoutOutput erbt von Node und stellt 
  Blätter der Graphenstruktur da, das heißt Module, die nicht Voraussetzung 
  für ein anderes Modul des Studienplans sind.\end{texdocclassintro}
\begin{texdocclassconstructors}
\texdocconstructor{public}{NodeWithoutOutput}{()}{}{}
\end{texdocclassconstructors}
\end{texdocclass}


\begin{texdocclass}{class}{NodeWithOutput}
\label{texdoclet:edu.kit.informatik.studyplan.server.generation.standard.NodeWithOutput}
\begin{texdocclassintro}
Die Klasse NodeWithOutput erbt von Node und stellt einen 
 inneren Knoten der Graphenstruktur, also ein Modul, welches Vorraussetzung
 für andere Module ist, dar.\end{texdocclassintro}
\begin{texdocclassconstructors}
\texdocconstructor{public}{NodeWithOutput}{()}{}{}
\end{texdocclassconstructors}
\begin{texdocclassmethods}
\texdocmethod{public}{void}{addChild}{(Node node)}{Die Methode addChild für der Liste children vom Typ Node den übergebenen 
 Knoten hinzu, zu der der Knoten eine Ausgangskante hat.}{\begin{texdocparameters}
\texdocparameter{node}{der hinzuzufügende Knoten}
\end{texdocparameters}
}
\texdocmethod{public}{Collection\textless{}Node\textgreater{}}{getChildren}{()}{Die Methode getChildren gibt eine Liste mit allen Knoten zurück, 
 zu denen der Knoten Ausgangskanten hat.}{\texdocreturn{children-moduls}
}
\texdocmethod{public}{Collection\textless{}Node\textgreater{}}{removeChild}{(Node node)}{Die Methode removeChild löscht den 
 übergebenen Knoten aus der Liste children vom Typ Node.
 Zurückgegeben wird die dann aktuelle Liste Children.}{\begin{texdocparameters}
\texdocparameter{node}{der zu löschende Knoten}
\end{texdocparameters}
\texdocreturn{die jetzt aktuelle Liste children.}
}
\end{texdocclassmethods}
\end{texdocclass}


\begin{texdocclass}{class}{SimpleGenerator}
\label{texdoclet:edu.kit.informatik.studyplan.server.generation.standard.SimpleGenerator}
\begin{texdocclassintro}
Die Klasse SimpleGenerator, welche Generator implementiert,
 stellt einen konkreten Generator da. Dieser nutzt die Graphenstruktur, 
 die Zielfunktion und die WeightFunction.\end{texdocclassintro}
\begin{texdocclassconstructors}
\texdocconstructor{public}{SimpleGenerator}{()}{}{}
\end{texdocclassconstructors}
\begin{texdocclassmethods}
\texdocmethod{public}{Plan}{generate}{(PartialObjectiveFunction objectiveFunction, Plan currentPlan, ModuleDao moduleDAO)}{}{}
\end{texdocclassmethods}
\end{texdocclass}


\begin{texdocclass}{class}{WeightFunction}
\label{texdoclet:edu.kit.informatik.studyplan.server.generation.standard.WeightFunction}
\begin{texdocclassintro}
Die Klasse Weightfunction enthält eine Methode, welche die Anzahl an Creditpoints
 eines Knotens und aller seiner innerer Knoten bestimmt.\end{texdocclassintro}
\begin{texdocclassconstructors}
\texdocconstructor{public}{WeightFunction}{()}{}{}
\end{texdocclassconstructors}
\begin{texdocclassmethods}
\texdocmethod{public}{int}{getWeight}{(Node node)}{Die Methode getWeight errechnet die Punktanzahl eines Knoten (Node) und
 aller innerer Knoten dieses Knotens, addiert sie und gibt diese int-Zahl zurück.}{\begin{texdocparameters}
\texdocparameter{node}{der Knoten, dessen Creditpoint-Anzahl bestimmt werden soll.}
\end{texdocparameters}
\texdocreturn{die Gesamtanzahl an Creditpoints des Knotens.}
}
\end{texdocclassmethods}
\end{texdocclass}


\end{texdocpackage}



\begin{texdocpackage}{server.model}
\label{texdoclet:edu.kit.informatik.studyplan.server.model}

\begin{texdocclass}{class}{HibernateUtil}
\label{texdoclet:edu.kit.informatik.studyplan.server.model.HibernateUtil}
\begin{texdocclassintro}
Fabrik zur Erzeugung von SessionFactories. Diese sind das Hibernate-Äquivalent zu Datenbankverbindungen.\end{texdocclassintro}
\begin{texdocclassconstructors}
\texdocconstructor{}{HibernateUtil}{()}{}{}
\end{texdocclassconstructors}
\begin{texdocclassmethods}
\texdocmethod{static}{SessionFactory}{getModuleDataSessionFactory}{()}{Erzeugt die SessionFactory zum Zugriff auf die Modul-Datenbank aus der entsprechenden Konfigurationsdatei}{\texdocreturn{die SessionFactory}
}
\texdocmethod{static}{SessionFactory}{getUserDataSessionFactory}{()}{Erzeugt die SessionFactory zum Zugriff auf die Nutzer-Datenbank aus der entsprechenden Konfigurationsdatei}{\texdocreturn{die SessionFactory}
}
\end{texdocclassmethods}
\end{texdocclass}


\end{texdocpackage}



\begin{texdocpackage}{server.model.moduledata}
\label{texdoclet:edu.kit.informatik.studyplan.server.model.moduledata}

\begin{texdocclass}{class}{Category}
\label{texdoclet:edu.kit.informatik.studyplan.server.model.moduledata.Category}
\begin{texdocclassintro}
Modelliert eine Modul-Kategorie.\end{texdocclassintro}
\begin{texdocclassconstructors}
\texdocconstructor{public}{Category}{()}{}{}
\end{texdocclassconstructors}
\begin{texdocclassmethods}
\texdocmethod{public}{int}{getCategoryId}{()}{}{\texdocreturn{gibt die eindeutige Kategorie-ID zurück}
}
\texdocmethod{public}{String}{getName}{()}{}{\texdocreturn{gibt den Namen der Kategorie zurück}
}
\texdocmethod{public}{boolean}{isSubject}{()}{}{\texdocreturn{gibt zurück ob es sich bei der Kategorie um ein Vertiefungsfach handelt}
}
\end{texdocclassmethods}
\end{texdocclass}


\begin{texdocclass}{enum}{CycleType}
\label{texdoclet:edu.kit.informatik.studyplan.server.model.moduledata.CycleType}
\begin{texdocclassintro}
Turnus eines Moduls\end{texdocclassintro}
\begin{texdocenums}
\texdocenum{BOTH}{Modul wird in jedem Semester angeboten}
\texdocenum{SUMMER\_TERM}{Modul wird nur im Sommersemester angeboten}
\texdocenum{WINTER\_TERM}{Modul wird nur im Wintersemester angeboten}
\end{texdocenums}
\begin{texdocclassmethods}
\texdocmethod{public static}{CycleType}{valueOf}{(String name)}{}{}
\texdocmethod{public static}{CycleType}{values}{()}{}{}
\end{texdocclassmethods}
\end{texdocclass}


\begin{texdocclass}{class}{Discipline}
\label{texdoclet:edu.kit.informatik.studyplan.server.model.moduledata.Discipline}
\begin{texdocclassintro}
Modelliert ein Studienfach\end{texdocclassintro}
\begin{texdocclassconstructors}
\texdocconstructor{public}{Discipline}{()}{}{}
\end{texdocclassconstructors}
\begin{texdocclassmethods}
\texdocmethod{public}{String}{getDescription}{()}{}{\texdocreturn{gibt die Kategoriebeschreibung zurück}
}
\texdocmethod{public}{int}{getDisciplineId}{()}{}{\texdocreturn{gibt die eindeutige Studienfach-ID zurück}
}
\end{texdocclassmethods}
\end{texdocclass}


\begin{texdocclass}{class}{Module}
\label{texdoclet:edu.kit.informatik.studyplan.server.model.moduledata.Module}
\begin{texdocclassintro}
Modelliert ein Modul\end{texdocclassintro}
\begin{texdocclassconstructors}
\texdocconstructor{public}{Module}{()}{}{}
\end{texdocclassconstructors}
\begin{texdocclassmethods}
\texdocmethod{public}{List\textless{}Category\textgreater{}}{getCategories}{()}{}{\texdocreturn{gibt die Kategorien, denen das Modul angehört, zurück}
}
\texdocmethod{public}{List\textless{}ModuleConstraint\textgreater{}}{getConstraints}{()}{}{\texdocreturn{gibt die Abhängigkeiten des Moduls zu anderen Modulen zurück}
}
\texdocmethod{public}{int}{getCreditPoints}{()}{}{\texdocreturn{gibt die ECTS-Zahl des Moduls zurück}
}
\texdocmethod{public}{CycleType}{getCycleType}{()}{}{\texdocreturn{gibt den Turnus des Moduls zurück}
}
\texdocmethod{public}{Discipline}{getDiscipline}{()}{}{\texdocreturn{gibt den Studiengang, dem das Modul angehört, zurück \texdocbr{}

 Ein Modul ist immer eindeutig einem Studiengang zugeordnet. Wird ein Modul
 in der Realität für mehrere Studiengänge angeboten, so handelt es sich jeweils
 um unterschiedliche Module, denn in einem anderen Studiengang können Modulabhängigkeiten
 ggf. varieren.}
}
\texdocmethod{public}{String}{getIdentifier}{()}{}{\texdocreturn{gibt den eindeutigen Identifier-String zurück}
}
\texdocmethod{public}{ModuleDescription}{getModuleDescription}{()}{}{\texdocreturn{gibt die Modul-Beschreibung des Moduls zurück}
}
\texdocmethod{public}{int}{getModuleId}{()}{}{\texdocreturn{gibt die eindeutige Modul-ID zurück}
}
\texdocmethod{public}{String}{getName}{()}{}{\texdocreturn{gibt die Modulbezeichnung zurück}
}
\texdocmethod{public}{boolean}{isCompulsory}{()}{}{\texdocreturn{gibt zurück, ob es sich um ein Pflichtmodul handelt}
}
\end{texdocclassmethods}
\end{texdocclass}


\begin{texdocclass}{class}{ModuleDescription}
\label{texdoclet:edu.kit.informatik.studyplan.server.model.moduledata.ModuleDescription}
\begin{texdocclassintro}
Modelliert eine Modulbeschreibung \texdocbr{}

 Eine Modulbeschreibung kann mehreren Modulen zugeordnet sein\end{texdocclassintro}
\begin{texdocclassconstructors}
\texdocconstructor{public}{ModuleDescription}{()}{}{}
\end{texdocclassconstructors}
\begin{texdocclassmethods}
\texdocmethod{public}{int}{getDescriptionId}{()}{}{\texdocreturn{gibt die eindeutige Beschreibungs-ID zurück}
}
\texdocmethod{public}{String}{getDescriptionText}{()}{}{\texdocreturn{gibt den Beschreibungstext zurück}
}
\texdocmethod{public}{String}{getLecturer}{()}{}{\texdocreturn{gbit den Dozenten zurück}
}
\texdocmethod{public}{ModuleType}{getModuleType}{()}{}{\texdocreturn{gibt den Modul-Typ zurück}
}
\end{texdocclassmethods}
\end{texdocclass}


\begin{texdocclass}{enum}{ModuleOrientation}
\label{texdoclet:edu.kit.informatik.studyplan.server.model.moduledata.ModuleOrientation}
\begin{texdocclassintro}
Beschreibt die Richtung eines Moduleconstraints bei der Übergabe an die
 verify(first,second,orientation) Methode von ModuleConstraintType.\end{texdocclassintro}
\begin{texdocenums}
\texdocenum{LEFT\_TO\_RIGHT}{Für ein gegebenes Tupel an Parametern (first,second) für die
 verify-Funktion mit gegebenen Constraint-Modulen modul1 und modul2 gilt:
 \texdocbr{}

 first:=modul1\texdocbr{}

 second:=modul2}
\texdocenum{RIGHT\_TO\_LEFT}{Für ein gegebenes Tupel an Parametern (first,second) für die
 verify-Funktion mit gegebenen Constraint-Modulen modul1 und modul2 gilt:
 \texdocbr{}

 second:=modul1\texdocbr{}

 first:=modul2}
\end{texdocenums}
\begin{texdocclassmethods}
\texdocmethod{public static}{ModuleOrientation}{valueOf}{(String name)}{}{}
\texdocmethod{public static}{ModuleOrientation}{values}{()}{}{}
\end{texdocclassmethods}
\end{texdocclass}


\begin{texdocclass}{class}{ModuleType}
\label{texdoclet:edu.kit.informatik.studyplan.server.model.moduledata.ModuleType}
\begin{texdocclassintro}
Modelliert einen Modultyp, wie beispielsweise Vorlesung, Seminar, etc.\end{texdocclassintro}
\begin{texdocclassconstructors}
\texdocconstructor{public}{ModuleType}{()}{}{}
\end{texdocclassconstructors}
\begin{texdocclassmethods}
\texdocmethod{public}{String}{getName}{()}{}{\texdocreturn{gibt den Namen des Typs zurück}
}
\texdocmethod{public}{int}{getTypeId}{()}{}{\texdocreturn{gibt die eindeutige Typ-ID zurück}
}
\end{texdocclassmethods}
\end{texdocclass}


\end{texdocpackage}



\begin{texdocpackage}{server.model.moduledata.constraint}
\label{texdoclet:edu.kit.informatik.studyplan.server.model.moduledata.constraint}

\begin{texdocclass}{class}{ModuleConstraint}
\label{texdoclet:edu.kit.informatik.studyplan.server.model.moduledata.constraint.ModuleConstraint}
\begin{texdocclassintro}
Diese Klasse modelliert eine Abhängigkeit zwischen zwei Modulen.\end{texdocclassintro}
\begin{texdocclassconstructors}
\texdocconstructor{public}{ModuleConstraint}{()}{}{}
\end{texdocclassconstructors}
\begin{texdocclassmethods}
\texdocmethod{public}{ModuleConstraintType}{getConstraintType}{()}{}{\texdocreturn{gibt den Typ der Abhängigkeit zurück}
}\begin{texdocsees}{See also}
\texdocsee{ModuleConstraintType}{texdoclet:edu.kit.informatik.studyplan.server.model.moduledata.constraint.ModuleConstraintType}
\end{texdocsees}

\texdocmethod{public}{Module}{getFirstModule}{()}{}{\texdocreturn{gibt das erste Modul der Abhängigkeitsrelation zurück}
}
\texdocmethod{public}{Module}{getSecondModule}{()}{}{\texdocreturn{gibt das zweite Modul der Abhängigkeitsrelation zurück}
}
\end{texdocclassmethods}
\end{texdocclass}


\begin{texdocclass}{class}{ModuleConstraintType}
\label{texdoclet:edu.kit.informatik.studyplan.server.model.moduledata.constraint.ModuleConstraintType}
\begin{texdocclassintro}
Modelliert eine Abhängkeitstyp einer Modulabhängigkeit\end{texdocclassintro}
\begin{texdocclassconstructors}
\texdocconstructor{public}{ModuleConstraintType}{()}{}{}
\end{texdocclassconstructors}
\begin{texdocclassmethods}
\texdocmethod{public}{String}{getDescription}{()}{}{\texdocreturn{gibt die textuelle Beschreibung der Abhängigkeit zurück}
}
\texdocmethod{public}{String}{getFormalDescription}{()}{}{\texdocreturn{gibt die Abhängigkeitsbeschreibung in Form eines logischen Ausdrucks zurück}
}
\texdocmethod{public abstract}{boolean}{isValid}{(ModuleEntry first, ModuleEntry second, ModuleOrientation orientation)}{Überprüft, ob für zwei gegebene Moduleinträge die Abhängigkeit dieses
 Typs erfüllt ist.}{\begin{texdocparameters}
\texdocparameter{first}{der erste Moduleintrag (Subjekt des aktuellen
            Verifikationsschritts)}
\texdocparameter{second}{der zweite Moduleintrag (Zweites zusätzlich geladenes Modul
            des aktuellen Verifikationsschritts)}
\texdocparameter{orientation}{Richtung in die die Relation überprüft werden soll. \texdocbr{}

            Jedes Constraint besitzt eine Richtung. Diese ist über die
            Eintragung der Module in Modul1 und Modul2 gegeben. Wenn nun
            gilt: first:=Modul1 UND second:=Modul2 so ist die orientation
            LEFT\_TO\_RIGHT; Wenn aber gilt: first:=Modul2 UND
            second:=Modul1 so ist die orientation RIGHT\_TO\_LEFT\texdocbr{}

            Dieses zusätzliche Attribut ist notwendig, da Informationen
            über zwei gerichtete Relationen übergeben werden müssen:
            \begin{enumerate}

            \item Welches Modul aktuell untersucht wird und welches "nur"
            das zweite Modul des Constraints ist
            \item Welches Modul im Constraint Modul1 und welches Modul2 ist
            
            
\end{enumerate}
            1. wird über die Anordnung der Parameter übergeben, 2. über
            das Parameter orientation.}
\end{texdocparameters}
\texdocreturn{Ergebnis der Überprüfung}
}
\texdocmethod{public}{void}{setDescription}{(String description)}{}{\begin{texdocparameters}
\texdocparameter{description}{die textuelle Beschreibung}
\end{texdocparameters}
}
\texdocmethod{public}{void}{setFormalDescription}{(String formalDescription)}{}{\begin{texdocparameters}
\texdocparameter{formalDescription}{der logische Ausdruck}
\end{texdocparameters}
}
\end{texdocclassmethods}
\end{texdocclass}


\begin{texdocclass}{class}{OverlappingModuleConstraintType}
\label{texdoclet:edu.kit.informatik.studyplan.server.model.moduledata.constraint.OverlappingModuleConstraintType}
\begin{texdocclassintro}
Modelliert eine zeitliche Abhängigkeit zwischen zwei Modulen:\texdocbr{}

 Die beiden Module können nicht im gleichen Semester belegt werden, da die Veranstaltung zur gleichen Zeit stattfindet.\end{texdocclassintro}
\begin{texdocclassconstructors}
\texdocconstructor{public}{OverlappingModuleConstraintType}{()}{}{}
\end{texdocclassconstructors}
\begin{texdocclassmethods}
\texdocmethod{public}{boolean}{isValid}{(ModuleEntry first, ModuleEntry second, ModuleOrientation orientation)}{}{}
\end{texdocclassmethods}
\end{texdocclass}


\begin{texdocclass}{class}{PlanLinkModuleConstraintType}
\label{texdoclet:edu.kit.informatik.studyplan.server.model.moduledata.constraint.PlanLinkModuleConstraintType}
\begin{texdocclassintro}
Modelliert eine Zusammenhangsabhängigkeit zwischen Modulen: \texdocbr{}

 Es müssen immer beide Module im Plan enthalten sein.\end{texdocclassintro}
\begin{texdocclassconstructors}
\texdocconstructor{public}{PlanLinkModuleConstraintType}{()}{}{}
\end{texdocclassconstructors}
\begin{texdocclassmethods}
\texdocmethod{public}{boolean}{isValid}{(ModuleEntry first, ModuleEntry second, ModuleOrientation orientation)}{}{}
\end{texdocclassmethods}
\end{texdocclass}


\begin{texdocclass}{class}{PrerequisiteModuleConstraintType}
\label{texdoclet:edu.kit.informatik.studyplan.server.model.moduledata.constraint.PrerequisiteModuleConstraintType}
\begin{texdocclassintro}
Modelliert eine Voraussetzungsabhängigkeit zwischen Modulen:\texdocbr{}

 Das erste Modul ist Voraussetzung für das zweite Modul.\end{texdocclassintro}
\begin{texdocclassconstructors}
\texdocconstructor{public}{PrerequisiteModuleConstraintType}{()}{}{}
\end{texdocclassconstructors}
\begin{texdocclassmethods}
\texdocmethod{public}{boolean}{isValid}{(ModuleEntry first, ModuleEntry second, ModuleOrientation orientation)}{}{}
\end{texdocclassmethods}
\end{texdocclass}


\begin{texdocclass}{class}{SemesterLinkModuleConstraintType}
\label{texdoclet:edu.kit.informatik.studyplan.server.model.moduledata.constraint.SemesterLinkModuleConstraintType}
\begin{texdocclassintro}
Modelliert eine Zusammenhangsbeziehung zwischen zwei Modulen:\texdocbr{}

 Beide Module müssen im gleichen Semester belegt werden.\end{texdocclassintro}
\begin{texdocclassconstructors}
\texdocconstructor{public}{SemesterLinkModuleConstraintType}{()}{}{}
\end{texdocclassconstructors}
\begin{texdocclassmethods}
\texdocmethod{public}{boolean}{isValid}{(ModuleEntry first, ModuleEntry second, ModuleOrientation orientation)}{}{}
\end{texdocclassmethods}
\end{texdocclass}


\end{texdocpackage}



\begin{texdocpackage}{server.model.moduledata.dao}
\label{texdoclet:edu.kit.informatik.studyplan.server.model.moduledata.dao}

\begin{texdocclass}{class}{HibernateModuleDao}
\label{texdoclet:edu.kit.informatik.studyplan.server.model.moduledata.dao.HibernateModuleDao}
\begin{texdocclassintro}
Ein konkretes ModulDao, welches die Datenbankverbindung über Hibernate herstellt.
 Es kann nur auf Module, Kategorien und Vertiefungsfächer 
 des im Konstruktur angebenen Studiengangs zugreifen\end{texdocclassintro}
\begin{texdocclassconstructors}
\texdocconstructor{public}{HibernateModuleDao}{(Discipline discipline)}{initialisiert ein neues DAO für den angebenen Studiengang}{\begin{texdocparameters}
\texdocparameter{discipline}{der Studiengang}
\end{texdocparameters}
}
\end{texdocclassconstructors}
\begin{texdocclassmethods}
\texdocmethod{public}{List\textless{}Category\textgreater{}}{getCategories}{()}{}{}
\texdocmethod{public}{List\textless{}Discipline\textgreater{}}{getDisciplines}{()}{}{}
\texdocmethod{public}{Module}{getModuleById}{(String id)}{}{}
\texdocmethod{public}{List\textless{}Module\textgreater{}}{getModulesByFilter}{(Filter filter)}{}{}
\texdocmethod{public}{List\textless{}Module\textgreater{}}{getModulesByFilter}{(Filter filter, int start, int end)}{}{}
\texdocmethod{public}{Module}{getRandomModuleByFilter}{(Filter filter)}{}{}
\texdocmethod{public}{List\textless{}Category\textgreater{}}{getSubjects}{()}{}{}
\end{texdocclassmethods}
\end{texdocclass}


\begin{texdocclass}{class}{ModuleAttributeNames}
\label{texdoclet:edu.kit.informatik.studyplan.server.model.moduledata.dao.ModuleAttributeNames}
\begin{texdocclassintro}
Kapselt die für die Filterarchitektur nötigen Module-Attribut-Namen als Stringkonstanten.
 Diese werden bei der Anwendung der Filter in den ModuleDao-Methoden für die eindeutige
 Identifzierung von Module-Eigenschaften benutzt.\end{texdocclassintro}
\begin{texdocsees}{See also}
\texdocsee{edu.kit.informatik.studyplan.server.filter}{texdoclet:edu.kit.informatik.studyplan.server.filter}
\texdocsee{ModuleDao}{texdoclet:edu.kit.informatik.studyplan.server.model.moduledata.dao.ModuleDao}
\end{texdocsees}
\begin{texdocclassfields}
\texdocfield{public static final}{String}{CATEGORY}{Stringkonstante, die die Kategorie-Eigenschaft eines Moduls repräsentiert.}
\texdocfield{public static final}{String}{CREDIT\_POINTS}{Stringkonstante, die die ECTS-Eigenschaft eines Moduls repräsentiert.}
\texdocfield{public static final}{String}{CYCLE\_TYPE}{Stringkonstante, die die Turnus-Eigenschaft eines Moduls repräsentiert.}
\texdocfield{public static final}{String}{DISCIPLINE}{Stringkonstante, die die Fachrichtungs-Eigenschaft eines Moduls repräsentiert.}
\texdocfield{public static final}{String}{IS\_COMPULSORY}{Stringkonstante, die die Wahl-$/$Pflicht-Veranstaltungs-Eigenschaft eines Moduls repräsentiert.}
\texdocfield{public static final}{String}{MODULE\_TYPE}{Stringkonstante, die die Modultyp-Eigenschaft eines Moduls repräsentiert.}
\texdocfield{public static final}{String}{NAME}{Stringkonstante, die die Modulnamens-Eigenschaft eines Moduls repräsentiert.}
\end{texdocclassfields}
\begin{texdocclassconstructors}
\texdocconstructor{public}{ModuleAttributeNames}{()}{}{}
\end{texdocclassconstructors}
\end{texdocclass}


\begin{texdocclass}{interface}{ModuleDao}
\label{texdoclet:edu.kit.informatik.studyplan.server.model.moduledata.dao.ModuleDao}
\begin{texdocclassintro}
DataAccessObject zum Zugriff auf die Modul-Datenbank\end{texdocclassintro}
\begin{texdocclassmethods}
\texdocmethod{public}{List\textless{}Category\textgreater{}}{getCategories}{()}{}{\texdocreturn{gibt eine Liste der verfügbaren Kategorien zurück}
}
\texdocmethod{public}{List\textless{}Discipline\textgreater{}}{getDisciplines}{()}{}{\texdocreturn{gibt eine Liste der verfügbaren Studiengänge zurück}
}
\texdocmethod{public}{Module}{getModuleById}{(String id)}{}{\begin{texdocparameters}
\texdocparameter{id}{der String-Identifier des zu suchenden Moduls}
\end{texdocparameters}
\texdocreturn{das Modul mit dem entsprechenden Identifier, \texttt{null} wenn kein Modul gefunden}
}
\texdocmethod{public}{List\textless{}Module\textgreater{}}{getModulesByFilter}{(Filter filter)}{Sucht alle Module die den angegebenen Filterkritierien entsprechen und gibt diese zurück}{\begin{texdocparameters}
\texdocparameter{filter}{der Modulfilter}
\end{texdocparameters}
\texdocreturn{die Modulliste}
}
\texdocmethod{public}{List\textless{}Module\textgreater{}}{getModulesByFilter}{(Filter filter, int start, int end)}{Sucht alle Module die den angegebenen Filterkritierien entsprechen und gibt die Einträge Nr.
 \texttt{start} bis \texttt{end} zurück.}{\begin{texdocparameters}
\texdocparameter{filter}{der Modulfilter}
\texdocparameter{start}{Start-Index}
\texdocparameter{end}{End-Index}
\end{texdocparameters}
\texdocreturn{die Modulliste}
}
\texdocmethod{public}{Module}{getRandomModuleByFilter}{(Filter filter)}{Gibt ein zufälliges Modul, welches den angebenen Filterkriterien entspricht, zurück}{\begin{texdocparameters}
\texdocparameter{filter}{der Modulfilter}
\end{texdocparameters}
\texdocreturn{das Modul}
}
\texdocmethod{public}{List\textless{}Category\textgreater{}}{getSubjects}{()}{}{\texdocreturn{gibt eine Liste der verfügbaren Vertiefungsfächer zurück}
}
\end{texdocclassmethods}
\end{texdocclass}


\begin{texdocclass}{class}{ModuleDaoFactory}
\label{texdoclet:edu.kit.informatik.studyplan.server.model.moduledata.dao.ModuleDaoFactory}
\begin{texdocclassintro}
Factory zur ModuleDao-Erzeugung\end{texdocclassintro}
\begin{texdocclassconstructors}
\texdocconstructor{public}{ModuleDaoFactory}{()}{}{}
\end{texdocclassconstructors}
\begin{texdocclassmethods}
\texdocmethod{public static}{ModuleDao}{getModuleDao}{(Discipline discipline)}{}{\begin{texdocparameters}
\texdocparameter{discipline}{der Studiengang}
\end{texdocparameters}
\texdocreturn{liefert das für die verwendete Datenbankschnittstelle benötigte DAO zurück \texdocbr{}

 Das DAO wird mit dem übergebenen Studiengang initialisiert.}
}
\end{texdocclassmethods}
\end{texdocclass}


\end{texdocpackage}



\begin{texdocpackage}{server.model.userdata.authorization}
\label{texdoclet:edu.kit.informatik.studyplan.server.model.userdata.authorization}

\begin{texdocclass}{class}{AuthorizationContext}
\label{texdoclet:edu.kit.informatik.studyplan.server.model.userdata.authorization.AuthorizationContext}
\begin{texdocclassintro}
Modelliert einen Authorisierungskontext.\texdocbr{}

 Er enthält die benötigten Informationen für einen authorisierten Benutzer
 siehe Kapitel ???\end{texdocclassintro}
\begin{texdocclassconstructors}
\texdocconstructor{public}{AuthorizationContext}{()}{}{}
\end{texdocclassconstructors}
\begin{texdocclassmethods}
\texdocmethod{public}{String}{getAccessToken}{()}{}{\texdocreturn{gibt den Access-Token zurück}
}
\texdocmethod{public}{Date}{getExpiryDate}{()}{}{\texdocreturn{gibt das Verfallsdatum des Access-Tokens zurück}
}
\texdocmethod{public}{String}{getRefreshToken}{()}{}{\texdocreturn{gibt den Refresh-Token zurück}
}
\texdocmethod{public}{RESTClient}{getRestClient}{()}{}{\texdocreturn{gibt den zugehörigen REST-Client zurück}
}
\texdocmethod{public}{AuthorizationScope}{getScope}{()}{}{\texdocreturn{gibt die Berechtigung des Nutzers zurück}
}
\texdocmethod{public}{User}{getUser}{()}{}{\texdocreturn{gibt den Nutzer zurück}
}
\texdocmethod{public}{void}{setAccessToken}{(String accessToken)}{}{\begin{texdocparameters}
\texdocparameter{accessToken}{der Access-Token}
\end{texdocparameters}
}
\texdocmethod{public}{void}{setExpiryDate}{(Date date)}{}{\begin{texdocparameters}
\texdocparameter{date}{das Verfallsdatum}
\end{texdocparameters}
}
\texdocmethod{public}{void}{setRefreshToken}{(String refreshToken)}{}{\begin{texdocparameters}
\texdocparameter{refreshToken}{der Refresh-Token}
\end{texdocparameters}
}
\texdocmethod{public}{void}{setRestClient}{(RESTClient client)}{}{\begin{texdocparameters}
\texdocparameter{client}{der REST-Client}
\end{texdocparameters}
}
\texdocmethod{public}{void}{setScope}{(AuthorizationScope scope)}{}{\begin{texdocparameters}
\texdocparameter{scope}{die Berechtigung}
\end{texdocparameters}
}
\texdocmethod{public}{void}{setUser}{(User user)}{}{\begin{texdocparameters}
\texdocparameter{user}{der Nutzer}
\end{texdocparameters}
}
\end{texdocclassmethods}
\end{texdocclass}


\begin{texdocclass}{enum}{AuthorizationScope}
\label{texdoclet:edu.kit.informatik.studyplan.server.model.userdata.authorization.AuthorizationScope}
\begin{texdocclassintro}
Berechtigungen die ein Nutzer anfragen kann\end{texdocclassintro}
\begin{texdocenums}
\texdocenum{STUDENT}{Berechtigung Student. Kann alle Kernfunktionen nutzen.}
\end{texdocenums}
\begin{texdocclassmethods}
\texdocmethod{public static}{AuthorizationScope}{valueOf}{(String name)}{}{}
\texdocmethod{public static}{AuthorizationScope}{values}{()}{}{}
\end{texdocclassmethods}
\end{texdocclass}


\begin{texdocclass}{class}{RESTClient}
\label{texdoclet:edu.kit.informatik.studyplan.server.model.userdata.authorization.RESTClient}
\begin{texdocclassintro}
Modelliert einen Klienten, der auf die REST-Schnittstelle zugreifen kann
 siehe Kapitel ????\end{texdocclassintro}
\begin{texdocclassconstructors}
\texdocconstructor{public}{RESTClient}{()}{}{}
\end{texdocclassconstructors}
\begin{texdocclassmethods}
\texdocmethod{public}{String}{getApiKey}{()}{}{\texdocreturn{gibt die eindeutige ID des Klienten zurück}
}
\texdocmethod{public}{String}{getApiSecret}{()}{}{\texdocreturn{gibt die nur dem Klienten bekannte Kennung zurück}
}
\texdocmethod{public}{String}{getOrigin}{()}{}{\texdocreturn{gibt die Domain, von welcher aus der Client auf Ressourcen zugreifen kann als regulären Ausdruck zurück.}
}
\texdocmethod{public}{URL}{getRedirectUrl}{()}{}{\texdocreturn{gibt die Weiterleitungs-URL zurück}
}
\texdocmethod{public}{List}{getScopes}{()}{}{\texdocreturn{gibt eine Liste aller Berechtigungen zurück, die vom Client angefragt werden können}
}
\texdocmethod{public}{void}{setApiKey}{(String apiKey)}{}{\begin{texdocparameters}
\texdocparameter{apiKey}{die ID des Klienten}
\end{texdocparameters}
}
\texdocmethod{public}{void}{setApiSecret}{(String apiSecret)}{}{\begin{texdocparameters}
\texdocparameter{apiSecret}{die Kennung}
\end{texdocparameters}
}
\texdocmethod{public}{void}{setOrigin}{(String origin)}{}{\begin{texdocparameters}
\texdocparameter{der}{reguläre Ausdruck}
\end{texdocparameters}
}
\texdocmethod{public}{void}{setRedirectUrl}{(URL redirectUrl)}{}{\begin{texdocparameters}
\texdocparameter{redirectUrl}{die Weiterleitungs-URL}
\end{texdocparameters}
}
\end{texdocclassmethods}
\end{texdocclass}


\end{texdocpackage}



\begin{texdocpackage}{server.model.userdata.dao}
\label{texdoclet:edu.kit.informatik.studyplan.server.model.userdata.dao}

\begin{texdocclass}{class}{AbstractSecurityProvider}
\label{texdoclet:edu.kit.informatik.studyplan.server.model.userdata.dao.AbstractSecurityProvider}
\begin{texdocclassintro}
Diese Klasse ermöglicht die Authentifizierung, indem sie Methoden zur Access-Token-Generierung, sowie zur Abfrage von Authentifizierungs-Kontexten bereitstellt.\end{texdocclassintro}
\begin{texdocclassconstructors}
\texdocconstructor{public}{AbstractSecurityProvider}{()}{}{}
\end{texdocclassconstructors}
\begin{texdocclassmethods}
\texdocmethod{public abstract}{AuthorizationContext}{generateAuthorizationContext}{(User user)}{Erzeugt zu einem Nutzer einen neuen Authentifizierungs-Kontext inklusive Access-Token und speichert diesen in der Datenbank.\texdocbr{}

 Die Gültigkeitsdauer wird auf 4 Stunden gesetzt.}{\begin{texdocparameters}
\texdocparameter{user}{der User}
\end{texdocparameters}
\texdocreturn{den generierten Authentifizierungs-Kontext}
}
\texdocmethod{public abstract}{AuthorizationContext}{getAuthorizationContext}{(String accessToken)}{Liefert zu einem Access-Token den enstprechenden Authentifizierungs-Kontext.\texdocbr{}

 Ist kein Kontext vorhanden, so wird \texttt{null} zurückgegeben.}{\begin{texdocparameters}
\texdocparameter{accessToken}{der Access-Token}
\end{texdocparameters}
\texdocreturn{der Authentifizierungs-Kontet}
}
\texdocmethod{public static final}{AbstractSecurityProvider}{getSecurityProviderImpl}{()}{}{\texdocreturn{gibt eine konkrete SecurityProvider-Implementierung zurück}
}
\end{texdocclassmethods}
\end{texdocclass}


\begin{texdocclass}{class}{HibernatePlanDao}
\label{texdoclet:edu.kit.informatik.studyplan.server.model.userdata.dao.HibernatePlanDao}
\begin{texdocclassintro}
Ein konkretes PlanDao, welches die Datenbankverbindung über Hibernate herstellt.\end{texdocclassintro}
\begin{texdocclassconstructors}
\texdocconstructor{}{HibernatePlanDao}{()}{}{}
\end{texdocclassconstructors}
\begin{texdocclassmethods}
\texdocmethod{public}{void}{deletePlan}{(Plan plan)}{}{}
\texdocmethod{public}{Plan}{getPlanById}{(String id)}{}{}
\texdocmethod{public}{void}{updatePlan}{(Plan plan)}{}{}
\end{texdocclassmethods}
\end{texdocclass}


\begin{texdocclass}{class}{HibernateUserDao}
\label{texdoclet:edu.kit.informatik.studyplan.server.model.userdata.dao.HibernateUserDao}
\begin{texdocclassintro}
Ein konkretes UserDao, welches die Datenbankverbindung über Hibernate herstellt.\end{texdocclassintro}
\begin{texdocclassconstructors}
\texdocconstructor{}{HibernateUserDao}{()}{}{}
\end{texdocclassconstructors}
\begin{texdocclassmethods}
\texdocmethod{public}{void}{deleteUser}{(User user)}{}{}
\texdocmethod{public}{User}{findUser}{(User user)}{}{}
\texdocmethod{public}{void}{updateUser}{(User user)}{}{}
\end{texdocclassmethods}
\end{texdocclass}


\begin{texdocclass}{interface}{PlanDao}
\label{texdoclet:edu.kit.informatik.studyplan.server.model.userdata.dao.PlanDao}
\begin{texdocclassintro}
DataAccessObject zum Zugriff auf Studienpläne aus der Datenbank\end{texdocclassintro}
\begin{texdocclassmethods}
\texdocmethod{public}{void}{deletePlan}{(Plan plan)}{Löscht den Plan aus der Datenbank.}{\begin{texdocparameters}
\texdocparameter{plan}{der Plan}
\end{texdocparameters}
}
\texdocmethod{public}{Plan}{getPlanById}{(String id)}{Sucht einen Plan nach seinem String-Identifier.}{\begin{texdocparameters}
\texdocparameter{id}{der Identifier-String}
\end{texdocparameters}
\texdocreturn{den gefundenen Plan oder \texttt{null} falls nichts gefunden}
}
\texdocmethod{public}{void}{updatePlan}{(Plan plan)}{Speichert alle Änderungen am Plan in der Datenbank, legt ihn an, wenn noch nicht vorhanden.}{\begin{texdocparameters}
\texdocparameter{plan}{der Plan}
\end{texdocparameters}
}
\end{texdocclassmethods}
\end{texdocclass}


\begin{texdocclass}{class}{PlanDaoFactory}
\label{texdoclet:edu.kit.informatik.studyplan.server.model.userdata.dao.PlanDaoFactory}
\begin{texdocclassintro}
Factory zur UserDao-Erzeugung\end{texdocclassintro}
\begin{texdocclassconstructors}
\texdocconstructor{public}{PlanDaoFactory}{()}{}{}
\end{texdocclassconstructors}
\begin{texdocclassmethods}
\texdocmethod{public static}{PlanDao}{getPlanDao}{()}{}{\texdocreturn{liefert das für die verwendete Datenbankschnittstelle benötigte DAO zurück}
}
\end{texdocclassmethods}
\end{texdocclass}


\begin{texdocclass}{class}{SecurityProvider}
\label{texdoclet:edu.kit.informatik.studyplan.server.model.userdata.dao.SecurityProvider}
\begin{texdocclassintro}
Konkrete Implementierung eines Security-Providers, der Hibernate verwendet.\end{texdocclassintro}
\begin{texdocclassconstructors}
\texdocconstructor{}{SecurityProvider}{()}{}{}
\end{texdocclassconstructors}
\begin{texdocclassmethods}
\texdocmethod{public}{AuthorizationContext}{generateAuthorizationContext}{(User user)}{}{}
\texdocmethod{public}{AuthorizationContext}{getAuthorizationContext}{(String accessToken)}{}{}
\end{texdocclassmethods}
\end{texdocclass}


\begin{texdocclass}{interface}{UserDao}
\label{texdoclet:edu.kit.informatik.studyplan.server.model.userdata.dao.UserDao}
\begin{texdocclassintro}
DataAccessObject zum Zugriff auf Nutzer in der Datenbank\end{texdocclassintro}
\begin{texdocclassmethods}
\texdocmethod{public}{void}{deleteUser}{(User user)}{Löscht den übergebenen Nutzer aus der Datenbank}{\begin{texdocparameters}
\texdocparameter{user}{der Nutzer}
\end{texdocparameters}
}
\texdocmethod{public}{User}{findUser}{(User user)}{Sucht nach dem ensprechenden Nutzer in der Datenbank. \texdocbr{}

 So kann ein Nutzer über die ID oder seinen Nutzernamen gefunden werden.}{\begin{texdocparameters}
\texdocparameter{user}{der zu suchende Nutzer}
\end{texdocparameters}
\texdocreturn{der gefundene Nutzer}
}
\texdocmethod{public}{void}{updateUser}{(User user)}{Speichert die Änderungen am übergebenen Nutzer in der Datenbank\texdocbr{}

 Handelt es sich um einen neuen Nutzer, so wird dieser angelegt.}{\begin{texdocparameters}
\texdocparameter{user}{der Nutzer}
\end{texdocparameters}
}
\end{texdocclassmethods}
\end{texdocclass}


\begin{texdocclass}{class}{UserDaoFactory}
\label{texdoclet:edu.kit.informatik.studyplan.server.model.userdata.dao.UserDaoFactory}
\begin{texdocclassintro}
Factory zur UserDao-Erzeugung\end{texdocclassintro}
\begin{texdocclassconstructors}
\texdocconstructor{public}{UserDaoFactory}{()}{}{}
\end{texdocclassconstructors}
\begin{texdocclassmethods}
\texdocmethod{public static}{UserDao}{getUserDao}{()}{}{\texdocreturn{liefert das für die verwendete Datenbankschnittstelle benötigte DAO zurück}
}
\end{texdocclassmethods}
\end{texdocclass}


\end{texdocpackage}



\begin{texdocpackage}{server.model.userdata}
\label{texdoclet:edu.kit.informatik.studyplan.server.model.userdata}

\begin{texdocclass}{class}{ModuleEntry}
\label{texdoclet:edu.kit.informatik.studyplan.server.model.userdata.ModuleEntry}
\begin{texdocclassintro}
Modelliert einen Moduleintrag in einem Studienplan\end{texdocclassintro}
\begin{texdocclassconstructors}
\texdocconstructor{public}{ModuleEntry}{()}{}{}
\end{texdocclassconstructors}
\begin{texdocclassmethods}
\texdocmethod{public}{Module}{getModule}{()}{}{\texdocreturn{gibt das Modul zurück}
}
\texdocmethod{public}{int}{getSemester}{()}{}{\texdocreturn{gibt die Nummer des Semesters zurück, dem der Eintrag zugeordnet wurde}
}
\texdocmethod{public}{void}{setModule}{(Module module)}{}{\begin{texdocparameters}
\texdocparameter{module}{das Modul}
\end{texdocparameters}
}
\texdocmethod{public}{void}{setSemester}{(int semester)}{}{\begin{texdocparameters}
\texdocparameter{semester}{die Semesternummer}
\end{texdocparameters}
}
\end{texdocclassmethods}
\end{texdocclass}


\begin{texdocclass}{class}{ModulePreference}
\label{texdoclet:edu.kit.informatik.studyplan.server.model.userdata.ModulePreference}
\begin{texdocclassintro}
Modelliert eine Modulpräferenz. 
 Eine Modulpräferenz bezeichnet die Bewertung eines Moduls durch einen Nutzer.\end{texdocclassintro}
\begin{texdocclassconstructors}
\texdocconstructor{public}{ModulePreference}{()}{}{}
\end{texdocclassconstructors}
\begin{texdocclassmethods}
\texdocmethod{public}{Module}{getModule}{()}{}{\texdocreturn{gibt das Modul zurück, zu dem die Präferenz gehört}
}
\texdocmethod{public}{PreferenceType}{getPreference}{()}{}{\texdocreturn{gibt den Typ der Präferenz zurück}
}\begin{texdocsees}{See also}
\texdocsee{PreferenceType}{texdoclet:edu.kit.informatik.studyplan.server.model.userdata.PreferenceType}
\end{texdocsees}

\texdocmethod{public}{void}{setModule}{(Module module)}{}{\begin{texdocparameters}
\texdocparameter{module}{das Modul}
\end{texdocparameters}
}
\texdocmethod{public}{void}{setPreference}{(PreferenceType preferenceType)}{}{\begin{texdocparameters}
\texdocparameter{preferenceType}{der Präferenztyp}
\end{texdocparameters}
}
\end{texdocclassmethods}
\end{texdocclass}


\begin{texdocclass}{class}{Plan}
\label{texdoclet:edu.kit.informatik.studyplan.server.model.userdata.Plan}
\begin{texdocclassintro}
Modelliert einen Studienplan\end{texdocclassintro}
\begin{texdocclassconstructors}
\texdocconstructor{public}{Plan}{()}{}{}
\end{texdocclassconstructors}
\begin{texdocclassmethods}
\texdocmethod{public}{int}{getCreditPoints}{()}{}{\texdocreturn{gibt die ECTS-Summe des Plans zurück}
}
\texdocmethod{public}{String}{getIdentifier}{()}{}{\texdocreturn{gibt den eindeutigen Plan-Identifier zurück}
}
\texdocmethod{public}{List\textless{}ModuleEntry\textgreater{}}{getModuleEntries}{()}{}{\texdocreturn{gibt alle Moduleinträge des Plans zurück}
}
\texdocmethod{public}{String}{getName}{()}{}{\texdocreturn{gibt den Namen des Plans zurück}
}
\texdocmethod{public}{int}{getPlanId}{()}{}{\texdocreturn{gibt die eindeutige Plan-ID zurück}
}
\texdocmethod{public}{PreferenceType}{getPreferenceForModule}{(Module module)}{Gibt für ein übergebenes Modul die Präferenz zurück. \texdocbr{}

 \texttt{null}, falls keine Präferenz vorhanden}{\begin{texdocparameters}
\texdocparameter{module}{das Modul}
\end{texdocparameters}
\texdocreturn{die Präferenz}
}
\texdocmethod{public}{ModulePreference}{getPreferences}{()}{}{\texdocreturn{gibt eine List der Modulpräferenzen zurück}
}
\texdocmethod{public}{User}{getUser}{()}{}{\texdocreturn{gibt den Eigentümer des Plans zurück}
}
\texdocmethod{public}{VerificationState}{getVerificationState}{()}{}{\texdocreturn{gibt den Verifizierungsstatus des Plans zurück}
}
\texdocmethod{public}{void}{setCreditPoints}{(int creditPoints)}{}{\begin{texdocparameters}
\texdocparameter{creditPoints}{die ECTS-Summe}
\end{texdocparameters}
}
\texdocmethod{public}{void}{setIdentifier}{(String identifier)}{}{\begin{texdocparameters}
\texdocparameter{identifier}{der Plan-Identifier}
\end{texdocparameters}
}
\texdocmethod{public}{void}{setName}{(String name)}{}{\begin{texdocparameters}
\texdocparameter{name}{der Name}
\end{texdocparameters}
}
\texdocmethod{public}{void}{setPlanId}{(String planId)}{}{\begin{texdocparameters}
\texdocparameter{planId}{die Plan-ID}
\end{texdocparameters}
}
\texdocmethod{public}{void}{setUser}{(User user)}{}{\begin{texdocparameters}
\texdocparameter{user}{der Eigentümer}
\end{texdocparameters}
}
\texdocmethod{public}{void}{setVerificationState}{(VerificationState verificationState)}{}{\begin{texdocparameters}
\texdocparameter{verificationState}{der Verifizierungsstatus}
\end{texdocparameters}
}
\end{texdocclassmethods}
\end{texdocclass}


\begin{texdocclass}{enum}{PreferenceType}
\label{texdoclet:edu.kit.informatik.studyplan.server.model.userdata.PreferenceType}
\begin{texdocclassintro}
Modelliert den Typ einer Modulpräferenz\end{texdocclassintro}
\begin{texdocenums}
\texdocenum{NEGATIVE}{das Modul wurde negativ bewertet}
\texdocenum{POSITIVE}{das Modul wurde positiv bewertet}
\end{texdocenums}
\begin{texdocclassmethods}
\texdocmethod{public static}{PreferenceType}{valueOf}{(String name)}{}{}
\texdocmethod{public static}{PreferenceType}{values}{()}{}{}
\end{texdocclassmethods}
\end{texdocclass}


\begin{texdocclass}{class}{Semester}
\label{texdoclet:edu.kit.informatik.studyplan.server.model.userdata.Semester}
\begin{texdocclassintro}
Modelliert ein Semester\end{texdocclassintro}
\begin{texdocclassconstructors}
\texdocconstructor{public}{Semester}{()}{}{}
\end{texdocclassconstructors}
\begin{texdocclassmethods}
\texdocmethod{public}{int}{getDistanceToCurrentSemester}{()}{Berechnet die Anzahl an Semester, die seit diesem Semester vergangen sind (inkl. aktuelles)}{\texdocreturn{die Semesterzahl}
}
\texdocmethod{public}{SemesterType}{getSemesterType}{()}{}{\texdocreturn{gibt den Typ des Semester zurück}
}\begin{texdocsees}{See also}
\texdocsee{edu.kit.informatik.studyplan.server.model.userdata.SemesterType}{texdoclet:edu.kit.informatik.studyplan.server.model.userdata.SemesterType}
\end{texdocsees}

\texdocmethod{public}{int}{getYear}{()}{}{\texdocreturn{gibt das Jahr zurück in dem das Semester begonnen hat}
}
\texdocmethod{public}{void}{setSemesterType}{(SemesterType semesterType)}{}{\begin{texdocparameters}
\texdocparameter{semesterType}{der Semestertyp}
\end{texdocparameters}
}
\texdocmethod{public}{void}{setYear}{(int year)}{}{\begin{texdocparameters}
\texdocparameter{year}{das Jahr in dem das Semester begonnen hat}
\end{texdocparameters}
}
\end{texdocclassmethods}
\end{texdocclass}


\begin{texdocclass}{enum}{SemesterType}
\label{texdoclet:edu.kit.informatik.studyplan.server.model.userdata.SemesterType}
\begin{texdocclassintro}
Modelliert Semester-Typen\end{texdocclassintro}
\begin{texdocenums}
\texdocenum{SUMMER\_TERM}{Sommersemester}
\texdocenum{WINTER\_TERM}{Wintersemester}
\end{texdocenums}
\begin{texdocclassmethods}
\texdocmethod{public static}{SemesterType}{valueOf}{(String name)}{}{}
\texdocmethod{public static}{SemesterType}{values}{()}{}{}
\end{texdocclassmethods}
\end{texdocclass}


\begin{texdocclass}{class}{User}
\label{texdoclet:edu.kit.informatik.studyplan.server.model.userdata.User}
\begin{texdocclassintro}
Modelliert einen Nutzer.\end{texdocclassintro}
\begin{texdocclassconstructors}
\texdocconstructor{public}{User}{()}{}{}
\end{texdocclassconstructors}
\begin{texdocclassmethods}
\texdocmethod{public}{Discipline}{getDiscipline}{()}{}{\texdocreturn{gibt den Studiengang zurück}
}
\texdocmethod{public}{List\textless{}ModuleEntry\textgreater{}}{getPassedModules}{()}{}{\texdocreturn{gibt eine Liste von Modul-Einträgen der bestandenen Module zurück}
}
\texdocmethod{public}{List\textless{}Plan\textgreater{}}{getPlans}{()}{}{\texdocreturn{gibt eine Liste der Studienpläne des Nutzers zurück}
}
\texdocmethod{public}{Semester}{getStudyStart}{()}{}{\texdocreturn{gibt das Semester des Studienstarts zurück}
}
\texdocmethod{public}{int}{getUserId}{()}{}{\texdocreturn{gibt die eindeutige ID des Nutzers zurück}
}
\texdocmethod{public}{String}{getUserName}{()}{}{\texdocreturn{gibt den eindeutigen Nutzernamen zurück}
}
\texdocmethod{public}{void}{setDiscipline}{(Discipline discipline)}{}{\begin{texdocparameters}
\texdocparameter{discipline}{der Studiengang}
\end{texdocparameters}
}
\texdocmethod{public}{void}{setStudyStart}{(Semester semester)}{}{\begin{texdocparameters}
\texdocparameter{semester}{das Semester des Studienstarts}
\end{texdocparameters}
}
\texdocmethod{public}{void}{setUserId}{(int userId)}{}{\begin{texdocparameters}
\texdocparameter{userId}{Wert, auf den die ID gesetzt wird}
\end{texdocparameters}
}
\texdocmethod{public}{void}{setUserName}{(String userName)}{}{\begin{texdocparameters}
\texdocparameter{userName}{der Nutzername}
\end{texdocparameters}
}
\end{texdocclassmethods}
\end{texdocclass}


\begin{texdocclass}{enum}{VerificationState}
\label{texdoclet:edu.kit.informatik.studyplan.server.model.userdata.VerificationState}
\begin{texdocclassintro}
Modelliert den Verifikationsstatus eines Studienplans\end{texdocclassintro}
\begin{texdocenums}
\texdocenum{INVALID}{der Plan enhält Fehler, d.h. Constraints sind verletzt}
\texdocenum{NOT\_VERIFIED}{der Plan wurde noch nicht verifiziert}
\texdocenum{VALID}{der Plan ist gültig}
\end{texdocenums}
\begin{texdocclassmethods}
\texdocmethod{public static}{VerificationState}{valueOf}{(String name)}{}{}
\texdocmethod{public static}{VerificationState}{values}{()}{}{}
\end{texdocclassmethods}
\end{texdocclass}


\end{texdocpackage}



\begin{texdocpackage}{server.pluginmanager}
\label{texdoclet:edu.kit.informatik.studyplan.server.pluginmanager}

\begin{texdocclass}{class}{GenerationManager}
\label{texdoclet:edu.kit.informatik.studyplan.server.pluginmanager.GenerationManager}
\begin{texdocclassintro}
Verwaltet den Zugriff auf das Generierungsplug-in.  
 Das Generierungsplug-in umfasst sowohl die Generierer-Schnittstelle als auch 
 die Zielfunktionen-Schnittstelle. Beide Schnittstellen werden mittels diese Klasse adaptiert.\end{texdocclassintro}
\begin{texdocclassconstructors}
\texdocconstructor{public}{GenerationManager}{()}{Erstellt einen GenerationManager.}{}
\end{texdocclassconstructors}
\begin{texdocclassmethods}
\texdocmethod{public}{double}{evaluate}{(Plan plan)}{Diese Methode ruft die evaluate Methode der 
 edu.kit.informatik.studyplan.server.generation.objectivefunction.PartialObjectiveFunction (siehe \ref{texdoclet:edu.kit.informatik.studyplan.server.generation.objectivefunction.PartialObjectiveFunction}).}{\begin{texdocparameters}
\texdocparameter{plan}{der zu bewertende Plan.}
\end{texdocparameters}
\texdocreturn{Wert zwischen 0 und 1 der den Plan evaluiert.}
}
\texdocmethod{public}{Plan}{generate}{(PartialObjectiveFunction objectiveFunction, Plan currentPlan, ModuleDao moduleDAO)}{Diese Methode ruft die generate Methode des
 edu.kit.informatik.studyplan.server.generation.Generator  (siehe \ref{texdoclet:edu.kit.informatik.studyplan.server.generation.Generator}).}{\begin{texdocparameters}
\texdocparameter{objectiveFunction}{Die Zielfunktion, anhand der optimiert werden soll}
\texdocparameter{currentPlan}{der bereits bestehende Plan}
\texdocparameter{moduleDAO}{die Module}
\end{texdocparameters}
\texdocreturn{ein vollständiger, korrekter und optimierter Studienplan vom Typ Plan.}
}
\texdocmethod{public}{Generator}{getGenerator}{()}{Gibt den Generator zurück.}{\texdocreturn{generator : der Generator}
}
\texdocmethod{public}{Collection\textless{}PartialObjectiveFunction\textgreater{}}{getObjectiveFunction}{()}{Gibt die Liste der Zielfunktionen zurück.}{\texdocreturn{objectiveFunction : die Liste der Zielfunktionen}
}
\end{texdocclassmethods}
\end{texdocclass}


\begin{texdocclass}{class}{VerificationManager}
\label{texdoclet:edu.kit.informatik.studyplan.server.pluginmanager.VerificationManager}
\begin{texdocclassintro}
Verwaltet den Zugriff auf das Verifizierungsplug-in, das die Verifizierer-Schnittstelle enthält.
 Diese Schnittstelle wird von dem VerificationManager adaptiert.\end{texdocclassintro}
\begin{texdocclassfields}
\texdocfield{public}{Verifier}{verifier}{Der Verifizierer.}\begin{texdocsees}{See also}
\texdocsee{edu.kit.informatik.studyplan.server.verification.Verifier}{texdoclet:edu.kit.informatik.studyplan.server.verification.Verifier}
\end{texdocsees}

\end{texdocclassfields}
\begin{texdocclassconstructors}
\texdocconstructor{public}{VerificationManager}{()}{Erstellt einen VerificationManager.}{}
\end{texdocclassconstructors}
\begin{texdocclassmethods}
\texdocmethod{public}{Verifier}{getVerifier}{()}{Gibt den Verifizierer zurück.}{\texdocreturn{verifier : der Verifizierer}
}
\texdocmethod{public}{VerificationResult}{verify}{(Plan plan)}{Diese Methode ruft die verify Methode des
 edu.kit.informatik.studyplan.server.verification.Verifier  (siehe \ref{texdoclet:edu.kit.informatik.studyplan.server.verification.Verifier}).}{\begin{texdocparameters}
\texdocparameter{plan}{Ein zu verifizierender Studienplan wird übergeben.}
\end{texdocparameters}
\texdocreturn{invalid Ein VerificationResult wird als Ergebnis der Verifizierung zurückgegeben.}
}
\end{texdocclassmethods}
\end{texdocclass}


\end{texdocpackage}



\begin{texdocpackage}{server.rest.authorization.endpoint}
\label{texdoclet:edu.kit.informatik.studyplan.server.rest.authorization.endpoint}

\begin{texdocclass}{class}{AuthorizationCodeGrant}
\label{texdoclet:edu.kit.informatik.studyplan.server.rest.authorization.endpoint.AuthorizationCodeGrant}
\begin{texdocclassintro}
Diese Klasse repräsentiert einen AuthorizationCodeGrant  $\{$ @see RFC 6749 Kapitel 1.3.1$\}$.
 AuthorizationCodeGrant wird in der ersten Version des Systems nicht benötigt aber zur möglichen 
 Erweiterungen vorgesehen.
  * Bei dem Versuch, eine Authentifizierung mittels diese Klasse durchzuführen wird eine Fehlermeldung
 zurückgegeben.\end{texdocclassintro}
\begin{texdocclassconstructors}
\texdocconstructor{public}{AuthorizationCodeGrant}{()}{Erstellt einen AuthorizationCodeGrant.}{}
\end{texdocclassconstructors}
\begin{texdocclassmethods}
\texdocmethod{public}{void}{getLogin}{(String clientId, String scope, String state)}{}{}
\texdocmethod{public}{void}{postToken}{(MultivaluedMap\textless{}StringString\textgreater{} params)}{}{}
\end{texdocclassmethods}
\end{texdocclass}


\begin{texdocclass}{class}{AuthResource}
\label{texdoclet:edu.kit.informatik.studyplan.server.rest.authorization.endpoint.AuthResource}
\begin{texdocclassintro}
Diese Klasse repräsentiert die Authentifizierung-Ressource.
 Der Resource owner wird zurück zu dieser Ressource weitergeleitet, nachdem er 
 den Zugang zur Anwendung gewährt hat.\end{texdocclassintro}
\begin{texdocclassconstructors}
\texdocconstructor{public}{AuthResource}{()}{Erstellt eine Authentifizierungs-Ressource.}{}
\end{texdocclassconstructors}
\begin{texdocclassmethods}
\texdocmethod{public}{GrantType}{getGrantType}{()}{Gibt den GrantType zurück.}{\texdocreturn{den grantType}
}
\texdocmethod{public}{void}{getLogin}{(String clientID, String scope, String state)}{GET-Anfrage:
 Gibt den Authorization Endpoint RFC 6749 Kapitel 3.1 (siehe \ref{texdoclet:RFC}) zurück.}{\begin{texdocparameters}
\texdocparameter{clientID}{den api\_key des Klienten.}
\texdocparameter{scope}{in den ersten Versionen des Systems immer „student“.}
\texdocparameter{state}{ein Schlüssel, der vom REST-Webservice in der Antwort mitgesendet wird.}
\end{texdocparameters}
}\begin{texdocsees}{See also}
\texdocsee{Kapitel 3.2 Tabelle 5.}{texdoclet:Kapitel}
\end{texdocsees}

\texdocmethod{public}{void}{postToken}{(MultivaluedMap\textless{}StringString\textgreater{} params)}{POST-Anfrage:
 Setzt den Token Endpoint RFC 6749 Kapitel 3.1 (siehe \ref{texdoclet:RFC}).}{\begin{texdocparameters}
\texdocparameter{params}{eine mehrwertige Zuordnung}
\end{texdocparameters}
}
\end{texdocclassmethods}
\end{texdocclass}


\begin{texdocclass}{interface}{GrantType}
\label{texdoclet:edu.kit.informatik.studyplan.server.rest.authorization.endpoint.GrantType}
\begin{texdocclassintro}
Diese Schnittstelle repräsentiert eine Fabrik zur erstellung von Typen von Authorisation Grant
 $\{$ @see RFC 6749 Kapitel 1.3$\}$\end{texdocclassintro}
\begin{texdocclassmethods}
\texdocmethod{public}{void}{getLogin}{(String clientId, String scope, String state)}{GET-Anfrage:
 Gibt den Authorization Endpoint $\{$ @see RFC 6749 Kapitel 3.1$\}$ zurück.}{\begin{texdocparameters}
\texdocparameter{clientID}{den api\_key des Klienten.}
\texdocparameter{scope}{in den ersten Versionen des Systems immer „student“.}
\texdocparameter{state}{ein Schlüssel, der vom REST-Webservice in der Antwort mitgesendet wird.}
\end{texdocparameters}
}\begin{texdocsees}{See also}
\texdocsee{Kapitel 3.2 Tabelle 5.}{texdoclet:Kapitel}
\end{texdocsees}

\texdocmethod{public}{void}{postToken}{(MultivaluedMap\textless{}StringString\textgreater{} params)}{POST-Anfrage:
 Setzt den Token Endpoint $\{$ @see RFC 6749 Kapitel 3.1$\}$.}{\begin{texdocparameters}
\texdocparameter{params}{eine mehrwertige Zuordnung}
\end{texdocparameters}
}
\end{texdocclassmethods}
\end{texdocclass}


\begin{texdocclass}{class}{ImplicitGrantType}
\label{texdoclet:edu.kit.informatik.studyplan.server.rest.authorization.endpoint.ImplicitGrantType}
\begin{texdocclassintro}
Diese Klasse repräsentiert einen ImplicitGrantType $\{$ @see RFC 6749 Kapitel 1.3.2$\}$.\end{texdocclassintro}
\begin{texdocclassconstructors}
\texdocconstructor{public}{ImplicitGrantType}{()}{Erstellt einen ImplicitGrantType.}{}
\end{texdocclassconstructors}
\begin{texdocclassmethods}
\texdocmethod{public}{void}{getLogin}{(String clientId, String scope, String state)}{}{}
\texdocmethod{public}{void}{postToken}{(MultivaluedMap\textless{}StringString\textgreater{} params)}{}{}
\end{texdocclassmethods}
\end{texdocclass}


\begin{texdocclass}{class}{PasswordCredentialsGrant}
\label{texdoclet:edu.kit.informatik.studyplan.server.rest.authorization.endpoint.PasswordCredentialsGrant}
\begin{texdocclassintro}
Diese Klasse repräsentiert einen PasswordCredentialsGrant  $\{$ @see RFC 6749 Kapitel 1.3.3$\}$.
 PasswordCredentialsGrant wird in der ersten Version des Systems nicht benötigt aber zur möglichen 
 Erweiterungen vorgesehen.
  * Bei dem Versuch, eine Authentifizierung mittels diese Klasse durchzuführen wird eine Fehlermeldung
 zurückgegeben.\end{texdocclassintro}
\begin{texdocclassconstructors}
\texdocconstructor{public}{PasswordCredentialsGrant}{()}{Erstellt einen PasswordCredentialsGrant.}{}
\end{texdocclassconstructors}
\begin{texdocclassmethods}
\texdocmethod{public}{void}{getLogin}{(String clientId, String scope, String state)}{}{}
\texdocmethod{public}{void}{postToken}{(MultivaluedMap\textless{}StringString\textgreater{} params)}{}{}
\end{texdocclassmethods}
\end{texdocclass}


\begin{texdocclass}{class}{RefreshGrant}
\label{texdoclet:edu.kit.informatik.studyplan.server.rest.authorization.endpoint.RefreshGrant}
\begin{texdocclassintro}
Diese Klasse repräsentiert einen RefreshGrant: beim Ablaufen der Access-Token, schickt dieser Granttype
 eine Refresh-Token $\{$ @see RFC 6749 Kapitel 1.5$\}$ an den Klient als Antwort .\end{texdocclassintro}
\begin{texdocclassconstructors}
\texdocconstructor{public}{RefreshGrant}{()}{Erstellt einen RefreshGrant.}{}
\end{texdocclassconstructors}
\begin{texdocclassmethods}
\texdocmethod{public}{void}{getLogin}{(String clientId, String scope, String state)}{}{}
\texdocmethod{public}{void}{postToken}{(MultivaluedMap\textless{}StringString\textgreater{} params)}{}{}
\end{texdocclassmethods}
\end{texdocclass}


\begin{texdocclass}{class}{UnknownGrant}
\label{texdoclet:edu.kit.informatik.studyplan.server.rest.authorization.endpoint.UnknownGrant}
\begin{texdocclassintro}
Diese Klasse repräsentiert einen unbekannten Granttype. Beim Instanziierung von dieser Klasse wird 
 eine Fehlermeldung zurückgegeben, da der Granttype ungültig ist.\end{texdocclassintro}
\begin{texdocclassconstructors}
\texdocconstructor{public}{UnknownGrant}{()}{Gibt fehler zurück.}{}
\end{texdocclassconstructors}
\begin{texdocclassmethods}
\texdocmethod{public}{void}{getLogin}{(String clientId, String scope, String state)}{}{}
\texdocmethod{public}{void}{postToken}{(MultivaluedMap\textless{}StringString\textgreater{} params)}{}{}
\end{texdocclassmethods}
\end{texdocclass}


\end{texdocpackage}



\begin{texdocpackage}{server.rest}
\label{texdoclet:edu.kit.informatik.studyplan.server.rest}

\begin{texdocclass}{class}{AuthorizationRequestFilter}
\label{texdoclet:edu.kit.informatik.studyplan.server.rest.AuthorizationRequestFilter}
\begin{texdocclassintro}
Klasse für das Filtern von Authentifizierungs-Anfragen.\end{texdocclassintro}
\begin{texdocclassconstructors}
\texdocconstructor{public}{AuthorizationRequestFilter}{()}{Erstellt einen Filter für die Authentifizierung-Anfragen.}{}
\end{texdocclassconstructors}
\begin{texdocclassmethods}
\texdocmethod{public}{void}{filter}{()}{}{}
\end{texdocclassmethods}
\end{texdocclass}


\begin{texdocclass}{class}{FilterResource}
\label{texdoclet:edu.kit.informatik.studyplan.server.rest.FilterResource}
\begin{texdocclassintro}
Diese Klasse repräsentiert die Filter-Ressource.\end{texdocclassintro}
\begin{texdocclassconstructors}
\texdocconstructor{public}{FilterResource}{()}{Erstellt eine Filter-Ressource.}{}
\end{texdocclassconstructors}
\begin{texdocclassmethods}
\texdocmethod{public}{List\textless{}JSONObject\textgreater{}}{getAllFilters}{()}{GET Anfrage:
 Gibt eine Liste aller vorhandenen Filtern zurück.}{\texdocreturn{eine Liste von Filtern.}
}
\end{texdocclassmethods}
\end{texdocclass}


\begin{texdocclass}{class}{GetParameters}
\label{texdoclet:edu.kit.informatik.studyplan.server.rest.GetParameters}
\begin{texdocclassintro}
Platzhalter für mehrere GET-Anfrage-Parameter.\end{texdocclassintro}
\begin{texdocclassconstructors}
\texdocconstructor{public}{GetParameters}{()}{}{}
\end{texdocclassconstructors}
\end{texdocclass}


\begin{texdocclass}{class}{JSONObject}
\label{texdoclet:edu.kit.informatik.studyplan.server.rest.JSONObject}
\begin{texdocclassintro}
Represäntiert einen unmodifizierbaren JSON-Objekt Wert und bietet eine unmodifizierbare Name$/$Werte Zuordnung an.\end{texdocclassintro}
\begin{texdocclassconstructors}
\texdocconstructor{public}{JSONObject}{()}{}{}
\end{texdocclassconstructors}
\end{texdocclass}


\begin{texdocclass}{class}{MainApplication}
\label{texdoclet:edu.kit.informatik.studyplan.server.rest.MainApplication}
\begin{texdocclassintro}
Hilfsklasse um Ressource Klassen festzulegen.\end{texdocclassintro}
\begin{texdocclassconstructors}
\texdocconstructor{public}{MainApplication}{()}{}{}
\end{texdocclassconstructors}
\end{texdocclass}


\begin{texdocclass}{class}{ModuleResource}
\label{texdoclet:edu.kit.informatik.studyplan.server.rest.ModuleResource}
\begin{texdocclassintro}
Diese Klasse repräsentiert die Modul-Ressource.\end{texdocclassintro}
\begin{texdocclassconstructors}
\texdocconstructor{public}{ModuleResource}{()}{Erstellt eine Module-Ressource.}{}
\end{texdocclassconstructors}
\begin{texdocclassmethods}
\texdocmethod{public}{List\textless{}JSONObject\textgreater{}}{getAllDisciplines}{()}{GET Anfrage:
 Gibt eine Liste der JSON-Representationen von allen Fachrichtungen zurück.}{\texdocreturn{eine Liste der JSON-Representationen von allen Fachrichtungen.}
}
\texdocmethod{public}{List\textless{}JSONObject\textgreater{}}{getAllSubjects}{()}{GET Anfrage:
 Gibt eine Liste der JSON-Representationen von allen Vertiefungsfächer zurück.}{\texdocreturn{eine Liste der JSON-Representationen von allen Vertiefungsfächer.}
}
\texdocmethod{public}{JSONObject}{getModule}{(String moduleID)}{Get Anfrage:
 Gibt eine JSON-Representation von dem Modul mit der gegebenen ID zurück.}{\begin{texdocparameters}
\texdocparameter{moduleID}{ID des erforderten Modul.}
\end{texdocparameters}
\texdocreturn{eine JSON-Representation von dem Modul.}
}
\texdocmethod{public}{List\textless{}JSONObject\textgreater{}}{getModules}{(GetParameters jsonFilter)}{GET Anfrage:
 Gibt eine Liste der JSON-Representationen von Modulen, die dem gegebenen Filter entsprechen, zurück.}{\begin{texdocparameters}
\texdocparameter{jsonFilter}{die benutzeten Filtern als Get-Parameter.}
\end{texdocparameters}
\texdocreturn{eine Liste der JSON-Representationen von Modulen.}
}
\end{texdocclassmethods}
\end{texdocclass}


\begin{texdocclass}{class}{ObjectiveFunctionResource}
\label{texdoclet:edu.kit.informatik.studyplan.server.rest.ObjectiveFunctionResource}
\begin{texdocclassintro}
Diese Klasse repräsentiert die Zielfunktion-Ressource.\end{texdocclassintro}
\begin{texdocclassconstructors}
\texdocconstructor{public}{ObjectiveFunctionResource}{()}{Erstellt eine Zielfunktion-Ressource.}{}
\end{texdocclassconstructors}
\begin{texdocclassmethods}
\texdocmethod{public}{List\textless{}JSONObject\textgreater{}}{getAllObjectiveFunctions}{()}{GET-Anfrage:
 Gibt eine Liste mit allen vorhandenen Zielfunktionen als JSON Objekte zurück.}{\texdocreturn{Liste mit allen vorhandenen Zielfunktionen als JSON Objekte.}
}
\texdocmethod{public}{GenerationManager}{getGenerationManager}{()}{Gibt den generierungsmanager zurück.}{\texdocreturn{der generationManager}
}
\end{texdocclassmethods}
\end{texdocclass}


\begin{texdocclass}{class}{PlanConverterResource}
\label{texdoclet:edu.kit.informatik.studyplan.server.rest.PlanConverterResource}
\begin{texdocclassintro}
Diese Klasse repräsentiert die Plankonverter-Ressource.\end{texdocclassintro}
\begin{texdocclassconstructors}
\texdocconstructor{public}{PlanConverterResource}{()}{Erstellt eine PlanKonverter-Ressource.}{}
\end{texdocclassconstructors}
\begin{texdocclassmethods}
\texdocmethod{public}{PDF}{convertplanToPDF}{(String planID, String accessToken)}{GET-Anfrage:
 Gibt die PDF-Version des Plans mit den gegebenen ID zurück.}{\begin{texdocparameters}
\texdocparameter{planID}{ID des zu konvertierenden Plans.}
\texdocparameter{accessToken}{Ein Token, zur Authentifizierung der Klient.}
\end{texdocparameters}
\texdocreturn{die PDF-Version des Plans.}
}
\end{texdocclassmethods}
\end{texdocclass}


\begin{texdocclass}{class}{PlanGeneratorResource}
\label{texdoclet:edu.kit.informatik.studyplan.server.rest.PlanGeneratorResource}
\begin{texdocclassintro}
Diese Klasse repräsentiert die Plangenerierer-Ressource.\end{texdocclassintro}
\begin{texdocclassconstructors}
\texdocconstructor{public}{PlanGeneratorResource}{()}{Erstellt eine Plangenerierer-Ressource.}{}
\end{texdocclassconstructors}
\begin{texdocclassmethods}
\texdocmethod{public}{JSONObject}{generatePlan}{(String planID, GetParameters jsonSettings)}{GET-Anfrage:
 Erstellt und gibt einen auf Basis des Plans mit der gegebenen ID generierten Plan 
 als JSON-Objekt zurück.}{\begin{texdocparameters}
\texdocparameter{planID}{ID des Basis-Plans.}
\texdocparameter{jsonSettings}{die gesetzten Einstellungen des Plans als Get-Parameter.}
\end{texdocparameters}
\texdocreturn{den generierten Plan als JSON Objekt.}
}
\texdocmethod{public}{GenerationManager}{getGenerationManager}{()}{Gibt den Generierungsmanager zurück.}{\texdocreturn{der generationManager}
}
\end{texdocclassmethods}
\end{texdocclass}


\begin{texdocclass}{class}{PlanModulesResource}
\label{texdoclet:edu.kit.informatik.studyplan.server.rest.PlanModulesResource}
\begin{texdocclassintro}
Diese Klasse repräsentiert die Planmodule-Ressource.\end{texdocclassintro}
\begin{texdocclassconstructors}
\texdocconstructor{public}{PlanModulesResource}{()}{Erstellt eine Planmodule-Ressource.}{}
\end{texdocclassconstructors}
\begin{texdocclassmethods}
\texdocmethod{public}{Filter}{getFilterFromRequest}{( params)}{GET-Anfrage:
 Gibt den angefragten Filter zurück.}{\begin{texdocparameters}
\texdocparameter{params}{Anfrage als eine mehrwertige Zuordnung von Strings.}
\end{texdocparameters}
\texdocreturn{den angefragten Filter.}
}
\texdocmethod{public}{JSONObject}{getModule}{(String planID, String moduleID)}{GET Anfrage:
 Gibt eine JSON-Representation von dem Modul mit der gegebenen ID, der in dem Plan mit der gegebenen ID ist, zurück.}{\begin{texdocparameters}
\texdocparameter{planID}{ID des zu bearbeitenden Plans.}
\texdocparameter{moduleID}{ID des erforderten Modul.}
\end{texdocparameters}
\texdocreturn{eine JSON-Representation von dem Modul als JSON Objekt.}
}
\texdocmethod{public}{JSONObject}{getModules}{(String plan\_id, GetParameters jsonFilter)}{GET Anfrage:
 Gibt die Liste der JSON-Representationen von Modulen, die :
     -in dem Plan mit der gegebenen ID sind und 
     -den gegebenen Filtern entsprechen, zurück.}{\begin{texdocparameters}
\texdocparameter{planID}{ID des zu bearbeitenden Plans.}
\texdocparameter{jsonFilter}{die benutzeten Filtern als Get-Parameter.}
\end{texdocparameters}
\texdocreturn{eine Liste der JSON-Representationen von Modulen.}
}
\texdocmethod{public}{JSONObject}{putModuleSemester}{(String planID, String moduleID, GetParameters jsonPutModule)}{PUT Anfrage: 
 fügt das Modul als JSON Objekt zur Plan mit den gegebenen ModulID bzw. PlanID hinzu.}{\begin{texdocparameters}
\texdocparameter{planID}{ID des zu bearbeitenden Plans.}
\texdocparameter{moduleID}{ID des hinzuzufügenden Modul.}
\texdocparameter{jsonPutModule}{das Modul als Get-Parameter.}
\end{texdocparameters}
\texdocreturn{JSON-Representation des Moduls als JSON Objekt.}
}
\texdocmethod{public}{void}{removeModuleSemester}{(String planID, String moduleID)}{DELETE-Anfrage:
 entfernt das Modul von dem Plan mit den gegebenen ModulID bzw. PlanID.}{\begin{texdocparameters}
\texdocparameter{planID}{ID des zu bearbeitenden Plans.}
\texdocparameter{moduleID}{ID des zu entfernenden Modul.}
\end{texdocparameters}
}
\texdocmethod{public}{JSONObject}{setModulePreference}{(String planID, String moduleID, GetParameters jsonModulePreference)}{PUT-Anfrage:
 setzt eine Bewertung als JSON Objekt für den Modul mit der gegebenen ID, der im Plan mit der gegebenen ID.}{\begin{texdocparameters}
\texdocparameter{planID}{ID des zu bearbeitenden Plans.}
\texdocparameter{moduleID}{ID des zu bewertenden Modul.}
\texdocparameter{jsonModulePreference}{die zu setzenden Bewertung des Moduls als GetParameters.}
\end{texdocparameters}
\texdocreturn{die gesetzten Bewertung des Moduls als JSON Objekt.}
}
\end{texdocclassmethods}
\end{texdocclass}


\begin{texdocclass}{class}{PlansResource}
\label{texdoclet:edu.kit.informatik.studyplan.server.rest.PlansResource}
\begin{texdocclassintro}
Diese Klasse repräsentiert die Pläne-Ressource.\end{texdocclassintro}
\begin{texdocclassconstructors}
\texdocconstructor{public}{PlansResource}{()}{Erstellt eine Module-Ressource.}{}
\end{texdocclassconstructors}
\begin{texdocclassmethods}
\texdocmethod{public}{JSONObject}{createPlan}{()}{POST-Anfrage:
 Erstellt einen neuen Studienplan.}{\texdocreturn{jsonPlan der erstellte Plan als JSON Objekt.}
}
\texdocmethod{public}{void}{deletePlan}{(String planID)}{DELETE-Anfrage:
 Löscht den Plan mit dem gegebenen ID.}{\begin{texdocparameters}
\texdocparameter{planID}{ID des zu löschenden Plans.}
\end{texdocparameters}
}
\texdocmethod{public}{JSONObject}{duplicatePlan}{(String planID)}{POST-Anfrage:
 Dupliziert der Plan mit der gegebenen ID}{\begin{texdocparameters}
\texdocparameter{planID}{ID des zu duplizierenden Plans.}
\end{texdocparameters}
\texdocreturn{jsonPlan Plan als JSON Objekt.}
}
\texdocmethod{public}{JSONObject}{editPlan}{(String planID, GetParameters jsonPlan)}{PATCH-Anfrage:
 Bearbeitet den Plan mit der gegebenen ID.}{\begin{texdocparameters}
\texdocparameter{planID}{ID des zu bearbeitenden Plans.}
\texdocparameter{jsonPlan}{der Plan als Get-Parameter.}
\end{texdocparameters}
\texdocreturn{jsonChangedPlan JSON Objekt des bearbeiteten Plans.}
}
\texdocmethod{public}{JSONObject}{getPlan}{(String plan\_id)}{GET-Anfrage:
 Gibt den Plan mit der gegebenen ID zurück.}{\begin{texdocparameters}
\texdocparameter{planID}{ID des angefragten Plans.}
\end{texdocparameters}
\texdocreturn{jsonPlan der Plan als JSON Objekt.}
}
\texdocmethod{public}{List\textless{}JSONObject\textgreater{}}{getPlans}{(Collection\textless{}JSONObject\textgreater{} jsonPlanList)}{GET-Anfrage:
 Gibt eine Liste aller vorhandenen StudienPläne zurück.}{\begin{texdocparameters}
\texdocparameter{jsonPlanList}{einen Array aller vorhandenen StudienPläne als JSON Objekte.}
\end{texdocparameters}
\texdocreturn{jsonPlanList eine Liste aller vorhandenen StudienPläne als JSON Objekte.}
}
\texdocmethod{public}{JSONObject}{replacePlan}{(String planID, JSONObject jsonPlan)}{PUT-Anfrage:
 Ersetzt den Plan mit der gegebenen ID mit den gegeben Plan .}{\begin{texdocparameters}
\texdocparameter{planID}{ID des zu entfernenden Plans.}
\texdocparameter{jsonPlan}{der zu speichernden Plan als JSON Objekt.}
\end{texdocparameters}
\texdocreturn{jsonNewPlan der gespeicherte Plan.}
}
\end{texdocclassmethods}
\end{texdocclass}


\begin{texdocclass}{class}{PlanVerifierResource}
\label{texdoclet:edu.kit.informatik.studyplan.server.rest.PlanVerifierResource}
\begin{texdocclassintro}
Diese Klasse repräsentiert die Planverifizierer-Ressource.\end{texdocclassintro}
\begin{texdocclassconstructors}
\texdocconstructor{public}{PlanVerifierResource}{()}{Erstellt eine Planverifizierer-Ressource.}{}
\end{texdocclassconstructors}
\begin{texdocclassmethods}
\texdocmethod{public}{VerificationManager}{getVerificationManager}{()}{Gibt den Verifizierungsmanager zurück.}{\texdocreturn{der verificationManager}
}
\texdocmethod{public}{JSONObject}{verifyPlan}{(String planID)}{GET-Anfrage:
 Verifiziert den Planmit den gegebenen ID, gibt den verifizierten Plan zurück und speichert ihn 
 in der Datenbank.}{\begin{texdocparameters}
\texdocparameter{planID}{ID des plans.}
\end{texdocparameters}
\texdocreturn{den verifizierten Plan als JSON Objekt.}
}
\end{texdocclassmethods}
\end{texdocclass}


\begin{texdocclass}{class}{StudentResource}
\label{texdoclet:edu.kit.informatik.studyplan.server.rest.StudentResource}
\begin{texdocclassintro}
Diese Klasse repräsentiert dir Student-Resource.\end{texdocclassintro}
\begin{texdocclassconstructors}
\texdocconstructor{public}{StudentResource}{()}{Erstellt eine Student-Resource.}{}
\end{texdocclassconstructors}
\begin{texdocclassmethods}
\texdocmethod{public}{void}{deleteStudent}{()}{DELETE-Anfrage:
 Löscht den Student.}{}
\texdocmethod{public}{JSONObject}{getInformation}{()}{GET-Anfrage:
 Gibt die Studentinformationen zurück.}{\texdocreturn{die Studentinformationen als JSON Objekt.}
}
\texdocmethod{public}{JSONObject}{replaceInformation}{(JSONObject jsonStudentInformation)}{PUT-Anfrage:
 Ersetzt Informationen über einen Student und löscht die Verifikationsinformationen.}{\begin{texdocparameters}
\texdocparameter{jsonStudentInformation}{die Informationen über den Student als JSON Objekt.}
\end{texdocparameters}
\texdocreturn{Student mit neue Informationen als JSON Objekt.}
}
\end{texdocclassmethods}
\end{texdocclass}


\end{texdocpackage}



\begin{texdocpackage}{server.verification}
\label{texdoclet:edu.kit.informatik.studyplan.server.verification}

\begin{texdocclass}{class}{VerificationResult}
\label{texdoclet:edu.kit.informatik.studyplan.server.verification.VerificationResult}
\begin{texdocclassintro}
Die Klasse VerificationResult ist das Ergebins einer Verifizierung.\end{texdocclassintro}
\begin{texdocclassconstructors}
\texdocconstructor{public}{VerificationResult}{()}{}{}
\end{texdocclassconstructors}
\begin{texdocclassmethods}
\texdocmethod{public}{Collection\textless{}ModuleConstraint\textgreater{}}{getViolations}{()}{Gibt die verletzten Modul-Constraints zurück.}{\texdocreturn{die verletzten Modul-Constraints}
}
\texdocmethod{public}{boolean}{isCorrect}{()}{is correct prüft anhand von violations, ob der Studienplan erfolgreich verifiziert wurde.}{\texdocreturn{zurückgegeben wird false, falls der Studienplan fehlerhaft ist und true,
 wenn er zu einem erfolreichen Studienabschluss führt.}
}
\end{texdocclassmethods}
\end{texdocclass}


\begin{texdocclass}{interface}{Verifier}
\label{texdoclet:edu.kit.informatik.studyplan.server.verification.Verifier}
\begin{texdocclassintro}
Das Interface Verifier bietet die allgemeine Sturktur eines Verifizierers.\end{texdocclassintro}
\begin{texdocclassmethods}
\texdocmethod{public}{VerificationResult}{verify}{(Plan plan)}{Die Methode verify verifiziert einen übergebenen Studienplan. 
 Das heißt sie überprüft an Hand der gegebenen System-Constraints, 
 ob der Plan zu einem erfolreichen Studienabschluss führen kann.}{\begin{texdocparameters}
\texdocparameter{plan}{Ein zu verifizierender Studienplan wird übergeben.}
\end{texdocparameters}
\texdocreturn{invalid Ein VerificationResult wird als Ergebnis der Verifizierung zurückgegeben.}
}
\end{texdocclassmethods}
\end{texdocclass}


\end{texdocpackage}



\begin{texdocpackage}{server.verification.standard}
\label{texdoclet:edu.kit.informatik.studyplan.server.verification.standard}

\begin{texdocclass}{class}{StandardVerifier}
\label{texdoclet:edu.kit.informatik.studyplan.server.verification.standard.StandardVerifier}
\begin{texdocclassintro}
Der StandardVerifier importiert den Verifier. 
 Er ist eine konkret verifizierende Klasse.\end{texdocclassintro}
\begin{texdocclassconstructors}
\texdocconstructor{public}{StandardVerifier}{()}{}{}
\end{texdocclassconstructors}
\begin{texdocclassmethods}
\texdocmethod{public}{VerificationResult}{verify}{(Plan plan)}{}{}
\end{texdocclassmethods}
\end{texdocclass}


\end{texdocpackage}



