% restusecase environment
% #1: Title of the use case, like "A{n}: Title"
\newenvironment{restusecase}[1]{
	\begin{longtable}{|p{.43\linewidth}|p{.55\linewidth}|}
		\multicolumn{2}{@{}l}{\textsf{\textbf{#1}}} \vspace{2pt} \\
		\hhline{|=|=|}
		\textbf{Aufruf} & \textbf{Antwort} \\
		\hline 
		\endfirsthead
		
		\hline
		\textbf{Aufruf} & \textbf{Antwort} \\
		\endhead
		
		\hhline{|=|=|}
		\endlastfoot 
}{
	\end{longtable} \vspace{-12pt}
}

\newcommand{\case}[1]{\textit{#1}}



%%%%%%%%%%%%%%%%%%%%%%%%%%%%%%%%%%%%%%%%%%%%%%%%%%%%%%%%%%%%%

\begin{comment}

\FloatBarrier
\subsection{Umsetzung der Anwendungsfälle mittels REST-Kommunikation}

These are provisional stubs and are going to be removed or replaced by sequence diagrams in future versions of this document. (Warning: They are not updated yet. Do not use them as reference for sequence diagrams unless you use your brain and the official REST command description. If in doubt, shout – some REST experts might come to your rescue.)

\begin{restusecase}{A10: Erstanmeldung}
	GET /auth/login
	& (Weiterleitung) \\
	\hline
	GET /student
	& (Statuscode für Erstanmeldung) \\
	\hline
	GET /filters
	& \jsonlist{\jsonobj{Filter}} \\
	\hline
	\multicolumn{2}{|c|}{(Bei Bedarf: Modulfilterung (s. A110))} \\
	\hline
	PUT /student \newline
	\jsonobj{Student}
	& (Bestätigung) \\
\end{restusecase}

\begin{restusecase}{A20: Login}
	GET /auth/login
	& (Weiterleitung) \\
	\hline
	GET /student
	& \jsonobj{Student} \\
	\hline
	GET /plans
	& \jsonlist{\jsonobj{Studienplan}}
\end{restusecase}

\begin{restusecase}{A30: Logout}
	(Session beenden? Wie?) & 
\end{restusecase}

\begin{restusecase}{A40: Profil bearbeiten}
	GET /student
	& \jsonobj{Student} \\
	\hline
	GET /filters
	& \jsonlist{\jsonobj{Filter}} \\
	\hline
	\multicolumn{2}{|c|}{(Bei Bedarf: Modulfilterung (s. A110))} \\
	\hline
	PUT /student \newline \jsonobj{Student}
	& (Bestätigung) \\
\end{restusecase}

\begin{restusecase}{A50: Neuen Studienplan anlegen}
	POST /plans \newline \jsonatom{Studienplan-Name}
	& (Bestätigung/Fehler) \\
	\hline
	GET /filters
	& [\jsonobj{Filter}, ...] \\
\end{restusecase}

\begin{restusecase}{A55: Studienplan anzeigen}
	GET /plans/\jsonatom{Plan-ID}
	& \jsonobj{Studienplan} TODO: Sind da schon Modulinfos wie Bewertungen mit drin? \\
	\hline
	GET /filters
	& \jsonlist{\jsonobj{Filter} }\\
\end{restusecase}


\begin{restusecase}{A57: Schließen der Studienplan-Ansicht mit Wechsel zur Hauptansicht}
	GET /plans
	& \jsonlist{\jsonpart{Studienplan}{name, id, verified}}
\end{restusecase}

\begin{restusecase}{A60: Studienplan umbenennen}
	PATCH /plans/\jsonatom{Plan-ID} \newline \jsonpart{Studienplan}{name}
	& (Bestätigung/Fehler)
\end{restusecase}

\begin{restusecase}{A70: Studienplan duplizieren}
	POST /plans/\jsonatom{Plan-ID} \newline \jsonpart{Studienplan}{name}
	& (Bestätigung/Fehler)
\end{restusecase}

\begin{restusecase}{A80: Studienplan löschen}
	DELETE /plans/\jsonatom{Plan-ID}
	& (Bestätigung/Fehler)
\end{restusecase}

\begin{restusecase}{A85: Studienplan exportieren}
	GET /plans/\jsonatom{Plan-ID}/pdf
	& (Weiterleitung auf on-the-fly generiertes PDF-Dokument)
\end{restusecase}

\begin{restusecase}{A90: Mehrere Studienpläne duplizieren/löschen}
	\multicolumn{2}{|c|}{Für alle markierten Studienpläne wird nacheinander aufgerufen:} \\
	\hline
	\case{Duplizieren/Löschen}: s. A70 bzw. A80. 
	& \case{Duplizieren/Löschen}: (Bestätigung/Fehler)
\end{restusecase}

\begin{restusecase}{A100: Vergleichsansicht für Studienpläne}
	\case{Plan 1}: GET /plans/\jsonatom{Plan-ID}
	& \jsonobj{Studienplan} \\
	\hline
	\case{Plan 2}: GET /plans/\jsonatom{Plan-ID}
	& \jsonobj{Studienplan}
\end{restusecase}

\begin{restusecase}{A110: Module in der Suchleiste filtern}
	GET /plan/\jsonatom{Plan-ID}/\+modules\+?\jsonatom{filter}=\jsonatom{value}...
	& \jsonlist{\jsonpart{Modul}{id, name, creditpoints, lecturer, preference}}
\end{restusecase}

\begin{restusecase}{A130: Info-Leiste zu einem Modul anzeigen}
	\multicolumn{2}{|c|}{--- (bereits bei A110 abgerufen).}
\end{restusecase}

\begin{restusecase}{A140: Modul in Studienplan einfügen}
	PUT /plans/\jsonatom{Plan-ID}/\+modules/\+\jsonatom{Modul-ID}  \newline  \jsonpart{Modul}{id, semester}
	& (Bestätigung/Fehler)
\end{restusecase}

\begin{restusecase}{A150: Modul aus Studienplan löschen}
	DELETE /plans/\jsonatom{Plan-ID}/\+modules/\jsonatom{Modul-ID} 
	& (Bestätigung/Fehler)
\end{restusecase}

\begin{restusecase}{A160: Modul innerhalb Studienplan verschieben}  
	DELETE /plans/\jsonatom{Plan-ID}/\+modules/\+\jsonatom{Modul-ID}  \newline  \jsonobj{PlanModuleDeleteRequest}
	& (Bestätigung/Fehler) \\
	\hline
	PUT /plans/\jsonatom{Plan-ID}/\+modules/\+\jsonatom{Modul-ID}  \newline
	\jsonobj{PlanModulePutRequest}
	& (Bestätigung/Fehler)
\end{restusecase}

\begin{restusecase}{A170: Studienplanänderung rückgängig machen}
	\multicolumn{2}{|c|}{(abhängig von letzter Änderung)}
\end{restusecase}

\begin{restusecase}{A180: Modul positiv bewerten / A190: Modul negativ bewerten}
	PUT /plans/\jsonatom{Plan-ID}/\+modules/\+\jsonatom{Modul-ID}/\+preference \newline \jsonpart{Modul}{id, preference}
	& (Bestätigung/Fehler)
\end{restusecase}

\begin{restusecase}{A200: Semester im Studienplan hinzufügen}
	\multicolumn{2}{|c|}{--- (werden nicht explizit gespeichert)}
\end{restusecase}

\begin{restusecase}{A210: Semester aus Studienplan löschen}
	\multicolumn{2}{|c|}{--- (ggfs. Module löschen (A150))}
\end{restusecase}

\begin{restusecase}{A215: Abgeschlossene Semester im Studienplan ein-/ausblenden}
	\multicolumn{2}{|c|}{---}
\end{restusecase}

\begin{restusecase}{A220: Studienplan auf Korrektheit überprüfen}
	GET /plans/\jsonatom{Plan-ID}/verification 
	& \jsonpart{Studienplan}{id, status, violations}
\end{restusecase}

\begin{restusecase}{A230: Studienplan vervollständigen lassen}
	GET /objective-functions
	& \jsonlist{\jsonobj{Objective}} \\
	\hline
	\multicolumn{2}{|l|}{(Zum Festlegen von Präferenzen siehe A110, A180, A190)} \\
	\hline
	GET /plans/\jsonatom{Plan-ID}/proposal/\+\jsonatom{Zielfunktions-ID}?\jsonatom{prop}=\jsonatom{value}... 
	& \jsonpart{Studienplan}{id, modules} oder Fehlercode. \\
	\hline
	\case{Verwerfen}: --- 
	& --- \\
	\case{Übernehmen}: PUT /plans/\jsonatom{Plan-ID} \newline \jsonobj{Studienplan}
	& (Bestätigung/Fehler) \\
	\case{Unter neuem Namen speichern}: \newline 
	1. POST /plans \newline \jsonatom{Studienplan-Name}
	& (Bestätigung/Fehler) \\
	2. PUT /plans/\jsonatom{Plan-ID} \newline \jsonobj{Studienplan}
	& (Bestätigung/Fehler) \\
\end{restusecase}

\end{comment}

