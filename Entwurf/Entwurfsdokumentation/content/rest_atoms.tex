\FloatBarrier
\subsection{Atomare Werte}

Es folgt die Beschreibung der für die REST-Kommunikation notwendigen atomaren Werte.

\begin{longtable}{p{.22\linewidth} p{.73\linewidth}}
	Bezeichner
	& Erklärung \\
	\hline
	\endfirsthead
	
	Bezeichner
	& Erklärung \\
	\hline 
	\endhead
	
	\hline
	\endlastfoot
	
	\lbljsonatom{Modul-Turnus} 
	& $ \in \{\verb|"WS"|, \verb|"SS"|, \verb|"both"|\}$. \\
	\lbljsonatom{Semester-Typ} 
	& $ \in \{\verb|"WS"|, \verb|"SS"|\}$. \\
	\lbljsonatom{Jahr}
	& Jahreszahl (Ganzzahl). \\
	\lbljsonatom{Studienfach-ID}
	& ID, welche ein eindeutiges Studienfach repräsentiert. \\
	\lbljsonatom{Studienfach-Name}
	& Name eines Studienfachs. \\
	\lbljsonatom{Modul-ID}
	& ID, welche ein eindeutiges Modul repräsentiert. \\
	\lbljsonatom{Modul-Name}
	& Name eines Moduls. \\
	\lbljsonatom{Modul-Kategorie}
	& Name der Kategorie eines Moduls. \\
	\lbljsonatom{Modul-Semester}
	& Das Semester, in welchem sich ein Modul befindet (Ganzzahl). \\
	\lbljsonatom{Modul-Creditpoints}
	& ECTS-Zahl eines Moduls. \\
	\lbljsonatom{Modul-Dozent}
	& Name des Dozenten des Moduls. \\
	\lbljsonatom{Modul-Präferenz}
	& $ \in \{\verb|"positive"|, \verb|"negative"|, \verb|""|\} $. \\
	\lbljsonatom{Modul-Beschreibung}
	& Beschreibungstext eines Moduls. \\
	\lbljsonatom{Constraint-Name}
	& Bezeichner eines Constraints. \\
	\lbljsonatom{Constraint-Typ}
	& Kurzer Text, der den Constraint-Typ beschreibt (für UI). \\
	\lbljsonatom{Filter-ID}
	& ID, welche einen eindeutigen Filter repräsentiert. \\
	\lbljsonatom{Filter-Name}
	& Name des Filters (Titel für UI). \\
	\lbljsonatom{Filter-Default}
	& Für List-Filter: Nummer des standardmäßig ausgewählten Elements. \newline
	Für Contains-Filter: Standardmäßiger Suchstring. \newline
	Für Range-Filter: \newline
	{\begin{tbljson}
{
	"min": (*\jsonatom{Filter-Minimum}*),
	"max": (*\jsonatom{Filter-Maximum}*)	
} 	
	\end{tbljson}} \\
	\lbljsonatom{Filter-Tooltip}
	& Kurzer Beschreibungstext des Filters für ein UI-Tooltip. \\
	\lbljsonatom{Filter-Typ}
	& $ \in \{\verb|"range"|, \verb|"list"|, \verb|"contains"|\} $. \\
	\lbljsonatom{Filter-Minimum}
	& Ganze Zahl, die eine untere Schranke eines Range-Filters beschreibt. \\
	\lbljsonatom{Filter-Maximum}
	& Ganze Zahl, die eine obere Schranke eines Range-Filters beschreibt. \\
	\lbljsonatom{Item-ID}
	& Nummer des \jsonobj{Filter-Wahlitem} eines UI-Filter-Auswahlfeldes (die Wahlitems sind nullbasiert fortlaufend nummeriert). \\
	\lbljsonatom{Item-Text}
	& Anzeigetext der Wahlmöglichkeit eines UI-Filter-Auswahlfeldes (String). \\
	\lbljsonatom{Zielfunktion-ID}
	& ID, welche eine eindeutige Zielfunktion repräsentiert. \\
	\lbljsonatom{Zielfunktion-Name}
	& Name einer Zielfunktion. \\
	\lbljsonatom{Zielfunktion-Beschreibung}
	& Beschreibungstext einer Zielfunktion für die UI. \\
	\lbljsonatom{Studienplan-ID}, \lbljsonatom{Plan-ID}
	& ID, welche einen eindeutigen Studienplan repräsentiert. \\
	\lbljsonatom{Studienplan-Status}
	& $ \in \{\verb|"valid"|, \verb|"invalid"|, \verb|"not-verified"|\} $. \\
	\lbljsonatom{Studienplan-Name}
	& Name eines Studienplans. \\
	\lbljsonatom{Studienplan-Gesamt-Creditpoints}
	& Gesamtzahl aller ECTS eines Studienplans. \\
	\lbljsonatom{Untere-ECTS-Schranke}/\lbljsonatom{Obere-ECTS-Schranke}
	& \jsonatom{Filter-Minimum}/\jsonatom{Filter-Maximum} für ECTS-Filter. \\
	\lbljsonatom{Semester-Minimum}/\lbljsonatom{Semester-Maximum}
	& Minimale/Maximale Anzahl an Semestern, die der Generierungsvorschlag haben soll. \\
	\lbljsonatom{Semester-ECTS-Minimum}/\lbljsonatom{Semester-ECTS-Maximum}
	& Minimale/Maximale Anzahl von ECTS pro Semester, die der Generierungsvorschlag haben soll. \\
	\lbljsonatom{Access-Token}
	& Das in Kapitel \ref{subsec:api-auth}, Tabelle \ref{tab:api-auth-login-res-success} spezifizierte Access-Token. \\
\end{longtable}