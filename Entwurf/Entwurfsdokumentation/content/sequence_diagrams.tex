\subsection{Abläufe}

In den folgenden Sequenzdiagrammen wird beispielhaft vereinfachte* Umsetzungen mehrerer Anwendungsfälle gezeigt.\\
*Die Parameter fallen hier weg.\\

Die Abbildungen \ref{seq:register_1} bis \ref{seq:register_4} beschreiben den Registrierungsablauf, der Teil der Umsetzung des Anwendungsfalls „A40: Profil bearbeiten“ ist.

\begin{figure}[H]
	\resizebox{\textwidth}{!} {
	\begin{tikzpicture}  
	\begin{umlseqdiag}
	  	\umlobject[no ddots]{BackboneHistory}
		\umlobject[class=MainRouter]{router} 
		\umlobject[class=WizardPage]{wizPage} 
		\begin{umlcall}[op={signUpWizard()}, dt=5]{BackboneHistory}{router} 		
			\begin{umlcall}[op={create}]{router}{wizPage} 
				\umlcreatecall[class=SignUpWizardComponent1]{wizPage}{curView}  
			\end{umlcall}
			\begin{umlcall}[op={render()}, dt=5]{router}{curView}		
			\end{umlcall} 
			\umlcreatecall[class=MainView]{router}{mainView}
			\umlobject[no ddots]{SessionInformation}  
			\begin{umlcall}[op={setHeader()}]{router}{mainView}  
				\begin{umlcall}[op={getInstance()}, return={sessionInformation}]{mainView}{SessionInformation}  
					\umlcreatecall[class=SessionInformation]{SessionInformation}{sessionInfo}  
				\end{umlcall}			
			\end{umlcall}
			\begin{umlcall}[op={setContent()}]{router}{mainView}  
				\begin{umlcall}[op={getInstance()}, return={sessionInformation}]{mainView}{SessionInformation}  
				\end{umlcall}	
			\end{umlcall}
		\end{umlcall}
	\end{umlseqdiag}
\end{tikzpicture}
}
	\caption{1. Registrierungsseite anzeigen.}
	\label{seq:register_1}
\end{figure}

\begin{figure}[H]
	\resizebox{\textwidth}{!} {
	\begin{tikzpicture}  
	\begin{umlseqdiag}
		\umlobject[class=MainRouter]{router} 
		\umlobject[class=SignUpWizardComponent1, x=4.8]{firstPage} 
		\umlobject[class=Student, x=9.5]{student} 
		\umlobject[class=Sync]{sync} 
		\umlobject[class=SessionInformation, x=15.3]{sesInfo} 
		\umlobject[class=StudentResource]{studentRes}
		\begin{umlcall}[op={onChange()}, dt=5]{router}{firstPage}		
			\begin{umlcall}[op={setStudyStartCycle()}]{firstPage}{student}   
			\end{umlcall}
			\begin{umlcall}[op={save()}]{firstPage}{student}  
				\begin{umlcall}[op={OAuthSync()}]{student}{sync}   
					\begin{umlcall}[op={getAccessToken()}, return={AccessToken}]{sync}{sesInfo}   
					\end{umlcall} 
					\begin{umlcall}[op={replaceInformation()}, dt=5]{sync}{studentRes}
					\end{umlcall} 					
				\end{umlcall} 
			\end{umlcall}
		\end{umlcall}
	\end{umlseqdiag}
	\end{tikzpicture}
}
	\caption{Die verarbeiteten Informationen speichern(Clientseitig)}
	\label{seq:register_2}
\end{figure}

\begin{figure}[H]
	\resizebox{\textwidth}{!} {
	\begin{tikzpicture}  
	\begin{umlseqdiag}
		\umlobject[class=Sync]{sync} 
 		\umlobject[class=StudentResource]{studentRes}
		\umlobject[class=SessionInformation, x=9]{sesInfo} 
		\umlobject[class=AuthorizationContext, x=13.8]{auth}
		\umlobject[class=User, x=17.1]{user}
		\begin{umlcall}[op={replaceInformation()}, dt=5]{sync}{studentRes}
			\begin{umlcall}[op={getAccessToken()}, return={AccessToken}]{studentRes}{sesInfo}   
			\end{umlcall} 
			\begin{umlcall}[op={getUser()}, return={user}, dt=5]{studentRes}{auth}   
			\end{umlcall}
			\begin{umlcall}[op={setStudyStartCycle()}, dt=5]{studentRes}{user}   
			\end{umlcall} 
		\end{umlcall}
	\end{umlseqdiag}
	\end{tikzpicture}
}
	\caption{Die verarbeiteten Informationen speichern(Serverseitig)}
	\label{seq:register_3}
\end{figure}

\begin{figure}[H]
	\resizebox{\textwidth}{!} {
	\begin{tikzpicture}  
	\begin{umlseqdiag}
		\umlobject[class=MainRouter]{router} 
		\umlobject[class=WizardPage, x=3.5]{wizPage} 
		\umlobject[class=SignUpWizardComponent1, x=8.5]{firstPage} 
		\umlobject[class=SignUpWizardComponent2, x=15]{secondPage} 
		\begin{umlcall}[op={next()}, dt=5]{router}{wizPage}	
			\begin{umlcall}[op={next()}, return={secondPage}]{wizPage}{firstPage}
				\begin{umlcall}[op={create()}]{firstPage}{secondPage}   
				\umlcreatecall[class=ModulFinder, x=18.6]{secondPage}{modulFinder}  
				\end{umlcall} 
			\end{umlcall}
			\begin{umlcall}[op={render()}, dt=5]{wizPage}{secondPage}   
				\begin{umlcall}[op={render()}]{secondPage}{modulFinder}   
				\end{umlcall} 
			\end{umlcall} 
		\end{umlcall}
	\end{umlseqdiag}
	\end{tikzpicture}
}
	\caption{Zur 2. Registrierungsseite wechseln}
	\label{seq:register_4}
\end{figure}

Die Abbildungen \ref{seq:addmodul_1} bis \ref{seq:addmodul_3} beschreiben einen Teil der Umsetzung des Anwendungsfalls „A140: Modul in Studienplan einfügen “.

\begin{figure}[H]
	\resizebox{\textwidth}{!} {
	\begin{tikzpicture}  
	\begin{umlseqdiag}
	  	\umlobject[no ddots]{BackboneHistory}
		\umlobject[class=MainRouter]{router} 
		\umlobject[class=PlanEditPage]{editPage} 
		\begin{umlcall}[op={editPage()}, dt=5]{BackboneHistory}{router}
			\umlcreatecall[class=PlanHeadBar]{router}{headBar} 		
			\umlcreatecall[class=BackboneView]{router}{sideBar} 		
			\umlcreatecall[class=Plan]{router}{plan} 		
			\begin{umlcall}[op={render()}]{router}{editPage} 
				\begin{umlcall}[op={render()}]{editPage}{headBar}		
				\end{umlcall}
				\begin{umlcall}[op={render()}]{editPage}{sideBar}	
				\end{umlcall}
				\begin{umlcall}[op={render()}]{editPage}{plan}			
				
				\end{umlcall}
 			\end{umlcall}
		\end{umlcall}
	\end{umlseqdiag}
\end{tikzpicture}
}
	\caption{Bearbeitungsansicht anzeigen}
	\label{seq:addmodul_1}
\end{figure}

\begin{figure}[H]
	\resizebox{\textwidth}{!} {
	\begin{tikzpicture}  
	\begin{umlseqdiag}
		\umlobject[no ddots]{Backbone} 
		\umlobject[class=Semester, x=3.2]{semester} 
		\umlobject[class=Plan, x=6.2]{plan} 
		\umlobject[class=Sync, x=8.8]{sync} 
		\umlobject[class=SessionInformation, x=12.5]{sesInfo} 
		\umlobject[class=PlanResource, x=17]{planRes}
		\begin{umlcall}[op={onDrop()}, dt=5]{Backbone}{semester}
			\begin{umlcall}[op={onChange()}]{semester}{plan}	
				\begin{umlcall}[op={OAuthSync()}]{plan}{sync}   
					\begin{umlcall}[op={getAccessToken()}, return={AccessToken}]{sync}{sesInfo}   
					\end{umlcall} 
					\begin{umlcall}[op={editPlan()}, dt=5]{sync}{planRes}
					\end{umlcall} 					 
				\end{umlcall}
			\end{umlcall}
		\end{umlcall}	
	\end{umlseqdiag}
	\end{tikzpicture}
}
	\caption{Modul im Plan einfügen: Informationen speichern(Clientseitig)}
	\label{seq:addmodul_2}
\end{figure}

\begin{figure}[H]
	\resizebox{\textwidth}{!} {
	\begin{tikzpicture}  
	\begin{umlseqdiag}
		\umlobject[no ddots]{Sync} 
 		\umlobject[class=planResource, x=2.5]{planRes}
		\umlobject[class=SessionInformation, x=6.8]{sesInfo} 
		\umlobject[class=AuthorizationContext, x=11.5]{auth}
		\umlobject[class=User, x=14.9]{user}
		\umlobject[class=Plan, x=16.8]{plan}
		\umlobject[class=List<ModulEntry>, x=20.6]{modules}
		\begin{umlcall}[op={editPlan()}, dt=5]{Sync}{planRes}
			\begin{umlcall}[op={getAccessToken()}, return={AccessToken}]{planRes}{sesInfo}   
			\end{umlcall} 
			\begin{umlcall}[op={getUser()}, return={user}, dt=5]{planRes}{auth}   
			\end{umlcall}
			\begin{umlcall}[op={getPlans().get()},  return={plan}, dt=5]{planRes}{user}   
			\end{umlcall} 
			\begin{umlcall}[op={getModulEntries()},  return={modules}, dt=5]{planRes}{plan}   
			\end{umlcall} 
			\begin{umlcall}[op={add()}]{planRes}{modules}   
			\end{umlcall} 
		\end{umlcall}
	\end{umlseqdiag}
	\end{tikzpicture}
}
	\caption{Modul im Plan einfügen: Informationen speichern(Serverseitig)}
	\label{seq:addmodul_3}
\end{figure}

