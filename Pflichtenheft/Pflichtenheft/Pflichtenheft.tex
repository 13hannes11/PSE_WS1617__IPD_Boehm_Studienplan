\documentclass[titlepage=true, parskip=full]{scrartcl}
\usepackage[utf8]{inputenc} % use utf8 file encoding for TeX sources
\usepackage[T1]{fontenc}    % avoid garbled Unicode text in pdf
\usepackage[german]{babel}  % german hyphenation, quotes, etc
\usepackage{hyperref}       % detailed hyperlink/pdf configuration
\hypersetup{                % ‘texdoc hyperref‘ for options
	pdftitle={PSE: Pflichtenheft},%
	bookmarks=true,%
}
\usepackage{graphicx}       % provides commands for including figures
\usepackage{csquotes}       % provides \enquote{} macro for "quotes"
\usepackage[nonumberlist]{glossaries}     % provides glossary commands
\usepackage{enumitem}

\makeglossaries

\title{Studienplanung als Generierung von Workflows mit Compliance-Anforderungen: Planerstellung und Visualisierung}
\subtitle{Pflichtenheft}
\author{Daniel Jungkind \and Ulrike Rheinheimer \and Hannes Kuchelmeister \and Samuel Teuber \and Nada Chatti \and Tim Niklas Uhl}
\date{30. November 2016}


\begin{document}

\maketitle
\tableofcontents
\pagebreak

%
% % Hier beginnt die Gliederung des Pflichtenhefts
\section{Einleitung}

\section{Zielbestimmung}

\subsection{Musskriterien}

\subsection{Wunschkriterien}

\subsection{Abgrenzungskriterien}

\section{Produkteinsatz}

\subsection{Anwendungsbereiche}

\subsection{Zielgruppen}

\subsection{Betriebsbedingungen}

\section{Produktumgebung}

\subsection{Software}

\subsection{Hardware}

\subsection{Produkt-Schnittstellen}

\section{Funktionale Anforderungen}
\begin{itemize}[nosep]
	\item[FA10]
\end{itemize}

\section{Produktdaten}
\begin{itemize}[nosep]
	\item[PD10]
\end{itemize}

\section{Nichtfunktionale Anforderungen}
\begin{itemize}[nosep]
\item[NF10]
\end{itemize}

\section{Globale Testfälle}
\section{Systemmodelle}

\subsection{Szenarien}

\subsection{Anwendungsfälle}

\subsection{Objektmodell}

\subsection{Dynamische Modelle}

\subsection{Benutzerschnittstelle}




%
% % Automatisch generiertes Glossar
%
%\glsaddall % das sorgt dafür, dass alles Glossareinträge gedruckt werden, nicht nur die verwendeten. Das sollte nicht nötig sein!
%\printglossaries
%
% % Glossareinträge
%
\newglossaryentry{Plug-In}
{
	name=Plug-In,
	plural=Plug-Ins,
	description={Erweiterung für \enquote{\gls{iMage}}, die zusätzliche Funktionalitäten bietet}
}

\newglossaryentry{Kunde}
{
	name=Kunde,
	plural=Kunden,
	description={Person, die den \gls{Online-Shop} besucht}
}

\newglossaryentry{Plug-In-Paket}
{
	name=Plug-In-Paket,
	plural=Plug-In-Pakete,
	description={Paket bestehend aus mehreren \glspl{Plug-In}}
}

\newglossaryentry{iMage}
{
	name={iMage},
	description={Bildbearbeitungssoftware der Firma SWT. Bietet im Basispaket nur Funktionalitäten zur Skalierung und Drehung von Bildern}
}

\newglossaryentry{Internetbrowser}
{
	name={Internetbrowser},
	description={Programm, mit dem Websites gefunden, gelesen und verwaltet werden können}
}

\newglossaryentry{Online-Shop}
{
	name={Online-Shop},
	description={Internetseite, die Produkte zum Kauf anbietet}
}


\end{document}
