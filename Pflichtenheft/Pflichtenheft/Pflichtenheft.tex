\documentclass[titlepage=true, parskip=full]{scrartcl}
\usepackage[utf8]{inputenc} % use utf8 file encoding for TeX sources
\usepackage[T1]{fontenc}    % avoid garbled Unicode text in pdf
\usepackage{palatino}	      % because "Computer Modern" standard font is illegible
\usepackage[ngerman]{babel}  % german hyphenation, quotes, etc
\usepackage{hyperref}       % detailed hyperlink/pdf configuration
\hypersetup{                % ‘texdoc hyperref‘ for options
	pdftitle={PSE: Pflichtenheft},%
	bookmarks=true,%
}
\usepackage{graphicx}       % provides commands for including figures
\usepackage{csquotes}       % provides \enquote{} macro for "quotes"
\usepackage[nonumberlist, numberedsection]{glossaries}     % provides glossary commands
\usepackage{enumitem}
\usepackage{tikz}
\usepackage{tikz-uml}
\usepackage{comment}

\makeglossaries

\title{Studienplanung als Generierung von Workflows mit Compliance-Anforderungen: Planerstellung und Visualisierung}
\subtitle{Pflichtenheft}
\author{Daniel Jungkind \and Ulrike Rheinheimer \and Hannes Kuchelmeister \and Samuel Teuber \and Nada Chatti \and Tim Niklas Uhl}
\date{30. November 2016}

\begin{document}

\maketitle
\tableofcontents
\pagebreak

%
% % Hier beginnt die Gliederung des Pflichtenhefts
\section{Einleitung}
\paragraph{Aufgabenstellung}
Ziel der Praxis der Software-Entwicklung (PSE) ist die Entwicklung eines mittelgroßen Systems im Team mit objektorientierter Softwaretechnik. Dieses soll circa 10~kLOC umfassen. Hierzu wird ein objektorientierter Entwurf auf Basis von UML-Diagrammen geschaffen und unter Nutzung von Java, JavaScript, HTML und SQL implementiert. Im Anschluss soll mit JCov oder JUnit die Qualitätssicherung durchgeführt werden.\\
Das Projekt \enquote{Studienplanung als Generierung von Workflows mit Compliance"=Anforderungen: Planerstellung und Visualisierung} umfasst die Entwicklung einer webbasierten Benutzeroberfläche eines ebenfalls zu entwickelnden Systems zur Studienplanung. Sinn dieses Systems ist das Aufstellen von Studienplänen, angepasst an Bedürfnisse, bereits erbrachte Leistungen und Zeitmöglichkeiten des Studenten. Dies soll sowohl manuell, als auch automatisch möglich sein. Die Algorithmen zur Generierung und Verifizierung von Workflows unter Berücksichtigen von Constraints sind in Form von Black Boxes gegeben und können auf die Studienpläne angewendet werden. Das gesamte System soll modular und gut erweiterbar sein.\\
\paragraph{Im Detail heißt das:}
Die graphische Oberfläche des Systems soll intuitiv bedienbar und benutzerzentriert gestaltet werden. Als Benutzer sind Studenten zu erwarten. Eine zusätzliche Dozentenoberfläche soll modular hinzugefügt werden können. Das System soll dem Nutzer auf dem bisherigen Studienverlauf basierend Vorschläge in Form von Studienplänen zur Planung der nächsten Semester liefern. Der Studienplan soll als Ablauf(Workflow) aufgefasst werden, so dass die Black-Boxes zur Generierung und Verifizierung von Workflows eingebunden werden können. Die gegebenen Daten müssen hierzu passend umgewandelt und erhaltene Ergebnisse für den Nutzer verständlich verwertet werden. Als Aktivitäten der Workflows werden die Module und andere Lehrveranstaltungen aufgefasst. Module werden mit ihrem Namen, ECTS-Punkten, Angebot im Winter- oder Sommersemester und Art der Veranstaltung modelliert. Die Module sollen zu einem sinnvollen Workflow zusammengefügt werden. Die Workflows werden von Constraints beeinflusst. Dies sind: die Unterscheidung zwischen Pflicht- und Wahlveranstaltungen, Wahl eines Vertiefungsfaches, Abhängigkeiten zwischen Modulen (Voraussetzungen, Zusammenhänge, Überschneidungen), zur Verfügung stehende Semesteranzahl, gewünschte Module, bisheriger Studienverlauf, weitere gewünschte Eigenschaften.\\
Auch graphisch sollen die Studienpläne als Workflows dargestellt werden. 
\section{Zielbestimmung}

\subsection{Musskriterien}
\begin{itemize}[nosep]
	\item Webbasierte Benutzeroberfläche:
	\begin{itemize}[nosep]
		\item Benutzerorientiert 
		\item Modular erweiterbar
		\item Nutzer: Studenten
	\end{itemize}
	\item Funktionen:
	\begin{itemize}[nosep]
		\item Studienpläne manuell bearbeiten ( Module einfügen / löschen) ?
		\item Generierung $\rightarrow$ Erstellung von möglichen Studienplänen
		\item Berücksichtigung von Constraints:
		\begin{itemize}[nosep]
			\item Unterscheidung zwischen Pflicht- und Wahlveranstaltungen
			\item Wahl eines Vertiefungsfaches
			\item Abhängigkeiten zwischen Modulen (Voraussetzungen, Zusammenhänge, Überschneidungen)
			\item zur Verfügung stehende Semesteranzahl
			\item gewünschte Module
			\item bisheriger Studienverlauf
		\end{itemize}
		\item Verifizierung von Studienplänen
		\begin{itemize}[nosep]
			\item Ermöglicht dieser Studienplan ein erfolgreiches Abschließen des Studiums?
			\item Sind alle Constraints erfüllt?
		\end{itemize}
		\item Berücksichtigung des bisherigen Studienverlaufs
	\end{itemize}
	\item Gegebene Algorithmen zur Generierung und Verifizierung einbinden
	\item Modularität
	\item Objektorientierung
	\item Studienplan als Workflow modellieren
	\begin{itemize}[nosep]
		\item Module/Lehrveranstaltungen als Aktivitäten
		\item Graphische Darstellung als Workflow
	\end{itemize}
	\item Module modellieren mit folgenden Details: Name, ECTS-Punkte, Angebot im Sommer- oder Wintersemester, Art der Veranstaltung
\end{itemize}

\subsection{Wunschkriterien}
\begin{itemize}[nosep]
	\item Mögliches Login über Shibboleth
	\begin{itemize}[nosep]
		\item Speichern von: Studiengang, Studienbeginn, bestandene und begonnene Prüfungsleistungen, bereits erstellte Studienpläne
	\end{itemize}
	\item Speichern und löschen von Studienplänen
	\item Benennung von Studienplänen
	\item Duplizieren von Studienplänen
	\item Studienpläne exportieren
	\item Vergleichsansicht für 2 (mehrere?) Pläne 
	\item Modulübersicht
	\begin{itemize}[nosep]
		\item Alle Module in einer Liste
		\begin{itemize}[nosep]
			\item Durchstöbern können
			\item Suche nach Namen
			\item Anklicken für Details
			\item In Stundenplan ziehen können
		\end{itemize}
		\item Filterbar mit Filtern: 
		\begin{itemize}[nosep]
			\item Angebotenes Semester
			\item Veranstaltungsart
			\item Fachrichtung
			\item Pflicht-/Wahlmodul
			\item Kategorie?
		\end{itemize}
	\end{itemize}
	\item Rückgängig-Button
	\item Weitere Constraints
	\begin{itemize}[nosep]
		\item benötigte ECTS-Punkte
		\item ausgeschlossene Module
	\end{itemize}
\end{itemize}
\subsection{Abgrenzungskriterien}
\begin{itemize}[nosep]
	\item kein Notenportal
	\item keine Vernetzung zwischen Studenten
	\item keine Vernetzung zum Prüfungsportal
	\item keine Unterstützung von parallelen Studiengängen
\end{itemize}
\section{Produkteinsatz}

\subsection{Anwendungsbereiche}

\subsection{Zielgruppen}

\subsection{Betriebsbedingungen}

\section{Produktumgebung}

\subsection{Software}

\subsection{Hardware}

\subsection{Produkt-Schnittstellen}

\section{Funktionale Anforderungen}
\begin{itemize}[nosep]
	\item[FA10]
\end{itemize}

\section{Produktdaten}
\begin{itemize}[nosep]
	\item[PD10]
\end{itemize}

\section{Nichtfunktionale Anforderungen}
\begin{itemize}[nosep]
\item[NF10]
\end{itemize}

\section{Globale Testfälle}
\section{Systemmodelle}

\subsection{Szenarien}

\subsection{Anwendungsfälle}

\begin{center}
\resizebox{\textwidth}{!} {
	\begin{tikzpicture}
	\begin{umlsystem}{Studienplan-Verifizierung}
		\umlusecase[x=-3]{Studienplan anlegen}
		\umlusecase[x=3, y=-1]{Modul hinzufügen}
		\umlusecase[x=3, y=1]{Modul suchen}
		\umlusecase[x=-3, y=-2]{Studienplan verifizieren}
		\umlusecase[x=-3, y=-4]{Konlifkte anzeigen}
	\end{umlsystem}

	\umlactor[x=-8]{Nutzer}

	\umlassoc{Nutzer}{usecase-1}
	\umlassoc{Nutzer}{usecase-3}
	\umlassoc{Nutzer}{usecase-4}
	\umlassoc{Nutzer}{usecase-5}

	\umlinclude{usecase-1}{usecase-2}
	\umlinclude{usecase-2}{usecase-3}
\end{tikzpicture}
}
\end{center}

\subsection{Objektmodell}

\subsection{Benutzerschnittstelle}

%
% % Automatisch generiertes Glossar
%
%\glsaddall % das sorgt dafür, dass alles Glossareinträge gedruckt werden, nicht nur die verwendeten. Das sollte nicht nötig sein!
%
% % Glossareinträge
%
\newglossaryentry{Rest}
{
	name=REST,
	description={Abk. für Representational State Transfer, Programmierparadigma für \glspl{Webservice} auf Basis des HTTP-Protokolls}
}

\newglossaryentry{ECTS-Punkte}
{
	name=ECTS-Punkte,
	description={Leistungspunkte, die für ein erfolgreich absolviertes \gls{Modul} von der Hochschule auf Basis des ECTS-Punktesystems vergibt werden, und mit denen der Arbeitsaufwand gemessen wird},
}

\newglossaryentry{Generierungs-Tool}
{
	name=Generierungs-Tool,
	description={Tool, für die automatische Erstellung bzw. Vervollständigung von Studienpläne},
}


\newglossaryentry{Webservice}
{
	name=Webservice,
	plural=Webservices,
	description={Softwareanwendung, die über ein Netzwerk bereitgestellt wird}
}

\newglossaryentry{Plug-In-Paket}
{
	name=Plug-In-Paket,
	plural=Plug-In-Pakete,
	description={Paket bestehend aus mehreren \glspl{Plug-In}}
}

\newglossaryentry{iMage}
{
	name={iMage},
	description={Bildbearbeitungssoftware der Firma SWT. Bietet im Basispaket nur Funktionalitäten zur Skalierung und Drehung von Bildern}
}

\newglossaryentry{Internetbrowser}
{
	name={Internetbrowser},
	description={Programm, mit dem Websites gefunden, gelesen und verwaltet werden können, mit aktiviertem JavaScript}
}

\newglossaryentry{Online-Shop}
{
	name={Online-Shop},
	description={Internetseite, die Produkte zum Kauf anbietet}
}

\newglossaryentry{KIT}
{
	name=KIT,
	description={Das Karlsruher Institut für Technologie ist die Forschungsuniversität in der Helmholtz-Gemeinschaft. Standort der Universität ist Karlsruhe. }
}

\newglossaryentry{SCC}
{
	name=SCC,
	plural=SCC,
	description={Das Steinbuch Center for Computing ist ein Institut und das zentrale Rechenzentrum des \gls{KIT}s.}
}

\newglossaryentry{Benutzer}
{
	name=Benutzer,
	plural=Benutzer,
	description={Ein am \gls{KIT} eingeschriebener Student, der über ein gültigen Account beim \gls{SCC} verfügt.}
}

\newglossaryentry{Wizard}
{
	name=Wizard,
	plural=Wizards,
	description={Ein Wizard ist ein Subsystem, welches einen \gls{Benutzer} visuell durch eine Systemfunktionalität führt und dabei vom \gls{Benutzer} bestimmte Interaktionen mit dem System fordert.}
}

\newglossaryentry{Drag-and-Drop}
{
	name=Drag-and-Drop,
	description={(deutsch: "Ziehen und Ablegen") Eine Methode zur Bedienung grafischer Benutzeroberflächen bei der grafische Elemente mittels eines Mauszeigers bewegt werden.}
}

\newglossaryentry{Shibboleth Identity Provider}
{
	name=Shibboleth Identity Provider,
	description={Ein genau spezifiziertes System zum Login mittels einer von einer dritten Instanz bereitgestellten Identität.}
}

\newglossaryentry{Studiengang}
{
	name=Studiengang,
	description={Ein vom KIT angebotener, auf einer Studien- und Prüfungsordnung und einem Modulhandbuch basierender Studiengang.}
}

\newglossaryentry{Semester des Studienbeginns}
{
	name=Semester des Studienbeginns,
	description={Das Semester in welchem der \gls{Benutzer} im ersten Fachsemester des \gls{Studiengang}s war.}
}

\newglossaryentry{Modul}
{
	name=Modul,
	plural=Module,
	description={Ein Modul ist ein Teilblock des Studiums, welcher aus verschiedenartigen Veranstaltungen (genannt \glspl{Teilmodul}) bestehen kann und für welchen man nach Ablegung eventueller \glspl{Modulpruefung} eine festgelegte Anzahl an ECTS-Punkten erhält.}
}
\newglossaryentry{Teilmodul}
{
	name=Teilmodul,
	plural=Teilmodule,
	description={Ein Teilmodul ist eine universitäre Veranstaltung, welche als Teil eines Moduls besucht werden kann und mittels einer (wie auch immer gearteten) \gls{Modulpruefung} bestanden werden kann.}
}
\newglossaryentry{Modulpruefung}
{
	name=Modulprüfung,
	plural=Modulprüfungen,
	description={Eine Modulprüfung ist eine Prüfung, welche abgelegt werden muss um ein Modul \glslink{Modul abgeschlossen}{abzuschließen}.}
}
\newglossaryentry{Zur Pruefung angetreten}
{
	name=Zur Pruefung angetreten,
	description={Ein \gls{Benutzer} ist zu einer \gls{Modulpruefung} angetreten, wenn er sich fristgerecht für selbige angemeldet und nicht fristgerecht abgemeldet hat.}
}

\newglossaryentry{Modul abgeschlossen}
{
	name=Modul abgeschlossen,
	description={Ein \gls{Modul} gilt als abgeschlossen, wenn der \gls{Benutzer} alle nach Modulhandbuch notwendigen \glspl{Modulpruefung} bestanden hat.}
}

\newglossaryentry{Modul begonnen}
{
	name=Modul begonnen,
	description={Ein \gls{Modul} gilt als begonnen, wenn der \gls{Benutzer} zu mindestens einer \gls{Modulpruefung} \glslink{Zur Pruefung angetreten}{angetreten} ist.}
}

\newglossaryentry{Studienplan}
{
	name=Studienplan,
	plural=Studienpläne,
	description={Eine Zusammenstellung von Modulen, in welcher enthalten ist wann welches Modul planmäßig \glslink{Modul begonnen}{begonnen} werden soll.}
}

\end{document}
