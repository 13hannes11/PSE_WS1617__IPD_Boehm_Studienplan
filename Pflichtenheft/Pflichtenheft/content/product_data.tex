\section{Produktdaten}

\subsection{Systemdaten}
	
\begin{itemize}[nosep]
	\item[PD10]Anleitung zur Benutzung der Website
	\item[PD20] \gls{Modul}daten: Für jedes \gls{Modul} sind folgende Daten zu speichern	
	\begin{itemize}
		\item Modulname
		\item Dozent
		\item Beschreibung
		\item \gls{ECTS-Punkte}
		\item Veranstaltungsart
		\item Abhängigkeiten zu anderen \glspl{Modul}n
		\item Wintersemester bzw. Sommersemester
		\item betroffene Fachrichtungen 
		\item Pflicht-/Wahlbestandteile
	\end{itemize}	 
\end{itemize}

\subsection{Benutzerdaten}
\label{subsec:product_data-benutzerdaten}  % needed for cross referencing

\begin{itemize}[nosep]
	\item[PD30]Profildaten: Daten, die den aktuellen Zustand des Studiums des Benutzers beschreiben. Dazu gehört: 
	\begin{itemize}
		\item Studienfach
		\item Studienbeginn
		\item Begonnene \glspl{Modul}
		\item Bestandene \glspl{Modul} 
		\item Summe der gesammelten \gls{ECTS-Punkte}
	\end{itemize}
	\item[PD40] Studienpläne:
	\begin{itemize}
		\item 	vom Benutzer erstellte Studienpläne 
		\item	vom \gls{Generierungs-Tool} vorgeschlagene Studienpläne 
	\end{itemize} 
	\item[PD50] Präferenzen: Module, die vom Benutzer positiv/negativ bewertet wurden
\end{itemize}