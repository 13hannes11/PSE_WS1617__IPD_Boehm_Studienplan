\section{Zielbestimmung}

\subsection{Musskriterien}
\begin{itemize}[nosep]
	\item Webbasierte Benutzeroberfläche:
	\begin{itemize}[nosep]
		\item Benutzerorientiert 
		\item Modular erweiterbar
		\item Nutzer: Studenten
	\end{itemize}
	\item Funktionen:
	\begin{itemize}[nosep]
		\item Studienpläne manuell bearbeiten ( Module einfügen / löschen) ?
		\item Generierung $\rightarrow$ Erstellung von möglichen Studienplänen
		\item Berücksichtigung von Constraints:
		\begin{itemize}[nosep]
			\item Unterscheidung zwischen Pflicht- und Wahlveranstaltungen
			\item Wahl eines Vertiefungsfaches
			\item Abhängigkeiten zwischen Modulen (Voraussetzungen, Zusammenhänge, Überschneidungen)
			\item zur Verfügung stehende Semesteranzahl
			\item gewünschte Module
			\item bisheriger Studienverlauf
		\end{itemize}
		\item Verifizierung von Studienplänen
		\begin{itemize}[nosep]
			\item Ermöglicht dieser Studienplan ein erfolgreiches Abschließen des Studiums?
			\item Sind alle Constraints erfüllt?
		\end{itemize}
		\item Berücksichtigung des bisherigen Studienverlaufs
	\end{itemize}
	\item Gegebene Algorithmen zur Generierung und Verifizierung einbinden
	\item Modularität
	\item Objektorientierung
	\item Studienplan als Workflow modellieren
	\begin{itemize}[nosep]
		\item Module/Lehrveranstaltungen als Aktivitäten
		\item Graphische Darstellung als Workflow
	\end{itemize}
	\item Module modellieren mit folgenden Details: Name, ECTS-Punkte, Angebot im Sommer- oder Wintersemester, Art der Veranstaltung
\end{itemize}

\subsection{Wunschkriterien}
\begin{itemize}[nosep]
	\item Mögliches Login über Shibboleth
	\begin{itemize}[nosep]
		\item Speichern von: Studiengang, Studienbeginn, bestandene und begonnene Prüfungsleistungen, bereits erstellte Studienpläne
	\end{itemize}
	\item Speichern und löschen von Studienplänen
	\item Benennung von Studienplänen
	\item Duplizieren von Studienplänen
	\item Studienpläne exportieren
	\item Vergleichsansicht für 2 (mehrere?) Pläne 
	\item Modulübersicht
	\begin{itemize}[nosep]
		\item Alle Module in einer Liste
		\begin{itemize}[nosep]
			\item Durchstöbern können
			\item Suche nach Namen
			\item Anklicken für Details
			\item In Stundenplan ziehen können
		\end{itemize}
		\item Filterbar mit Filtern: 
		\begin{itemize}[nosep]
			\item Angebotenes Semester
			\item Veranstaltungsart
			\item Fachrichtung
			\item Pflicht-/Wahlmodul
			\item Kategorie?
		\end{itemize}
	\end{itemize}
	\item Rückgängig-Button
	\item Weitere Constraints
	\begin{itemize}[nosep]
		\item benötigte ECTS-Punkte
		\item ausgeschlossene Module
	\end{itemize}
\end{itemize}
\subsection{Abgrenzungskriterien}
\begin{itemize}[nosep]
	\item kein Notenportal
	\item keine Vernetzung zwischen Studenten
	\item keine Vernetzung zum Prüfungsportal
	\item keine Unterstützung von parallelen Studiengängen
\end{itemize}