\section{Globale Testfälle}

\renewcommand{\arraystretch}{1.24}  % for proper row spacing
\newenvironment{usecase}[1]{
	\subsubsection*{#1} 
	\hspace{1pt}\vspace{-2\baselineskip}  % ugly orphan-heading fix
	\begin{longtable}{|>{\hspace{0pt}}p{.18\linewidth}|p{.34\linewidth}|p{.44\linewidth}|}   % >{\hspace{0pt}}  necessary to allow hyphenation of the first word of the column's paragraph. (YTF isn't this standard behaviour?!)
		\hline
		\textbf{Stand} & \textbf{Aktion} & \textbf{Reaktion} \\
		\hline 
		\endfirsthead
		
		\hline
		\textbf{Stand} & \textbf{Aktion} & \textbf{Reaktion} \\
		\endhead
		
		\hline
		\endlastfoot
}{
	\end{longtable}
}





\subsection{Testszenarien}






\subsection{Anwendungsfälle}

\begin{center}
	\resizebox{\textwidth}{!} {
		\begin{tikzpicture}
	\begin{umlsystem}{Studienplan-Verifizierung}
		\umlusecase[x=-3]{Studienplan anlegen}
		\umlusecase[x=3, y=-1]{Modul hinzufügen}
		\umlusecase[x=3, y=1]{Modul suchen}
		\umlusecase[x=-3, y=-2]{Studienplan verifizieren}
		\umlusecase[x=-3, y=-4]{Konlifkte anzeigen}
	\end{umlsystem}

	\umlactor[x=-8]{Nutzer}

	\umlassoc{Nutzer}{usecase-1}
	\umlassoc{Nutzer}{usecase-3}
	\umlassoc{Nutzer}{usecase-4}
	\umlassoc{Nutzer}{usecase-5}

	\umlinclude{usecase-1}{usecase-2}
	\umlinclude{usecase-2}{usecase-3}
\end{tikzpicture}
	}
\end{center}



\begin{usecase}{A10: Erstanmeldung}
	Geöffnete Seite, unangemeldet
	& Nutzer loggt sich zum ersten Mal via Shibboleth ein
	& Dem Nutzer erscheint eine Willkommensseite; mitsamt der Eingabe"=Formulare wie in /A40/ beschrieben.\\ 
	\hline
	& Nutzer füllt die Formulare aus.
	& Dem Nutzer erscheint die Hauptseite, auf der bislang keine Studienpläne vorhanden sind. 	
\end{usecase}

\begin{usecase}{A20: Login}
	Geöffnete Seite, unangemeldet
	& Nutzer loggt sich via Shibboleth ein
	& Dem Nutzer erscheint die Hauptseite mit ggfs. bereits angelegten Studienplänen
\end{usecase}

\begin{usecase}{A30: Logout}
	Geöffnete Seite, angemeldet, beliebige Ansicht
	& Nutzer klickt auf den Logout"=Knopf.
	& Dem Nutzer erscheint die Login-Seite, er wird bis zur nächsten Anmeldung nicht mehr als eingeloggt erkannt.
\end{usecase}

\begin{usecase}{A40: Profil bearbeiten}
	Geöffnete Seite, angemeldet, beliebige Ansicht
	& Nutzer klickt auf den Profil"=Knopf.
	& Dem Nutzer erscheint ein Formular zur Eingabe von Studienbeginn und Studiengang. \\ 
	\hline
	& Nutzer gibt diese Informationen ein und drückt auf den Weiter"=Knopf.
	& Dem Nutzer erscheint ein Formular zur Eingabe des aktuellen Studienstands: Er kann festlegen, welche Prüfungsleistungen er schon begonnen und in welchem Semester er sie bestanden hat. \\
	\hline
	& Nutzer kann nun Module als begonnen markieren, indem er sie in den Studienplan ins jeweilige Semester hineinzieht (/A140/), und sie als bestanden markieren, indem er den entsprechenden Knopf im jeweiligen Modul anwählt. Modulfilterung (/A110/) ist in dieser Ansicht möglich.
	Nach Eingabe dieser Informationen drückt er den Weiter"=Knopf.
	& Dem Nutzer erscheint die Ansicht, in welcher er den Profil"~Knopf betätigt hat, oder im Fall /A10/ die Hauptansicht.
\end{usecase}

\begin{usecase}{A50: Neuen Studienplan anlegen}
	Hauptansicht
	& Nutzer klickt auf den Knopf „Neuen Studienplan erstellen“.
	& Dem Nutzer erscheint ein Popup, welches ihn nach dem Studienplan-Namen fragt; voreingestellt ist „Neuer~Studienplan~1“. \\
	\hline
	& Nutzer gibt gewünschten Namen ein und bestätigt.
	& Der neue Studienplan öffnet sich in der Bearbeitungsansicht und wird zur Liste der bereits erstellten Pläne hinzugefügt.
\end{usecase}
	
\begin{usecase}{A60: Studienplan umbenennen}
	Bearbeitungsansicht mit offenem Studienplan
	& Nutzer klickt auf den Namen des Studienplans
	& Dem Nutzer erscheint ein Popup, welches ihn nach dem neuen Studienplan"=Namen fragt; voreingestellt ist der alte Name. \\
	\hline
	& Nutzer gibt gewünschten Namen ein und bestätigt.
	& Der Nutzer kehrt zur Bearbeitungsansicht zurück, der Name des Studienplans hat sich geändert.
\end{usecase}

\begin{usecase}{A70: Studienplan duplizieren}
	Hauptansicht, es exist. mind. ein Studienplan
	& Nutzer klickt rechts eines Studienplans „\$Name“ auf „Duplizieren“.
	& Ein neuer Studienplan namens „\$Name – Kopie~\#n“ taucht unter dem Ersten auf (\#n beschreibt die kleinste Zahl $\ge 1$, die keine Namenskollisionen hervorruft).
\end{usecase}

\begin{usecase}{A80: Studienplan löschen}
	Hauptansicht, es exist. mind. ein Studienplan
	& Nutzer klickt rechts eines Studienplans auf „Löschen“.
	& Nutzer wird mittels Dialog gebeten, das Löschen des Studienplans zu bestätigen. \\
	\hline
	& Nutzer entscheidet sich für Bestätigung oder Abbruch.
	& Der Dialog verschwindet, dem Nutzer erscheint die Hauptansicht. Falls er das Löschen bestätigt hat, existiert der genannte Studienplan nun nicht mehr.
\end{usecase}

\begin{usecase}{A90: Mehrere Studienpläne duplizieren/löschen}
	Hauptansicht, es exist. mind. ein Studienplan
	& Nutzer wählt einen oder mehrere Studienpläne mittels der Anwahlkästchen aus (oder auch alle durch Wählen des obersten Hakens in der Leiste).
	Danach wählt der Nutzer im Aktions"=Wahlfeld „Duplizieren“ oder „Löschen“.
	& Es folgt das Vorgehen wie in /A70/ bzw. /A80/, zusammengefasst angewandt auf die markierten Studienpläne.
\end{usecase}

\begin{usecase}{A100: Vergleichsansicht für Studienpläne}
	Hauptansicht, es exist. mind. zwei Studienpläne
	& Nutzer wählt genau zwei Studienpläne mittels Anwahlkästchen aus. Im Aktions"=Wahlfeld betätigt er die Vergleichsansicht.
	& Dem Nutzer erscheint die Vergleichsansicht. \\
	\hline
	& Nutzer schließt die Vergleichsansicht. 
	& Der Nutzer kehrt zur Hauptansicht zurück.
\end{usecase}

% TODO Fix ugly orphan-heading (workaround fails unfortunately)
\begin{usecase}{A110: Module in der Suchleiste filtern}
	Bearbeitungsansicht mit offenem Studienplan, Suchleiste wird angezeigt
	& Der Nutzer kann in der Suchleiste Module filtern durch Wählen...
	\begin{itemize}[nosep,leftmargin=12pt]
		\item des ECTS"=Intervalls (Klick auf „ECTS“ und Ziehen an den Reglern)
		\item der Veranstaltungsart $\in$ \{Vorlesung, Praktikum, Seminar\} (Klick auf „Art“ und Auswahl)
		\item der Kategorie (bzw. des Themenbereiches) (Klick auf „Kategorie“ und Auswahl)
		\item des Turnus $\in$ \{WS, SS, WS/SS\} (Klick auf „WS/SS“ und Auswahl)
		\item ob Pflicht-, Wahlveranstaltungen oder beides anzuzeigen ist (Klick auf „Pflicht/Wahl“ und Auswahl)
		\item der Fachrichtung (Klick auf „Fachrichtung“ und Auswahl)
		\item ob bereits platzierte Module anzuzeigen sind (Klick auf „mit Platzierten?“ und Auswahl)
		\item eines Suchbegriffes, nach welchem die Titel der Module gefiltert werden (Eingabe von Text ins Suchfeld)
	\end{itemize}\hspace{0pt}\vspace{-\baselineskip}
	& In der Suchleiste werden entspr. der Nutzerfilterung alle Module angezeigt, die...
	\begin{itemize}[nosep,leftmargin=12pt]
		\item im gewählten ECTS"=Intervall liegen
		\item der gewählten Veranstaltungsart entsprechen
		\item zur gewählten Kategorie gehören
		\item im gewählten Turnnus stattfinden
		\item Pflicht-, Wahlveranstaltungen oder beides sind
		\item zur gewählten Fachrichtung gehören
		\item bereits platziert worden sind oder nicht
		\item den gewählten Suchbegriff im Titel enthalten
	\end{itemize}\hspace{0pt}\vspace{-\baselineskip} \\
	\hline
	& Der Nutzer kann gesetzte Filter durch Klicken auf das Kreuzchen im Filter"-Knopf wieder zurücksetzen.
	& Entsprechende Filter treten außer Kraft und die Suchleiste aktualisiert sich wie oben.
\end{usecase}

\begin{usecase}{A120: Info-Popup anzeigen}
	
\end{usecase}

\begin{usecase}{A130: Info-Leiste anzeigen}
	
\end{usecase}

\begin{usecase}{A140: Modul in Plan einfügen}
	
\end{usecase}

\begin{usecase}{A150: Multimodul in Plan einfügen}
	
\end{usecase}

\begin{usecase}{A160: Modul aus Plan löschen}
	
\end{usecase}

\begin{usecase}{A170: Modul innerhalb Plan verschieben}
	
\end{usecase}

\begin{usecase}{A180: Modul liken}
	
\end{usecase}

\begin{usecase}{A190: Modul disliken}
	
\end{usecase}

\begin{usecase}{A200: Semester im Plan hinzufügen}
	
\end{usecase}

\begin{usecase}{A210: Semester aus Plan löschen}
	
\end{usecase}

\begin{usecase}{A215: Abgeschlossene Semester im Plan ausblenden}
	
\end{usecase}

\begin{usecase}{A220: Studienplan auf Korrektheit überprüfen}
	
\end{usecase}

\begin{usecase}{A230: Studienplan vervollständigen lassen}
	
\end{usecase}


	

% reset
\renewcommand{\arraystretch}{1.0}