\section{Globale Testfälle}

%%%%%%%%%%%%%%%%%%%%%%%%%%%%%%%%%%%%%%%%%%%%%%%%%
\renewcommand{\arraystretch}{1.24}  % for proper row spacing
\setlength{\LTpre}{3pt}  % adjust top margin of longtables
\setlength{\LTpost}{0pt} % adjust bottom margin of longtables

% scenario environment
% #1: Scenario title
\newenvironment{scenario}[1]{
	\vspace{-\baselineskip}
	\subsubsection{#1} 
	\vspace{-\baselineskip}
	\begin{enumerate}[nosep]
}{
	\end{enumerate}
}

% scenario* environment
% #1: Scenario title
% #2: Initial description
\newenvironment{scenario*}[2]{
	\vspace{-\baselineskip}
	\subsubsection{#1} \vspace{-\baselineskip}
	\hspace{0pt}#2  \vspace{-\baselineskip}
	\begin{enumerate}[nosep]
}{
	\end{enumerate}
}


% usecase environment
% #1: Title of the use case, like "\lA{n}: Title"
% #2: Initial state description
\newenvironment{usecase}[2]{
	\subsubsection*{#1}
	%\addcontentsline{toc}{subsubsection}{#1} 
	\vspace{-\baselineskip}\textbf{Ausgangs-Stand: } #2
	\begin{longtable}{|p{.44\linewidth}|p{.55\linewidth}|}
		\hhline{|=|=|}
		\textbf{Aktion} & \textbf{Reaktion} \\
		\hline 
		\endfirsthead
		
		\hline
		\textbf{Aktion} & \textbf{Reaktion} \\
		\endhead
		
		\hhline{|=|=|}
		\endlastfoot
}{
	\end{longtable} \vspace{-12pt}
}

% tblitemize environment
% Provides itemize optimized for table cells
\newenvironment{tblitemize}{
	\begin{itemize}[nosep,leftmargin=12pt]
}{
	\end{itemize}\hspace{0pt}\vspace{-\baselineskip}
}

% lA command (label A...)
% declares a usecase label
% #1: number of the A-abbrev. (without leading A)
\newcommand{\lA}[1]{\label{A#1}A#1}

% refA command (refer to A...)
% creates a linked reference to the given usecase
% #1: number of the A-abbrev. (without leading A)
\newcommand{\refA}[1]{\hyperref[A#1]{/A#1/}}

\newcommand{\opt}{*} 

\newcommand{\case}[1]{\textit{#1}}

%%%%%%%%%%%%%%%%%%%%%%%%%%%%%%%%%%%%%%%%%%%%%%%%%

Im Folgenden werden Testszenarien und Anwendungsfälle beschrieben, die das Systemverhalten benutzungsorientiert beschreiben.

Eine Stern"=Markierung (\opt) weist in diesem Kapitel auf Kann"=Anforderungen hin, deren spätere Umsetzung nicht garantiert wird.

\subsection{Testszenarien}
Folgende Funktionssequenzen müssen überprüft werden:

\begin{scenario}{Erststart mit „halbherziger Bedienung“}
	\item \nameref{A10} (ohne Angabe bereits bestandener \glspl{Modul})
	\item \nameref{A50}
	\item \nameref{A220} – Ergebnis „fehlerhaft“ (da unvollständig)
	\item \nameref{A230} mit anschließendem Verwerfen
	\item \nameref{A57}
	\item \nameref{A80} 
	\item \nameref{A30}
\end{scenario}

\begin{scenario}{Einfache Vervollständigung}
	\item \nameref{A20}
	\item \nameref{A40} – erste zwei Semester anschließend belegt
	\item \nameref{A50}
	\item \nameref{A230} mit anschließendem Übernehmen des \glslink{Studienplan}{Studienplans}
	\item \nameref{A60} 
	\item \nameref{A57}
	\item \nameref{A85}
	\item \nameref{A30}
\end{scenario}

\begin{scenario*}{Bearbeitung eines Studienplans}
	{\glslink{Benutzer}{Nutzer} ist bereits eingeloggt und hat mind. einen \gls{Studienplan} angelegt.}
	\item \nameref{A55}
	\item \nameref{A110} 
	\item \nameref{A130} und wieder schließen
	\item \nameref{A140}
	\item \nameref{A150}
	\item \nameref{A215} (danach: ausgeblendet)
	\item \nameref{A160}
	\item \nameref{A190}
	\item \nameref{A180} (selbes \gls{Modul} – es ist dann positiv bewertet) 
	\item \nameref{A140}
	\item \nameref{A170}
	\item \nameref{A57}
\end{scenario*}

\begin{scenario*}{Profil bearbeiten}
	{\glslink{Benutzer}{Nutzer} ist bereits eingeloggt und hat mind. einen \gls{Studienplan} angelegt.}
	\item \nameref{A55}
	\item \nameref{A215} (danach: eingeblendet)
	\item \nameref{A110}
	\item \nameref{A40} – dabei Änderung der Semester-Belegung
	\item Anschließend sollte die Änderung im Studienplan gezeigt werden und die Suchleiste sich im selben Zustand befinden wie vor Schritt 4.
\end{scenario*}

\begin{scenario}{Vervollständigung mit mehreren Alternativen}
	\item \nameref{A20}
	\item \nameref{A50}
	\item \nameref{A110}
	\item \nameref{A140}
	\item Schritte 3–4 mehrmals wiederholen, sodass Abhängigkeitsfehler vorhanden sind und der \gls{Studienplan} noch unvollständig ist
	\item \nameref{A220}: Es werden Abhängigkeitsfehler gemeldet
	\item Mittels \enquote{\nameref{A150}} und \enquote{\nameref{A160}} Abhängigkeitsfehler beheben
	\item \nameref{A220}: Der Studienplan ist unvollständig
	\item \nameref{A230} mit anschließendem Speichern unter neuem Namen
	\item \nameref{A57}
	\item \nameref{A55} (den in Schritt 2 erstellten Studienplan)
	\item \nameref{A230} mit anderen Zielkriterien als in Schritt 9 und anschließendem Speichern unter neuem Namen
	\item \nameref{A57}
	\item \nameref{A100} (die zwei \glslink{Generierung}{generierten} Studienpläne)
	\item \nameref{A57}
	\item \nameref{A90}: Teilen eines Studienplans
\end{scenario}

\begin{scenario}{Studienpläne duplizieren und löschen}
	\item \nameref{A20}
	\item \nameref{A50}
	\item \nameref{A57}
	\item \nameref{A70}
	\item \nameref{A90}: Duplizieren aller vorhandenen \glslink{Studienplan}{Studienpläne}
	\item Mehrmaliges Wiederholen von Schritt 5.
	\item \nameref{A90}: Alle Studienpläne löschen.
\end{scenario}

\begin{scenario}{Semester"=Zeilen anpassen}
	\item \nameref{A20}
	\item \nameref{A50}
	\item Mehrmals \nameref{A210}, bis keine mehr vorhanden sind.
	\item Mehrmals \nameref{A200}
	\item \nameref{A57}
\end{scenario}


%\needspace{1.1\textwidth}
\subsection{Anwendungsfälle}

Die folgenden Anwendungsfalldiagramme bieten einen Überblick über die im Anschluss beschriebenen Anwendungsfälle.

\begin{center}
	\resizebox{\textwidth}{!} {
		\begin{tikzpicture}
	\begin{umlsystem}{Profil- und Studienplan-Verwaltung}
		\umlusecase[x=-3, y=0]{\nameref{A10}}
		\umlusecase[x=2.5, y=-1.5]{\nameref{A20}}
		\umlusecase[x=6, y=-1.5]{\nameref{A30}}
		\umlusecase[x=4.5, y=0, width=0.2\linewidth]{\nameref{A40}}
		\umlusecase[x=-2.5, y=-1.5, width=.2\linewidth]{\nameref{A50}}
		\umlusecase[x=-2.5, y=-3.4, width=0.3\linewidth]{\nameref{A55} /
			\hyperref[A57]{A57: Studienplan-Ansicht schließen}}
		\umlusecase[x=-2.3, y=-5.2]{\nameref{A60}}
		\umlusecase[x=4.5, y=-5.5, width=0.20\linewidth]{\nameref{A70}}
		\umlusecase[x=4.2, y=-7.5, width=0.20\linewidth]{\nameref{A80}}
		\umlusecase[x=-2, y=-8.5]{\nameref{A85}}
		\umlusecase[x=-3, y=-6.85, width=0.25\linewidth]{\nameref{A90}}
		\umlusecase[x=4, y=-3.5, width=0.20\linewidth]{\nameref{A100}}
	\end{umlsystem}	
	
	\umlactor[x=-8.5, y=-5]{Nutzer}
	
	\umlassoc{Nutzer}{usecase-1}
	\umlassoc{Nutzer}{usecase-2}
	\umlassoc{Nutzer}{usecase-3}
	\umlassoc{Nutzer}{usecase-4}
	\umlassoc{Nutzer}{usecase-5}
	\umlassoc{Nutzer}{usecase-6}
	\umlassoc{Nutzer}{usecase-7}
	\umlassoc{Nutzer}{usecase-8}
	\umlassoc{Nutzer}{usecase-9}
	\umlassoc{Nutzer}{usecase-10}
	\umlassoc{Nutzer}{usecase-11}
	\umlassoc{Nutzer}{usecase-12}
	
	\umlinclude{usecase-1}{usecase-4}
	\umlinclude{usecase-11}{usecase-8}
	\umlinclude{usecase-11}{usecase-9}
\end{tikzpicture}

\begin{comment}
	\begin{umlsystem}{Profil- und Studienplanverwaltung}
		\umlusecase[x=-3]{Studienplan anlegen}
		\umlusecase[x=3, y=-1]{Modul hinzufügen}
		\umlusecase[x=3, y=1]{Modul suchen}
		\umlusecase[x=-3, y=-2]{Studienplan verifizieren}
		\umlusecase[x=-3, y=-4]{Konlifkte anzeigen}
	\end{umlsystem}
	
	\umlactor[x=-8]{Nutzer}
	
	\umlassoc{Nutzer}{usecase-1}
	\umlassoc{Nutzer}{usecase-3}
	\umlassoc{Nutzer}{usecase-4}
	\umlassoc{Nutzer}{usecase-5}
	
	\umlinclude{usecase-1}{usecase-2}
	\umlinclude{usecase-2}{usecase-3}
\end{comment}
	}
\end{center}

\begin{center}	
	\resizebox{\textwidth}{!} {
		\begin{tikzpicture}
\begin{umlsystem}{Studienplan-Bearbeitung}
\umlusecase[x=-3, y=-9, width=0.3\linewidth]{\nameref{A110}}
\umlusecase[x=5.5, y=-8, width=0.2\linewidth]{\nameref{A130}}
\umlusecase[x=-3, y=-1, width=0.3\linewidth]{\nameref{A140}}
\umlusecase[x=4.5, y=-1, width=0.3\linewidth]{\nameref{A150}}
\umlusecase[x=-3, y=-2.8, width=0.3\linewidth]{\nameref{A160}}
\umlusecase[x=4.5, y=-2.8, width=0.3\linewidth]{\nameref{A170}}
\umlusecase[x=-3.3, y=-7.3, width=0.3\linewidth]{Module \hyperref[A180]{positiv (A180)} / \hyperref[A190]{negativ (A190)} bewerten}
\umlusecase[x=-3, y=1, width=0.3\linewidth]{\nameref{A200} / \hyperref[A210]{A210: löschen}}
\umlusecase[x=5, y=1, width=0.25\linewidth]{\nameref{A215}}
\umlusecase[x=-3, y=-5, width=0.2\linewidth]{\nameref{A220}}
\umlusecase[x=5.5, y=-5, width=0.2\linewidth]{\nameref{A230}}
\end{umlsystem}	

\umlactor[x=-8.5, y=-4]{Nutzer}

\umlassoc{Nutzer}{usecase-13}
\umlassoc{Nutzer}{usecase-14}
\umlassoc{Nutzer}{usecase-15}
\umlassoc{Nutzer}{usecase-16}
\umlassoc{Nutzer}{usecase-17}
\umlassoc{Nutzer}{usecase-18}
\umlassoc{Nutzer}{usecase-19}
\umlassoc{Nutzer}{usecase-20}
\umlassoc{Nutzer}{usecase-21}
\umlassoc{Nutzer}{usecase-22}
\umlassoc{Nutzer}{usecase-23}


\umlinclude{usecase-14}{usecase-19}
\umlinclude{usecase-13}{usecase-14}
\umlinclude{usecase-23}{usecase-14}
\umlinclude{usecase-23}{usecase-19}

\begin{comment}
	includes:
	130 –> 180/190, 
	110 -> 130
	230 -> 180/190, 130
	
\end{comment}

\end{tikzpicture}
	}
\end{center}


\subsubsection{Profil- und Studienplan-Verwaltung}

\begin{usecase}{\lA{10}: Erstanmeldung}
	{Geöffnete Seite, unangemeldet}
	\glslink{Benutzer}{Nutzer} loggt sich zum ersten Mal via \gls{Shibboleth Identity Provider}\opt ein
	& Dem \glslink{Benutzer}{Nutzer} wird eine Willkommensseite angezeigt; mitsamt der Eingabe"=Formulare wie in \refA{40} beschrieben.\\ 
	\hline
	\glslink{Benutzer}{Nutzer} füllt die Formulare aus.
	& Dem \glslink{Benutzer}{Nutzer} wird die Hauptansicht angezeigt, auf der bislang keine \glslink{Studienplan}{Studienpläne} vorhanden sind. 	
\end{usecase}

\begin{usecase}{\lA{20}: Login}
	{Geöffnete Seite, unangemeldet}
	\glslink{Benutzer}{Nutzer} loggt sich via \gls{Shibboleth Identity Provider}\opt ein
	& Dem \glslink{Benutzer}{Nutzer} wird die Hauptansicht mit ggf. bereits angelegten \glslink{Studienplan}{Studienplänen} angezeigt.
\end{usecase}

\begin{usecase}{\lA{30}: Logout}
	{Geöffnete Seite, angemeldet, beliebige Ansicht}
	\glslink{Benutzer}{Nutzer} klickt auf den Logout"=Button.
	& Dem \glslink{Benutzer}{Nutzer} wird die Login-Seite angezeigt, er wird bis zur nächsten Anmeldung nicht mehr als eingeloggt erkannt.
\end{usecase}

\begin{usecase}{\lA{40}: Profil bearbeiten}
	{Geöffnete Seite, angemeldet, beliebige Ansicht außer Erstanmeldung (\refA{10})}
	\glslink{Benutzer}{Nutzer} klickt auf den Profil"=Button.
	& Dem \glslink{Benutzer}{Nutzer} wird ein Formular zur Eingabe von \glslink{Semester des Studienbeginns}{Studienbeginn} und \gls{Studiengang} angezeigt. \\ 
	\hline
	\glslink{Benutzer}{Nutzer} gibt diese Informationen ein und drückt auf den Weiter"=Button.
	& Dem \glslink{Benutzer}{Nutzer} wird ein Formular zur Eingabe des aktuellen Studienstands angezeigt: Er kann festlegen, welche Prüfungsleistungen er in welchem Semester \glslink{Modul abgeschlossen}{bestanden} hat. Als Startwert sind so viele Semester"=Zeilen vorgegeben, wie der \glslink{Benutzer}{Nutzer} laut vorheriger Angabe bereits abgeschlossen hat.\\
	\hline
	Bei Bedarf markiert der \glslink{Benutzer}{Nutzer} \glspl{Modul} als bestanden, indem er sie in den \gls{Studienplan} ins jeweilige Semester hineinzieht (\refA{140}). Modulfilterung (\refA{110}) ist in dieser Ansicht möglich.
	Nach Eingabe dieser Informationen drückt er den Fertig"=Button.
	& Dem \glslink{Benutzer}{Nutzer} wird die Ansicht angezeigt, in welcher er den Profil"=Button betätigt hat, oder im Fall \refA{10} die Hauptansicht. Sofern er seine \glslink{Modul abgeschlossen}{bestandenen Prüfungsleistungen} verändert hat, gelten alle bereits vorhandenen Studienpläne als „nicht \glslink{Verifizierung}{überprüft}“.
\end{usecase}

\begin{usecase}{\lA{50}: Neuen Studienplan anlegen}
	{Hauptansicht}
	\glslink{Benutzer}{Nutzer} klickt auf den Button „Neuen Studienplan erstellen“.
	& Dem \glslink{Benutzer}{Nutzer} wird ein \gls{Popup} angezeigt, welches ihn nach dem \glslink{Studienplan}{Studienplan"=Namen} fragt; voreingestellt ist „Neuer~Studienplan~1“. \\
	\hline
	\glslink{Benutzer}{Nutzer} gibt gewünschten Namen ein und bestätigt.
	& Der neue Studienplan öffnet sich in der Bearbeitungsansicht und wird zur Liste der bereits erstellten Studienpläne hinzugefügt. Er gilt als bislang nicht \glslink{Verifizierung}{überprüft} und enthält bereits die in der Profilansicht hinzugefügten bereits belegten \glspl{Modul}.
\end{usecase}

\begin{usecase}{\lA{55}: Studienplan anzeigen}
	{Hauptansicht, es exist. mind. ein \gls{Studienplan}}
	\glslink{Benutzer}{Nutzer} klickt auf den Namen eines Studienplans oder rechts davon auf „Anzeigen“.
	& Der gewählte Studienplan öffnet sich in der Bearbeitungsansicht.
\end{usecase}

\begin{usecase}{\lA{57}: Schließen der Studienplan-Ansicht mit Wechsel zur Hauptansicht}
	{Bearbeitungsansicht/Vergleichsansicht}
	\glslink{Benutzer}{Nutzer} betätigt den Schalter zur Hauptansicht.
	& Die Hauptansicht wird angezeigt. Falls zuvor ein \gls{Studienplan} in der Bearbeitungsansicht geändert wurde, trägt dieser nun ggf. auch einen neuen Namen (\refA{60}) und einen neuen \glslink{Verifizierung}{Überprüfungsstatus} (\refA{220}).
\end{usecase}
	
\begin{usecase}{\lA{60}: Studienplan umbenennen\opt}
	{Bearbeitungsansicht mit offenem \gls{Studienplan}}
	\glslink{Benutzer}{Nutzer} klickt auf den Namen des Studienplans
	& Die Namensanzeige wird zu einem Eingabefeld, welches ihn nach dem neuen Studienplan"=Namen fragt; voreingestellt ist der alte Name. \\
	\hline
	\glslink{Benutzer}{Nutzer} gibt gewünschten Namen ein und bestätigt mit Enter.
	& Das Eingabefeld wird wieder zur herkömmlichen Namensanzeige, wobei der Name des Studienplans sich geändert hat.
\end{usecase}

\begin{usecase}{\lA{70}: Studienplan duplizieren*}
	{Hauptansicht, es exist. mind. ein \gls{Studienplan}}
	\glslink{Benutzer}{Nutzer} klickt neben einem Studienplan „$\langle \textit{Name} \rangle$“ auf „Duplizieren“.
	& Eine Kopie des Studienplans namens „$\langle \textit{Name} \rangle$ – Kopie~$\langle n \rangle$“ taucht in der Studienplanliste auf.
\end{usecase}

\begin{minipage}{\linewidth}\vspace{.2\baselineskip} % To f***ingly avoid widows/orphans (Hell, LaTeX, WTF is wrong with you!?)
	\textit{Hinweis:} $\langle \textit{Name} \rangle$ bezeichnet den Namen des gewählten \glslink{Studienplan}{Studienplans}, $\langle n \rangle$ die kleinste Zahl $\ge 1$, die keine Namenskollisionen hervorruft.
\end{minipage}

\begin{usecase}{\lA{80}: Studienplan löschen}
	{Hauptansicht, es exist. mind. ein \gls{Studienplan}}
	\glslink{Benutzer}{Nutzer} klickt neben einem Studienplan auf „Löschen“.
	& \glslink{Benutzer}{Nutzer} wird mittels Dialog gebeten, das Löschen des Studienplans zu bestätigen. \\
	\hline
	\glslink{Benutzer}{Nutzer} entscheidet sich für Bestätigung oder Abbruch.
	& \case{Bestätigung}: Der Dialog verschwindet, dem \glslink{Benutzer}{Nutzer} wird die Hauptansicht angezeigt. Der genannte Studienplan existiert nun nicht mehr. \newline
	\case{Abbruch}: Der Dialog verschwindet, dem \glslink{Benutzer}{Nutzer} wird die unveränderte Hauptansicht angezeigt. 
\end{usecase}

\begin{usecase}{\lA{85}: Studienplan exportieren\opt}
	{Hauptansicht, es exist. mind. ein \gls{Studienplan}}
	\glslink{Benutzer}{Nutzer} klickt neben einem Studienplan auf „Exportieren“.
	& Das System \glslink{Generierung}{generiert} eine PDF"=Zusammenfassung des Studienplans, welche dem \glslink{Benutzer}{Nutzer} vom Browser zum Download angeboten wird.
\end{usecase}

\begin{usecase}{\lA{90}: Mehrere Studienpläne duplizieren\opt/löschen/teilen\opt}
	{Hauptansicht, es exist. mind. ein \gls{Studienplan}}
	\glslink{Benutzer}{Nutzer} wählt einen oder mehrere Studienpläne mittels der Anwahlkästchen aus (oder auch alle durch Wählen des obersten Hakens in der Leiste).
	Danach wählt der \glslink{Benutzer}{Nutzer} im Aktions"=Wahlfeld „Duplizieren“\opt, „Löschen“ oder „Teilen“\opt.
	& \case{„Duplizieren“/„Löschen“}: Es folgt das Vorgehen wie in \refA{70} bzw. \refA{80}, zusammengefasst angewandt auf die markierten Studienpläne. \newline
	\case{„Teilen“}: Dem \glslink{Benutzer}{Nutzer} wird ein \gls{Popup} angezeigt, in welchem URLs zum Teilen der gewählten Studienpläne kopierfähig aufgelistet werden. Die derart geteilten Studienpläne sind nur schreibgeschützt von außen einsehbar. \\
	\hline
	\case{„Teilen“}: Der \glslink{Benutzer}{Nutzer} kopiert sich bei Bedarf URLs und schließt das \gls{Popup}.
	& Die Hauptansicht wird unverändert angezeigt.
\end{usecase}

\begin{usecase}{\lA{100}: Vergleichsansicht für Studienpläne\opt}
	{Hauptansicht, es exist. mind. zwei \glslink{Studienplan}{Studienpläne}}
	\glslink{Benutzer}{Nutzer} wählt genau zwei Studienpläne mittels Anwahlkästchen aus. Im Aktions"=Wahlfeld betätigt er die Vergleichsansicht.
	& Dem \glslink{Benutzer}{Nutzer} wird die Vergleichsansicht angezeigt. \\
	\hline
	\glslink{Benutzer}{Nutzer} schließt die Vergleichsansicht. 
	& Dem \glslink{Benutzer}{Nutzer} wird die Hauptansicht angezeigt.
\end{usecase}

\subsubsection{Studienplan-Bearbeitung}

\begin{usecase}{\lA{110}: Module in der Suchleiste filtern}
	{Bearbeitungsansicht, \gls{Studienplan} geöffnet, Suchleiste wird angezeigt}
	Der \glslink{Benutzer}{Nutzer} filtert bei Bedarf in der Suchleiste die aufgelisteten \glspl{Modul} durch Wählen...
	\begin{tblitemize}
		\item des \glslink{ECTS-Punkte}{ECTS"=Intervalls}\opt{}  (Klick auf „ECTS“ und Ziehen an den Reglern)
		\item der Veranstaltungsart\opt{} $\in\hspace{-3pt}\{$ Vorlesung, Praktikum, Seminar $\}$ (Klick auf „Art“ und Auswahl)
		\item der Kategorie\opt{} (bzw. des Themenbereiches) (Klick auf „Kategorie“ und Auswahl)
		\item des Turnus\opt{} $\in \{$ WS, SS, WS/SS $\}$ (Klick auf „WS/SS“ und Auswahl)
		\item ob Pflicht-, Wahlveranstaltungen oder beides anzuzeigen ist\opt{} (Klick auf „Pflicht/Wahl“ und Auswahl)
		\item der Fachrichtung\opt{} (Klick auf „Fachrichtung“ und Auswahl)
		\item ob bereits platzierte Module anzuzeigen sind\opt{} (Klick auf „mit Platzierten?“ und Auswahl)
		\item eines Suchbegriffes, nach welchem die Titel der Module gefiltert werden (Eingabe von Text ins Suchfeld)
	\end{tblitemize}
	& In der Suchleiste werden entspr. der \glslink{Benutzer}{Nutzerfilterung} alle Module angezeigt, die...
	\begin{tblitemize}
		\item im gewählten \glslink{ECTS-Punkte}{ECTS"=Intervall} liegen\opt
		\item der gewählten Veranstaltungsart entsprechen\opt
		\item zur gewählten Kategorie gehören\opt
		\item im gewählten Turnus stattfinden\opt
		\item Pflicht-, Wahlveranstaltungen oder beides sind\opt
		\item zur gewählten Fachrichtung gehören\opt
		\item bereits platziert worden sind oder nicht\opt
		\item den gewählten Suchbegriff im Titel enthalten
	\end{tblitemize} \\
	\hline
	Der \glslink{Benutzer}{Nutzer} setzt bei Bedarf gesetzte Filter durch Klicken auf das Kreuzchen im Filter"=Button wieder zurück.
	& Entsprechende Filter treten außer Kraft und die Suchleiste aktualisiert sich wie oben. \\
	\hline
	Der \glslink{Benutzer}{Nutzer} klickt bei Bedarf auf ein in der Suchleiste aufgelistetes Modul.
	& Reaktion wie in \refA{130}.
\end{usecase}

\begin{usecase}{\lA{130}: Info-Leiste zu einem Modul anzeigen\opt}
	{Bearbeitungsansicht, \gls{Studienplan} geöffnet, mind. ein \gls{Modul} in der Suchleiste aufgelistet bzw. im Studienplan verteilt}
	\glslink{Benutzer}{Nutzer} klickt auf ein Modul in der Suchleiste/im Studienplan.
	& Das Modul wird im Studienplan farblich hervorgehoben und die Suchleiste verwandelt sich in eine Info-Leiste, welche dem \glslink{Benutzer}{Nutzer} folgende Informationen anzeigt:
	\begin{tblitemize}
		\item Titel und Modulnummer
		\item Dozent, \glslink{ECTS-Punkte}{ECTS}, Modulbeschreibung, evtl. Turnus, Dauer (in Semestern)
		\item Buttons zum positiven (\refA{180}) und negativen Bewerten(\refA{190})  des Moduls
	\end{tblitemize} \\
	\hline
	Bei Bedarf bewertet der \glslink{Benutzer}{Nutzer} das Modul positiv/negativ (\refA{180} bzw. \refA{190}).
	& Reaktion wie in \refA{180} bzw. \refA{190}. \\
	\hline
	\glslink{Benutzer}{Nutzer} klickt in der Info-Leiste auf den Zurück-Button.
	& Die Leiste kehrt zur Ausgangs"=Suchansicht zurück.
\end{usecase}

\begin{usecase}{\lA{140}: Modul in Studienplan einfügen}
	{Bearbeitungsansicht, \gls{Studienplan} geöffnet, mind. eine nicht bereits abgeschlossene Semester"=Zeile vorhanden, mind. ein unplatziertes \gls{Modul} in der Suchleiste aufgelistet}
	Der \glslink{Benutzer}{Nutzer} hält seine linke Maustaste über einem Modul in der Suchleiste gedrückt, zieht es in eine Semester"=Zeile, die nicht bereits im Profil als abgeschlossenes Semester befüllt wurde, und lässt die Maustaste wieder los.
	& Das Modul wird in der Ziel"=Zeile eingefügt und gilt als platziert, Gesamt"= und Zeilen"=\glslink{ECTS-Punkte}{ECTS} erhöhen sich. Der \glslink{Verifizierung}{Überprüfungsstatus} des Studienplans ändert sich zu „nicht \glslink{Verifizierung}{überprüft}“.
\end{usecase}

\begin{usecase}{\lA{150}: Modul aus Studienplan löschen}
	{Bearbeitungsansicht, \gls{Studienplan} geöffnet, mind. ein nicht bereits \glslink{Modul abgeschlossen}{abgeschlossenes Modul} im Studienplan verteilt}
	Der \glslink{Benutzer}{Nutzer} klickt in einem nicht abgeschlossenen Modul in der Tabelle auf den Löschen-Button.
	& Das Modul verschwindet und gilt als nicht platziert, die \glslink{ECTS-Punkte}{ECTS} der entsprechenden Semester"=Zeile und die \glslink{ECTS-Punkte}{Gesamt"=ECTS} verringern sich. Der \glslink{Verifizierung}{Überprüfungsstatus} des Studienplans ändert sich zu „nicht \glslink{Verifizierung}{überprüft}“.
\end{usecase}

\begin{usecase}{\lA{160}: Modul innerhalb Studienplan verschieben}
	{Bearbeitungsansicht, \gls{Studienplan} geöffnet, mind. ein \gls{Modul} im Studienplan verteilt, mind. zwei Semester"=Zeilen in der Tabelle}
	Der \glslink{Benutzer}{Nutzer} hält seine linke Maustaste über einem Modul in der Tabelle gedrückt, zieht es in eine Semester"=Zeile ungleich der vorherigen und lässt die Maustaste wieder los.
	& Das Modul verschiebt sich dorthin, die \glslink{ECTS-Punkte}{Semester-ECTS} der Ausgangs"= und der Ziel"=Zeile ändern sich entsprechend und der \glslink{Verifizierung}{Überprüfungsstatus} des Studienplans ändert sich zu „nicht \glslink{Verifizierung}{überprüft}“.
\end{usecase}

\begin{usecase}{\lA{170}: Studienplanänderung rückgängig machen\opt}
	{Bearbeitungsansicht, \gls{Studienplan} geöffnet}
	Der \glslink{Benutzer}{Nutzer} klickt auf den „Rückgängig“"=Button.
	& Die letzte Einfüge-, Lösch- oder Verschiebe-Operation (\refA{140}, \refA{150}, \refA{160}) wird – falls existent – rückgängig gemacht, d.h., der Status vor dem Durchführen der Operation wird wiederhergestellt.
\end{usecase}

\begin{usecase}{\lA{180}: Modul positiv bewerten}
	{Bearbeitungsansicht, \gls{Studienplan} geöffnet, mind. ein \gls{Modul} in der Suchleiste aufgelistet}
	\glslink{Benutzer}{Nutzer} klickt auf das positive Bewertungs"=Symbol eines der in der Suchleiste aufgeführten Module.
	& Bewertung des Moduls... \newline
	\case{positiv?} $\Rightarrow$ Das Modul ist nicht mehr positiv (also neutral) bewertet. Das positive Bewertungs"=Symbol erscheint inaktiv. \newline
	\case{neutral/negativ?} $\Rightarrow$ Das Modul ist positiv bewertet. Das positive Bewertungs"=Symbol erscheint aktiv.
\end{usecase}

\begin{usecase}{\lA{190}: Modul negativ bewerten}
	{Bearbeitungsansicht, \gls{Studienplan} geöffnet, mind. ein \gls{Modul} in der Suchleiste aufgelistet}
	\glslink{Benutzer}{Nutzer} klickt auf das negative Bewertungs"=Symbol eines der in der Suchleiste aufgeführten Module.
	& Bewertung des Moduls... \newline
	\case{negativ?} $\Rightarrow$ Das Modul ist nicht mehr negativ (also neutral) bewertet. Das negative Bewertungs"=Symbol erscheint inaktiv. \newline
	\case{neutral/positiv?} $\Rightarrow$ Das Modul ist negativ bewertet. Das negative Bewertungs"=Symbol erscheint aktiv.
\end{usecase}

\begin{usecase}{\lA{200}: Semester im Studienplan hinzufügen}
	{Bearbeitungsansicht, \gls{Studienplan} geöffnet}
	Der \glslink{Benutzer}{Nutzer} klickt in der Semester"=Leiste auf „Weiteres Semester hinzufügen“.
	& In der Tabelle erscheint unten eine neue leere Semester"=Zeile mit 0~\glslink{ECTS-Punkte}{ECTS}.
\end{usecase}

\begin{usecase}{\lA{210}: Semester aus Studienplan löschen}
	{Bearbeitungsansicht, \gls{Studienplan} geöffnet, mind. eine nicht abgeschlossene Semester"=Zeile in der Tabelle vorhanden}
	Der \glslink{Benutzer}{Nutzer} klickt in einer nicht abgeschlossenen Semester"=Zeile auf „Semester löschen“.
	& Falls die \case{Zeile nicht leer} ist, wird der \glslink{Benutzer}{Nutzer} mittels \gls{Popup} gebeten, das Löschen der Zeile zu bestätigen. \newline 
	Falls die \case{Zeile leer} ist, entfällt das \gls{Popup} und es erfolgt sofort die nächste Reaktion.\\
	\hline
	Der \glslink{Benutzer}{Nutzer} entscheidet sich für Bestätigung oder Abbruch.
	& Der \glslink{Benutzer}{Nutzer} kehrt zur vorherigen Bearbeitungsansicht zurück. \newline 
	\case{Bestätigung}: Die Semester"=Zeile verschwindet; dadurch gelten alle darin enthaltenen \glspl{Modul} nicht mehr als platziert und der \glslink{Verifizierung}{Überprüfungsstatus} des Studienplans ändert sich zu „nicht \glslink{Verifizierung}{überprüft}“; ferner aktualisieren sich die \glslink{ECTS-Punkte}{Gesamt-ECTS}. \newline
	\case{Abbruch}: Die Bearbeitungsansicht bleibt unverändert.
\end{usecase}

\begin{usecase}{\lA{215}: Abgeschlossene Semester im Studienplan ein"=/ausblenden}
	{Bearbeitungsansicht, \gls{Studienplan} geöffnet}
	\glslink{Benutzer}{Nutzer} klickt auf „Abgeschlossene Semester ein-/ausblenden“.
	& Die Zeilen bereits abgeschlossener Semester werden (wieder) ein"=/ausgeblendet.
\end{usecase}

\begin{usecase}{\lA{220}: Studienplan auf Korrektheit überprüfen}
	{Bearbeitungsansicht, \gls{Studienplan} geöffnet mit \glslink{Verifizierung}{Überprüfungsstatus} „nicht \glslink{Verifizierung}{überprüft}“}
	\glslink{Benutzer}{Nutzer} klickt auf „Überprüfen“.
	& Zur Überbrückung der Wartezeit wird ein Ladekreis angezeigt. \newline 
	Nach Abschluss der \glslink{Verifizierung}{Überprüfung} erhält der Studienplan den Status „korrekt“ oder „fehlerhaft“, was dem \glslink{Benutzer}{Nutzer} auch durch eine \gls{Benachrichtigung} am Seitenrand gemeldet wird. \glspl{Modul}, die Konflikte hervorrufen, werden mit einer roten „Fehler“"=Markierung gekennzeichnet. \\
	\hline
	\glslink{Benutzer}{Nutzer} fährt bei Bedarf mit der Maus über fehlerhafte Module.
	& Daraufhin wird ein \gls{Tooltip} angezeigt, das den jeweiligen Konflikt erklärt. \\
	\hline
	Der \glslink{Benutzer}{Nutzer} nimmt bei Bedarf Änderungen am Studienplan vor (\refA{140} bis \refA{170}). 
	& Daraufhin verschwinden die „Fehler“"=Markierungen und der Studienplan erhält den Status „nicht \glslink{Verifizierung}{überprüft}“.
\end{usecase}

\begin{usecase}{\lA{230}: Studienplan vervollständigen lassen}
	{Bearbeitungsansicht, \gls{Studienplan} geöffnet, es können bereits \glspl{Modul} im Studienplan verteilt sein (s. \refA{140} und \refA{150}); Module können Präferenzen haben (\refA{180} und \refA{190})}
	\glslink{Benutzer}{Nutzer} klickt auf den Button „Plan vervollständigen“.
	& Dem \glslink{Benutzer}{Nutzer} wird das Vervollständigungs"=Formular angezeigt, in welchem folgende Daten abgefragt werden:
	\begin{tblitemize}
		\item Zieleigenschaft des vervollständigten \glslink{Studienplan}{Studienplans}:
		\begin{tblitemize}
			\item \glslink{ECTS-Punkte}{ECTS"=Minimum}
			\item Gewünschte Vertiefungsrichtung/positiv bewertete Module
			\item Möglichst schneller Studienabschluss
			\item Möglichst gleichmäßig über alle Semester verteilte \glslink{ECTS-Punkte}{ECTS}
		\end{tblitemize}
		\item Maximale \glslink{ECTS-Punkte}{ECTS-Zahl} pro Semester
		\item Minimale/Maximale Semesteranzahl
		\item Angabe der gewünschten Vertiefungsrichtungen
		\item Präferenzen für Module (positive oder negative Bewertungen)
	\end{tblitemize} \\
	\hline
	\glslink{Benutzer}{Nutzer} gibt geforderte Daten ein und bestätigt.
	& Das System \glslink{Generierung}{generiert} – sofern möglich – einen vollständigen, den Kriterien des \glslink{Benutzer}{Nutzers} und des zugrundeliegenden Datensatzes entsprechenden \gls{Studienplan}. Dabei werden auch Modulpräferenzen, bereits belegte/bestandene sowie eingeplante Module berücksichtigt. \newline
	\case{Im Erfolgsfall} wird dieser Studienplan dem \glslink{Benutzer}{Nutzer} mitsamt einer entsprechenden \gls{Benachrichtigung} am Seitenrand angezeigt. Der vervollständigte Studienplan hat den Status „korrekt“. Dem \glslink{Benutzer}{Nutzer} wird in der Seitenleiste angeboten, den vorgeschlagenen Studienplan...
	\begin{tblitemize}
		\item zu verwerfen,
		\item zu übernehmen,
		\item oder unter neuem Namen zu speichern.
	\end{tblitemize} \newline
	\case{Im Fehlerfall} wird eine entspr. \gls{Benachrichtigung} am Seitenrand angezeigt. Der Ausgangsplan wird angezeigt und hat den Status „fehlerhaft“; der \glslink{Benutzer}{Nutzer} wird wie in \refA{220} über „Fehler“"=Markierungen auf Konflikte hingewiesen. \\
	\hline
	Im Erfolgsfall sieht der \glslink{Benutzer}{Nutzer} nun den vervollständigten \gls{Studienplan}. Bei Bedarf klickt er auf ein Modul im Studienplan.
	& Die entspr. Info"=Leiste (\refA{130}*) öffnet sich. \\
	\hline
	Er entscheidet sich anschließend zwischen den drei genannten Optionen. 
	& \case{„Verwerfen“}: Der \glslink{Benutzer}{Nutzer} wird gebeten, das Verwerfen des Vorschlags zu bestätigen. \newline
	\case{„Übernehmen“}: Der Vorschlag wird in den Ausgangsplan übernommen. Dem \glslink{Benutzer}{Nutzer} wird die Bearbeitungsansicht mit dem vervollständigten „korrekten“ Studienplan angezeigt. \newline
	\case{„Unter neuem Namen speichern“}:  Der \glslink{Benutzer}{Nutzer} wird via \gls{Popup} nach einem Namen für den Vorschlag gefragt. \\
	\hline
	\case{„Verwerfen“}: Der \glslink{Benutzer}{Nutzer} wählt Bestätigung oder Abbruch. \newline
	\case{„Unter neuem Namen speichern“}: Der \glslink{Benutzer}{Nutzer} gibt den Namen ein. 
	& \case{„Verwerfen“}: Im Falle der Bestätigung kehrt der \glslink{Benutzer}{Nutzer} in die ursprüngliche Bearbeitungsansicht mit dem \glslink{Studienplan}{Ausgangsplan} im alten Status zurück. Lehnt der \glslink{Benutzer}{Nutzer} dies ab, so kehrt er zur Vorschlagsansicht zurück, wo ihm wieder die drei Optionen angeboten werden. \newline
	\case{„Unter neuem Namen speichern“}: Der Vorschlag wird in einen neuen Studienplan mit angegebenem Namen gespeichert. Dem \glslink{Benutzer}{Nutzer} wird die Bearbeitungsansicht mit dem neuen, vervollständigten Studienplan angezeigt. Dieser Studienplan hat den Status „korrekt“.
\end{usecase}


\bigskip

% reset changes
\renewcommand{\arraystretch}{1.0}
\setlength{\LTpre}{\bigskipamount}
\setlength{\LTpost}{\bigskipamount}