\section{Funktionale Anforderungen}

	\subsection{Grundfunktionen}
	\begin{itemize}[nosep]	
		\item [FA10] Anzeigen eines Login-Bereichs
		\item[FA20] Login mit den KIT-Daten über den \gls{Shibboleth Identity Provider}
		\item [FA30]Eingabe (in einem Registrierungs-\gls{Wizard}) und Speicherung von
		\begin{itemize}[nosep]
			\item Studiengang
			\item \gls {Semester des Studienbeginns}
			\item bestandene Prüfungsleistungen
		\end{itemize}
		\item [FA40] Erstellen von \glslink{Studienplan}{Studienplänen}
		\item [FA41]Anzeigen einer Übersicht aller gespeicherten \glspl{Studienplan}
		\item [FA50] Generierung von \glslink{Studienplan}{Studienplänen} unter Berücksichtigung folgender \glspl{Constraint}:
			\begin{itemize}[nosep]
			\item von Pflichtveranstaltungen
			\item der Wahl eines Vertiefungsfaches
			\item von Abhängigkeiten zwischen \glspl{Modul}n
			\item der zur Verfügung stehenden Semesteranzahl
			\item gewünschter \glspl{Modul}
			\item des bisherigen Studienverlaufs
			\end{itemize}
		\item [FA60] \gls{Verifizierung} von \glslink{Studienplan}{Studienplänen}
		\item [FA70] \glspl{Studienplan} manuell bearbeiten
		\item[FA80] Bearbeitungsansicht mit einem \gls{Studienplan} und Modulübersicht anzeigen  
		\item [FA90]Speichern und Löschen von \glslink{Studienplan}{Studienplänen}	
		\item[FA110] \glspl{Modul} suchen (Volltextsuche)
		\item[FA120] \glspl{Modul} einem \gls{Studienplan} zufügen
		\item[FA125] \glspl{Modul} studienplanbezogen positiv oder negativ bewerten (berücksichtigt \gls{Generierung})
		\end{itemize}
	\subsection{Erweiterte Funktionen}
		\label{subsec:func_requirements-erweitert}
		\begin{itemize}[nosep]
		\item [FA130] \glspl{Modul} filtern
		\begin{itemize}
		\item Angebotenes Semester
		\item Veranstaltungsart
		\item Fachrichtung
		\item Pflicht-/Wahlmodul
		\item Kategorie
		\item ECTS-Bereich
		\end{itemize}
		\item[FA135] einzelne \gls{Modul} anzeigen (Detailansicht)
		\item [FA140]	Benennung von
		\glslink{Studienplan}{Studienplänen}
		\item [FA150] Duplizieren von \glslink{Studienplan}{Studienplänen}
		\item [FA160] \glspl{Studienplan} exportieren
		\item [FA170] \glspl{Studienplan} mit anderen teilen
		\item[FA180] \glspl{Studienplan} vergleichen
		\item [FA190] "Rückgängig machen" der jeweils letzten Aktivität
		\end{itemize}

