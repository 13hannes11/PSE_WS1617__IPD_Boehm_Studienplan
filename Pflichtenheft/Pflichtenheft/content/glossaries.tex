%
% % Automatisch generiertes Glossar
%
%\glsaddall % das sorgt dafür, dass alles Glossareinträge gedruckt werden, nicht nur die verwendeten. Das sollte nicht nötig sein!
\printglossaries
%
% % Glossareinträge
%
\newglossaryentry{Plug-In}
{
	name=Plug-In,
	plural=Plug-Ins,
	description={Erweiterung für \enquote{\gls{iMage}}, die zusätzliche Funktionalitäten bietet}
}

\newglossaryentry{Kunde}
{
	name=Kunde,
	plural=Kunden,
	description={Person, die den \gls{Online-Shop} besucht}
}

\newglossaryentry{Plug-In-Paket}
{
	name=Plug-In-Paket,
	plural=Plug-In-Pakete,
	description={Paket bestehend aus mehreren \glspl{Plug-In}}
}

\newglossaryentry{iMage}
{
	name={iMage},
	description={Bildbearbeitungssoftware der Firma SWT. Bietet im Basispaket nur Funktionalitäten zur Skalierung und Drehung von Bildern}
}

\newglossaryentry{Internetbrowser}
{
	name={Internetbrowser},
	description={Programm, mit dem Websites gefunden, gelesen und verwaltet werden können}
}

\newglossaryentry{Online-Shop}
{
	name={Online-Shop},
	description={Internetseite, die Produkte zum Kauf anbietet}
}