%
% % Automatisch generiertes Glossar
%
%\glsaddall % das sorgt dafür, dass alles Glossareinträge gedruckt werden, nicht nur die verwendeten. Das sollte nicht nötig sein!
%
% % Glossareinträge
%
\newglossaryentry{Rest}
{
	name=REST,
	description={Abk. für Representational State Transfer, Programmierparadigma für \glspl{Webservice} auf Basis des HTTP-Protokolls}
}

\newglossaryentry{ECTS-Punkte}
{
	name=ECTS-Punkte,
	description={Leistungspunkte, die für ein erfolgreich absolviertes \gls{Modul} von der Hochschule auf Basis des ECTS"=Punktesystems vergibt werden, und mit denen der Arbeitsaufwand gemessen wird},
}

\newglossaryentry{Generierungs-Tool}
{
	name=Generierungs-Tool,
	description={Tool, für die automatische Erstellung bzw. Vervollständigung von Studienpläne},
}


\newglossaryentry{Webservice}
{
	name=Webservice,
	plural=Webservices,
	description={Softwareanwendung, die über ein Netzwerk bereitgestellt wird}
}

\newglossaryentry{Internetbrowser}
{
	name={Internetbrowser},
	description={Programm, mit dem Websites gefunden, gelesen und verwaltet werden können, mit aktiviertem JavaScript}
}

\newglossaryentry{Online-Shop}
{
	name={Online-Shop},
	description={Internetseite, die Produkte zum Kauf anbietet}
}

\newglossaryentry{KIT}
{
	name=KIT,
	description={Das Karlsruher Institut für Technologie ist die Forschungsuniversität in der Helmholtz"=Gemeinschaft. Standort der Universität ist Karlsruhe}
}

\newglossaryentry{SCC}
{
	name=SCC,
	plural=SCC,
	description={Das Steinbuch Center for Computing ist ein Institut und das zentrale Rechenzentrum des \gls{KIT}s}
}

\newglossaryentry{Benutzer}
{
	name=Benutzer,
	plural=Benutzer,
	description={Ein am \gls{KIT} eingeschriebener Student, der über ein gültigen Account beim \gls{SCC} verfügt}
}

\newglossaryentry{Wizard}
{
	name=Wizard,
	plural=Wizards,
	description={Ein Wizard ist ein Subsystem, welches einen \gls{Benutzer} visuell durch eine Systemfunktionalität führt und dabei vom \gls{Benutzer} bestimmte Interaktionen mit dem System fordert}
}

\newglossaryentry{Drag-and-Drop}
{
	name=Drag-and-Drop,
	description={(deutsch: „Ziehen und Ablegen“) Eine Methode zur Bedienung grafischer Benutzeroberflächen bei der grafische Elemente mittels eines Mauszeigers bewegt werden}
}

\newglossaryentry{Shibboleth Identity Provider}
{
	name=Shibboleth Identity Provider,
	description={Ein genau spezifiziertes System zum Login mittels einer von einer dritten Instanz bereitgestellten Identität (in diesem Fall vom \gls{SCC})}
}

\newglossaryentry{Studiengang}
{
	name=Studiengang,
	description={Ein vom KIT angebotener, auf einer Studien"= und Prüfungsordnung und einem Modulhandbuch basierender Studiengang}
}

\newglossaryentry{Semester des Studienbeginns}
{
	name=Semester des Studienbeginns,
	description={Das Semester in welchem der \gls{Benutzer} im ersten Fachsemester des \gls{Studiengang}s war}
}

\newglossaryentry{Modul}
{
	name=Modul,
	plural=Module,
	description={Ein Modul ist ein Teilblock des Studiums, welcher aus verschiedenartigen Veranstaltungen (genannt \glspl{Teilmodul}) bestehen kann und für welchen man nach Ablegung eventueller \glspl{Modulpruefung} eine festgelegte Anzahl an ECTS-Punkten erhält}
}
\newglossaryentry{Teilmodul}
{
	name=Teilmodul,
	plural=Teilmodule,
	description={Ein Teilmodul ist eine universitäre Veranstaltung, welche als Teil eines Moduls besucht werden kann und mittels einer (wie auch immer gearteten) \gls{Modulpruefung} bestanden werden kann}
}
\newglossaryentry{Modulpruefung}
{
	name=Modulprüfung,
	plural=Modulprüfungen,
	description={Eine Modulprüfung ist eine Prüfung, welche abgelegt werden muss um ein Modul \glslink{Modul abgeschlossen}{abzuschließen}}
}
\newglossaryentry{Zur Pruefung angetreten}
{
	name=Zur Prüfung angetreten,
	description={Ein \gls{Benutzer} ist zu einer \gls{Modulpruefung} angetreten, wenn er sich fristgerecht für selbige angemeldet und nicht abgemeldet (fristgerecht) hat}
}

\newglossaryentry{Modul abgeschlossen}
{
	name=Modul abgeschlossen,
	description={Ein \gls{Modul} gilt als abgeschlossen, wenn der \gls{Benutzer} alle nach Modulhandbuch notwendigen \glspl{Modulpruefung} bestanden hat}
}

\newglossaryentry{Modul begonnen}
{
	name=Modul begonnen,
	description={Ein \gls{Modul} gilt als begonnen, wenn der \gls{Benutzer} zu mindestens einer \gls{Modulpruefung} \glslink{Zur Pruefung angetreten}{angetreten} ist}
}

\newglossaryentry{Studienplan}
{
	name=Studienplan,
	plural=Studienpläne,
	description={Eine Zusammenstellung von Modulen, in welcher enthalten ist wann welches Modul planmäßig \glslink{Modul begonnen}{begonnen} werden soll}
}
\newglossaryentry{Weboberflaeche}{
	name={Weboberfl{\"a}che},
	description={Ein HTML-Dokument, das mit einem \gls{Internetbrowser} geöffnet wird. Der \gls{Internetbrowser} stellt die Informationen des HTML-Dokuments graphisch dar}
}


\newglossaryentry{hovern}
{
	name=hovern,
	description={etwas hovern: Mit der Maus über etwas fahren}
}



\newglossaryentry{Popup}
{
	name=Popup,
	description={Eine auf der \gls{Weboberflaeche} eingeblendete Box, die vom \gls{Benutzer} Informationen abfragen kann, indem sie Knöpfe oder Eingabefelder anbietet}
}

\newglossaryentry{Notification}
{
	name=Notification,
	plural=Notifications,
	description={(deutsch: Benachrichtigung) Eine Benachrichtigung an den \gls{Benutzer}, die auf der \gls{Weboberflaeche} am Seitenrand eingeblendet wird und mit der Maus weggeklickt werden kann bzw. nach vorgegebenem Zeitlimit von selbst verschwindet}
}

\newglossaryentry{Tooltip}
{
	name=Tooltip,
	description={Eine kleine Box mit Informationen, die beim \Gls{hovern} nahe der Maus auf der \gls{Weboberflaeche} eingeblendet wird}
}