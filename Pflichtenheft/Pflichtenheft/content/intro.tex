\section{Einleitung}
\paragraph{Aufgabenstellung}
Ziel der Praxis der Software-Entwicklung (PSE) ist die Entwicklung eines mittelgroßen Systems im Team mit objektorientierter Softwaretechnik. Dieses soll circa 10~kLOC umfassen. Hierzu wird ein objektorientierter Entwurf auf Basis von UML-Diagrammen geschaffen und unter Nutzung von Java, JavaScript, HTML und SQL implementiert. Im Anschluss soll mit JCov oder JUnit die Qualitätssicherung durchgeführt werden.\\
Das Projekt \enquote{Studienplanung als Generierung von Workflows mit Compliance"=Anforderungen: Planerstellung und Visualisierung} umfasst die Entwicklung einer webbasierten Benutzeroberfläche eines ebenfalls zu entwickelnden Systems zur Studienplanung. Sinn dieses Systems ist das Aufstellen von Studienplänen, angepasst an Bedürfnisse, bereits erbrachte Leistungen und Zeitmöglichkeiten des Studenten. Dies soll sowohl manuell, als auch automatisch möglich sein. Die Algorithmen zur Generierung und Verifizierung von Workflows unter Berücksichtigen von Constraints sind in Form von Black Boxes gegeben und können auf die Studienpläne angewendet werden. Das gesamte System soll modular und gut erweiterbar sein.\\
\paragraph{Im Detail heißt das:}
Die graphische Oberfläche des Systems soll intuitiv bedienbar und benutzerzentriert gestaltet werden. Als Benutzer sind Studenten zu erwarten. Eine zusätzliche Dozentenoberfläche soll modular hinzugefügt werden können. Das System soll dem Nutzer auf dem bisherigen Studienverlauf basierend Vorschläge in Form von Studienplänen zur Planung der nächsten Semester liefern. Der Studienplan soll als Ablauf(Workflow) aufgefasst werden, so dass die Black-Boxes zur Generierung und Verifizierung von Workflows eingebunden werden können. Die gegebenen Daten müssen hierzu passend umgewandelt und erhaltene Ergebnisse für den Nutzer verständlich verwertet werden. Als Aktivitäten der Workflows werden die Module und andere Lehrveranstaltungen aufgefasst. Module werden mit ihrem Namen, ECTS-Punkten, Angebot im Winter- oder Sommersemester und Art der Veranstaltung modelliert. Die Module sollen zu einem sinnvollen Workflow zusammengefügt werden. Die Workflows werden von Constraints beeinflusst. Dies sind: die Unterscheidung zwischen Pflicht- und Wahlveranstaltungen, Wahl eines Vertiefungsfaches, Abhängigkeiten zwischen Modulen (Voraussetzungen, Zusammenhänge, Überschneidungen), zur Verfügung stehende Semesteranzahl, gewünschte Module, bisheriger Studienverlauf, weitere gewünschte Eigenschaften.\\
Auch graphisch sollen die Studienpläne als Workflows dargestellt werden.