\section{Einleitung}
\paragraph{Aufgabenstellung}
Das Projekt \enquote{Studienplanung als Generierung von Workflows mit Compliance"=Anforderungen: Planerstellung und Visualisierung} umfasst die Entwicklung eines Systems zur Studienplanung mit einer webbasierten Benutzeroberfläche. Zweck des Systems ist das Erstellen von \glslink{Studienplan}{Studienplänen}, angepasst an Bedürfnisse, bereits erbrachte Leistungen und die zeitlichen Möglichkeiten des Studenten. Dies soll sowohl manuell, als auch automatisch möglich sein. Die Algorithmen zur \gls{Generierung} und \gls{Verifizierung} von Workflows unter Berücksichtigen von \glspl{Constraint} sollen in einfacher Form implementiert und auf die \glspl{Studienplan} angewendet werden. Das gesamte System soll \gls{modular} und gut erweiterbar sein.\\
\paragraph{Im Detail heißt das:}
Die graphische Oberfläche des Systems soll intuitiv bedienbar und benutzerzentriert gestaltet werden. Als \gls{Benutzer} sind Studenten zu erwarten. Eine zusätzliche Dozentenoberfläche soll \gls{modular} hinzugefügt werden können. Das System soll dem \gls{Benutzer}, auf dem bisherigen Studienverlauf basierend, Vorschläge in Form von \glslink{Studienplan}{Studienplänen} zur Planung der nächsten Semester liefern. Der \gls{Studienplan} soll als Ablauf(Workflow) aufgefasst werden, um Algorithmen zur \gls{Generierung} und \gls{Verifizierung} von Workflows nutzen zu können. Die gegebenen Daten müssen hierzu passend umgewandelt und erhaltene Ergebnisse für den \gls{Benutzer} verständlich verwertet werden. Als Aktivitäten der Workflows werden die \glspl{Modul} aufgefasst. \glspl{Modul} werden mit ihrem Namen, \glslink{ECTS-Punkte}{ECTS-Punkten}, Angebot im Winter- oder Sommersemester und Art der Veranstaltung modelliert. Die \glspl{Modul} sollen zu einem sinnvollen Workflow zusammengefügt werden. Die Workflows müssen \glspl{Constraint} erfüllen. Dies sind: die Unterscheidung zwischen Pflicht- und Wahlveranstaltungen, Wahl eines Vertiefungsfaches, Abhängigkeiten zwischen \glslink{Modul}{Modulen} (Voraussetzungen, Zusammenhänge, Überschneidungen), zur Verfügung stehende Semesteranzahl, gewünschte \glspl{Modul}, bisheriger Studienverlauf, weitere gewünschte Eigenschaften.\\