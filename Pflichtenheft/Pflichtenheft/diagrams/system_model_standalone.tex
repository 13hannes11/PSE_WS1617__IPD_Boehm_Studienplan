\documentclass{standalone}
\usepackage[utf8]{inputenc} % use utf8 file encoding for TeX sources
\usepackage[T1]{fontenc}    % avoid garbled Unicode text in pdf
\usepackage{palatino}	      % because "Computer Modern" standard font is illegible
\usepackage{mathpazo}
\usepackage[ngerman]{babel}  % german hyphenation, quotes, etc
\usepackage{hyperref}       % detailed hyperlink/pdf configuration
\hypersetup{                % ‘texdoc hyperref‘ for options
	pdftitle={PSE: Pflichtenheft},%
	bookmarks=true,%
}
\usepackage{graphicx}       % provides commands for including figures
\usepackage{csquotes}       % provides \enquote{} macro for "quotes"
\usepackage[nonumberlist, numberedsection]{glossaries}     % provides glossary commands
\usepackage{enumitem}
\usepackage{tikz}
\usepackage{tikz-uml}
\usepackage{comment}
\usepackage{array}
\usepackage{longtable}
\usepackage{hhline}
\usepackage{placeins}
\usepackage{needspace}
\usepackage[toc, page]{appendix}
\begin{document}
\usetikzlibrary{shapes.geometric}
\begin{tikzpicture}[
	>=stealth,
	node distance=3cm,
	database/.style={
		cylinder,
		cylinder uses custom fill,
		cylinder body fill=yellow!50,
		cylinder end fill=yellow!50,
		shape border rotate=90,
		aspect=0.25,
		draw
	}
	]
	\begin{umlpackage}{Server}
	\begin{umlcomponent}[x=0, y=0]{Programm-Logik}
	\end{umlcomponent}
	\node[database] (db1) at (-2,-2) {{\tiny Benutzerdaten}};
	\node[database] (db2) at (2,-2) {{\tiny Moduldaten}};
	\umlVHassemblyconnector[last arm, dashed]{Programm-Logik}{db1}
	\umlVHassemblyconnector[last arm, dashed]{Programm-Logik}{db2}
	\end{umlpackage}
	
	\begin{umlpackage}[x=-8, y=0]{Client}
	\begin{umlcomponent}[name=browser]{Internetbrowser}
	\begin{umlcomponent}[name=ob]{Weboberflaeche}
	\end{umlcomponent}
	\end{umlcomponent}
	\end{umlpackage}
	\umlassemblyconnector[interface=REST-Webservice]{Programm-Logik}{ob}
\end{tikzpicture}
\end{document}