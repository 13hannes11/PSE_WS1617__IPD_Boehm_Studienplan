\section{Virtuelle Maschine}
Die Virtuelle Maschine (VM) kann mit VirtualBox importiert werden. Es muss nur der Webbrowser innerhalb der VM geöffnet werden.

\section{Client}
Firefox und Chrome müssen auf dem System bereits installiert sein!
\subsection{Linux (Ubuntu)}
\underline{\textbf{Es könnte sein, dass die Befehle als root ausgeführt werden müssen (\texttt{sudo [Befehl]})}}

\begin{enumerate}
	\item Installiere Apache: 	
	\begin{json}
		apt-get install apache2 
	\end{json}
	\item Ersetze Zeile mit \texttt{DocumentRoot [...]} in 
		\begin{json}
			/etc/apache2/sites-enabled/000-default.conf
		\end{json}
	durch
	
	\begin{json}
	DEFINE WEBAPPROOT "[Pfad zu Repository]/Implementierung/Client/WebApp"
	Include [Pfad zu Repository]/Implementierung/Client/Conf/apache.conf 
	\end{json}

	\setcounter{enumi}{2}
	\item Ordnerberechtigungen: \texttt{chmod -R +x [Pfad zu Repository] }
	\item Apache Service neustarten: \texttt{service apache2 restart}
	\item Installiere Ant \texttt{apt-get install ant}
	\item Installiere ruby \texttt{apt-get install ruby}
	\item Installiere npm \texttt{apt-get install npm}
	\item Installiere grunt via npm \texttt{npm install -g grunt}
	\item Erstelle Verknüpfung: \texttt{ln -s /usr/bin/nodejs /usr/bin/node}
	\item Enable mod-rewrite \texttt{ln -s /etc/apache2/mods-available/rewrite.load /etc/apache2/mods-enabled/rewrite.load}
\end{enumerate}
Im Verzeichnis WebApp:
\begin{enumerate}
	\item Installiere sass via gem \texttt{gem install sass}
	\item \texttt{npm install}
	\item \texttt{npm install -g karma-cli}
\end{enumerate}
Brackets installieren
\begin{enumerate}
	\item Brackets ppa hinzufügen \texttt{add-apt-repository ppa:webupd8team/brackets}
	\item \texttt{apt-get update}
	\item Brackets installieren \texttt{apt-get install brackets}
	\item Öffne Brackets und, wenn erwünscht, installiere die Plugins 
	\item \textit{Annotate} und \textit{Brackets Git}		
\end{enumerate}
\subsection{Windows}
\begin{enumerate}
\item Lade \url{http://www.microsoft.com/de-de/download/details.aspx?id=29} herunter und installiere es
\item Lade Apache 2.4.25 (wenn 64-Bit"=Betriebssystem, dann unbedingt 64-Bit-Version) von \url{http://www.apachehaus.com/cgi-bin/download.plx} herunter
\item Extrahiere das Verzeichnis Apache24 in C:\textbackslash oder äquivalent
\item Setze in \texttt{C:/Apache24/conf/httpd.conf} in der Zeile \texttt{Define SRVROOT[...]} den Wert \texttt{Define SRVROOT "[Verzeichnis von Apache24]"} 
\subitem (wenn in C:\textbackslash{}Apache24, setze \texttt{Define SRVROOT C:\textbackslash{}Apache24})
\item Eventuell Zeile mit \texttt{LoadModule rewrite\_module modules/mod\_rewrite.so} (o.ä.) 
\item Ersetze in httpd.conf (siehe 4.) die Zeilen

\begin{json}
DocumentRoot "${WEBAPPROOT}/htdocs"
<Directory "${WEBAPPROOT}/htdocs">
#
# Possible values for the Options directive are "None", "All",
# or any combination of:
#   Indexes Includes FollowSymLinks SymLinksifOwnerMatch ExecCGI MultiViews
#
# Note that "MultiViews" must be named *explicitly* --- "Options All"
# doesn't give it to you.
#
# The Options directive is both complicated and important.  Please see
# http://httpd.apache.org/docs/2.4/mod/core.html#options
# for more information.
#
Options Indexes FollowSymLinks

#
# AllowOverride controls what directives may be placed in .htaccess files.
# It can be "All", "None", or any combination of the keywords:
#   Options FileInfo AuthConfig Limit
#
AllowOverride None

#
# Controls who can get stuff from this server.
#
Require all granted
</Directory>
\end{json}

(oder so ähnlich -- wichtig ist das /htdocs, darauf kann man sich verlassen) durch
\begin{json}
DEFINE WEBAPPROOT "[Pfad zu Repository]/Implementierung/Client/WebApp"
Include "[Pfad zu Repository]/Implementierung/Client/Conf/apache.conf"
\end{json}
Aktiviere mod\_rewrite in der httpd.conf.
\item Öffne CMD als Administrator in Verzeichnis \texttt{[...]/Apache24/bin/} führe aus: \texttt{httpd -k install} (dies installiert Apache 2.4 als Service) 
\item Öffne ApacheMonitor (befindet sich in \texttt{[...]/Apache24/bin/}) und starte Apache 2.4 (ggf. alles bejahen was kommt). \newline\textbf{Achtung:} Hierfür darf kein Programm den Port 80 blockieren -- insbesondere Skype macht das in der Standardeinstellung gerne mal! 
\item Öffne im Webbrowser \url{http://localhost}: Ein Fenster mit dem Inhalt \texttt{It works: Studyplan Client} sollte sich öffnen
\end{enumerate}
\begin{enumerate}
\item Installiere Brackets von \url{http://brackets.io/}
\item Öffne Brackets und installiere die Plugins \textit{Annotate} und, wenn erwünscht, \textit{Brackets Git}
\item Installiere ant von \url{http://ant.apache.org/bindownload.cgi} und füge das bin-Verzeichnis der PATH"=Umgebungsvariable hinzu
\item Installiere nodejs von \url{https://nodejs.org/en/download/} und wähle bei Installation aus, dass die PATH"=Variable gesetzt werden soll
\item Installiere \url{https://github.com/petetnt/brackets-sass-lint\#readme} mit der brackets-npm-registry
\item Für SASS/SCSS benötigen wir Ruby, deshalb installieren wir es von \url{http://rubyinstaller.org/} und fügen es während der Installation dem PATH hinzu
\item Öffne CMD als Administrator in Verzeichnis \newline\texttt{[Pfad zu Repository]/Implementierung/Client/WebApp}
\item Führe aus: \texttt{gem install sass} (Das installiert sass)
\item \texttt{npm install}
\item \texttt{npm install -g karma-cli}
\end{enumerate}

\subsection{Kompilieren}
\begin{enumerate}
	\item Kompilieren mit 
	\begin{json}
	ant -DAPI_DOMAIN=[server_ip]:[port]/studplan/rest 
	\end{json}
	\item Mögliche Flags:
	\begin{json}
	- DAPI_DOMAIN -> server url
	- DAPI_KEY -> client_id
	- DAPI_SCOPE -> scope bei Login-Anfrage
	- DAPI_TOKEN -> default access token
	- DDEBUG_ALWAYS_LOGIN -> beim Start eingeloggt mit default access token
	\end{json}
\end{enumerate}

