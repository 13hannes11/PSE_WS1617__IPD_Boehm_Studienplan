\section{Testfälle}
In der Implementierungsphase wurden bereits erste Integrationstests durchgeführt. So wurden fast alle möglichen Nutzeraktionen einmal durchgespielt, um eine eine grundlegende Qualität der Implementierung zu gewährleisten.
\subsection{Server}
Auf dem Server wurden noch nicht für alle Klassen Unittests erstellt, da es sich hauptsächlich um Klassen zur Darstellung von REST-Schnittstellen und simple Java-Objekte handelt, die nur Getter und Setter besitzen. Die tatsächliche Funktionalität kann erst in Verbindung mit einer Datenbank getestet werden. Diese Integrationstests werden in Qualitätssicherungsphase stattfinden.


Unittests:

\begin{longtable}{| >{\hspace{0pt}} p{.26\textwidth} | >{\hspace{0pt}} p{.45\textwidth} | >{\hspace{0pt}} p{.19\textwidth} |}
	\hline
	\textbf{Testklasse} & \textbf{Beschreibung} & \textbf{Status} \\ 
	\hhline{|=|=|=|}  
	\endfirsthead
	\endhead
	CategoryTest & Getter und Setter für \texttt{Category} getestet & ERFOLGREICH \\
	\hline
	DisciplineTest & Getter und Setter für \texttt{Discipline} getestet & ERFOLGREICH \\
	\hline
	SemesterTest & Test der Semester-Abstandsberechnung zwischen:
	\begin{itemize}
		\item zwei Wintersemestern
		\item zwei Sommersemestern
		\item zwischen Winter- und Sommersemester
		\item zwischen Sommer- und Wintersemester
	\end{itemize}
	Test der compareTo-Methoden & ERFOLGREICH \\
	\hline
	StandardVerifierTest & Testet, ob der \texttt{StandardVerifier} folgendes erkennt:
	\begin{itemize}
		\item fehlende Pflichtmodule
		\item Verletzung von Rule-Group"=Bedingungen
		\item Verletzung von Bereichsbedingungen (Fields)
	\end{itemize} & ERFOLGREICH \\
	\hline
	StandardVerifierConstraintTest & Testet ob der die Verletzung folgender \texttt{ConstraintTypes} in unterschiedlichen Ausartungen erkennt:
	\begin{itemize}
		\item Verletzung von Überlappung 
		\item Verletzung von Plan"=Zusammengehörigkeit 
		\item Verletzung von Semester"=Zusammengehörigkeit 
		\item Verletzung von Voraussetzungen
	\end{itemize}
	Dadurch wurden auch die \texttt{isValid}-Methoden der entsprechenden \texttt{ConstraintTypes} getestet & ERFOLGREICH \\
	\hline
		SimpleGeneratorTest & Testet einzelne Methoden des \texttt{SimpleGenerator}. 
	Ein Plan mit einem einzelnen \texttt{ModuleEntry} wird:
	\begin{itemize}
		\item anhand der Benutzer"=Präferenzen und \texttt{Constraint}s verschiedener Art vervollständigt und modifiziert
		\item anhand einer Zielfunktion optimiert
	\end{itemize}
	Dafür werden sog. Mock-Objekte benutzt, die nur zum Testen erstellt wurden, d.h. bei der echten Verwendung des Systems werden diese nicht benutzt werden, sondern stattdessen die Daten aus der Datenbank. Das Verhalten von Methoden der Klassen \texttt{ModuleDao} und \texttt{Plan} wurde hierfür mittels Mockito angepasst. & ERFOLGREICH\\
	\hline
	NodesListTest & Testet die topologische Sortierung der Klasse \texttt{NodesList}. & ERFOLGREICH\\
	\hline
	VerificationManager & Getter für \texttt{Verifier} getestet. & ERFOLGREICH\\
	\hhline{|=|=|=|} 
\end{longtable}

\subsection{Client}
Auf der Client"=Seite wurde mit Hilfe der Frameworks Karma und Jasmine.js getestet.
Während der Implementierungsphase wurde insbesondere das Modell mit Unit- und Integration-Tests überprüft.
Dies war notwendig, da zu diesem Zeitpunkt noch nicht auf eine REST-Schnittstelle zu gegriffen werden konnte.
Bei diesen Tests erreichten wir für die Modell-Klassen eine Testabdeckung von ca. 65~Prozent. Weitere Tests werden in der Qualitätssicherungsphase folgen.
Für die Integrationstests war es die Zielsetzung, dass jede Anfrage bzw. Antwort an bzw. vom Server durch einen Testfall abgedeckt ist.
Dieses Ziel wurde erreicht, wobei alle Tests erfolgreich laufen.

\begin{longtable}{| >{\hspace{0pt}} p{.2\textwidth} | >{\hspace{0pt}} p{.45\textwidth} | >{\hspace{0pt}} p{.25\textwidth} |}
	\hline
	\textbf{Testfall} & \textbf{Fragestellung} & \textbf{Status} \\ 
	\hhline{|=|=|=|}  
	\endfirsthead
	\endhead
	
	%===============================================================================================
	ModuleCollection"=Initialisierung & Wird eine ModuleCollection für gegebene Daten erfolgreich initialisiert?  & ERFOLGREICH \\ \hline
	ModuleConstraint"=Initialisierung & Wird ein ModuleConstraint für gegebene Daten erfolgreich initialisiert? & ERFOLGREICH \\ \hline
	Module"=Initialisierung & Wird ein Modul für gegebene Daten erfolgreich initialisiert? & ERFOLGREICH \\ \hline
	Preference"=Initialisierung & Wird eine Preference für gegebene Daten erfolgreich initialisiert? & ERFOLGREICH \\ \hline
	Plan"=Initialisierung & Wird ein Plan für gegebene Daten erfolgreich initialisiert? & ERFOLGREICH \\ \hline
	Discipline"=Initialisierung & Wird eine Discipline für gegebene Daten erfolgreich initialisiert? & ERFOLGREICH \\ \hline
	Filter"=Initialisierung & Wird ein Filter für gegebene Daten erfolgreich initialisiert? & ERFOLGREICH \\ \hline
	LanguageManager"=Funktion & Gibt der Language Manager die richtigen Textbausteine zurück? & ERFOLGREICH \\ \hline
	NotificationCollection"=Initialisierung & Gibt getInstance() das richtige Objekt zurück? & ERFOLGREICH \\ \hline
	ObjectiveFunctionCollection"=Initialisierung & Wird eine ObjectiveFunctionCollection für gegebene Daten erfolgreich initialisiert? & ERFOLGREICH \\ \hline
	TemplateManager"=Funktion & Gibt der TemplateManager die richtige HTML"=Ausgabe zurück? & ERFOLGREICH \\ \hline
	CookieSync"=Funktion & Speichert/Lädt CookieSync ein Modell erfolgreich? & ERFOLGREICH \\ \hline
	OAuthSync"=Funktion & Übergibt OAuthSync beim Speichern/Laden die richtigen Header? & ERFOLGREICH \\ \hline
	MainView"=Funktion & Lädt der MainView die richtigen Views und zeigt diese an? & ERFOLGREICH \\	 
	\hhline{|=|=|=|}   
	\caption{Geschriebene Unit-Tests}
\end{longtable}