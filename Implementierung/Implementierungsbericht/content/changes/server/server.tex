\subsubsection{Model-Paket}

\paragraph{Paket moduledata.*} Es wurden die Klassen Field und RuleGroup hinzugefügt, um die Modelländerungen abzubilden. Im ModuleDao-Interface wurden einige Bequemlichkeitsmethoden zum Finden von Kategorien, Bereichen, Vertiefungsfächern u. ä. hinzugefügt. Eine ConditionQueryConverter-Klasse unterstützt nun die Umwandlung von Condition-Objekten in SQL"=Statements, die dann mit Hibernate an die Datenbank geschickt werden können.



\subsubsection{Filter-Paket}
\label{subsubsec:filter}

Das Filter-Paket wurde grundlegend umgestaltet: Ein Filter enthält mittlerweile nur die Daten, die zum Filtern benötigt werden (wie z.B. konkret selektierte Elemente). Client-bezogene Daten (wie beispielsweise zulässige Intervallschranken oder eine Liste aller Auswahlelemente) liegen ausschließlich in den Filter-Deskriptoren. Diese kümmern sich auch um De- und Serialisierung von Filtern bei Anfragen.

\paragraph{Paket condition} Da sich die jOOQ-Condition-Klasse als nicht sonderlich weiterverwertbar bzw. handhabbar erwies, wurde sie durch eine eigene Condition-Klasse ersetzt, welche die drei nötigen Bedingungstypen als Binäroperatoren modelliert. Jeder Filter liefert nun eine Liste von Conditions zurück. Dadurch lassen sich SQL-Statements einfacher aus den Condition-Objekten zusammenbauen.


\subsubsection{Generation-Paket}

Die NodeWithoutOutput-Klasse wurde entfernt (da wir zwischen den Knoten mit und ohne Kindern nicht zu unterscheiden brauchen). 
Dafür wurde die Klasse NodesList hinzugefügt (zur Speicherung der Knoten).


\subsubsection{Pluginmanager-Paket}

Auf Implementierung eines Plugin-Systems wurde verzichtet.
Standard-Generierer und -Verifizierer sind somit fest in die Software eingebaut (können aber nach wie vor später durch Black-Boxes ersetzt werden). GenerationManager und VerificationManager enthalten also fest verdrahtet unsere Nachbildungen, da nicht davon auszugehen ist, das eine Änderung des Generierungstools im laufenden Betrieb stattfindet.


\subsubsection{Verification-Paket}

Durch die sich geänderten Anforderungen wegen der Modelländerungen wurde VerificationResult um Attribute zum Festhalten von Field-, Rule-Group- und Compulsory-Violations erweitert und die Verifizierung entsprechend angepasst.


\subsubsection{REST-Paket}

Zur Verwaltung von Datenbank-Sessions wurden die Klassen SessionOpenFilter und SessionCloseFilter hinzugefügt, welche bei ankommender Anfrage bzw. vor abgeschickter Antwort eine Datenbank-Session eröffnen bzw. wieder schließen. Damit ist gewährleistet, dass während der Verarbeitung einer Anfrage stets auf die Datenbank zugegriffen werden kann und DAOs stets verfügbar und einsatzbereit sind.

Ferner wurde die Annotation @AuthorizationNeeded hinzugefügt, die eine REST-Ressourcenklasse bzw. deren Methoden derart markiert, dass Jersey beim Verarbeiten einer Anfrage mithilfe des AuthorizationRequestFilters User-Informationen bereitstellen kann, die zur Abwicklung einer solchen Anfrage notwendig sind. Eine AuthorizationContextFactory-Klasse hilft dabei beim Injizieren eines AuthorizationContexts in die Ressource.

Die Klasse MyObjectMapperProvider konfiguriert einen ObjectMapper, der die De-/Serialisierung abwickelt. ValidationConfigContextResolver aktiviert BeanValidation, welche die Überprüfung speziell annotierter Attribute beim Deserialisieren von JSON-Objekten veranlasst.


\paragraph{Paket authorization.endpoint} Die Klasse GrantTypeFactory dient nun als Multiplexer für die verschiedenen GrantTypes.


\paragraph{Paket resources.*} In diesem Paket liegen die REST-Ressourcen-Klassen. Mehrere Ressourcen-Klassen wurden aus technischen Gründen als Sub-Ressourcen in Oberklassen eingegliedert, bspw. PlanVerificationResource in PlansResource. Es sind -- in Zusammenhang mit den Änderungen in \ref{subsec:changes_rest} -- neue Ressourcen-Klassen hinzugekommen, welche auf neue Anfragen reagieren, wie z.~B. FieldsResource oder MainResource. \newline
Im Sub-Paket „json“ befinden sich Klassen zur De- und Serialisierung von JSON-Datenklassen, sogenannte Data-Transfer-Objects (DTOs). Diese dienen der korrekten Einhaltung der JSON-Spezifikationen und ermöglichen eine reibungslose Umwandlung von bzw. zu ihren Modell-Gegenstücken. \newline
Da der PDF-Export einer HTML-Generierung gewichen ist, wurden hierfür die Klassen DisplayablePlan und PlanConverter angelegt, welche mithilfe der Apache VelocityEngine und einem Report-Template die HTML- und CSS-Ausgabe generieren.



\subsubsection{Sonstiges}

Die Klassen HibernateSessionFactoryListener und SessionCloseListener wurden hinzugefügt, um beim Starten bzw. Beenden des Webservices die Datenbankverbindung einmalig herzustellen bzw. zu beenden.

In der Klasse Utils befinden sich einige Hilfsmethoden, u.a. zur kontrollierten DAO-Nutzung.