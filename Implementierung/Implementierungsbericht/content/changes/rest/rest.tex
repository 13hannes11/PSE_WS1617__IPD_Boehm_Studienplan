Nachfolgend eine Auflistung aller atomaren Werte und JSON"=Datenklassen, deren Definitionen sich geändert haben.

\subsubsection*{JSON-Atome}

\begin{longtable}{p{.22\linewidth} p{.73\linewidth}}
	Bezeichner
	& Erklärung \\
	\hline
	\endfirsthead
	
	Bezeichner
	& Erklärung \\
	\hline 
	\endhead
	
	\hline
	\endlastfoot
	
	\lbljsonatom{Modul-Turnus} 
	& $ \in \{\verb|"WT"|, \verb|"ST"|, \verb|"both"|\}$. \\
	\lbljsonatom{Semester-Typ} 
	& $ \in \{\verb|"WT"|, \verb|"ST"|\}$. \\
	\lbljsonatom{Semester-Zahl}
	& Zahl des aktuellen Semesters des Benutzers. \\
	\lbljsonatom{Filter-URI-Identifier}
	& String, mit welchem der Filter in GET-Parametern identifiziert wird. \\

\end{longtable}

\subsubsection*{JSON-Datenklassen}

\begin{lstlisting}[language=json]
(*\lbljsonobj{Student}*) = {
	"discipline": (*\jsonpart{Studienfach}{id}*),
	"study-start": (*\jsonobj{Studienbeginn}*),
	"passed-modules": (*\jsonlist{\jsonpart{Modul}{id, name, creditpoints, lecturer, semester}}*),
	"current-semester": (*\jsonatom{Semester-Zahl}*)
}
\end{lstlisting}

\begin{lstlisting}[language=json]
(*\lbljsonobj{Modul}*) = {
	"id": (*\jsonatom{Modul-ID}*),
	"name": (*\jsonatom{Modul-Name}*),
	"categories" : (*\jsonlist{\jsonobj{Kategorie}}*),
	"semester": (*\jsonatom{Modul-Semester}*),
	"cycle-type": (*\jsonatom{Modul-Turnus}*),
	"creditpoints": (*\jsonatom{Modul-Creditpoints}*),
	"lecturer": (*\jsonatom{Modul-Dozent}*),
	"preference": (*\jsonatom{Modul-Präferenz}*),
	"compulsory": (*$\langle$*)true(*$\vert$*)false(*$\rangle$*),
	"description": (*\jsonatom{Modul-Beschreibung}*),	
	"constraints": (*\jsonlist{\jsonobj{Constraint}}*)
}
\end{lstlisting}

\begin{lstlisting}[language=json]
(*\lbljsonobj{Filter}*) = {
	"id": (*\jsonatom{Filter-ID}*),
	"name": (*\jsonatom{Filter-Name}*),
	"uri-name": (*\jsonatom{Filter-URI-Identifier}*),
	"default-value": (*\jsonatom{Filter-Default}*),
	"tooltip": (*\jsonatom{Filter-Tooltip}*),
	"specification": (*\jsonobj{Filter-Eigenschaften}*)
}
\end{lstlisting}


\begin{lstlisting}[language=json]
(*\lbljsonobj{Studienplan}*) = {
	"id": (*\jsonatom{Studienplan-ID}*),
	"status": (*\jsonatom{Studienplan-Status}*),
	"creditpoints-sum": (*\jsonatom{Studienplan-Gesamt-Creditpoints}*),
	"name": (*\jsonatom{Studienplan-Name}*),
	"modules": (*\jsonlist{\jsonpart{Modul}{id, name, semester, creditpoints, cycle-type, lecturer, preference}}*),
	"violations": (*\jsonlist{\jsonobj{Constraint}}*),
	"field-violations": (*\jsonlist{\jsonpart{Field}{name, min-ects}}*),
	"rule-group-violations": (*\jsonlist{\jsonobj{Rule-Group}}*),
	"compulsory-violations": (*\jsonlist{\jsonpart{Modul}{name}}*)
}
\end{lstlisting}

\begin{lstlisting}[language=json]
(*\lbljsonobj{Field}*) = {
	"id": (*\jsonatom{Field-ID}*),
	"name": (*\jsonatom{Field-Name}*),
	"min-ects": (*\jsonatom{Field-Mindest-ECTS}*),
	"categories": (*\jsonlist{\jsonobj{Kategorie}}*),
}
\end{lstlisting}

\begin{lstlisting}[language=json]
(*\lbljsonobj{Rule-Group}*) = {
	"name": (*\jsonatom{Rule-Group-Name}*),
	"min-num": (*\jsonatom{Rule-Group-Modul-Mindestanzahl}*),
	"max-num": (*\jsonatom{Rule-Group-Modul-Höchstanzahl}*)
}
\end{lstlisting}

\begin{json}
(*\lbljsonobj{ModulesResult}*) = {
	"modules": (*\jsonlist{\jsonpart{Modul}{id, name, creditpoints, lecturer, cycle-type}}*)
}	
\end{json}

\begin{json}
(*\lbljsonobj{ModuleResult}*) = {
	"module": (*\jsonpart{Modul}{id, name, categories, cycle-type,  creditpoints, lecturer, compulsory, description, constraints}*)
}	
\end{json}

\begin{json}
(*\lbljsonobj{StudentPutRequest}*) = {
	"student": (*\jsonpart{Student}{discipline, study-start, passed-modules}*)
}	
\end{json}

\begin{json}
(*\lbljsonobj{PlanResult}*) = {
	"plan": (*\jsonpart{Studienplan}{id, name, status, creditpoints-sum, modules}*)
}	
\end{json}

\begin{json}
(*\lbljsonobj{PlanPutRequest}*) = {
	"plan": (*\jsonpart{Studienplan}{id, name, modules[id, semester]}*)
}	
\end{json}

\begin{json}
(*\lbljsonobj{PlanPutResult}*) = {
	"plan": (*\jsonpart{Studienplan}{id, name, modules[id, semester], status}*)
}	
\end{json}


\begin{json}
(*\lbljsonobj{PlanModulesResult}*) = {
	"modules": (*\jsonlist{\jsonpart{Modul}{id, name, creditpoints, lecturer, cycle-type, preference}}*)
}	
\end{json}

\begin{json}
(*\lbljsonobj{PlanModuleResult}*) = {
	"module": (*\jsonpart{Modul}{id, name, categories, cycle-type, creditpoints, lecturer, preference, compulsory, description, constraints}*)
}
\end{json}

\begin{json}
(*\lbljsonobj{PlanVerificationResult}*) = {
	"plan": (*\jsonpart{Studienplan}{id, status, violations, field-violations, rule-group-violations, compulsory-violations}*)
}
\end{json}

\begin{json}
(*\lbljsonobj{PlanProposalResult}*) = {
	"plan": (*\jsonpart{Studienplan}{status, modules}*)
}
\end{json}

\begin{json}
(*\lbljsonobj{FieldsResult}*) = {
	"fields": (*\jsonlist{\jsonobj{Field}}*)
}
\end{json}
\subsubsection*{REST-Zugriffsstruktur}

Hinzugefügt:
\begin{longtable}{| >{\hspace{0pt}} p{.11\textwidth} | >{\hspace{0pt}} p{.17\textwidth} | >{\hspace{0pt}} p{.28\textwidth} | >{\hspace{0pt}} p{.34\textwidth} |}
	\hline
	\textbf{Methode} & \textbf{URI} & \textbf{Beschreibung} & \textbf{Anfrage-Parameter\+/Rückgabewerte} \\ 
	\hhline{|=|=|=|=|}  
	\endfirsthead
	
	\hline
	\textbf{Methode} & \textbf{URI} & \textbf{Beschreibung} & \textbf{Anfrage-Parameter\+/Rückgabewerte} \\ 
	\hline  
	\endhead
	
	\hhline{|=|=|=|=|}   
	\endlastfoot
	%===============================================================================================
	GET & / & Abfrage, ob der REST-Service antwortet & Anfrage: --- \newline Rückgabe: --- (200 OK) \\
	\hhline{|=|=|=|=|} 
	GET & /fields & Gibt alle Bereiche des zum Studenten gehörenden Studiengangs zurück & Anfrage: --- \newline Rückgabe: \jsonobj{FieldsResult} 
\end{longtable}

Entfernt:
\begin{itemize}
	\item POST /plans/\jsonatom{Plan-ID} \quad (Plan duplizieren): \newline
		  Kann mit DELETE, POST und PUT erreicht werden (dabei geht aber der Verifikationsstatus verloren, s.u.).
	\item GET /subjects: \newline
		  Rückgabe in GET /fields verfügbar.
\end{itemize}

Geändert:
\begin{itemize}
	\item GET /modules: \quad \jsonobj{ModulesResult} geändert (s.o.)
	\item GET /modules/\jsonatom{Module-ID}: \quad \jsonobj{ModuleResult} geändert (s.o.)
	\item PUT /plans/\jsonatom{Plan-ID} \quad (Plan ersetzen): \newline
		  Nimmt jetzt nur noch Namen und Modulbelegung entgegen, der Verifikationsstatus wird auf „nicht verifiziert“ gesetzt. \newline
		  Anfrage: \jsonobj{PlanPutRequest} (geändert, s.o.) \newline
		  Rückgabe: \jsonobj{PlanPutResult} (geändert, s.o.)
	\item GET /plans/\jsonatom{Plan-ID}: \quad \jsonobj{PlanResult} geändert (s.o.)
	\item GET /plans/\jsonatom{Plan-ID}/modules: \quad \jsonobj{PlanModulesResult} geändert (s.o.)
	\item GET /plans/\jsonatom{Plan-ID}/modules/\jsonatom{Modul-ID}: \quad \jsonobj{PlanModuleResult} geändert (s.o.)
	\item GET /plans/\jsonatom{Plan-ID}/verification: \quad \jsonobj{PlanVerificationResult} geändert (s.o.)
	\item GET /plans/\jsonatom{Plan-ID}/proposal/\jsonatom{Zielfunktion-ID}: \newline
		  Da sich die Modellierung der Vertiefungsfächer geändert hat (siehe Kap.~\ref{subsec:changes_database}), werden auch die GET-Parameter wie folgt angepasst: \newline
		  Anfrage-Parameter: 
		  \subitem max-semester-ects=\jsonatom{Semester-ECTS-Maximum} 
		  \subitem fields=\jsonatom{Field-ID},...,\jsonatom{Field-ID}  
		  \subsubitem (Auflistung gewählter Bereiche) 
		  \subitem field-$k$=\jsonatom{Vertiefungsfach-ID} 
		  \subsubitem (für jeden gewählten Bereich mit ID $k$) \newline
		  Rückgabe: \jsonobj{PlanProposalResult} (geändert, s.o.)
	\item GET /plans/\jsonatom{Plan-ID}/pdf: \newline
		  Zurückgeschickt wird ein Dokument vom Typ ``text/html''. Der Nutzer kann dieses entweder manuell in ein PDF umwandeln (von gängigen Browsern unterstützt) oder direkt ausdrucken.
	\item PUT /student: \newline
		  Beim Ersetzen der Studenteninformationen werden die bestandenen Module aus allen Plänen gelöscht, in denen sie vorkommen. Alle Pläne werden weiterhin als „nicht verifiziert“ markiert. \newline
		  Anfrage: \jsonobj{StudentPutRequest} (geändert, s.o.) 
	\item GET /student: \newline
		  Es wird zusätzlich die \jsonatom{Semester-Zahl} zurückgeschickt (s. \jsonobj{Student}, geändert)
\end{itemize}


